\documentclass[pdftex,12pt,a4paper]{article}

\input{/home/boris/Dropbox/Public/tex_general/title_bor_utf8}

%\usepackage{showkeys} % показывать метки

% \input{/home/boris/Dropbox/Public/tex_general/prob_and_sol_utf8}

\begin{document}
\parindent=0 pt

Теория игр. Game theory. \\
Неофициальная программа курса для студентов 2 курса ГУ-ВШЭ.\\

Требования к уровню предварительной подготовки включают умения:\\
\begin{itemize}
\item решать квадратные уравнения;
\item строить отрезки;
\end{itemize}


Из курса теории игр при желании ты узнаешь:\\
\begin{itemize}
\item почему Кортес сжег корабли;
\item что получение правдивой информации может снизить выигрыш игрока;
\item как воспользоваться предсказанием гадалки;
\item почему эмоции способствуют выживанию человека;
\item в какой ситуации последнему достанется лучший кусок;
\item что фильм <<Игры разума>> бесполезен при подготовке к экзамену;
\item почему фирмы обещают компенсировать разницу в цене;
\end{itemize}


Из курса теории игр ты, скорее всего, не узнаешь:
\begin{itemize}
\item сколько стоит один левый ботинок, по сравнению с двумя правыми;
\item кто же, все-таки, победил в битве полов;
\item кому достался доллар Рубинштейна;
\item кто выигрывает в шахматах, белые или черные;
\end{itemize}

Несмотря на почти полное отсутствие требований к уровню предварительной подготовки и свое название, теория игр - это серьезная математическая дисциплина. Здесь принято решать задачи.\\

В меню входит: две контрольных и домашние задачи\\

\end{document}