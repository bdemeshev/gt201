\documentclass[a4paper,12pt]{article}
\usepackage[autogenerated]{ucs} 
\usepackage[utf8]{inputenc}
\usepackage[russian,french]{babel}
\usepackage{graphicx}
\usepackage{amssymb}
\usepackage{epstopdf}
\usepackage{babel}
\newcommand{\eps}{\varepsilon}
\newcommand{\ff}{ F\hspace{-3mm} f }
\textheight=25cm \textwidth=16cm \oddsidemargin=0cm
\topmargin=-1.5cm \tolerance=5000
\parindent=0mm
\parskip=2mm
\def\eqdef{\buildrel\rm def\over=}
\def\R{{\bf R}}
\def\N{{\bf N}}
\def\лк {``}
\def\пк{''}
\newcommand{\Arg}{\mathop{\rm Arg}\nolimits}
\def\dd#1#2{\frac{\displaystyle\partial
#1}{\displaystyle\partial #2}}
\def\dr#1#2{\frac{\displaystyle#1\mathstrut}{\displaystyle#2\mathstrut}}
\unitlength=1mm
\def\nsk{\vspace{-2mm}}
\def\hi{\vphantom{\Big)}}
\def\hidr#1#2{\frac{\displaystyle#1\hi}{\displaystyle#2\hi}}
\def\phm{\hphantom{-}}
\def\st{{\rm st}\,}
%\input{tcilatex}
%\usepackage{latexcad}
\begin{document}
\selectlanguage{russian}
%\documentclass[12pt,openany]{book}
%\usepackage{game}
%\pagestyle{plain}
%\begin{document}{
%   begin zadachnik, March 2002 - June 2003

\title{ ...: {\huge Большой Задачник Игр РЭШ-НГУ-ВШЭ-ЕУСпб}}
\author{С. Коковин, A.Тонис, A. Савватеев и все-все-все  \thanks{Благодарность
Асе Каменской за помощь в переводе и техническую.} }
\maketitle

{\bf Концепция.} Этот задачник составляется для
нужд преподавания составителей (РЭШ, НГУ, ВШЭ,
Европ.Ун.СПб) из всех задач, доступных откуда-то.
То есть, задачник хочется "полный" по темам, как
энциклопедия, или хотя бы "большой", чтоб не
чувствовать недостатка в задачах всех типов. Цели
задачника: 1)задачи для наших семинаров, домашних и
контрольных работ, 2)задачи для самоподготовки,
прежде всего для наших курсов лекций, но не только,
for everybody. В том числе, могут быть задачи для
домашней работы на день-два, типа реферата.\medskip

{\bf Структура:} Есть разделы задачника по темам:
равновесие Нэша, и др. Внутри такого раздела задачи
усложняются от типовых до доказатльных. Также есть
тематические разделы задачника, где каждый касается
некоторой {\em содержательной} темы (сюжета). Ибо
часто разбираемый содержательный сюжет просится
протянуть его по всем понятийным темам (все типы
решений). Так, понятийная тема может идти
"пунктиром" по содержательным разделам, и наоборот.
Для гипертекстового обращения к подборкам по
понятийным темам, их надо размечать условными
обозначениями.\medskip

Задачи (или вопросы в задачах) будут разного
уровня/назначения: 1 уровень/тип: Просто находить разные
типы решений в конечных и непрерывных (по множеству
стратегий) играх, выбирать подходящую концепцию по
контексту, сопоставлять концепции между собой. 2
уровень/тип: Требуется выводить существование и свойства
решений игры (множественность, компактность...) из условий.
3 уровень/тип: По текстовой ситуации достраивать модель,
решать неполные или некорректные задачи, дополняя
недоданные условия или концепции решений.\vspace{2mm}

{\small suggestions from S.Kok:

{\bf Классификация задач и ее обозначения.}

0. {\em Имя игры и происхождение}, например, "Пираты"
(Мулен), или "Печенье" (РЭШ).

1. {\em Содержательный сюжет} игры формулируется по
разделам: абстрактные, спортивные (детские) и бытовые,
политические, Экономические: совершенные рынки и аукционы
(включая модель обмена и вероятностные), олигополии= рынки
с несовершенством конкуренции (включая вероятностные,
качество и рекламу), контракты, неполные контракты.

2. {\em Пространство стратегий} (вроде как оценка сверху
СЛОЖНОСТИ этого пространства) классифицируем так:

Например, 3[2(2*3*2)- (4*3*2)] означает: 3(игрока), [2(слоя
дерева) (2*3*2)- (4*3*2) (указано максимальное число
разветвлений в каждом слое для каждого игрока)].

Эту запись можно упрощать, опуская очевидное, например:

"Перекресток" = 2[2(2*2)]= [2*2] = "Дилемма заключенных"

"Террорист" = 2[1(2-2)]= [2-2]

"Дележ пирога по Рубинштейну" = 2[$N$*(R*R)...]

3. {\em Концепция решения} (понятийная тема)

Это тесно связано с информационной структурой: игры
популяционные, или повторяющиеся, или с информацией о
целях, и др.

Например, NForm, MM, NE, IND, Tree, SPE,
Irrationality, Bluff, }

\vspace{2mm}

- - - - - - - - - -
\newpage
\tableofcontents

\newpage

\section{NSU Kokovin}

\subsection{Обозначения игр и определения}

Далее всюду  \bedscr\item  $ \I := \{1,...,m \}$
---  множество участников $i=1,...,m$, \item  $X:=
(X_i)_{i\in \I}  := \times_{i} X_i = (X_1\times
X_2\times ...  \times  X_m)$ --- профиль допустимых
множеств стратегий участников,\index{стратегия}
 \item $u:= (u_i)_\I = (u_i)_{i\in \I }$ --- набор (профиль)
целевых функций участников (заметим: каждая целевая функция
$u_i:~X_i\mapsto \R$ зависит, вообще говоря, от всех
$(x_j)_{j\in I}$). \edscr


Проиллюстрируем используемые всюду далее принципы обозначений в
последовательных и одновременных играх на примере (используемом
впоследствии). В частности, введем "ожидания", "дерево игры" и
"таблицу выигрышей".

\begin{exmp}[``Футбол
или кино'' $\approx$ ``Battle of Sexes'': Luce, Raiffa, 1953]

{\rm Играют Анна (персонаж, который далее во всех
обсуждаемых динамических играх ходит первым и индексируется
как $i=A$) и Боб или Виктор  (персонажи, которые в других
играх, не как здесь, ходят {\em позже} Анны и,
соответственно, обозначаются буквами B, V стоящими {\em
позже} в латинском алфавите). Здесь Анна и Виктор ходят
одновременно, после хода "Природы", сформировавшей у них
какие-то "ожидания" (beliefs) о поведении партнера. Они не
имеют возможности переговариваться (возможно, это период
ухаживания ДО знакомства  ), и каждый выбирает, пойти ли
вечером на футбол или в кино. Они предпочли бы пойти
куда-нибудь вместе, что отражено в {\em таблице выигрышей}
на Рис.~\ref{BatSex}. А именно, совместное попадание в кино
$(~(x_A,x_V)=(c_A,c_V)=(cinema_A,cinema_V)~)$ дало бы
вектор удовлетворения или выигрышей равный
$(u_A(c_A,c_V),u_V(c_A,c_V))=(3,2)$, а совместное попадание
на футбол $(football_A,football_V)$ дает выигрыши
$(u_A(f_A,f_V),$ $u_V(f_A,f_V))=$ $(1,4)$.

\begin{center}
\begin{figure}[h]
%\unitlength 1.00mm .~~~~~~~~~~~~~~~\input{AnnVict.lp}
\caption{Пример обозначений в играх. Игра ``Футбол или
кино'' (``Battle of Sexes''). {Приносим извинения за
английские надписи на рисунках (они вынуждены и
используемыми программными средствами и латынью в
обозначениях переменных).} \label{BatSex}}
\end{figure}
\end{center}

В овалах "дерева игры" отмечены участники принимающие
решения в этой позиции, на стрелках - переменные, выбранные
участниками. В каждой клетке "таблицы выигрышей",
соответствующей одному из 4-х возможных исходов игры,
помещен сначала субъективный выигрыш строчного игрока --
Анны (измеренный в некоторых единицах полезности), затем -
выигрыш  столбцового игрока - Виктора. Стрелки отражают
последовательность ходов, в данном случае - то, что игроки
вынуждены принять решения одновременно, не зная выбора
другого, а только имея какие-то {\em ``ожидания''}
(beliefs) об этом выборе, предопределенные природой
(случаем).\vspace{2mm}

{\small Далее, как и здесь, мы будем большими буквами
обозначать участников (или множества), малыми латинскими
буквами -- переменные стратегий, стараясь сохранять
преемственность: какая переменная кому принадлежит.
Греческие буквы используются для ожиданий или вероятностей,
в данном случае  $\beta_V$ - это вектор ожиданий об Анне и
о себе (ожидания о себе нетривиальны, только если
рациональность неполная), а индекс указывает -- чье это
ожидание. Нечетные цифры выигрышей во многих наших примерах
соответствуют первому игроку, четные -
второму.}\vspace{2mm}

В данном случае, для предсказания исхода можно использовать
простейшую концепцию решения -- {\em решение с заданными заранее
ожиданиями ходов}, известными откуда-то нам -- предсказывающему
наблюдателю, причем ожидания участников могут не быть
``согласованными''. В частности, если оба ожидают от партнера
выбор ``футбол'', то есть ожидания о партнере имеют значения
$\nu_A=footb_V$, $\alpha_V=footb_A$,
%\footnote{Ожидания обычно будут обозначаться греческими буквами.}
тогда рациональный выбор каждого - присоединиться к выбору
партнера, и исходом будет счастливая (более счастливая для
Виктора) встреча на футболе: $x_A=footb_A,~x_V=footb_V$.
Аналогично, совпадающие гипотезы о кино привели бы к счастливой
(особенно для Анны) встрече в кино, а несовпадающие гипотезы --
к развлечениям порознь.

}\end{exmp}

\subsection{Наиболее употребительные определения}

\textbf{Максимин (ММ)} - исход игры (профиль
стратегий) при осторожном поведении всех, то есть
при максимизации гарантированных выигрышей, не
учитывая в своих расчетах целей и текущих решений
партнеров.\vspace{2mm}

\textbf{Решение в (слабо-) доминирующих стратегиях
(WDE)} или слабо-доминирующее равновесие - исход
игры в случае наличия у каждого
"абсолютно-оптимальной" стратегии, то есть
стратегии, (слабо) доминирующей над всеми другими
его стратегиями независимо от ходов партнеров, их
целей и текущих решений. [Аналогично и определение
сильно-доминирующего равновесия
\textbf{SDE}.]\vspace{2mm}

\textbf{Решение в итерационно-
(слабо-)недоминируемых стратегиях (IND${}_W$)} -
исход игры в случае одновременного итерационного
отбрасывания (слабо-) доминируемых стратегий каждым
игроком и соответствующего редуцирования игры:
исключения отброшенных стратегий из рассмотрения
ВСЕМИ игроками. Требует знания или целей партнеров
или факта отбрасывания стратегий. [Аналогично
определяется Решение в итерационно-
сильно-недоминируемых стратегиях
(IND${}_S$).]\vspace{2mm}

\textbf{Равновесие Нэша (NE}) -  исход игры
(профиль стратегий), при котором ни у одного игрока
нет стимула отступить от своей текущей стратегии,
при знании текущих стратегий партнеров и гипотезе,
что партнеры не отступят. [Эквивалентный вариант:
Равновесие Нэша - исход, когда все сходили
одновременно вслепую, имея лишь некоторые ожидания
о запланированном ходе партнеров, а когда карты
открылись, то все ожидания
оправдались.]\vspace{2mm}

\textbf{Совершенное в Подыграх Равновесие (Нэша)
(SPE} = SPNE) - это равновесие Нэша в развернутой
форме игры, являющееся также равновесием Нэша во
всех ее подыграх. (Внимание: оно может не являться
NE этой же игры в нормальной форме, поэтому не
всегда $SPE\subseteq NE$!)\vspace{2mm}

\textbf{Слабый оптимум Парето} ($W\Par$) -
возможный исход, который нельзя улучшить для всех
игроков сразу, даже согласовав их ходы.
\textbf{Сильный оптимум Парето} ($\Par$) - исход,
который нельзя улучшить для кого-то, не ухудшив для
других. \vspace{2mm}

\textbf{Элемент (слабого) Ядра игры} (${\bf C}$) -
возможный исход, который не блокируется ни одной
коалицией в переговорах. Коалиция блокирует в
переговорах (отвергает) вариант, если имеет другой,
строго более желательный для всех своих членов,
среди СВОИХ возможностей (среди вариантов,
достижимых независимо от действий вне-коалиционных
игроков). Т.е. Ядро - множество вариантов, вне
которого соглашений быть не может. \vspace{2mm}

{\bf Сокращения:}  $MM$ -- MaxiMin, $DE$ --
Dominant Equilibrium, $SDE$ -- Strong Dominant
Equilibrium, $IND_W$ -- Iterative (Weakly)
Non-Dominant Equilibrium,  $SoE$ -- Sophisticated
Equilibrium, $NE$ - Nash Equilibrium,  $NE_m$ --
Nash Equilibrium in Mixed stratagies, $SP(N)E$ --
Subgame Perfect (Nash) Equilibrium, $StE$ --
Stackelberg Equilibrium, $\Par$ - Pareto, $C$ --
Core. \vspace{2mm}

\newpage

Пока что сортировка по разделам и классификация не
проведены, задачи собраны по ПРОИСХОЖДЕНИЮ.

{\Large Задачи из НГУ:}

\subsection{Задачи статических (одновременных) игр}

\begin{exmp}
{\bf : Абстрактная ``биматричная'' игра}. {\rm Каждый из
2-х игроков имеет 3 стратегии: $a_i, b_i, c_i~ (i=1,2)$.
Взяв свое имя как бесконечную последовательность символов
типа ``иваниваниван...'', задайте выигрыши первого игрока
так:~ $u_1(a_1,a_{2}) = $``и'', $ u_1(a_1,b_{2}) = $``в'',$
u_1(a_1,c_{2}) = $``а'', $ u_1(b_1,a_{2}) = $``н''$,
u_1(b_1,b_{2}) = $``и'', $ u_1(b_1,c_{2}) = $``в'',$
u_1(c_1,a_{2}) = $``а'', $ u_1(c_1,b_{2}) = $``н'',$
u_1(c_1,c_{2}) = $``и''$~ $.

Подставьте вместо каждой буквы имени число = ее номер в
алфавите. Аналогично используя свою фамилию, задайте
выигрыши второго игрока $u_2(.)$. Найти все известные вам
типы некооперативных решений в нормальной форме: $DE$,
$MM$, $NE$, $StE$, $SE$, $NE_m$ (для $NE_m$, для простоты
решения, превратить игру в антагонистическую, заменив
выигрыши второго игрока --- выигрышами первого с обратным
знаком), а из кооперативных концепций --- сильную и слабую
Парето-границы $P$, $P'$ и обычное $C$-ядро (за уровень
полезности достижимый индивидуально
--- взять максимально-гарантированный выигрыш).

(Сокращения: $DE$ -- Dominant Equilibrium, $MM$ -- MaxiMin,
$NE$ - Nash Equilibrium, $StE$ -- Stackelberg Equilibrium,
$SE$ -- Sophisticated Equilibrium, $NE_m$ -- Nash
Equilibrium in Mixed stratagies.) }\end{exmp}


\subsection{Задачи последовательных игр, $SPE$, $INDW_{\Gamma}$}

\begin{exmp}
{\bf ``Лабиринт''}. {\rm 2-а игрока играют на графе
(дереве) игры: первый игрок с именем $K$ выбирает вверх или
вниз, затем второй игрок с именем $S$ - влево или вправо, и
т.д. Это дерево с корнем $<K1>$, с 12-ю вершинами (типа
up-righ-up-right), глубиной 4 хода можно представить так:

\begin{figure}[h]
\begin{verbatim}
 К   О        К    О
 С   Е        Р    Г
  \  /         \  /
  S4ulu       S4uru
    |           |
    |           |
   K3ul<--S2u-->K3ur
    |     |     |
    В     |     И
    Е     |     Й
          |
        < K1 >
    Н     |     К
    С     |     Е
    |     |     |
   K3dl<--S2d-->K3dr
    |           |
    |           |
  S4dld        S4drd
  /  \         /  \
 /    \       /    \
О     К      О      В
Р     Г      Е      Й
\end{verbatim}\caption{``Лабиринт.''}
\end{figure}

Взяв свое имя как бесконечную последовательность символов
типа ``коковинкокови...'', свое имя типа
``сергейсергейсер...'' задайте выигрыши первого и вторго
игроков в вершинах, как на рисунке; пара букв в каждой
вершине.

Подставьте вместо каждой буквы имени число --- ее номер в
алфавите, либо решайте непосредственно в буквах: более
поздняя больше ст'оит.

1)Найти $SPE$, $SE$. 2)Найти $PBE$ в альтернативном
предположении, что второй ход (вершины $S2d$, $S2u$)
производит природа, случайно, с вероятностями 0.5, и оба
игрока не способны определить, влево или вправо произошел
выбор. }\end{exmp} \vspace{2mm}


%%%%%%%%%%%%%%

\begin{exmp}{\rm
Построить пример игры в развернутой и/или в нормальной
форме, где $INDW \not\subseteq SPE$ или доказать, что это
невозможно. }\end{exmp}

\begin{exmp}{\rm В задаче "террорист" из учебника,
рассмотрим, как выглядит SPE в более сложном случае: при
совпадении некоторых значений выигрышей и/или при
несовершенной информации о ходах. Случай ``С" описанной
игры возможен, если террорист --- психически особенный
человек (что с ними бывает): ему все равно, жить или нет,
но приятнее умереть или жить на Кубе. Тогда возникает много
равновесий $SPE$ (все стратегии), но ни одного $SE$,
поскольку множество IND итерационно (слабо) недоминируемых
стратегий включает неэквивалентные исходы: $IND=\{
(N_Y,NB)\hence (2,1);~ (N_Y,B)\hence (0,1)\}$.

\begin{figure}[h]
\begin{verbatim}

C. Case, when second          D. Case, when second player
can understand,        player can not understand, what the
1-st move was:       what the 1-st move was:


   Pilot    .  2                  Pilot    .   2
           / \ 1                          / \ 0.5
      N-Y /   \ Cuba            2, 0 N-Y /   \ Cuba 1  0
   2  0  /     \  1  0         0.5,1    /     \    2, 0
   1, 1 .       . 2, 2    Terr-st      .........
 NoBomb/ \B  NB/ \Bombing           NB/ \B  NB/ \Bombing
      /   \   /   \                  /   \   /   \
UP=  2     0 1     0           UP=  2     0 1     0
UT=  1     1 2     2           UT=  0.5   1 2     0
(perfect information)       (imperfect information on moves)
\end{verbatim}\caption{``Camicadze''.}
\end{figure}

В случае ``D" выигрыши различны, но неинформированность
террориста позволяет ожидать от него любых ходов. В
результате много равновесий $SPE$ (все стратегии), но ни
одного $SE$, поскольку множество IND итерационно (слабо)
недоминируемых стратегий включает неэквивалентные исходы:
$IND=\{ (N_Y,NB)\hence (2,0.5);~ (N_Y,B)\hence (0,1)\}$.
}\end{exmp}

\begin{exmp}\rm
{\bf Пираты.} {\em (Мулен, 1985,p.-85).} Пусть на пиратском
корабле 50 разного старшинства пиратов делят 100 кусков
золота по следующему обычаю. Старший предлагает дележ --
кому сколько. Если хотя бы половина команды (включая его)
согласна, то так и будет, иначе его выбросят за борт, а
оставшийся старшим предложит дележ, и так далее.

Предскажите, кто сколько получит, вплоть до младшего юнги
(SPE, INDW).
\end{exmp}

\begin{exmp}\rm
{\bf Камешки.} {\em } Пусть Анна и Борис договорились, что
из лежащих перед ними 10 камушков Анна возьмет 1 или 2, по
желанию. Потом Борис 1 или 2, и так далее, а взявший
последний камень проиграл.

Кто выиграет при идеальной игре обоих? Сохранится ли
результат, если можно брать 1 или 2 или 3? Каков общий
метод решения всех таких задач?
\end{exmp}


\begin{exmp}
[SPE в непрерывной игре: Дележ убывающего пирога].\newline
(A.Rubinstein, 1959, see also H.Varian
``Microec.Analysis").\newline
 {\em
Уезжая из дома, мать оставила двум жадным сыновьям пирог, с
таким условием. Сначала Андрей предложит дележ $a_1\in
[0,1]$ (свою долю), если Борис согласен, то так и будет,
иначе через час Борис предложит дележ $b_2\in [0,1]$ (свою
долю). Если Андрей не согласен, он через час предложит
новый дележ $a_3\in [0,1]$, и так далее. Но с каждым часом
полезность пирога убывает (возможно, от нетерпения и от
засыхания пирога) некоторым темпом, то есть через час
остается $\alpha\in (0,1)$ для Андрея и $\beta\in (0,1)$
для Бориса. То есть, если, скажем, на третьей итерации они
согласились на дележ $(a_3, b_3); ~ a_3+ b_3=1$, то
полезность Андрея от него будет $A(a_3)=\alpha^3 a_3$,
Бориса
--- $B(b_3)=\beta^3 b_3$. Зная конечный период $T$, в течение
которого пирог остается съедобен, нужно предсказать, на
какой итерации и как (рациональные и жадные) братья
поделятся (подобная игра очень типична в ситуации, когда
две фирмы способны осуществить взаимовыгодный проект, но
надо договориться о разделе прибыли, а время переговоров
означает упущенную прибыль).

Для простоты будем считать $\alpha = \beta =1/2$, Т=4, и
нарисуем дерево (если можно так выразиться) этой
непрерывной по стратегиям игры:

Решение игры в конечном простом случае (общий случай, и
бесконечный вариант игры рассмотрите самостоятельно) легко
найти с помощью ступенчатой диграммы уровней полезностей,
алгоритмом обратной индукции.

Упражнение. Предположите, что дисконты (``кофициенты
терпения") Андрея $\alpha$ и Бориса $\beta$ разные, как и в
чью пользу (терпеливого ли) изменится решение? Обобщите
решение для произвольного числа периодов Т, и для
бесконечного $T=\infty$, например, переходом к пределу.
(Проверьте $A=(1-\beta)/(1-\alpha\beta)$,
$B=(1-\alpha)\beta/(1-\alpha\beta)$.)}
\end{exmp}

\begin{exmp}
{\bf Игра входа в отрасль.}\rm Пусть есть отрасль с
функцией обратного спроса (ценой от суммарного объема) вида
$p(Y)=9-Y$ и монополистом - старожилом в этой отрасли, с
постоянными предельными издержками $\dot{c}_1(y_1)=1$
(проверьте, что монопольная цена $p^M=5$). Пусть
потенциальный новичок входя в отрасль должен сделать
невозвратные начальные капиталовложения $K=1$ и ожидает
предельные издержки $\dot{c}_2(y_2)=2$. Пусть отрасль может
просуществовать два периода (можно обобщить на $n$) и
дисконта нет: прибыли сегодня и завтра равноценны,
альтернативное вложение капитала $K$ невозможно. Старожил
обещает новичку в случае входа добиться (повышением
выпуска) снижения цены достаточно низко ($<2$), чтобы
заставить новичка прекратить производство, предполагая, что
после этого новичок банкрот и во втором периоде можно
сохранять монополию. Если же новичок войдет, то ожидается
решение Штакельберга (т.е. SPE$_{Old,New}$): лидер-
страрожил установит выпуск раньше. Стоит ли верить этой
угрозе или он блефует и можно входить? Обобщите задачу для
различных $\dot{c}_2(y_2)\neq 2, ~K\neq 1$.
\end{exmp}

%\begin{exmp}{\bf Игра ``Возьми или оставь" (``сороконожка")}:
%Пусть первый из
%двух игроков может взять 2/3 общей прибыли (то есть 2 из 3-х) на
%шаге 1, тогда игра закончится, а второму останется 1/3 (то есть 1
%из 3-х). На шаге 2 прибыль удваивается, и черед 2-го выбирать:
%взять ли 2/3 прибыли (то есть 4 из 6-х) и закончить тем самым игру,
%или оставить, и т.д. Предсказывая исход для конечной игры по
%принципу PE, мы увидим, что игра тривиально закончится на 1-м шаге
%с суммой прибыли 3. А по принципу $PBE(\varepsilon)$ она может
%дойти до конца с суммой прибыли... Какое $\varepsilon$ нужно для
%продолжения игры?
%\end{exmp}
\begin{exmp}{\bf Игра ``Возьми или оставь" (``сороконожка")(by Rosental)}:\rm {}
Пусть первый из двух игроков может взять 4/5 общей прибыли
(то есть \$4 из \$5) на шаге 1, тогда игра закончится, а
второму останется \$1. На шаге 2 прибыль удваивается
(например, ведущим), и черед 2-го выбирать: взять ли 4/5
прибыли (то есть \$8 из \$10-и) и закончить тем самым игру,
или оставить, и т.д. Предсказывая исход для конечной
(скажем, по 3 хода каждого) игры по принципу SPE или PBE,
мы увидим, что игра тривиально закончится на 1-м шаге с
суммой прибыли 3. А по принципу решения $PBE(\varepsilon)$
она может дойти до конца с большой суммой прибыли. (Здесь
$\varepsilon$-- вероятность не ниже которой ожидается от
любого хода, благодаря случайному поведению типа
иррациональности).

1)Какое $\varepsilon$ достаточно для продолжения игры до
конца (точнее, продолжения рациональных ходов до узла
$V_3$)? Достаточно ли его также и в бесконечной игре? Какое
$\varepsilon$ необходимо для рациональности ходов типа
$leave_i$ в конечной и бесконечной играх?

2)Пусть, ситуация изменилась: игрок В слышал, что игрок А в
подобной игре из 10-ти ходов сделал 1 иррациональный
(невыгодный, ошибочный), и ожидает, соответственно,
вероятность иррациональности около $\alpha= 1/10$.
Аналогично, игрок А слышал, что игрок В в подобной игре из
30-ти ходов сделал 2 иррациональных хода и ожидает
вероятность иррациональности $\beta= 2/30$ (это окажется не
то же, что 1/15!). Предположим, игроки считают рациональным
брать банк, когда вероятность ошибки партнера больше 1/7 и
ожидают от партнера такого же мнения. Очевидно, при такой
"простоватой" рациональности, первый - А - на первом ходу
ВОЗЬМЕТ (если не ошибется). Но если он ошибется, возьмет ли
второй? Он может интерпретировать взятие первым как ошибку,
и тогда подправить свою субъективную вероятность ошибок А
до величины (1+1)/(10+1)=2/11. Либо считать случившееся
ОСТАВЛЕНИЕ рациональным ходом, и сделать отсюда вывод о
текущих гипотезах ($\beta=?$) игрока А относительно себя
(относительно В). Независимо от того, верны ли эти
гипотезы, выгодно ли теперь игроку В БРАТЬ и пойдет ли игра
до 6-го хода?

3)По сравнению с предыдущей ситуацией, оставим игрока В
"простым", а первого игрока предположим способным
рассчитать предыдущую ситуацию. Станет ли он на первом шаге
ОСТАВЛЯТЬ, независимо от своих гипотез о партнере
(БЛЕФОВАТЬ)? Пойдет ли игра до 6-го хода?

4)Что если теперь оба игрока "сложные", и В просчитывает
возможность блефа первого (считающего второго простым),
изменит ли это результат?
\end{exmp}


\begin{exmp}{\bf Игра ``Возьми или оставь'' (``сороконожка'' - повтор)}:\rm{}
%\begin{figure}[h]
%\unitlength 1mm
%\input{centip.lp}
%\caption{Игра ``Возьми или оставь''.}
%\end{figure}

Пусть первый из двух игроков (Анна) может взять 4/5 общей
прибыли (то есть \$4 из \$5 на ветви $take_{a1}$) на шаге
1, тогда игра закончится, а второму - Виктору - останется
\$1. Либо можно оставить банк на столе ($leave_{a1}$). На
шаге 2 прибыль удваивается (например, ведущим), и черед
2-го выбирать: взять ли 4/5 прибыли (то есть \$8 из \$10-и)
и закончить тем самым игру, или оставить, и т.д.
Предсказывая исход для конечной (скажем, по 3 хода каждого)
игры по принципу SPE, PBE, или THPE мы увидим, что игра
тривиально закончится на 1-м шаге $take_{a1}$ с выигрышами
(1,4). А по принципу решения $PBE(\varepsilon)$ она может
дойти до конца с большой суммой прибыли. (Здесь
$\varepsilon$-- вероятность не ниже которой ожидается от
любого хода, благодаря случайному поведению типа
иррациональности).

Покажите, что $\varepsilon >1/7$ достаточно для продолжения
игры до конца (точнее, продолжения рациональных ходов до
узла $V_3$). Какое $\varepsilon$ необходимо для этого же?

2)Пусть, ситуация изменилась: игрок Victor слышал, что Анна
в подобной игре из 10-ти ходов сделала 1 иррациональный
(невыгодный, ошибочный), и ожидает, соответственно,
вероятность иррациональности около $\alpha= 1/10$.
Аналогично,  Анна слышала, что Виктор в подобной игре из
30-ти ходов сделал 2 иррациональных хода, она ожидает
вероятность иррациональности $\beta= 2/30$ (это окажется не
то же, что 1/15!). Предположим, игроки считают рациональным
{\em брать} банк, когда вероятность ошибки партнера больше
1/7 и ожидают от партнера такого же мнения. Очевидно, при
такой ``простоватой'' рациональности, Анна на первом ходу
ВОЗЬМЕТ (если не ошибется). Но если он ошибется, возьмет ли
Виктор? Он может интерпретировать {\em оставление} Анной
как ошибку, и тогда подправить свою субъективную
вероятность ошибок А до величины (1+1)/(10+1)=2/11. Либо
считать случившееся {\em оставление} рациональным ходом, и
сделать отсюда вывод о текущих гипотезах ($\beta=?$) Анны
относительно себя (Виктора). Независимо от того, верны ли
эти гипотезы, выгодно ли теперь Виктору {\em оставлять} и
пойдет ли игра до узла $V_3$?

3)По сравнению с предыдущей ситуацией, оставим Виктора
``простым'', а первого игрока предположим способным
рассчитать предыдущую ситуацию. Станет ли он на первом шаге
ОСТАВЛЯТЬ, независимо от своих гипотез о партнере
(БЛЕФОВАТЬ)? Пойдет ли игра до 6-го хода?

4)Что если теперь оба игрока ``сложные'', и В просчитывает
возможность блефа первого (считающего второго простым),
изменит ли это результат?
\end{exmp}
.

\begin{exmp}
Колпаки (см. R.Myerson).\rm {} Царь решил испытать своих
четырех мудрецов. Он приказал им закрыть глаза и из мешка,
где было много черных и белых колпаков, вынул случайным
образом и надел на мудрецов 4 колпака. Сказал им открыть
глаза, сообщил, что в мешке были черные и белые и спросил:
может ли кто наверняка угадать на себе колпак глядя на
других? "Нет - это невозможно" - сказали мудрецы (и были
правы). А случилось, что все колпаки белые. "А ведь среди
одетых есть белый колпак, тогда что вы о себе скажете?" -
сказал царь. Мудрецы промолчали. Ведь ничего нового он не
сообщил: каждый уже видел троих соседей в белом. "Помолчите
минуту, потом можно говорить" - сказал царь. Но и во второй
раз они молчали. И когда третий раз он предложил говорить -
молчали. А вот на четвертый раз каждый сказал, что он в
белом колпаке, и однозначно объяснил это, и царь всех
наградил. 1)Объясните решение (SPE). В чем новизна
поступившей от царя информации? 2)Покажите, что так же
решается задача для любого числа мудрецов и любого варианта
выпавших черных/белых колпаков: кто-то определит свой
колпак и скажет. 3)Покажите, что если выпали неодинаковые
колпаки, то более слабой подсказки царя: "колпаки
неодинаковы" - достаточно для разрешения игры НЕ ПРИ ВСЯКОМ
числе черных и белых.

\end{exmp}

(=предыдущую игру можно как=Конкурс. (тема "общее
знание") Трое или четверо студентов закрывают глаза
и экзаменатор кладет на голову каждому белый
платок. Известно, что платки бывают красные или
белые. Разрешается открыть глаза. Угадать цвет
платка на себе невозможно. Но когда экзаменатор
скажет: "среди вас есть белый платок", угадать
возможно (хотя он ничего нового каждому не
сообщил). Потом командует: закрыть глаза, открыть
глаза (чтобы были этапы).

Выигрывает угадавший первым и объяснивший. Призером
также является - первым подавший корректную модель
игры.)

Примеры конкурсов. (Конкурс предлагается всей
аудитории, участие необязательно, но 1-2 победителя
получают значительные дополнительные очки к зачету)
Конкурс 1. (тема: рациональность поведения и "общее
знание") Каждому предлагается задумать и подать на
бумаге число от 1 до 100. Тот, чье число ближе всех
к половинке от среднего из названных - победитель.
Призером также является - первым подавший
корректную модель игры. Конкурс 2. (тема:
рациональность поведения, динамические игры)

Аукцион "Лохотрон". На столе лежит 100 руб. - дар
экзаменатора. Любой желающий участвовать в
розыгрыше этой купюры вносит на стол 1 рубль
(безвозвратно). Желающие продолжать розыгрыш вносят
еще по 1. И т.д. Победитель забирает все (возместив
экзаменатору 100) и получает "отл.". Призером также
является - первым подавший корректную модель игры.
Конкурс 3. ("общее знание") Трое или четверо из
занявших лучшие места в другом конкурсе закрывают
глаза и экзаменатор кладет на голову каждому белый
платок. Известно, что платки бывают красные или
белые. Разрешается открыть глаза. Угадать цвет
платка на себе невозможно. Но когда экзаменатор
скажет: "среди вас есть белый платок", угадать
возможно (хотя он ничего нового каждому не
сообщил). Вигрывает угадавший первым и объяснивший.
Призером также является - первым подавший
корректную модель игры.


\begin{exmp}\rm
{\rm Мосты Цезаря: ``commitment''.} Цезарь с войском
переправился по наведенным мостам через реку на сторону
неприятеля, войско Цезаря приготовилось к бою. Неприятель
приготовился к бою. Цезарь сжег за собой мосты. Увидев это,
неприятель бежал.

Составить две игры, годящихся для объяснения этой ситуации:
1)С полной информацией о целях трех игроков: Цезарь, войско
Цезаря, неприятель. 2)С двумя игроками (Цезарь,
неприятель), неполной информацией и выявлением целей Цезаря
через его поведение.

(Сходная ситуация ``commitment'' возникает при кредите с
залогом).
\end{exmp}

\begin{exmp}\rm
{\rm Залог или клятва: ``commitment''.}

Пусть, популяция российских бизнесменов имеет обыкновение
просить кредит размером в \$ 100 на год в банке, и при
гарантии возврата банк готов бы давать кредит под 5\%.
Пользование кредитом приносит среднему бизнесмену в год
$A$\%. В среднем, четверть бизнесменов кредит без залога не
отдают. Оставляя в гараже банка свой джип ценой в \$ 120  в
залог на год, бизнесмен имеет потерю полезности, равную
$10$\% годовых. Половина бизнесменов джипов не имеют, им
оставить нечего. Но половина из всех (поровну среди
джипо-владельцев и безлошадных) известны как истые
мусульмане, каждый может поклясться Аллахом, что кредит
отдаст, и клятвы эти всегда исполняются.

Добавив данных, составить игру, годящуюся для ситуации, и
найти, при каком параметре $A$ в равновесии банк дает
кредит не только джипо-владельцам и мусульманам и почем.
Если банк монополист, то для какой категории кредит
дешевле, а какой дороже (выгоднее ли быть собственником
джипа или репутированным мусульманином)?
\end{exmp}

\begin{exmp}[Цена репутации (клятва)]
Предположим, религиозный ростовщик в средневековом Багдаде
не считает богоугодным зарабатывать на каждой сделке больше
1 процента и не ограничен в деньгах. Его кредит может
принести любому торговцу 20\% годовых. В городе
ограниченное число купцов трех вероисповеданий. Мусульмане
известны как исполняющие клятву в $m$\% случаев, иудеи -- в
$j$\% случаев, христиане -- в $c$\% случаев. Кто из купцов
почем получает кредит (если получает)?

(Вариант: Мусульмане имеют неудовольствие от нарушение
клятвы размером в $m$\% годовых, иудеи -- в $j$\% годовых,
христиане -- в $c$\% годовых. Кто из купцов почем получает
кредит (если получает)?)
\end{exmp}

\begin{exmp}\rm
{\rm Веревки Одиссея: ``commitment'', мультиперсонное
представление, иррациональность.} Одиссей подплывая на
корабле к острову Сирен, хотел послушать их сладкое пение,
но знал, что всякий слушающий бросается в воду, плывет к
ним и не возвращается. Этого он не хотел. Он приказал
матросам залепить уши воском, а его самого привязать к
мачте. Так он услышал Сирен, но остался капитаном.

Составить  игру, годящуюся для объяснения этой ситуации.
\end{exmp}

\begin{exmp}[``Вор на базаре'']\rm

\begin{verbatim}


 1)Thief:         * --------------.
                 / \               \
    v1-to stole /   \ v2-to rob     \ v0- to rest
               /     \
 2)Seller:    *.......*
 to be silent/ \shout/ \ silence
      (1-k) /   \ k /   \ (1-k)
           /     \ /     \
 3)Policemen   ..*.*..
                /| |\
  to rest  ____/ |c| \_____ to rest
     (1-l)/      |l|       \ (1-l)
           0.5   1 2     0
           0
           1

\end{verbatim}

%\begin{figure}[hbt] \unitlength 1.00mm
%\input{bazar.lp}
% \caption{Игра ``Базар'': решения SBE.\label{bazar}}
%\end{figure}
На 
%Рис.\ref{bazar} 
рисунке представлена игра с несовершенной
информация о ходах: второй и третий игроки не способны
различать, какой ход сделан первым. Подразумевается
популяция трех ролей: Воров, Торговок, Полисменов. Базарный
вор может или отдыхать (быть честным), или воровать просто,
или воровать с оружием. Торговка может кричать или молчать,
когда у нее с лотка тянут товар. Полисмен может или бежать
на крик и ловить, или лениться (отдыхать). Записанные на
рисунке выигрыши берут за точку отсчета (0,0,0) вариант,
когда Вор отдыхает, и остальные - тоже. Когда торговка
что-то теряет, ей неприятно, но неприятно вдвойне, если она
еще и кричит при этом зря  (еще, она побаивается кричать,
когда вор вооружен, а не кричать о безоружном считает
стыдным, это отражает выигрыш -2 в этом варианте). Если же
ее врага-вора поймают - она довольна. О Полисмене,
предполагается, что он любит премии за поимку воров, но не
любит риска с вооруженными, хотя справится и с таким. О
Воре - что он больше отсидит, если пойман с оружием.

\rm Будем рассматривать смешанные стратегии игроков
$(\sigma_{thief}\in [0,1]^3, \sigma_{seller}\in
[0,1]^2,\sigma_{police}\in [0,1]^2)$ как вероятности, с
которыми эти ходы в среднем встречаются на описанном
базаре. Разыскивая равновесие (то есть стабильное поведение
каждого типа), предположим, что ОЖИДАНИЯ всех игроков
(предполагаемые вероятности ходов партнеров), а именно:
ожидания вора,
% $(\tilde{k},\tilde{l})$,
ожидания торговки, ожидания полисмена
--- соответствуют наблюдаемым частотам делаемых ходов. Но
этого мало, поскольку нужно еще и вне пути игры задать так
называемые ВЕРЫ, то есть ожидаемые вероятности нахождения в
том или ином узле информационного множества, если игра
вдруг, каким-нибудь чудом, туда попадет. Скажем, если все
Воры обычно отдыхают, то Торговке, а еще более - Полисмену,
все же любопытно знать, с ножом ли тот, кто стащил у нее
вещь с лотка, если это случится. Это и есть их (Торговки и
Полисмена) априорные, не проверенные жизнью, веры, которые
между собой могут не совпадать. Например, Торговка может
предполагать частоты верхнего и нижнего узлов графа типа
(0.2, 0.8), а Полисмен - (0.9, 0.1), и каждый объявлять
свою (возможно, ненаблюдаемую, но известную партнерам)
стратегию, исходя из этих априорных вер.


\begin{defn}{\rm
Совершенное Байесовское равновесие (PBE, называемое также
слабым секвенциальным равновесием) в игре $n$ лиц есть
набор $(\sigma,\mu)$ смешанных пошаговых (поведенческих)
стратегий $\sigma=(\sigma_1,...,\sigma_n)\in \Delta X$   и
вер $\mu=(\mu_1,...,\mu_n)\in \Delta M$
%(ожидания всех игроков $i\neq 1$ об игроке 1 одинаковы, так же и об остальных)
%$\beta_i\in A~~\forall i$ (beliefs)
всех игроков,\footnote{Множество вероятностных пошаговых
стратегий $\Delta X$ есть смешанное расширение чистых
стратегий - ходов $X$. Ожидания любого игрока о себе
считаем равными его стратегии: $\mu_{ii}=\sigma_i$.}
%Согласованность ожиданий ...}
таких что\newline 1) стратегии $\sigma$ являются
секвенциально-рациональными (то есть итеративно
недоминируемыми строго на графе), при данных верах
$\mu$;
\newline 2) веры $\mu$ слабо (только на пути
игры) согласованы с наблюдаемым стратегиям $\sigma$, в
смысле Байесовского правила условных
вероятностей.\footnote{ В частности, если все ходы из
некоторого узла оканчиваются в одном (последующем)
информационном множестве, то веры в нем должны совпадать с
вероятностями ходов: $\mu_{h}=\sigma_{h-1}$.} }\end{defn}

Проверим,  может ли быть решением
(Отдыхать,[Кричать,Ловить]) (в квадратных скобках, как
обычно, ходы вне пути игры) хоть при каких-либо верах.
Заметим, что стратегия торговки КРИЧАТЬ лучше
противоположной при ее ожидании от Полисмена хода "Ловить".
%поэтому ее можно считать ``болваном'' и рассматривать игру
%двух лиц: вора и милиционера (более нетривиальные варианты
%игры рассмотрите самостоятельно).
Полисмен же может продолжать объявлять (только на словах,
пока Вор не ворует) стратегию Ловить, только если он верит,
что если уж Вор сворует, то без оружия. Если же с
вероятностью более 2/5 он верит в противоположное, то
отступит от Ловить. Иначе, проверяемое решение может
оставаться SBE (а также SPE, NE) при вере более 3/5 в
безоружность, и при любых верах Торговки, возможно и
отличающихся от вер Полисмена!

Анализируя эту игру, можно найти, что в ней есть и другие
Совершенные Байесовские равновесия, но несовпадение вер
торговки и Полисмена в становится невозможным, если Вор
хоть иногда ворует: определение PBE не позволяет
несоответствие вер практике {\em на пути игры}.
\end{exmp}

{\em Упражнение.} Пример ``Масти и Картинки''.
 \begin{table}[hbt]  \caption{}
~~~~~~~~~~~~~~~~~~~~~~~~~~~~~~~~~~~~~~~~~~~~ Масти:  \\
{\begin{tabular}{cc|l|l|l|l}
%& \multicolumn{1}{cc}{       &       Инспекторы       } &&\\
                                                           %w\\
               &      &             &~~~          &    &     \\
               &     {\em Крас- }      &  {\em ные }      &  {\em Чер-}     &  {\em ные }  &      \\
\cline{2-5}%   &---------------&-------------&-------------&------------&------------\\
               & Черви $\downarrow$ &   Бубны $\downarrow$ & Крести $\downarrow$ &     Пики $
\downarrow$ &            \\
\cline{2-5}%   &---------------&-------------&-------------&------------&------------\\
             ~~~   &~~~ ~~~~~ 8 &~~~ ~~~~~ 0 &~~~~~~~~~  2 &~~~~~~~~~  0 &~~~~~~~~~   \\
   Старшие: Туз &~  1~~~ ~~&~ 3~~~ ~~ &~ 5         ~&~ 1        ~&~         \\
\cline{2-5}%   &---------------&-------------&-------------&------------&------------\\
                 &~~~ ~~~~~ 6&~~~~~~~~~ 0  &~~~~~~~~~  4 &~~~~~~~~~  2 &~~~~~~~~~   \\
   Старшие: Король &~ 7~~~~~~~~&~ 1~~~~~~~~  &~ 5~~~~ ~~ &~ 3~ ~~~~~~ &~  ~ \\
\cline{2-5}%   &---------------&-------------&-------------&------------&------------\\
                 &~ ~~ ~~ 2~&~~~~~~~~~ 6  &~~~~~~~~~  8 &~~~~~~~~~  0 &~~~~~~~~~   \\
   Младшие:  Дама &~ 3~~~~~~~&~ 9~~~~~~~~~ &~  3~~~~~ ~~ &~ 1~~~~~~~ &~  ~~~~~ \\
\cline{2-5}%   &---------------&-------------&-------------&------------&------------\\
\cline{2-5}%   &---------------&-------------&-------------&------------&------------\\
                  &~ ~~~ ~0~&~~~~~~~~~ 4  &~~~~~~~~~ 4 &~~~~~~~~~ 0  &~~~~~~~~~   \\
 Младшие:  Валет  &~ 3~~~~~~~&~ 7~~~~~~~~~ &~ 9~~~~~ ~~ &~ 1~~~~~~~ &~  ~~~~~~~ \\
\cline{2-5}%   &---------------&-------------&-------------&------------&------------\\
\end{tabular}}\label{cards}
\end{table}

Здесь предполагается, что строчный игрок выберет: Старшие
или Младшие, потом столбцовый - Красные или Черные, потом
строчный - конкретную картинку из уже названной группы (из
Старших или из Младших), потом столбцовый - конкретную
масть из уже названного цвета. Найти SPE, INDW.

Вариант 2: Усложнение задачи - найти SPE, INDW, PBE если
последний ход решается жребием - подбрасыванием монетки.

Вариант 3: То же, но результат подбрасывания известен до
ходов второму игроку, и только ему.

Вариант 4: Найти SPE, INDW, PBE если {\em первый} ход
решается жребием - подбрасыванием монетки, и никто не видит
его. \vspace{2mm}

{\bf Пример ``Trivial quize''} (упрощенный покер)?? Разыгрывая 1
рубль, Анна тянет карту из колоды, смотрит, и не показывая
Борису (у которого открыт Валет, а карты от 10-ки до Туза), или
удваивает ставку, или пасует и имеет -1, а Борис 1. Если
удвоено, Борис или пасует и имеет -2, а Анна 2, или удваивает, и
карта открывется. Если она больше, чем Валет, то Анна выиграла 4
у Бориса 4, иначе проигрывает 4. Найти SBE (частоты ходов при
каждой карте).

{\Large Задачи из ВШЭ} (...):

 From Alex
Debelov : во ВШЭ дали Тигр zadacha:

AD 1. Take-or-leave game - \$ 100 дележ. А предлагает, В
решает. Согласен -   получает долю, нет - оба с 0. Какое
решение - Subgame Perfect NE? Strictly speaking, only
(100,0) is a Subgame Perfect NE. Indeed, B has no reason to
switch to another 0, as well as A does not switch from 100.
Any other partition can be improved for A and B agree.(Why
you ask easy questions?) However, more realistically is to
suppose, that A prefer to be sure that B strictly prefere
taking his share than refusing. Then A can get as much as
\$ 99.99, leaving B with 0.01 only (or \$ 1 if it is the
smallest portion of money available). This may be
formulated in the concept of "epsilon-Bayesian NE", but not
in "Trembling hand". Even more realistically is to expect
(50,50), I have told you, but this is captured by another
model. AD> 2. А предлагает предмет В. А знает ценность
предмета, В - нет. В AD> думает, что предмет стоит 0 с
вер-тью р и 3 с (1-р). Это он стоит для игрока А, видимо!
AD> Оба знают, что AD> оценка В равна оценке А плюс 4 (вот
это меня несколько ставит в AD> тупик). И меня тоже. Лучше
бы узнать у друзей ясную интерпретацию задания, но одна из
возможных такая. Предмет приносит игроку А полезность (3) с
вер. (1-р),  и (0) с вер-тью р, то есть, от него хочется
избавиться. А у Бори ожид. полезность р*4+(1-р)*7.
 Скажем, есть у Анны старый диван, который то ли треснет в
первый день, то ли долго простоит, и Анна знает качество, и хотела бы,
чтоб Боб его вывез. Если она объявит цену около 0 (вывози даром) - это знак,
что диван плохой, и он не купит. (ситуация типа Акерлова).
Можно предложить цену р*4+(1-р)*7, тогда возможно SPE, когда он купит,
при вероятности Р:  р*4+(1-р)*7>=3.
Но искать-то, судя по тексту, надо SPBayesianE (популяция
игроков типа Ань и игроков типа Борь). Оно тоже будет в
этой точке, и является типа pooling.(Действительно, все Ани готовы
отдавать по этой цене, и все Бори готовы брать диван,
хотя и с ожидаемым выигрышем 0). А повысить цену нельзя, Бори
откажутся. Нету out-of-equilibrium beliefs,
так что это действительно SPBayesianE. А разделяющее (когда
хорошие Ани не продают по 3 или другой <= ожидаемой, а плохие продают)
есть при р*4+(1-р)*7<=3, но не при большей цене. Ведь если бы
Бори брали по цене 7, то плохие Ани тоже начали бы косить под хороших,
и оно бы разрушилось. Любопытно, что при разных beliefs
Борь (о том, что, скажем, доля=0.5 хороших Ань выходит на рынок)
и пограничной цене равной 3=(0.5+р*0.5)*4+(1-р)*0.5*7 можно тоже
конструировать равновесия... Некоторые из них будут с
out-of-equilibrium beliefs, тогда это всего лишь SPE, но не
SPBayesianE. То есть, оно случилось бы 1 раз, а назавтра разрушилось
бы в повторяемой игре.

AD А предлагает один раз - Take-or-leave game.

AD а) Описать игру и возможные стратегии А и В.

AD б) Существует ли разделяющее равновесие, в к-м всегда
происходит обмен
 по цене, равной оценке предмета игроком В?

AD в) Существует ли pooling равновесие? Если да, то что
можно сказать об

AD out-of-equilibrium beliefs?

\section{NES, A.Tonis, A.Savvateev}

\subsection{Развернутая и нормальная форма}

\begin{enumerate}

\item {\bf Крестики и нолики.} Рассматривается игра в
обычные ``крестики-нолики'' (${3\times 3}$). Все ли знают
ее правила? Для нас сейчас неважно, кто и когда в ней
выигрывает, нас интересуют лишь различные формы, в которых
может быть представлена игра. Для простоты можете считать,
что терминальными позициями являются только те, в которых
все 9 клеток заполнены, т.~е. игра продолжается даже если
уже ясно, кто выиграл.

\begin{enumerate}

\item\label{pos} Сколько {\it позиций} имеет игра, иными
словами, сколько существует расстановок крестиков и
ноликов, которые могут возникнуть по ходу игры (и в ее
конце)?

\item\label{razv} Сколько вершин в дереве игры
(в~развернутой форме)?  Сравните с~п.~\ref{pos}.

\item Подсчитайте, хотя бы приблизительно, сколько
стратегий имеется у каждого участника в каждом из
рассмотренных представлений игры (см. пп.~\ref{pos}
и~\ref{razv}).

\end{enumerate}

\item {\bf Захват рынка.} Две фирмы $A$ и $B$ производят
некоторый товар. В~каждый момент времени $t=1,\ldots,5$
каждая фирма может произвести единицу товара либо ничего не
производить. Затраты на производство равны~$3$, а цена
продажи определяется числом~$n$ активных фирм на рынке и
составляет~$6-2n$.

\begin{enumerate}

\item Опишите развернутую форму этой игры, стратегии
участников и функции выигрыша. Сколько стратегий у каждой
фирмы?

\item Те же вопросы, если, однажды выйдя с рынка, фирма уже
не может вернуться.

\end{enumerate}

\item {\bf Рабочий и управляющий.} В игре имеются два
участника: рабочий и управляющий. Если рабочий работает, он
теряет~$1$, а управляющий получает~$3$. Иначе рабочий
ничего не теряет а управляющий теряет~$1$. Управляющий
назначает рабочему зарплату~$w$.

\begin{enumerate}

\item Пусть рабочий и управляющий принимают свои решения
одновременно. Нарисовать развернутую и нормальную форму.
Внимание: правильно понять и формализовать игру~--- входит
в задание!


\item Тот же вопрос для случая, когда рабочему известно,
сколько ему будут платить. Как Вы думаете, чем кончится
игра и кто сколько выиграет?

\end{enumerate}

\item {\bf Делим пирог.} Рассмотрим две модели
``справедливой'' дележки пирога между двумя соискателями.

\begin{enumerate}

\item Один режет пирог на две части, другой выбирает себе
любую из них. Описать развернутую и нормальную форму. Каков
наиболее вероятный исход игры?

\item Один режет пирог на две части и пишет на них ``$1$''
и ``$2$''. Другой в это время, отвернувшись, говорит, какую
часть ему выдать. Описать развернутую и нормальную форму.
Как Вы думаете, может ли произойти неравный раздел?
Как-нибудь обоснуйте свой ответ.

\end{enumerate}

\end{enumerate}

\subsection{Решение по доминированию}

\begin{enumerate}

\item {\bf Решите по доминированию} игру с матрицей
выигрышей
\[\left[\begin{array}{cccc} 5,2&2,6&1,4&0,3\\ 4,1&3,4&2,1&1,2\\
1,0&1,1&1,5&5,1\\ 2,3&0,1&0,2&4,4\\
\end{array}\right]\]

\item {\bf Итерационные игры.} Играют $n$ участников.
Функция полезности участника с номером~$k$ имеет вид
$u_k(s_1,\ldots,s_k)$. Можно ли решить игру по
доминированию? Если надо, сделайте какие-нибудь разумные
предположения.

\end{enumerate}

\subsection{Игры с нулевой суммой}

\begin{enumerate}

\item {\bf Примеры игр с нулевой суммой.} Найти
максимин~$\alpha$ и минимакс~$\beta$ в чистых стратегиях, а
также все седловые точки (они же равновесия Нэша) и цену
игры в смешанных стратегиях:

\begin{enumerate}

\item \[\left[\begin{array}{lll} -1&\phm 2&-2\\ \phm 5&-1&\phm 1\\
\phm 2&\phm 1&\phm 3\phm \\
\end{array}\right]\]

\bigskip

\item \[\left[\begin{array}{lllll} \phm 1&\phm 3&\phm 2&\phm 0&\phm 4\\
\phm 4&\phm 1&\phm 1&\phm 5&\phm 0\\ \phm 6&\phm 0&\phm
4&\phm
9&-3\phm \\
\end{array}\right]\]

\end{enumerate}

\item {\bf Два пальца.} Есть такая замечательная игра.
Участники одновременно показывают один или два пальца.
Потом считают сумму~$s$ (она может получиться от двух до
четырех). Если $s$ четно, то второй игрок выиграл у первого
$s$~долларов, если же $s$ нечетно, то наоборот, выиграл
первый.

\begin{enumerate}

\item Найдите седловую точку в смешанных стратегиях и цену
игры. Справедлива ли игра, и, если нет, то кому лучше?

\item Те же вопросы для игры ``три пальца'' (можно
выбрасывать от одного до трех пальцев).

\end{enumerate}

\item {\bf Еще одна игра ``на пальцах''.} Двое играют на
деньги. Участники одновременно показывают сколько-то
пальцев (от $1$ до~$n$). Если оказалось поровну, то ничья.
Если число пальцев, показанных одним и другим игроком,
отличается на~$1$, то тот, у кого меньше, выигрывает~$\$2$.
В~остальных случаях тот, у кого больше, выигрывает~$\$1$.

\begin{enumerate}

\item Можно ли, не производя никаких вычислений, определить
цену игры в смешанных стратегиях? Приведите соображения,
обосновывающие ваш ответ.

\item Пусть $n=3$. Как надо играть?

\item Верно ли, что при любом~$n$ в игре имеется ровно одно
смешанное равновесие?

\end{enumerate}

\end{enumerate}

\subsection{Равновесие по Нэшу}

\begin{enumerate}

\item {\bf Задача про лыжников.} Двое бегут по лыжной
трассе навстречу друг другу. У~каждого две стратегии:
уступить или не уступить (для определенности предположим,
что лыжники руководствуются принципом правостороннего
движения). Уступивший дорогу теряет на этом 2~сек., а если
столкнулись, то будут распутываться 10~сек.

\begin{enumerate}

\item\label{a} Найдите все (и чистые, и смешанные)
равновесия в данной игре, предполагая, что проигрыш
участников определяется потерянным временем. Какие из
равновесий являются эффективными (оптимальными по Парето в
сильном или слабом смысле)?

\item Решить задачу \ref{a} в предположении, что можно еще
уступить половину лыжни~--- мы ведь бегаем классикой! При
этом теряется 1~сек. (в дополнение к возможным десяти).
Сколько будет равновесий? Что на сей раз можно сказать об
их эффективности?

\end{enumerate}

\medskip

\item {\bf Торговцы на станции.} На станции Тайга трое
местных предпринимателей, Александр, Василий и Семен
($A,B,C$), промышляют тем, что продают пассажирам,
соответственно, пиво, воблу и соленые орешки. Утром
приходят сразу два поезда, поэтому каждый спешит поставить
свою торговую точку на первой или второй платформе. Если
торговец работает на платформе в одиночку, его выручка
(в~рублях) от продажи товаров пассажирам соответствующего
поезда определяется из таблицы:
\begin{center}
\begin{tabular}{|c|ccc|}
\hline Платформа&$A$&$B$&$C$\\ \hline 1&80&60&60\\ 2&100&40&40\\
\hline
\end{tabular}
\end{center}
Если в одном месте продаются и пиво, и закуска, то этих
товаров удается продать на $50\%$ больше из-за эффекта
дополняемости. Впрочем, если продавцы закуски находятся на
одной платформе, то вследствие конкуренции оба выручают
вдвое меньше, чем когда они на разных платформах.
\begin{enumerate}

\item Формализуйте взаимодействие торговцев как игру в
нормальной форме, предполагая, что до установки торговой
точки никто из них не может получить информацию о том, где
будут другие.

\item Найдите все чистые и смешанные равновесия Нэша в этой
игре.

\item Что изменится, если Александр будет в~одиночку
зарабатывать на второй платформе не~$100$, а~$60$ рублей?

\end{enumerate}

\item {\bf ``Гладкая'' игра.} Играют двое. Стратегии
игроков задаются вещественными параметрами $s$ и $t$
(${s,t\in[-1,1]}$). Выигрыши равны, соответственно,
$2\alpha st-s^2$ и $t^3-3st$. Изобразите кривые реакции
обоих участников (если возможно, разными цветами или
стилями линий) и найдите все равновесия Нэша в чистых
стратегиях для $\alpha=-{3\over 4}$, $\alpha={1\over 4}$ и
$\alpha=2$.

\item {\bf Продавцы мороженого на пляже.} На городском
пляже стоят два ларька ($A$~и~$B$), торгующие мороженым.
Продавцы независимо устанавливают цены
$p_A,p_B\in[0,\infty)$; издержками пренебрегаем. Все
выглядит примерно так:
\begin{center}
\begin{figure}
\begin{picture}(60,10)
\put(3,5){\line(1,0){54}}
\multiput(0,5)(60,0){2}{\circle{6}} \put(-1.6,4){$A$}
\put(58.4,4){$B$} \put(26.4,1){$1$км}
\put(25,2.5){\vector(-1,0){21}}
\put(35,2.5){\vector(1,0){21}} \put(20,6){п\quad л\quad
я\quad ж}
\end{picture}\end{figure}\end{center}
Народ равномерно распределился по пляжу и загорает. В~этот
день очень жарко, поэтому каждый из отдыхающих готов
переплатить за лакомство рубль, только бы не идти лишние
100м по раскаленному песку.

\begin{enumerate}

\item\label{bb} Предположим, что каждый, во что бы то ни
стало, стремится приобрести себе стаканчик мороженого.
Полностью опишите отображения наилучшего ответа, изобразите
на плоскости $(p_A,p_B)$ кривые реакции и найдите все
равновесия Нэша. Естественно, все в чистых стратегиях.

\item Выполните задание п.~\ref{bb} в предположении, что
ценность стаканчика мороженого составляет~$v\ge 0$, т.~е.
каждый потребитель был бы готов заплатить за мороженое
цену~$v$, если бы оно продавалось рядом (как нетрудно
видеть, в~п.~\ref{aa} $v=\infty$). Рассмотрите все
возможные случаи. Указание: чтобы избежать лишней возни с
несущественными кусками кривых, можете исключить что-нибудь
по доминированию.

\end{enumerate}

\item {\bf Продавцы мороженого в Игарке.} Город Игарка весь
расположен вдоль одной улицы длиной 3~км. Два конкурирующих
продавца мороженого независимо выбирают места для своих
торговых точек. Покупатели, естественно, идут к ближайшему
ларьку (в Игарке $-50^{\rm o}$C). Если расстояния до
ларьков одинаковы (в частности, если ларьки находятся в
одной точке), то место покупки мороженого выбирается
покупателем случайно и равновероятно.

\begin{enumerate}

\item\label{aa} Пусть цена мороженого фиксирована и все
хотят его купить. Найти все равновесия в чистых стратегиях.

\item Предположим, что в задаче \ref{aa} покупатели из-за
риска замерзнуть не ходят дальше 1~км от дома. Найдите все
равновесия.

\item\label{b} А что будет, если в задаче \ref{aa}
продавцов не два, а три?

\item Решить задачу \ref{b}, предполагая, что три продавца
выбирают свое местоположение вдоль Московской кольцевой
автомобильной дороги.

\end{enumerate}

\item {\bf Дуополия с дифференцированными товарами.} Две
фирмы производят два различных товара. Эти товары частично
взаимозаменяемы и спрос на них формируется по закону
\begin{equation}\label{eq}
\begin{array}{l}
  q_1=\max\biggl\{\hidr{1-2p_1+p_2}{3},0\biggr\};\\
  q_2=\max\biggl\{\hidr{1+p_1-2p_2}{3},0\biggr\},
\end{array}\end{equation}
где $p_1,p_2\ge 0$~--- цены, а $q_1,q_2\ge 0$~--- выпуски.
Затраты на производство единицы продукции составляют,
соответственно, $c_1$ и~$c_2$ ($c_1,c_2\ge 0$).
\begin{enumerate}
  \item\label{za} Пусть фирмы независимо устанавливают свои цены,
  после чего удовлетворяют возникший на рынке спрос. При каких~$c_1$
  и~$c_2$ реализуется равновесие с положительными ценами и
  выпусками? Найдите для этого случая равновесные $p_1,p_2,q_1,q_2$.
  \item А теперь, наоборот, пусть фирмы выбирают, сколько они
  будут производить, а цены на рынке формируются в соответствии с
  законом спроса~(\ref{eq}) (если правая часть какой-либо из получающихся
  формул отрицательна, то считаем соответствующую цену нулевой).
  Ответьте на те же вопросы, что в п.~\ref{za}
  \item Сравните равновесия, реализующиеся при ценовой и количественной
  конкуренции. Отдельно рассмотрите случай $c_1=c_2$.
\end{enumerate}

\item {\bf Аукцион печенья.} Имеется пакет с печеньем,
который нужно поделить между $n$ участниками. Количество
печенья в пакете всем известно. Каждый участник тайно от
других пишет на листке бумаги свое имя и сколько продукта
он хотел бы получить. Все заявки упорядочиваются по
возрастанию, после чего ведущий по очереди выдает каждому
запрошенное им количество, начиная с самых ``скромных''.
Если в некоторый момент печенье кончается, то заявившие
слишком много, увы, остаются ни с чем (если оставшегося
печенья оказывается недостаточно, чтобы обслужить несколько
одинаковых заявок, то делим между ними поровну). Если же
остались лишние печенья, их съедает ведущий.

\begin{enumerate}

\item Найдите все симметричные равновесия в чистых
стратегиях. Дискретностью печенья можно пренебречь.

\item Есть ли в этой игре несимметричные чистые равновесия?
Если есть, приведите пример, а если нет, то объясните,
почему.

\end{enumerate}

\end{enumerate}

\subsection{Динамические игры с совершенной информацией}

\begin{enumerate}

\item {\bf Вариант игры Баше.} В~игре участвуют двое, ходя
по очереди (первый, второй, первый и~т.~д.). На столе лежит
$n$ камней. В~свой ход каждый из участников может изъять
$2^k$ камней ($k=0,1,2,\ldots$; естественно, нельзя снять
со стола больше, чем там есть). Выигрывает тот, кто своим
ходом оставляет пустой стол.

\begin{enumerate}

\item Пользуясь принципом Цермело, определите выигрышные и
проигрышные позиции в игре. Для этого, например, можно
исследовать несколько первых позиций и угадать
закономерность. Обязательно обоснуйте свой ответ.

\item Опишите выигрышную стратегию игрока (в~случае, если
она у него есть). Когда выигрывает первый, а когда второй?
Может ли быть ничья?

\item Пусть $n=50$ и Вам ходить. Ваши действия?

\end{enumerate}

\bigskip

\item {\bf Вариант игры Ним.} Снова два участника ходят по
очереди. Теперь на столе уже две кучки камней ($m$~камней
в~первой кучке и $n$~во~второй). В~свой ход игрок может
взять сколько угодно камней из одной кучки, либо одинаковое
количество из обеих. Надо обязательно взять хотя бы один
камень. Как и раньше, выигрывает тот, кто своим ходом
оставляет пустой стол.

\begin{enumerate}

\item  Опишите процесс последовательного определения
выигрышных и проигрышных позиций в этой игре. Найдите
выигрышные стратегии для малых $m$ и~$n$.

\item Пусть $m=15$, $n=7$ и Ваш ход. Как надо сыграть?

\item Тот же вопрос, если $m=8$ и $n=13$.

\end{enumerate}

\item {\bf Русская рулетка.} Два офицера русской армии
повздорили из-за одной барышни. Порешили так: в барабан
шестизарядного револьвера случайным образом помещается одна
пуля. После чего они по очереди пытаются выстрелить в себя.
Впрочем, можно и спасовать, отказавшись тогда от притязаний
на невесту. Выигравший идет делать предложение, проигравший
остается лежать убитым, спасовавший возвращается в свой
полк, что, конечно, для него предпочтительней.

\begin{enumerate}

\item Формализуйте этот конфликт как игру с {\rm
совершенной информацией}, т.~е., без информационных
множеств. При этом считайте, что застрелившийся получает
полезность~$-1$, спасовавший~---~$0$, а выигравший~---
соответственно, $a$ или~$b$, т.~е. участники по-разному
оценивают качества будущей спутницы жизни .
Предполагается, что $a>0$ и $b>0$.

\item  Решите игру по доминированию, используя алгоритм
Цермело --- Куна. Какими будут равновесные стратегии в
зависимости от параметров~$a$ и~$b$? Удобно дать ответ в
виде диаграммы в координатах $(a,b)$, на которой указаны
соответствующие области. Граничные случаи (когда кому-то
все равно, как действовать) рассматривать не надо.

\end{enumerate}

\item {\bf Когда алгоритм Цермело работает плохо.} Найдите
все совершенные по подыграм равновесия в игре
\begin{center}
\begin{picture}(90,18)
\multiput(6,17)(25,0){3}{\vector(3,-4){13}}
\multiput(6,17)(25,0){3}{\vector(1,0){23}}
\multiput(30,17)(25,0){2}{\circle*{2}}
\put(5,17){\circle{2}} \put(4,19){{\bf 1}} \put(29,19){{\bf
2}} \put(54,19){{\bf 1}} \put(20,-2){3\,,\,2}
\put(45,-2){4\,,\,1} \put(70,-2){2\,,\,3}
\put(80,16){2\,,\,0}
\end{picture}
\end{center}
Не забыли про смеси?

\end{enumerate}

\subsection{Равновесия, совершенные по подыграм}

\begin{enumerate}

\item{\bf Трагедия общины: динамическая модель (поход без
завхоза).} Имеется некий объем частных благ (например,
запас продуктов питания в походе), который может
потребляться $T$~периодов. Полезность от разового
потребления $c$~единиц продукта составляет $\sqrt{c}$.
Индивидуум ценит будущее потребление наравне с настоящим,
т.~е. дисконтирование отсутствует. Как известно, в этом
случае он стремится разделить продукт поровну между
$T$~периодами.

Предположим теперь, что участников два, а~не один, и
потребляют они из одного запаса. В~каждом периоде
потребление происходит одновременно, т.~е. игроки
независимо выбирают, сколько съесть (но не более половины
от имеющегося).

\begin{enumerate}

\item Представьте это в виде игры в развернутой форме. Как
устроено дерево?

\item Как бы вы искали совершенное по подыграм равновесие в
этой игре? Указание: изобретите что-нибудь типа функций
Беллмана, найдите равновесные стратегии потребления и
выведите рекуррентные соотношения на параметры. Является ли
равновесие оптимальным по Парето?

\item Пусть $T=3$ и вначале имеется 290 единиц продукта.
Какими будут равновесные траектории потребления?

\end{enumerate}

\end{enumerate}

\subsection{Повторяющиеся игры.}

\begin{enumerate}

\item {\bf Двукратное повторение.} Рассмотрим одновременную
игру~$G_0$ с таблицей выигрышей

\begin{center}
\begin{tabular}{|c|cc|}
\hline &$X$&$Y$\\ \hline x&5,5&1,6\\ y&6,1&0,0\\ \hline
\end{tabular}
\end{center}

Представим теперь, что эта игра повторяется дважды. Все,
что происходило в первом периоде, становится общим знанием
во втором. Выигрыши первого и второго периодов суммируются
(без дисконтирования). Обозначим эту двухпериодную
игру~$G$.

\begin{enumerate}

\item Как устроено дерево этой игры? Как задаются стратегии
участников?

\item Сколько имеется чистых совершенных по подыграм
равновесий в~$G$? Указание: нужно перебрать всевозможные
варианты равновесий в подыграх второго периода, для каждого
из них прибавить выигрыши к выигрышам первого периода и
исследовать полученную однопериодную игру (будем обозначать
ее~$\tilde G$).

\item\label{z} Обратимся к смешанному расширению игры~$G$
(в поведенческих стратегиях). Пусть равновесные исходы
подыгр второго периода зафиксированы и, таким образом, игра
сведена к некоторой однопериодной игре~$\tilde G$. Может ли
профиль чистых стратегий, неравновесный в исходной
игре~$G_0$, участвовать на первом шаге в совершенном
равновесии игры~$G$? Может ли в одной и той же игре~$\tilde
G$ быть более одного равновесия, обладающего таким
свойством?

\item Более сложный вопрос: сколько всего чистых и
смешанных совершенных равновесий в игре~$G$? Указание:
достаточно выяснить, какие из игр~$\tilde G$ имеют более
одного равновесия. В~этом может помочь пункт~\ref{z}.

\end{enumerate}

\item {\bf Суперравновесия.} Рассмотрим игру с таблицей
выигрышей \begin{center}
\begin{tabular}{|c|cc|}
\hline &$a$&$b$\\ \hline $A$&0,2&2,3\\ $B$&1,0&3,1\\ \hline
\end{tabular}
\end{center}

\begin{enumerate}

\item Пусть каждый участник применяет стратегию наказания
Nash reversion (``один раз отклонишься~--- впредь буду
всегда играть равновесие Нэша''). Какие профили {\em
коррелированных} выигрышей можно реализовать таким образом
как равновесия в бесконечно повторяющейся игре?

\item Изобразите на плоскости множество коррелированных
выигрышей, достижимых в суперравновесиях в силу Народной
теоремы.

\item Рассмотрим такую попытку реализовать
выигрыши~$(2,3)$: в начальный момент играется~$(A,b)$, а
далее игрок~1 играет~$A$, если на предыдущем ходу игрок~2
сыграл~$b$ и~$B$, если~$a$; игрок~2 действует симметричным
образом: $a$, если~$B$ и~$b$, если~$A$. Является ли эта
пара стратегий равновесием Нэша в бесконечно повторяющейся
игре (при~$\delta$, достаточно близком~к~$1$)? Если да, то
является ли равновесие совершенным по подыграм?

\end{enumerate}

\end{enumerate}

\subsection{Статические игры с неполной информацией}

\begin{enumerate}

\item {\bf Байесовский семейный спор или ``капризная
жена''.} Муж и жена решают, куда пойти~--- на
футбол~($Ф$\/) или на балет~($Б$\/). Все осложняется тем
обстоятельством, что жена может находиться в хорошем
настроении (и тогда стремится быть вместе с мужем), а может
и в плохом (и тогда видеть его не может). Короче, вот
таблица игры. Строки соответствуют мужу, а столбцы~---
жене, чей выигрыш зависит от настроения ($Х,П$):
\nsk\begin{center}
\begin{tabular}{|c|cc|}
\hline &$Ф$&$Б$\\ \hline $Ф$&3;\ 2,0&1;\ 1,3\\ $Б$&0;\
0,2&2;\ 3,1\\ \hline
\end{tabular}
\end{center}

\nsk Найдите все байесовские равновесия, предполагая, что
настроения $Х$~и~$П$ наступают с равными вероятностями. Не
забудьте про смеси.

\item{\bf Rendez-vous-1.} Александра и Виктор живут на
одной улице (считаем, что их места жительства являются
случайными точками, равномерно и независимо распределенными
на отрезке~$[-1,1]$). Для того, чтобы договориться о
встрече, они сообщают друг другу, где живут, и встречаются
ровно посередине между названными точками. Сообщения
делаются одновременно и независимо и необязательно правдивы
(но все же в пределах~$[-1,1]$). Полезность каждого из
участников равна пройденному расстоянию, взятому со знаком
``минус''.
\begin{enumerate}
 \item\label{br} Найдите оптимальный ответ Александры на правдивую
стратегию Виктора (когда он в любом случае сообщает свое
фактическое место жительства).
 \item Докажите, что если каждый из участников будет играть, как Александра
в п.~\ref{br}, то получится байесово равновесие. Будет ли
оно эффективным?
 \item Разработайте механизм наподобие схемы Гровса, делающий
равновесие ex-post эффективным.
\end{enumerate}

\item {\bf Аукцион первой и второй цены.} На аукционе, в
котором участвуют два покупателя, продается картина.
Ценность ее для покупателя~$i$ составляет~$x_i$, $i=1,2$,
где $x_1,x_2$~--- независимые случайные величины,
равномерно распределенные на отрезке~$[0,1]$. Информация
об~$x_i$ доступна только $i$-му участнику. Игрок~$i$ тайно
заявляет цену $b_i$ и запечатывает заявку в конверт, на
котором пишет свою фамилию.

\begin{enumerate}

\item Пусть картина достается тому, кто назвал наибольшую
цену, каковую он и платит. Предполагая стратегии игроков
такими, что размер заявки пропорционален субъективной
ценности картины, найти байесовское равновесие. Будут ли
игроки декларировать свои истинные оценки картины?

\item Тот же вопрос для аукциона ``второй цены'', когда
картина достается тому, кто назвал наибольшую цену, но
платит он цену, заявленную проигравшим аукцион.

\end{enumerate}

\end{enumerate}

\subsection{Секвенциальные (совершенные байесовские) рав\-но\-весия}

\begin{enumerate}


\item{\bf Соседи.} Антон и Борис, соседи по общежитию,
каждый день тратили по 20~мин. на поход в магазин за
продуктами. Им это надоело и, чтобы сэкономить время, они
договорились в течение ближайших четырех дней ходить в
магазин по очереди и покупать на всех: в первый день Антон,
затем Борис и~т.~д. Покупка продуктов для соседа отнимает
лишние 5~мин, поэтому у идущего в магазин появляется
соблазн нарушить соглашение и купить продукты только себе.
Если кто-то хоть раз так сделает, то в последующие дни
ходить в магазин снова будут по отдельности (да и в этот
день обманутому соседу придется прогуляться).

Антон всегда действует сообразно обстоятельствам (в каждый
из двух своих дней может выбирать, выполнить или нарушить
соглашение), а вот Борис может являться оппортунистом, как
Антон, а может быть и честным, т.~е. всегда выполнять
соглашение (в~этом случае у него оба раза по единственному
возможному действию). Антон верит, что Борис с
вероятностью~$p>0$ является честным.

\begin{enumerate}
  \item Как выглядит развернутая форма игры, если полезности
  участников равны выигранному времени?
  \item При каких~$p$ существует секвенциальное  равновесие,
  в котором Антон в первый день играет честно (выполняет
  соглашение)?
\end{enumerate}

\end{enumerate}

\subsection{Коррелированные равновесия}

\begin{enumerate}

\item {\bf Примеры на коррелированные равновесия.}
Изобразить на плоскости $(u_1,u_2)$ область коррелированных
равновесий для следующих игр:

\begin{enumerate}

\item Семейный спор:
\[\left[\begin{array}{ll}
4,3&2,2\\ 0,0&3,4\\
\end{array}\right]\]

\item Преследование:
\[\left[\begin{array}{ll}
2,0&0,1\\ 0,1&2,0\\
\end{array}\right]\]

\end{enumerate}

\item {\bf Возможности необязывающих соглашений.} Имеются
два участника, каждый из которых может вести себя
эгоистически ($Э$) или кооперативно ($К$). Если оба
играют~$К$, то получают по~$c$. Если один играет~$Э$, а
другой~$К$, то выигрыши составляют, соответственно, $a$
и~$b$ (${0<b<c<a}$). Если оба эгоисты, то по нулям.

\begin{enumerate}

\item С какими весами надо смешивать исходы игры, чтобы
получилось коррелированное равновесие? Задайте ответ в виде
выпуклой оболочки нескольких наборов весов.

\item Найдите максимально возможный суммарный выигрыш,
реализуемый в коррелированных равновесиях. В~каких случаях
он превышает то, что достижимо в равновесиях Нэша?

\end{enumerate}

\end{enumerate}

\subsection{Задача торга (для двух участников)}

\begin{enumerate}

\item {\bf Торг в модели обмена.} У агента~$1$ есть единица
товара~$x$, а у агента~$2$~--- единица товара~$y$.
Предпочтения участников задаются функциями полезности,
соответственно, $u_1=\sqrt{xy}$ и
$u_2=\min\left(x,\dr{3}{2}y\right)$.

\begin{enumerate}

\item\label{ras} Пусть агенты торгуются за распределение
товаров $x$ и~$y$. Найти (и изобразить на рисунке в
координатах $u_1,u_2$) область допустимых выигрышей
(Парето-границу) и решение Нэша. Изобразить в ящике
Эджворта множество оптимальных по Парето распределений
товаров и точку, соответствующую решению Нэша.

\item\label{p} Пусть теперь предметом торга является только
цена, по которой происходит обмен, а размер сделки
устанавливается после этого первым игроком единолично.
Снова найти область допустимых выигрышей и решение Нэша
(отобразить на тех же рисунках).

\item\label{ge} Указать на тех же рисунках распределение
товаров и пару выигрышей, соответствующие общему равновесию
в данной экономике обмена. Сравнить результаты
пунктов~\ref{ras}--\ref{ge} и прокомментировать.

\end{enumerate}

\item {\bf Rendez-vous-2.} Джон и Мэри живут на одной улице
(см. рис.) и по умолчанию встречаются у фонтана ($x_0$).
Они могут договориться о встрече в любом другом месте
улицы. Полезность участника равна (со знаком минус)
расстоянию, которое ему нужно пройти.

\begin{enumerate}

\item Нарисуйте множество допустимых выигрышей $(u_1,u_2)$.
Изобразите на нем положения статус-кво при всех возможных
значениях параметра~$x_0$.

\item Пусть $x_N$~--- место встречи, соответствующее
решению Нэша этой задачи торга. Найдите $x_N$ как функцию
от~$x_0\in\R$ и постройте ее график.

\item Тот же вопрос для $x_K$~--- решения
Калаи---Смородински.

\item А что можно сказать про эгалитарное и утилитарное
решения?

\end{enumerate}

\medskip

\begin{center}
\begin{picture}(85,10)
\multiput(1,5)(20,0){3}{\line(1,0){18}}
\put(61,5){\vector(1,0){20}} \put(30,4){\line(0,1){2}}
\multiput(20,5)(20,0){2}{\circle{2}}
\put(60,5){\circle*{2}} \put(17,7){$-1$} \put(29,7){$0$}
\put(39,7){$1$} \put(58,7){$x_0$} \put(15,0){Джон}
\put(35,0){Мэри} \put(53,0){Фонтан} \put(83,4){$x$}
\end{picture}
\end{center}

\end{enumerate}

\subsection{Кооперативные игры: ядро и Шепли}

\begin{enumerate}

\item {\bf Подземные музыканты.} Оркестр из трех музыкантов
($А$,~$В$,~$С$) играет в подземном переходе. Поодиночке они
могли бы заработать, соответственно, $6$, $18$ и $30$
рублей в час. Играя по двое, они бы получили:
$А$~и~$В$~---~36, $А$~и~$C$~---~48, $B$~и~$C$~---~54.
А~вместе они имеют~$72$.

\begin{enumerate}

\item Будет ли отвергнут равный дележ, и если будет, то
какими коалициями?

\item Найдите ядро игры, т.~е. все дележи, которые не будут
отвергнуты.

\item Является ли игра супераддитивной? Супермодулярной?

\item Найдите все точки Вебера и вектор Шепли. Что из
найденного принадлежит ядру?

\end{enumerate}

\item {\bf Парламент.} Конгресс и Сенат состоят из трех
членов каждый. Закон принимается только если в обеих
палатах набрано большинство.

\begin{enumerate}

\item Найдите ядро этой кооперативной игры.

\item\label{pres} Пусть вдобавок имеется еще президент,
одобрение которого обязательно. Сколько он получит в ядре?
В~векторе Шепли?

\item А потом две палаты объединили. Теперь нужно просто
$4$ голоса из~$6$ плюс президентское одобрение. Вроде бы,
парламент стал сильнее, так как больше выигрывающих
коалиций, чем в~\ref{pres}. Как изменилась ``зарплата''
президента?

\end{enumerate}

\item {\bf Симметричные игры.} Назовем кооперативную игру с
трансферабельной полезностью {\it симметричной}, если
выигрыш $v(K)$ коалиции $K$ зависит только от ее
численности $k=|K|$. Без ограничения общности считаем, что
$v(N)=N$. Какой должна быть функция $v(k)$ для того, чтобы
ядро было непустым? чтобы игра была супермодулярной?

\item {\bf Строительство дороги.} Четыре поселка $A,B,C,D$
расположены на берегу большого озера (см. рис.). Каждый
поселок нуждается в автомобильном сообщении с тремя
остальными, причем кратчайшим путем (так, незамкнутая
дорога~$B C D A$ не устраивает жителей поселка~$B$,
поскольку они хотят ездить в~$A$ напрямик). Местные власти
решили скинуться и построить кольцевую дорогу вокруг озера,
соединяющую поселки. Вопрос состоит в том, как разделить
между поселками издержки по строительству 120~км дороги.

\begin{enumerate}

\item Для каждой коалиции найдите минимальную протяженность
нужной ей дороги. Опишите ядро игры. Является ли игра
супермодулярной?

\item Пусть поселки равноправны в переговорном процессе, а
затраты делятся, исходя из вектора Шепли. Сколько
километров дороги должен профинансировать каждый поселок?

\item Пусть $n_A$, $n_B$, $n_C$, $n_D$~--- число жителей в
поселках, причем не обязательно $n_A=n_B=n_C=n_D$. Как в
этом случае разумно распределить затраты? В~каком
соотношении должны находиться числа~$n_A,n_B,n_C,n_D$,
чтобы все жители платили один и тот же налог на
строительство дороги?

\end{enumerate}

\begin{center}
\begin{picture}(40,38)
\put(20,20){\oval(40,30)} \put(15,35){\circle*{2}}
\put(28,35){\circle*{2}} \put(40,18){\circle*{2}}
\put(0.6,9){\circle*{2}}
\multiput(3,22)(8,0){4}{\multiput(0,0)(4,4){3}{\~}}
\multiput(3,5)(8,0){4}{\multiput(0,0)(4,4){3}{\~}}
\multiput(3,13)(0,17){2}{\multiput(0,0)(32,-8){2}{\~}}
\put(14,19){Озеро} \put(13,37){A} \put(27,37){B}
\put(42,17){C} \put(-3,6){D} \put(16.5,36){12\,км}
\put(41,26){24\,км} \put(22,1){48\,км} \put(-11,19){36\,км}
\end{picture}\par
Проект дороги.
\end{center}

\item {\bf Охрана.} Имеется $6$ производителей
$A,B,C,D,E,F$, каждый из которых может заработать~$\$1$, и
два охранника~$P$~и~$Q$, не производящих ничего. Коалиция
получает суммарную выручку ее участников, но только в том
случае, если среди них есть хотя бы один охранник. Иначе
приходят грабители и все забирают. Сколько нужно платить
охранникам? Ответьте на этот вопрос с точки зрения ядра и
вектора Шепли.

\item{\bf Ядро экономики.} Алиса, Берта и Виола имеют по
единице товара, который оценивают, соответственно, в 3, 6 и
8 долларов. Густав, Даниил, Евгений и Жорж могли бы купить
по единице этого товара и готовы заплатить за него,
соответственно, 2, 4, 7 и 9 долларов.
\begin{enumerate}
  \item Формализуйте эту ситуацию в виде кооперативной игры с
  побочными платежами (задайте выигрыши коалиций).
  \item Пусто ли ядро этой игры?
\end{enumerate}

\item{\bf Разложение по элементарным играм.}
В~кооперативной игре~$v$ с побочными платежами участвуют
игроки $1$,~$2$,~$3$. Выигрыши коалиций заданы следующим
образом:
\[\begin{array}{llll}
  v(1)=4,&v(2)=9,&v(3)=4,\hi\\
  v(1,2)=15,&v(1,3)=12,&v(2,3)=13,&v(1,2,3)=20.\\
\end{array}\]
\begin{enumerate}
  \item Найдите разложение этой игры по базису из элементарных игр
  $v_S$, $S\subseteq\{1,2,3\}$.
  \item Как из найденного разложения получить вектор Шепли?
\end{enumerate}

\end{enumerate}

A.Savvateev "большие" задачи-проекты по играм:

1. Дуополия Бертрана с неодинаковыми издержками,
и/или с ограничением целочисленности цены. Найти
все Нэшевские равновесия. Какие еще могут быть
равновесия?

2. (Пионерлагерь=) Сторож колхозного поля и вор. На
поле можно забраться 4 путями: A,B,C,D, удаленными
от сторожки на расстояния, соответственно, 2, 4, 6,
8 (сот метров). В случае поимки вора, сторож
ожидает с него штраф \$10, а неприятность каждой
сотни метров ходьбы от сторожки для себя оценивает
в \$1. Хоть один из путей сторож обязан покараулить
в любом случае. Найти все смешанные равновесия
Нэша: будет ли  вор пользоваться в разные дни
разными путями, и разные ли пути будет караулить
сторож?

\section{P.Ordeshook: Political games}

{\Large Задачи из книги P.Ordeshook по политическим
играм}



стр. 2

Вместо обзора литературы по политической теории, мы
сосредоточимся на тех идеях, которые дают ключ к
теоретическому пониманию некоторых специфических
политических процессов. В частности, мы подойдем к
исследованию ответов на следующие вопросы:

1. Как должны две главные партии в системе выборов
по правилу большинства отвечать на угрозу входа
третьей партии?

2. Почему процедура правила большинства в
американских выборах приводит, в сущности, к
конкуренции только между двумя главными партиями?

3. Является ли редкость смены партии у власти
обязательно симптомом чего-то неправильного в
демократии?

4. Должны ли партии всегда предпочитать как можно
большее число мест в законодательном органе?

5. Создает ли плохая информированность избирателей
неизбежное смещение поведения, выгодное для тех,
кто имеет нужную информацию относительно
правительственной политики?

стр 3

6. Могут ли результаты взаимодействия людей
оказаться иррациональными, даже если каждый человек
в обществе удовлетворяет почти любому определению
рационального?

7. Почему Конгресс часто посвящает больше времени и
усилий обсуждению процедур, чем обсуждению
фактического законодательства на своих заседаниях?

8. Всегда ли предпочтения, продемонстрированные
законодателями при голосовании, являются хорошим
отражением предпочтений их электората, даже среди
законодателей, стремящихся только к переизбранию?

9. Какие виды законодательства имеют тенденцию
стимулировать законодателей торговать голосами при
решении вопросов в конгрессе?

10. Ведет ли система комитетов в Конгрессе к
результатам, которые закономерно отличаются от
результатов, которые преобладали бы, если бы
Конгресс обсудил каждый в отдельности вопрос в
целом?

11. Почему налоговая реформа является особенно
проблематичной темой для рассмотрения в Конгрессе,
особенно в год выборов?

12. Почему одни и те же конгрессмены так часто
переизбираются в Конгресс, несмотря на тот факт,
что Конгресс как учреждение мало уважается
обществом?

13. Должны ли мы рассматривать конфликты между
президентом и членами Конгресса просто как
столкновения личностей, или существует объяснение,
основанное на факте, что президенты и члены
Конгресса избираются различными способами?

14. Помимо общего недоверия избирателей, какие
теоретические и логические обоснования существуют
для представительной демократии, завещанной нам
Отцами-основателями Конституции США, в частности,
для двухпалатной законодательной власти?

15. Почему правительство регулирует тарифы такси и
дальних перевозок, в то время как подобное
регулирование терпит неудачу в случае регулировки
цен на автомобили и большинство других главных
товаров?

16. Часто утверждается, что "группы влияния с
особыми интересами" покупают политических деятелей
и политику управления, но как насчет точки зрения,
что в ходе избирательной кампании политики вымогают
взносы из заинтересованных групп?

17. Когда выгоднее принять решение прежде других, а
когда выгоднее повременить?

стр 56

1. Пусть свободная конкуренция преобладает в
отрасли промышленности, состоящей из двух фирм,
каждая из которых продает по 20 миллионов единиц
изделий с общей прибылью \$1/единицу. Но если они
тайно сговорятся, чтобы установить цену повыше, то
каждый продаст по 15 миллионов единиц с прибылью по
два доллара за каждую. Если же одна фирма
отступится от договора и продаст свой товар по
более низкой цене, она выручит \$35 миллионов, в то
время как другая фирма не продаст ничего. Прежде
чем каждая фирма установит цену (что они делают
одновременно), Сенатор Гилли Боб предлагает
заключить соглашение о патентовании, по которому
каждый платит налог \$.20/ на единицу, и производит
изделие по фиксированным картелем ценам - явно,
чтобы застраховаться от "разрушительной
конкуренции", которая могла бы "оставить
трудолюбивых Американцев безработными." Создайте
развернутую форму этой ситуации, где каждая фирма
должна сначала одобрить или не одобрить соглашение
о патентовании, которое действует, только если обе
фирмы соглашаются на это.

2. Допустим, имеются три штата (1, 2, и 3) равного
размера, и подсчет голосов проходит сначала в 1
штате, затем во 2, а затем в 3, и что победитель на
выборах должен собрать большинство (два) штатов.
Выведите развернутую форму игры, представляющую
решение одного гражданина в государстве 3, который
должен решить, голосовать ли и за кого из двух
кандидатов. Допустим, этот человек рассматривает
решения всех других избирателей и результаты
голосования в других государствах как случайные.
(Допустим, что связи между государствами нет.)

\bigskip стр 57

3. Выведите развернутую форму повестки дня "А
против В, победитель против С " для
законодательного органа из трех людей, в котором
законодатель 3 наблюдает выбор 1-го, но в котором
ни один законодатель не наблюдает ни за какими
другими выборами.

4. Допустим, игроки 1 и 2 последовательно выбирают
из двух вариантов ходов двоичных решений, сначала
1, затем 2, затем 1, затем 2, и т.д, и
предположите, что все решения наблюдаются обоими
людьми. Изобразите минимальную развернутую далеко
идущую форму игры , которая позволит Вам
представить ситуацию, в которой 1 имеет совершенную
память, но 2 может вспомнить только свои последние
два шага и последний ход 1-го.

5. Бюрократ отдела защиты наблюдает за двумя
взаимосвязанными программами X и Y. Превышение
недавней стоимости заставляет его "хоронить" эти
издержки с учетом одной из этих программ. Два
агента из ведомства учета правительства, за которым
Вы наблюдаете, будут в конце года делать обзор его
записей. Каждый агент может быть назначен
ответственным за: (A) ревизию записей персонала
обеих программ; (В) ревизию описей обеих программ;
(С) ревизию расходов поставщиков частей для обеих
программ. Если один из ваших агентов обнаруживает
превышение, Вы получаете + 10, в противном случае,
Вы получаете -10, в то время как соответствующая
плата для бюрократа - -10 и + 10. Вероятность, что
индивидуальный ревизор обнаружит превышение, дана в
центре его ревизии, и программа, в которой
превышение "похоронено", следующая:

\bigskip

\begin{tabular}{lll}
& X & Y \\ \cline{2-3} A & \multicolumn{1}{|l}{.5}
& \multicolumn{1}{|l|}{.2} \\ \cline{2-3} B &
\multicolumn{1}{|l}{.3} & \multicolumn{1}{|l|}{.6}
\\ \cline{2-3} C & \multicolumn{1}{|l}{.1} &
\multicolumn{1}{|l|}{.8} \\ \cline{2-3}
\end{tabular}
\ \

Допустим, Вы можете распределять двух своих
ревизоров, как Вам угодно и что в противном случае
ревизоры действует независимо, изобразите эту
ситуацию в развернутой форме.

стр 58

7. Допустим, трое людей (1, 2, и 3) должны выбрать
одного кандидата из списка A, В, С и D. Допустим,
все люди наблюдают все предшествующие выборы,
выведите развернутую форму процедуры, в которой
сначала 1-ый удаляет кандидата, затем 2, и затем 3,
и в которой кандидат, который остается, избран.

8. Допустим, что до того, как сделать выбор,
бросается жребий, чтобы определить, какая из двух
следующих стратегических форм описывает выигрыши
Ваши и Вашего противника, (первый номер в каждой
ячейке обозначает Ваш выигрыш -- выбор из строки -
в то время как второй номер обозначает выигрыш
Вашего противника):

\bigskip \begin{tabular}{llllll}
& $b_{1}$ & $b_{2}$ &  & $b_{1}$ & $b_{2}$ \\
\cline{2-3}\cline{5-6} $a_{1}$ &
\multicolumn{1}{|l}{9,5} & \multicolumn{1}{|l}{1,
-3} & \multicolumn{1}{|l}{} &
\multicolumn{1}{|l}{3,0} & \multicolumn{1}{|l|}{9,
-2 } \\ \cline{2-3}\cline{5-6} $a_{2}$ &
\multicolumn{1}{|l}{3,7} & \multicolumn{1}{|l}{4,6}
& \multicolumn{1}{|l}{} & \multicolumn{1}{|l}{9,9}
& \multicolumn{1}{|l|}{3,8}
\\ \cline{2-3}\cline{5-6}
\end{tabular}

a. Допустим, никто не наблюдает за результатом
жребия и Вы оба должны выбрать одновременно,
изобразите развернутую форму ситуации.

b. Изобразите развернутую форму, допускающую, что
Вы можете тайно заплатить три доллара, чтобы узнать
результат жребия.

9. Изобразите следующую ситуацию в развернутой
форме: Вы ищете среди множества используемых
учебников книгу, которая наиболее близко
соответствует новому тексту. Однако, Вы не знаете
ни как выглядит "наилучший " текст, ни его
состояния, и поскольку время дорого, Вы думаете,
что неблагоразумно планировать просмотреть всю
груду с самого начала, чтобы найти наиболее
подходящую книгу. Таким образом, Вы просматриваете
каждую книгу один раз, пока не находите текст,
который является "приемлемым". Чтобы сделать эту
задачу решаемой, допустим, что есть только три
текста и что где-то от ноля до трех из них являются
приемлемыми.

10. Допустим, что ценность некоторого объекта =
$x_{1}+x_{2}$, где и $x_{1}$ и $x_{2}$ -
произвольные переменные, которые могут принимать
значения 0 или 1. Допустим, один человек наблюдает
$x_{1}$, но не $x_{2}$, в то время как второй
человек наблюдает $x_{2}$, но не $x_{1}$.
Изобразите в развернутой форме аукцион, в котором
люди 1 и 2, после наблюдения за $x_{1}$ и $x_{2}$,
соответственно, предлагают цену 5 или 1.5 за
объект, предоставляя свои предложения на
рассмотрение в запечатанных конвертах. Допустим,
что человек, предоставляющий на рассмотрение самое
высокое предложение, получает объект, но должен
заплатить цену, которую он предлагал, и учтите
также, что в случае равенства ставок победителя
определяет жребий.

стр 59

11. Конгрессмен Порк представляет район со
значительной долей промышленности электроники,
которая находится в стадии резкого экономического
спада из-за иностранной конкуренции.
Соответственно, он предоставляет запрос на
федеральную субсидию на \$100 миллиардов для
бытовой электроники, но он знает, что несколько
членов комитета во главе с Конгрессвуман Пэм Соник
противостоят его запросу. Закон будет обсужден на
следующей неделе, и Порк может предложить один из
двух аргументов в свою пользу: (1) Соединенные
Штаты нуждаются в технологии бытовой электроники
для возможного преобразования в военных целях, в
этом случае \$50 миллиардов были бы использованы.
Однако, если Соник ожидает этот аргумент, она может
найти экспертов извне, которые засвидетельствуют,
что такая технология не соответствует потребностям
защиты, опровергая аргумент Порка и отвергнув
запрос. (2) Порк мог бы подчеркнуть рабочие места,
которые федеральная субсидия обеспечит в
избирательных округах большинства членов комитета.
В этом случае, он мог бы получить \$40 миллиардов.
Контраргумент Соник, упомянутый выше, не имел бы
никакого эффекта, но Пэм Соник могла бы
организовать акцию протеста (собрать подписи)
против запроса. В этом случае, комитет пошел бы на
компромисс на \$25 миллиардов. Однако, акция
протеста не имела бы никакого эффекта, если военный
аргумент Порка будет неоспорим, так как комитет
понимает, что деньги не играют роли, когда дело
касается национальной безопасности. Изобразите эту
ситуацию в стратегической форме, приписывая
правдоподобные выигрыши (полезности) исходов и для
Порка, и для Соник.

12. Что касается законодательного примера повышения
платы в разделе 1.6, изобразите развернутые и
стратегические формы для каждого из следующих
обстоятельств:

a. Законодатель С, а не В, наблюдает выбор A.

b. Законодатель В, а не С, наблюдает выбор A.

с Законодатель С, а не A, наблюдает выбор B.

стр 90

1. Вы член комитета из трех человек, которые должны
выбрать один вариант из списка (A,B,C,D). Допустим,
у комитета есть следующие предпочтения
(упорядоченные от лучших к худшим):

\begin{tabular}{lllll}
Вы: & A & B & C & D \\ член 2: & D & C & A & B \\
член 3: & C & B & D & A\end{tabular}

Какую из следующих процедур Вы бы предпочли видеть
выполненной, если Вы полагаете, что два других
члена комитета были хитры: (1) повестка дня,
которая сначала ставит В против С, победителя
против A, победителя против D. (2) повестка дня,
которая сначала ставит С против A, победителя
против D, и победителя против В; (3) повестка дня,
которая сначала ставит В против D, победителя
против С, победителя против A; или (4), Вам все
равно, кто выбран.

стр 91

2. Допустим, предпочтения комитета из пяти членов
следующие:

\begin{tabular}{|lllll|}
\hline 1 & 2 & 3 & 4 & 5 \\ \hline A & B & C & C &
B \\ B & C & A & A & C \\ C & A & B & B & A \\
\hline
\end{tabular}

Каков будет результат повестки дня (A.B.C), если
только избиратель 2 стратегический? Если избиратель
2 может обучить по крайней мере одного другого
человека быть хитрым, кого должен он обучить,
допустив, что избиратель 1 слишком тупой, чтобы
понять, что от него требуется?

3. Допустим, большинство в законодательном органе
предпочитает В. Если Вы противопоставлены A, и если
за А и В должны проголосовать сначала, независимо
от поправок, и если каждый избиратель - хитрый,
какой вариант Вы бы предпочли представлять: С или
D? С создает большинство циклов по правилу \ "А
предпочитается В, который предпочитается С, который
предпочитается А" в то время как D наносит
поражение и А, и В. Ваши предпочтения: "В
предпочитается D, который предпочитается С, который
предпочитается A. "

4. Вы - помощник депутата, советующий главе
комитета, который собирается выбрать между
выдвижением законопроекта А или В из комитета. Если
А выбран, он наверняка проиграет варианту статус
кво Q, если В выбран, он будет редактироваться в
ходе голосования (возникнет вариант С), и Вы
сможете предложить альтерантиву D. Допустим,
оставшиеся члены законодательного органа поделятся
на три одинаковые группы со следующими
предпочтениями:

\begin{tabular}{llllll}
группа 1: & B & D & A & C & Q \\ группа 2: & C & Q & A & D & B \\
группа 3: & Q & B & C & A & D\end{tabular}

Вы - член второй группы. Допустим, законодательный
орган голосует, используя следующую повестку дня: "
заменитель против поправки, победитель против
закона, победитель против статуса кво." Вы
уверенны, что при сложившемся положении дел,
каждый, кроме Вас, проголосует искренне. Допустим,
что долларовое значение для Вас каждого варианта: С
= \$2,000, Q = \$1,500, А=- \$500, D = \$0, и В =
-\$3,000. Какова верхняя граница суммы, которую Вы
желали бы залатить, чтобы иметь кого-то для
обучения законодательного органа, чтобы Вы, как и
все, голосовали изощренно?

стр 92

5. Законодательный орган из трех членов
рассматривает четыре варианта, где каждый вариант
воздействует на количество денег, собирающихся в
районе законодателя. Пусть выигрыши (в тысячах)
каждого района по каждому варианту будут таковы:

\begin{tabular}{l|lll|}
\cline{2-4} & район1 & \multicolumn{1}{|l}{район2}
& \multicolumn{1}{|l|}{район3} \\ \hline
\multicolumn{1}{|l|}{A(статус кво)} & 300 & 0 & -400 \\
\multicolumn{1}{|l|}{B(закон комиссии)} & 500 & -600 & 0 \\
\multicolumn{1}{|l|}{C(возможно редактируемый)} & 0 & 800 & -900 \\
\multicolumn{1}{|l|}{D(возможно редактируемый)} &
-900 & 400 & 450 \\ \hline
\end{tabular}

Варианты А и В выдвинуты как предложения, и Вы
должны решить, предложить ли редактируемый закон.
Если Вы предлагаете С - повестка дня будет С против
В, победитель против A; если Вы предлагаете D -
повестка дня будет D против В, победитель против A.
D и С также предложены, повестка дня будет - D
против С, победитель против В, победитель против A.

a. Допустим, Вы - представитель района 2. Какую из
поправок должны Вы предложить, \$800 000 - которую
заплатит Ваш район или \$400 000, которую заплатит
D?

b. Допустим, Вы - председатель (и диктатор)
релевантной законодательной подкомиссии, и Вы
можете отправить от своей подкомиссии В, С или D
как закон, который законодательный орган должен
рассмотреть. Но Вы также уверены, что любые
варианты, в которых Вы потерпите неудачу, будут
представлены в собрании как поправки. Таким
образом,

Если Вы отправляете В, повестка дня будет "С против
D, победитель против В, победитель против А".

Если Вы отправляете С, повестка дня будет - "В
против D, победитель против С, победитель против A.
"

Если Вы отправляете D, повестка дня будет - "С
против В, победитель против D, победитель против A.
"

Что бы Вы выбрали как свой закон, если бы Вы были
представителем из района 2?

с. Учитывая часть (b), если подкомиссией
используется правило большинства, чтобы выбрать к
докладу В, С или D, что получится?

стр 93

6. Вы - игрок 1, и Ваш противник - игрок 2, имеете
по три лошади. Пусть скорости ваших лошадей будут
обозначены $a_{1}$, $a_{2}$, и $а _{3}$, и пусть
скорости лошадей вашего противника
будут$b_{1},b_{2}$и $b_{3}$. Пусть их скорости
будут следующими: $b_{1}>a_{1}>b_{2}>а _{2}>b_{3}>а
_{3}$. Пусть три гонки должны пройти
последовательно, и каждый из Вас должен решить,
какую лошадь ввести в каждую гонку (ни одна лошадь
не может участвовать в гонках дважды). Хотя Вам
позволяют решить, какую лошадь вводить после того,
как предыдущая гонка закончена. Вы и игрок 2 должны
одновременно выбрать лошадь для первой гонки, но
Ваш противник выбирает первым (и Вы наблюдаете этот
выбор) для второй гонки. Человек, который
выигрывает две или больше гонки, выигрывает долю
капитала в предприятии \$1000. Однако, так как
заранее известно, что Ваши лошади в среднем
медленнее, Вам платят \$250, чтобы Вы участвовали.
Изобразите развернутую форму ситуации и выведите
равновесную стратегию . Существует ли однозначный
конечный результат?

\bigskip

Страница 133

7. Выведите развернутую форму, соответствующую
следующему описанию, и найдите равновесные
cтратегии: " Есть два черных ящика. Игрок 1 прячет
жемчуг в одном из них: затем Игрок 2, не зная, в
каком ящике находится жемчуг, смотрит в одном из
них. Если жемчуг находится в ящике 1, и он смотрит
там, он видит это с половинной вероятностью. Если
жемчуг находится в ящике 2, и он смотрит туда, он
видит это с половинной вероятностью. Если он
смотрит в неправильном ящике, он не видит ничего (и
даже не сообщается, что ящик пустой). Выигрыш -
пять игроку 2 и минус пять -- игроку 1, если 2
находит жемчуг; иначе нет никакого выигрыша.

8. Рассмотрите следующий сценарий: Игрок А имеет
\$7,200, игрок В имеет \$5,000, и игрок С имеет
\$3,601. У них есть возможность сделать ставку один
раз, оставаясь в игре. Допустим, нет преимуществ
быть вторым против третьего, и игрок с наибольшей
суммой денег выигрывает, и сумма денег выдается
наличными. Каждый игрок должен решить, ставить ли
ему все или ничего. До ставок (которые они должны
сделать одновременно), бросают жребий, чтобы
определить, какое "состояние мира" произойдет: в
состоянии 1 - А и В выигрывают, но С проигрывает; в
состоянии 2 - А проигрывает, но выигрывает С.
Допустим, игрок ставит "все" и выигрывает, его
благосостояние удвоено. Если он проигрывает, его
благосостояние - ноль. Если игрок не делает ставок,
его благосостояние не изменяется. Игрок с
наибольшим благосостоянием выигрывает в конце игры.
Выведите развернутую форму этой ситуации и
покажите, что каждый игрок делает в равновесии.

9. "Кот" и "мышь" каждый стартуют в противоположных
углах простого лабиринта, показанного ниже. Обоим
животным требуется пять секунд для того, чтобы
пересечь один сегмент, но проходы достаточно узки,
и ни одно животное не может развернуться в
лабиринте. Если кот съедает мышь, его выигрыш - +1,
а мыши - -1; и наоборот. После двадцати секунд
мышь, если жива, будет спасена из лабиринта.

a. Допустим, стратегия - это цельный план,
относительно того, куда поворачивать (налево,
направо, прямо вперед) в каждом стечении
обстоятельств в лабиринте, обладает ли эта игра
равновесием в чистых cтратегиях, допуская, что ни
кот, ни мышь не могут видеть другого, пересекающего
лабиринт (пока конечно, еще не "слишком поздно")?

b. Изменится ли Ваш ответ, если мышь будет спасена
только после тридцати секунд?

с. Изменится ли Ваш ответ на часть (a), если и кот
и мышь могут наблюдать, что делает другой каждые 5
секунд? Если да, то как и почему?

\begin{tabular}{l|l|llll}
\cline{1-1}\cline{3-5} кот &  &  &  &  &  \\
\cline{1-1}\cline{3-5} &  &  &  &  &  \\ &  &  &  &
&  \\ \cline{3-6} &  &  &  &  & мышь \\ \cline{3-6}
\end{tabular}

стр 134

10. До того как начать игру, бросается жребий,
чтобы определить, в какую из следующих двух игр
будут играть:

\begin{tabular}{|l|l|l|l|l|}
\cline{1-2}\cline{4-5} 9,5 & 1,-3 &  & 3,0 & 9,-2
\\ \cline{1-2}\cline{4-5} 3,7 & 4,6 &  & 9,9 & 3,8
\\ \cline{1-2}\cline{4-5}
\end{tabular}

Если ни один из игроков не знает результата жребия,
сколько хотел бы 1 (Мистер Строка) заплатить, чтобы
узнать результат жребия ( учтите, что вся
выигранные игры - в терминах долларов и что
полезность и деньги являются эквивалентными) и
какая цена за эту информацию будет безразлична для
2?

11. Вы - член комитета, который должен выбрать
между А и В. Сейчас положение дел таково, что В
наносит поражение A, но Вы ненавидите В, поэтому Вы
предлагаете альтернативу, С, который будет
голосовать против В, затем победитель будет
выставлен против A. Вы надеетесь, что С победит В,
но проиграет A. Хотя результат зависит от Ваших
аргументов за С и аргументов Вашего противника.
Социальное упорядочение будет определено этими
аргументами. Ваше предпочтение - A$>$С$>$В, и предпочтение Вашего противника
- В$>$С$>$А. Вы (Мистер
Строка) и Ваш противник (Мистер Столбец) каждый
имеете по два альтернативных аргумента, и они
выдают следующий профиль предпочтений по правилу
большинства:

\begin{tabular}{lll}
& Аргумент1 & Аргумент2 \\ \cline{2-3} Аргумент1 &
\multicolumn{1}{|l}{$B>C>A$ (переходный)} &
\multicolumn{1}{|l|}{$C>A>B>C$ (круг)} \\
\cline{2-3} Аргумент2 &
\multicolumn{1}{|l}{$C>B>A>C$ (круг)} &
\multicolumn{1}{|l|}{$C>B>A$ (переходный)} \\
\cline{2-3}
\end{tabular}

Если Вы должны выбрать каждый аргумент, не зная что
выберет Ваш противник, и если каждый хитер, какой
результат возобладает?

стр 135

12. Допустим, два политических кандидата имеют по
две cтратегии, как показано в таблице ниже, и
допустим, что кандидат 1 получает специфический
электорат от четырех возможных объединенных выборов
стратегий, как обозначено. Если цель обоих
кандидатов максимизировать вероятность своей победы
на выборах, какое будет, если это вообще возможно,
равновесие в этой игре? Как это равновесие
изменится, если оба кандидата пытаются
максимизировать свои ожидаемые электораты?

\begin{tabular}{ll}
\hline
\multicolumn{1}{|l}{p (0) = 5/8} & \multicolumn{1}{|l|}{p (8) = 1/2} \\
\multicolumn{1}{|l}{p (400) = 1/8} & \multicolumn{1}{|l|}{p (0) = 1/8} \\
\multicolumn{1}{|l}{p (800) = 2/8} & \multicolumn{1}{|l|}{p(-80) = 3/8} \\
\hline &  \\ \hline \multicolumn{1}{|l}{p (-400) =
1/2 \ \ } & \multicolumn{1}{|l|}{p (0) = 3/4}
\\
\multicolumn{1}{|l}{p (400) = 1/2} & \multicolumn{1}{|l|}{p (-75) = 1/8} \\
\multicolumn{1}{|l}{} & \multicolumn{1}{|l|}{p
(-125) = 1/8} \\ \hline
\end{tabular}

13. Что касается нашего обсуждения Рисунка 3.10 \
??? и посредничества, покажите, что если оба игрока
полностью информированы относительно совета
посредника, данного другому игроку, то роль
посредника, как координатора стратегии, исчезает.

14. Рассмотрите следующую ложную "теорему" и
найдите контрпримеры с нулевой и ненулевой суммой.
"Если в игре с двумя игроками нет никакого
равновесия в чистых стратегиях, в ней есть
единственное равновесие в смешанных стратегиях."

15. Игроки 1 и 2, действуя одновременно, должны оба
сначала выбрать, играть ли L или R, если они оба
играют L, то затем они играют игру А (см. ниже)

(все выигрыши даются в долларах); в другом случае,
бросается жребий, чтобы определить, играть ли в
игру В или С. (Результат жребия показывается
сразу.) До того, как они сделают выбор, игрок 1
(Мистер Строка) может предложить 2 любых суммы
денег за право сделать еще один выбор из R или L.
Определите, какую сумму, если вообще хоть какую-то,
игрок 1 должен предложить 2, также как цену игры
игрокам 1 и 2.

\begin{tabular}{|l|ll|l|ll|l|l|}
\cline{1-2}\cline{4-5}\cline{7-8} 0,4 & 0,2 &
\multicolumn{1}{|l|}{} & 1,3 & 2,1 &
\multicolumn{1}{|l|}{} & 1,4 & 6,1 \\
\cline{1-2}\cline{4-5}\cline{7-8} 4,2 & 1,1 &
\multicolumn{1}{|l|}{} & 4,3 & 3,2 &
\multicolumn{1}{|l|}{} & 5,1 & 0,4 \\
\cline{1-2}\cline{4-5}\cline{7-8}
\multicolumn{2}{l}{\ \ \ \ \ A} & \ \ \  &
\multicolumn{2}{l}{B} &  & \multicolumn{2}{l}{\ \ \
\ \ \ C}\end{tabular}

стр 136

16. Помните игру из начальной школы, которая
называлась "бросить на пальцах"? Один из игроков
выигрывает при "четном", а другой - при "нечетном".
На счет "три" каждый из них одновременно
выбрасывает в воздух один или два пальца. Если
общее количество пальцев четно, то "четный"
выигрывает, в то время как если сумма нечетна, то
выигрывает "нечетный". Предположите, что выигрыш --
плюс один для победителя и минус один для
проигравшего.

a. Покажите развернутую и стратегическую формы для
этой игры.

b. Покажите, что эта игра не имеет никакой чистой
стратегии.

с. Докажите, что в равновесии обязательно
"действовать случайно".

d. Предположите, что правила игры изменены так,
чтобы на счет три человек еще мог колебаться. Если
оба колеблются, каждый получает выигрыш 2, в то
время как, если колеблется только один, тот игрок
проигрывает 1, а противник получает 0. Каким будет
результат?

стр. 137

17. Рассмотрите две следующие игры:

\begin{tabular}{|l|l|l|l|l|}
\cline{1-2}\cline{4-5} 10,4 & 4,0 &  & 14,14 & 4,16
\\ \cline{1-2}\cline{4-5} 8,2 & 8,6 &  & 10,2 &
0,20 \\ \cline{1-2}\cline{4-5}
\end{tabular}

a. Если бросается жребий, чтобы определить, в какую
из игр будут играть, каково будет равновесие Нэша в
ситуации, где ни один из игроков не знает
результата бросания жребия, и игроки выбирают свои
cтратегии одновременно?

b. Создайте ситуацию (a) в развернутой форме.

с. Если в ситуации (a) один из игроков может
заплатить тому, кто бросает жребий (бросающий
жребий не участвует в игре), чтобы открыть
результат жребия обоим игрокам прежде, чем они
выберут свои cтратегии, какой игрок (строчный или
столбцовый) будет больше хотеть заплатить за эту
информацию?

d. Предположите, что вместо жребия и одновременных
шагов одному из игроков позволяется выбрать игру, в
которую будут играть, и что после раскрытия жребия
другому игроку, игроки выбирают свои cтратегии
последовательно, первым Мистер Столбец, затем
Мистер Строка. Найдите равновесие Нэша для
следующих четырех игр, которые эти условия
допускают:

i. Мистер Строка выбирает игру и ходит первым.

ii. Мистер Строка выбирает игру, и Мистер Столбец
ходит первым.

iii. Мистер Столбец выбирает игру, и Мистер Строка
ходит первым.

iv. Мистер Столбец выбирает игру и ходит первым.

18. В игре с двумя игроками следующие пары действий
ведут к следующим результатам:

( Игрок 1 Действие, Игрок 2 Действие) $\rightarrow
$(Полезность 1, 2 Полезность)

\begin{tabular}{lll}
(L,L) &  & (3,6) \\ (R,R) &  & (1,5) \\ (C,C) &  & (6,8) \\
(L,R) &  & (9,6) \\ (R,L) &  & (9,9) \\ (L,C) &  & (6,1) \\
(C,L) &  & (2,2) \\ (R,C) &  & (4,4) \\ (C,R) &  &
(4,3)\end{tabular}

a. Каков результат этой игры, если 1 ходит первым,
и 2 видит его ход, а затем делает свой ход?

b. Каков результат, если 2 ходит первым, и 1 видит
его ход, а затем делает свой ход?

с. Каков результат, если каждый игрок должен
сделать ход прежде, чем он узнает ход другого
игрока?

стр 190

1. Допустим, два кандидата в президенты, которые
максимизируют свою возможность победы на выборах,
должны решить, как распределить три дня между
шестью штатами. Тот, кто посвящает штату больше
всего времени, завоевывает поддержку этого штата, а
тот, кто получает больше всего голосов -- побеждает
на выборах.

Количество голосов у штатов следующее: 27, 27, 24,
18, 2 и 2. Учтите, что все ситуации равенства
голосов разрешаются жребием, и что транспорт не
позволяет поделить день между двумя штатами.

a. Имеет ли соответствующая игра выборов с двумя
кандидатами равновесие Нэша в чистых cтратегиях?

b. Изменится ли Ваш ответ, если три самых больших
штата имеют одинаковый избирательный вес?

с. Изменится ли Ваш ответ на часть (a), если дни
можно поделить между штатами?

d. Изменится ли Ваш ответ на часть (a), если
кандидаты имеют четыре дня для распределения?

e. Как Вы думаете, какую долю времени получили бы
самые маленькие штаты при различных
обстоятельствах, описанных выше? Почему?

f. Как изменятся Ваши ответы на (a-e), если
кандидаты максимизируют ожидаемые электораты ?

2. Предположим, что шесть избирателей: 1, 2, 3, 4,
5 и 6, озабоченые одной и той же проблемой, имеют
наиболее предпочитаемые варианты, упорядоченные
также как их номера-метки (то есть, 1 больше всего
предпочитает самую левую позицию, а 6 - самую
правую), и у которых есть однопиковые предпочтения
по этой проблеме. Также, предположите, что
используется взвешенное голосование, и что
избиратель $i$ имеет $i$ голосов. Если два
максимизирующих электората кандидата могут принять
любую из шести позиций по проблеме как свою
избирательную платформу и если все условия для
Теоремы Медианного Избирателя удовлетворены,
идеальную отметку какого избирателя примут они как
свою избирательную платформу?

3. В дополнение к выбору стратегии на
действительной прямой предположите, что кандидаты
могут также пробовать маскировать свои позиции,
представляя себя в виде лотереи - как распределения
вероятности на допустимом пространстве стратегий.
Предположите, что два параметра - среднее значение
и дисперсия - характеризуют стратегию кандидата
(когда эта стратегия соответствует функции
нормального распределения), и что полезность
избирателя $i$ для позиции $х $ будет равна
$u_{i}\left( x\right) =-(x_{i}-$ $x)^{2}$. Опишите
равновесие выбора с двумя кандидатами, если
применяются все другие условия Теоремы Медианного
Избирателя.

4. Примените Теорему Медианного Избирателя, чтобы
решить следующие вопросы, используя эти данные:

стр.191

\begin{tabular}{|l|l|lll}
\hline голосующий & идеальная точка & кандидат &
\multicolumn{1}{|l}{обещанное} &
\multicolumn{1}{|l|}{мать} \\ \hline
A & 1 & ??? & \multicolumn{1}{|l}{3} & \multicolumn{1}{|l|}{A} \\
B & 2 &  & \multicolumn{1}{|l}{5} & \multicolumn{1}{|l|}{B} \\
C & 3 &  & \multicolumn{1}{|l}{7} & \multicolumn{1}{|l|}{C} \\
D & 4 &  & \multicolumn{1}{|l}{9} & \multicolumn{1}{|l|}{D} \\
E & 5 &  & \multicolumn{1}{|l}{9} &
\multicolumn{1}{|l|}{E} \\ \cline{3-5}
F & 7 &  &  &  \\ G & 9 &  &  &  \\ H & 11 &  &  &  \\
J & 13 &  &  &  \\ \cline{1-2}
\end{tabular}

а. Допустим, Вам нравится побеждать и только один
из кандидатов, перечисленных выше, конкурирует с
Вами, кем из вышеупомянутых кандидатов Вы бы хотели
быть?

b. Предположим, пять избирателей - фактически
матери кандидатов, которые, если есть возможность,
голосуют за своих детей, независимо от того, как
это влияет на их полезности. Кроме этого
исключения, допустим, все условия Теоремы
Медианного Избирателя соблюдаются. Если учесть
вышеупомянутые материнские отношения, как изменится
Ваш ответ?

5. Рассмотрите избирательный округ из двадцати пяти
избирателей, который распределен по пяти районам
(A, B, C, D, E). В каждом районе живут по пять
избирателей, и каждый избиратель идентифицирован
его районом (например, каждый а живет в A) и его
идеальной отметкой на стратегии X = [1,9].
Предположите, что идеальные точки избирателей
следующие:

\begin{tabular}{lllllllll}
$x=1$ & $x=2$ & $x=3$ & $x=4$ & $x=5$ & $x=6$ & $x=7$ & $x=8$ & $x=9$ \\
\hline
$b$ & $b$ & $b$ & $a$ & $a$ & $a$ & $a$ & $a$ & $d$ \\
$e$ & $c$ & $c$ & $b$ & $b$ & $c$ & $c$ & $d$ & $d$ \\
$e$ & $c$ & $e$ & $d$ & $d$ & $e$ &  &  &  \\ &  &
&  & $e$ &  &  &  &
\end{tabular}

192

У всех избирателей есть однопиковые предпочтения.
Избирательный округ выбирает стратегию на X,
используя следующую процедуру: сначала, используя
правило большинства, избиратели внутри каждого
района выбирают законодателя из двух возможных
кандидатов, где эти кандидаты конкурируют, выбирая
позиции на X. Потом пять законодателей, выбранные в
пяти округах, вынуждены голосовать за стратегию,
которую они защищали при избрании и предлагать ее
законодательному органу, где они встречаются и
должны решить правилом большинства, какова будет
общественная политика.

a. Скажите, какая стратегия будет выбрана,
учитывая, что все законодатели являются хитрыми и
выбирают повестку дня из набора возможных двоичных
повесток, и объясните почему.

b. Как Ваш ответ на (a) изменится, если штаты А и D
и избиратели в них исключены?

c. Как Ваш ответ на (a) изменится, если штаты С и E
и избиратели в них исключены?

6. Рассмотрите избирательный округ из девяти
избирателей, в котором идеальная точка
избирателя$i$по проблеме равняется $i$. Допустим,
делается выбор из двух кандидатов, и все условия,
необходимые для Медианного Избирателя, соблюдаются,
за исключением того, что кандидат 1 должен выбрать
идеальную точку избирателя 3, как свою
установленную стратегию. Кандидат 2 не имеет
никаких ограничений.

a. Кто выиграет выборы?

b. Ниже - четыре возможных описания результата.
Решите, какие из следующих утверждений являются
истинными:

i. Идеальная точка Медианного Избирателя должна
быть результатом.

ii. Идеальная точка Медианного Избирателя не может
быть результатом, но идеальная точка другого
избирателя может.

iii. Идеальная точка Медианного Избирателя не может
быть результатом, и идеальная точка любого другого
избирателя также не может.

iv. Идеальная точка или Медианного Избирателя или
другого избирателя может быть результатом.

7. Предположим, что избирательный округ состоит из
пяти голосующих со следующими предпочтениями
относительно трех платформ (A, В, или С), которые
каждый из двух кандидатов может выдвинуть:

\begin{tabular}{llll}
голосующие 1 и 2 & A $\succ $ & B $\succ $ & C \\
голосующий 3 & A $\succ $ & C $\succ $ & B \\
голосующие 4 и 5 & C $\succ $ & B $\succ $ &
A\end{tabular}

стр. 193

Предположим, каждый избиратель, после того, как
каждый кандидат (одновременно) выбирает свою
платформу, голосует за предпочтенного им кандидата
с вероятностью $p$ и за его противника с
вероятностью $1-p,p>1/2$. Если избирателю
безразлично, то $p=1/2$. Каков исход (A, В, или С)
если оба кандидата максимизируют свои ожидаемые
электораты?

\bigskip \bigskip

8. Рассмотрите следующую игру из двадцати девяти
игроков, где два игрока (1 и 2) -- кандидаты, а
другие игроки, (3,4,..,29), являются избирателями.
Кандидаты конкурируют за руководство ведомством
(учреждением), выдвигая одну из четырех возможных
cтратегий учреждения (A,B,C,D) принимать на работу
избирателей, которые голосуют за них (еще до
принятия на работу: можно считать, что избиратели
уже работают в ведомстве, а A,B,C,D \ - стратегии
увольнения). Исход выборов будет зависеть от того,
кто из двух кандидатов завербует больше
избирателей. В первой стадии игры два кандидата
одновременно выбирают одну из четырех cтратегий
избирательной кампании. Некоторые комбинации
выдвинутых программ стратегии учреждения (ВВ, CC,
DB) кандидаты могут обсудить, и выбирают,
участвовать ли в дебатах. Кандидаты выбирают
cтратегии дебатов одновременно, но после этого они
узнают и стратегию кампании противника. Ниже -
таблица, которая показывает результаты различных
комбинаций cтратегий кампании. Значения ячейки
представляют число избирателей, завербованных
кандидатом 1 (Мистер Строка).\ Тогда 27 минус
значение ячейки равняется числу избирателей,
завербованных кандидатом 2 (Мистер Столбец).

\begin{tabular}{ll|ll|l|ll}
Строка / Столбец: & A &
\multicolumn{2}{l}{B, \ \ В2} &
\multicolumn{2}{l}{C, \ \ С2} & D \\ \cline{2-7} A
& \multicolumn{1}{|l|}{15} &
\multicolumn{2}{|l|}{3} & \multicolumn{2}{|l}{ \ \
21} & \multicolumn{1}{|l|}{7} \\ \cline{2-7} B &
\multicolumn{1}{|l|}{9} & 12 &
\multicolumn{1}{|l|}{27} & \multicolumn{2}{|l}{\ \
\ 5} & \multicolumn{1}{|l|}{13} \\ \cline{3-4} (В2=
с дебатами) & \multicolumn{1}{|l|}{} & 0 &
\multicolumn{1}{|l|}{15} & \multicolumn{2}{|l}{} &
\multicolumn{1}{|l|}{} \\ \cline{2-7}\cline{6-7}
C & \multicolumn{1}{|l|}{17} & 19 &  & 24 & 17 & \multicolumn{1}{|l|}{23} \\
\cline{5-6} (С2= с дебатами)
& \multicolumn{1}{|l|}{} &  &  & 14 & 18 & \multicolumn{1}{|l|}{} \\
\cline{2-7}\cline{6-7} D & \multicolumn{1}{|l|}{11}
& 12 & \multicolumn{1}{|l|}{13} &
\multicolumn{2}{|l}{27} & \multicolumn{1}{|l|}{11}
\\ \cline{3-4} (D2= с дебатами) &
\multicolumn{1}{|l|}{} & 11 &
\multicolumn{1}{|l|}{14} & \multicolumn{2}{|l}{ } &
\multicolumn{1}{|l|}{} \\ \cline{2-7}
\end{tabular}

стр 194

После кампании, победитель выборов определяется по
правилу большинства.

Таким образом, кандидат, который принимает на
работу больше людей, выигрывает выборы.

a. Какой кандидат выиграет выборы?

b. Если законы кампании изменены так, что стратегию
D нельзя выбрать, какой кандидат выиграет выборы?

с. Предположите, что Вы наняты кандидатом, который
ожидает частично потерять (a), чтобы предложить ряд
районов и избирательного правила, которое
гарантирует различный избирательный результат
(после того, как новички завербованы). Вы не
ограничены в числе районов, которые Вы можете
использовать, или в том, как Вы можете распределять
сторонников кандидатов 1 и 2 внутри и между
районами; однако в каждом районе должно быть
одинаковое число избирателей.

9. Законодательный орган из девяти членов,
использующий правило большинства, стоит перед
бюджетом, который позволяет ему пройти две из трех
предложенных программ (A,B,C). Законодательный
орган имеет следующие предпочтения (оцениваемые от
более до менее привилегированных):

\begin{tabular}{lllllllll}
\multicolumn{9}{l}{\ \ \ \ \ \ \ \ \ \ \ \ \ \ Законодатель} \\
1 & 2 & 3 & 4 & 5 & 6 & 7 & 8 & 9 \\ B & A & B & A & C & C & B & A & C \\
C & B & A & C & A & B & A & C & B \\ A & C & C & B
& B & A & C & B & A\end{tabular}

Вы возглавляете законодательный орган и Вы - 1
законодатель. Процедура голосования следующая:

Сначала проголосуйте, сохранить ли вариант В или
наложить на него вето. Если на В накладывают вето,
голосование кончается, и варианты А и С
выполняются. Если В сохраняется, то проголосуйте,
сохранить ли вариант С или наложить на него вето.
Если на С накладывают вето, голосование кончается,
и варианты А и В выполняются. Если С сохраняется,
то проголосуйте, сохранить ли вариант А или
наложить на него вето. Если накладывается вето -
голосование кончается, и варианты В и С выполнены.
Если сохраняется, один член законодательного органа
должен выбрать, на какой из вариантов нужно
наложить вето.

a. Должны ли Вы, как председатель, сами наложить
это вето, или предоставить это 4-му законодателю?

b. Учитывая, что законодатель 4 решает, накладывать
ли вето, создайте список по вышепоказанному типу, в
котором В и С тем не менее проходятся.

стр 195

10. В избирательном округе из семи людей, все
результаты выборов зависят от позиций кандидата на
одной размерности стратегии, и идеальной отметке
избирателя $i$ на этой размерности, соответствует
$x_{i} $, который равняется $i$. Все функции
полезности избирателя имеют форму $u_{i}=\left\vert
W-x_{i}\right\vert ^{2}$, где $W$ - позиция
стратегии побеждающего кандидата, и $x_{i}$ --
идеальная отметка избирателя $i$. В этом
избирательном округе имеются две политических
стороны. Избиратели 1, 2, и 7 находятся в партии X.
Избиратели 4, 5, и 6 находятся в партии Y.
Избиратель 3 не находится ни в одной из партий.
Имеются четыре возможных канндидата -- Х1, X2, Y1,
и Y2. Партия X использует правило большинства чтобы
назначить Х1 или X2. Партия Y использует правило
большинства чтобы назначить Y1 или Y2. Затем все
избиратели выбирают среди двух назначенных
кандидатов, чтобы определить $W$, позицию стратегии
побеждающего кандидата. Учтите, что все
предположения, необходимые для Теорему Медианного
Избирателя проводятся в случае, если они находятся
в противоречии с утверждением вопроса. Используйте
концепцию равновесия Нэша, чтобы найти возможные
расположения $W$ в следующих двух ситуациях:

a. Избиратели неоперативны при назначении
кандидата.

b. Избиратели оперативны при назначении кандидата.

11. Допустим, четыре человека имеют следующие
оценки (выигрыши) по трем вариантам А,В и С:

\begin{tabular}{|l|l|l|l|}
\hline человек & A & B & C \\ \hline
1 & 30 & 0 & 50 \\ 2 & 45 & 65 & 0 \\ 3 & 10 & 20 & 45 \\
4 & 50 & 35 & 0 \\ \hline
\end{tabular}

Допустим, что каждый человек должен сообщить оценку
для каждого варианта и что выбранный вариант
является вариантом с самой высокой суммированной
оценкой. Допустим также, что налоги собраны как
описано в Разделе 4.5. ???

a. Сколько вмененного налога будет оплачено каждым
человеком?

b. Допустим, люди 2 и 3 могут нанимать агента,
который будет координировать их ответы (сообщенные
оценки), включая ложные. Стоит ли им нанимать
такого человека, если взнос будет не слишком
большим?

\bigskip

197

(11?) Вы управляете "Объединенным Смоуком" и должны
решить, согласиться или нет встретиться с
президентом "Акмэ Сладж", чтобы Вы оба могли
устанавить цены на Ваши одинаковые изделия. В этом
случае, обе Ваши корпорации получат по \$220
миллионов. Однако Вы оба понимаете, что эта
ситуация - дилемма Заключенных: по рыночным ценам
каждый из Вас получил бы по \$90 миллионов, в то
время как если только один отказывается от
соглашения, его корпорация получает \$300
миллионов, а другая корпорация ничего не получает.
Будучи конкурирующими предпринимателями с MBA, ни
один из Вас не доверяет другому, чтобы заключить
хоть какое-нибудь соглашение. Дополнительная
опасность - то, что федеральные антимонопольные
исследователи (с вероятностью .4) могут обнаружить
Ваше соглашение, опровергнуть цены, установленные
по соглашению, и наложить штраф на \$50 миллионов
на каждого. Конгрессмен И. M. Тупой, однако,
предлагает предложить законодательство, которое
сделает картель законным и осуществимый в суде, и
это обеспечит картелю защиту; но он требует
некоторую помощь в следующих выборах, скажем, по
\$50 миллионов с каждой фирмы. Проблема в том, что
он хочет деньги вперед, до выборов законодательного
органа, и он может обещать только возможность 50 на
50, что предложенное законодательство пройдет.
Допустим, что Вы примете самое лучшее решение
сложившейся ситуации, какова будет ожидаемая
прибыль Вашей фирмы?

17. Три студента 1, 2 и 3, записались на курс
Политической Теории, и инструктор объявил заранее,
что он наградит одного A, одного С, и одного F
(никаких недифференцированных зачетов в этой
четверти). Домашняяя работа не имеет значения на
заключительных экзаменах, и студенты получили по
55, 65, и 70 баллов соответственно на своем
полугодовом экзамене. Каждый студент связывает
вознаграждение 2, 1 и -1, соответственно с
получением A, С или F, и каждый знает, что, если
они учатся, прикладывая максимум усилий, они
получат 80 на заключительном экзамене, а умеренные
или малые усилия дадут результат 40. Предположите,
что, celeris paribus, каждый студент предпочитает
прикладывать настолько мало усилий, насколько
возможно, но считает, что стоит немного поучить,
чтобы получить оценку повыше. (???) Заключительная
оценка студента включает его относительное
положение, которое определено суммой полугодовых и
годовых оценок на экзаменах.

стр 198

a. Допустим, что все три студента должны решить,
учиться ли им, обладая информацией о действиях,
принимаемых любым другим студентом, или не обладая.
Изобразите ситуацию в стратегической форме с
оценками, обозначающими результаты в каждой ячейке.

b. Является ли заключительный результат решающим?

с. Допустим, студенты 1 и 2 могут наблюдать
действие 3-го заранее. Изменит ли это их решение и
заключительный результат?

18. Допустим, идеальные точки избирателя
распределены однородно на интервале [0,1] и что,
если это особо не отмечено, все условия Теорему
Медианного Избирателя исполняются. Допустим, три
кандидата 1, 2 и 3 должны выбрать свои позиции в
[0,1] последовательно, сначала 1, затем 2, затем 3.

a. Докажите, (без чрезмерного формализма, даже если
это требует срезки нескольких углов) что равновесие
соответствует выделенному в последнем примере
отделения 4.3, где кандидаты 1 и 2 выбирают свои
позиции одновременно. \ ???

b. Почему Вы думаете, что эти ответы одинаковы?

19. Два фермера должны совместно использовать
систему орошения, которую они используют, чередуя
свой доступ к ней день за днем. Фермер, чья сейчас
очередь извлекать воду, должен в определенный день
решить, взять ли ему выделенную часть(с прибылью 0
себе и стоимостью 0 другому фермеру) или взять
больше выделенной части (с прибылью В
рассматриваемому фермеру и стоимостью С другому
фермеру). Однако фермер, который должен сидеть
праздно в течение дня, может выбрать посмотреть на
действия соперника за сумму К. Если чрезмерное
извлечение обнаружено, фермера уполномачивают
оштрафовать обидчика на сумму F, которая может
сохраняться как компенсация за любой экономический
ущерб.

a. Допустим, что все значения параметров выше ноля,
и рассматривая только один день, какие значения
параметров будут являться чистым равновесием
стратегии, в которой фермер пристально изучается с
уверенностью?

стр 199

b. Допустим, что не имеется никакого чистого
равновесия стратегии, что является смешанным
равновесием стратегии?

c. Допустим, что один из фермеров должен быть
выбран наугад, чтобы использовать систему орошения,
и что параметры установлены так, что (берите больше
чем выделенную часть, осмотрите \ ??? ) есть чистое
равновесие стратегии. Можем ли мы поднять цену F
так, чтобы фермеры предпочли, чтобы не было
никакого чистого равновесия стратегии, а чтоб они
жили со старыми значениями параметров?

20. Два лоббиста 1 и 2, каждый ищет
законодательство, которое было бы диаметрально
противопоставленным законодательству,
разыскиваемому другим. Так как мы не можем
рассматривать это законодательство как действующее
"в интересах общества," они должны решить, когда
жертвовать деньги кампании законодателя как
средства на войну. (Допустим, законодатели
"обойдутся дешево," так что размер вклада может
игнорироваться при вычислении вознаграждения
лоббиста.) Законодатель не видит никакой угрозы
переизбрания и предполагает преобразовать эти
средства на войну в персональный пенсионный фонд.
Но в мире научной фантастики этой проблемы,
предположите, что закон не одобряет никаких
подобных действий и как следствие, ни один лоббист
не посмеет сделать второе предложение, если получит
отказ на первое. Точный характер поведения
законодателя, однако, неизвестен, так что оба
лоббиста отражают следующую модель: Если лоббисты
нарушают во времена $t_{1}$и
$t_{2}(0<t_{1},t_{2}<1),$ соответственно, и если
$t_{i}<t_{j}$, то вероятность, что вклад приведет
законодателя к поддержке $i$ -- $t1$. Если
законодатель не нарушает на вклад $i$ в это время,
он или поддерживает законодательство, разыскиваемое
$j$ с вероятностью $t_{j}$, или он не поддерживает
никого с вероятностью$1-t_{j}$. Тогда существует
возможность, что законодатель никогда не нарушит
??? и что ни один лоббист не получит, чего хочет.
Допустим, что вознаграждение лоббиста ни из одного
законодательства эквивалентно четной возможности,
что один или другой лоббист получает свой путь.

a. Если лоббист $i$ ($i$ = 1 или 2) делает первое
перемещение, выбирая $t_{i}<t_{j}$, лоббист $j$
узнает этот факт в то время и также узнает ответ
законодателя на предложение $i$. Допустим, что
стратегия - это время, за которое делается
предложение, имеет ли тогда эта ситуация чистое
равновесие стратегии, и если так, то что это?

b. Как изменится ситуация, если предложение и отказ
от него (но не его принятие) не может пронаблюдать
конкурирующий лоббист?

стр. 252

3. В случаях неудачных хирургических операций(когда
пациент умирает) существует возможность 50 на 50,
что Доктора Яна Компетентного не сочтут виновным.
Зная об этом, родственники его последней жертвы
просили о компенсации - \$l,000,000. Если
Компетентный (который аннулировал страховой полис,
но который, как акционер в локальной ассоциации
сбережений и ссуд, довольно богат, отказывается
платить, родственники могут обратиться в суд (или
забыть об этом). Когда-то в суде предположили, что
то, что произошло, было справедливо. (Это
упражнение не касается компетентности адвокатов).
Если Компетентный невинен, (и только он знает это
наверняка), он не теряет ничего, а родственники
теряют \$l,000,000 (судебные издержки остаются тем,
чем они являются). С другой стороны, если он
проигрывает, то он теряет \$3000,000, а
родственники жертвы извлекают выгоду \$2,000,000
(адвокаты снова берут свою долю).

a. Изобразите развернутые и стратегические формы
этой ситуации.

b. Определите равновесие игры.

с. Интерпретируйте это равновесие.

4. Страна 1 тайно предложила стране 2 план
разоружения, которое касается новой системы оружия,
разработанной 2. Однако эта система эффективна
только на семьдесят процентов, даже если 2 твердо
знает возможности системы. Страна 2 может
игнорировать предложение 1 и продолжить состязание
в оружии; и наоборот, она может публично сделать
такое же предложение, в то время как тогда страна 1
должна "клюнуть или сорваться с крючка", формально
принимая предложение 2. Скажите, что Вы
рассматриваете как приемлемые вознаграждения
результатам, изобразите развернутые и
стратегические формы этой ситуации, и найдите любое
равновесие с чистой стратегией. Интерпретируйте это
равновесие, если оно существует.

5. Создайте стратегическую форму нашего  анализа
повесток дня с тремя допустимыми типами
предпочтения в Разделе 5.5???, учтите, что комитет
состоит из трех избирателей. Сравните Ваш ответ с
предлагаемым в тексте, и покажите, что он не
противоречит ему.

6. Рассмотрите следующий последовательный список
(повестку дня): Варианты А и В сначала соединены.
Если А выигрывает, это - результат; но если
выигрывает В, то В ставится против С, и победитель
этого голосования - заключительный результат.
Допустим, что только эти три типа предпочтения
возможны (оцененные от первого предпочтения к
последнему):

\begin{tabular}{llll}
$t_{1}$ & A & B & C \\ $t_{2}$ & B & A & C \\
$t_{3}$ & C & B & A\end{tabular}

стр. 253

a. Допустим, что человек имеет предпочтения типа ti
с вероятностью pi, которые являются общепринятой
истиной, и что человек знает свои предпочтения,
покажите, что выбор победителя Кондорсе
соответствует равновесию.

b. Является ли это равновесие уникальным?

c. Изменятся ли Ваши ответы на части (a) и (b),
если мы добавим четвертый тип предпочтения, "В
предпочитается С, предпочитается А"?

7. Профессор просит, чтобы ТА дал следующую лекцию,
и подготовка лекции выполняется только за день
перед выступлением. Зная, что ТА читает лекции,
посещаемость будет 100 \%, но класс не знает типа
TA, который может быть двух сортов. Если ТА -- тип
Ответственный, он извлечет значительную пользу из
эффективной подготовки своей лекции в его
Ведомстве. Но если ТА - тип Серфингист, он
предпочел бы готовить свою лекцию на пляже, где он
работает не так эффективно, но со значительно
большим удовольствием. Существуют три состояния
мира, каждое известно ТА: (1) ТА - тип
Ответственный, (2) ТА сначала выбирает свое
расположение, затем бросается жребий, чтобы
определить его тип; и (3) ТА - тип Серфингист.
Класс не наблюдает состояние мира, но у него есть
более веский вероятности убеждений, что истинное
состояние мира может быть:

состояния мира "1" происходит с вероятностью .5.

состояния мира "2" происходит с вероятностью .2.

состояния мира "3" происходит с вероятностью .3.

Использование priors и действия TA (класс
наблюдает, где ТА готовит свою лекцию), класс
выбирает "Бездействовать" или "быть Активным" на
уроке. По описанию ниже, класс сопоставляет две
альтернативных стратегических формы, в зависимости
от типа TA: Если ТА - тип О, то

вставка таблицы

В то время как, если ТА - тип С, то

\begin{tabular}{lll}
& Офис & Пляж \\ \cline{2-3} Бодрствует &
\multicolumn{1}{|l}{10,10} &
\multicolumn{1}{|l|}{-10,0} \\ \cline{2-3} Спит &
\multicolumn{1}{|l}{-1,2} &
\multicolumn{1}{|l|}{8,3} \\ \cline{2-3}
\end{tabular}

стр 254

\begin{tabular}{lll}
& Офис & Пляж \\ \cline{2-3} Бодрствует &
\multicolumn{1}{|l}{10, -2} &
\multicolumn{1}{|l|}{-10,1} \\ \cline{2-3} Спит &
\multicolumn{1}{|l}{-1, -10} &
\multicolumn{1}{|l|}{8,5} \\ \cline{2-3}
\end{tabular}

a. Изобразите развернутую форму этой ситуации.

b. Определите чистые cтратегии, доступные каждому
игроку.

c. Определите, какие из следующих четырех
комбинаций стратегий являются равновесием
Байеса-Нэша:

\begin{tabular}{lll}
{?} пары стратегий & Стратегия для ТА & Стратегия
студентов \\ 1 & Если состояние a или b, выберите
"Офис", но если состояние c\ - Пляж & Если "Офис"
\\ 2 & ??? &
\\ 3 &  &  \\ 4 &  &
\end{tabular}

8. Рассмотрите следующую игру с двумя людьми, игру
с нулевой суммой. Случай сначала выбирает тип 1-го
игрока. Игрок 1 знает свой тип и выбирает "Да" или
"НЕТ". После того, как игрок 1 выбрал, игрок 2
выбирает "Да" или "НЕТ". Игрок 2, чей тип -- общее
знание, не наблюдает типа первого игрока, но
наблюдает его выбор "Да" или "НЕТ". Выигрыши игрока
определяются согласно следующей таблице:

стр 255

\begin{tabular}{|l|l|l|l|}
\hline 1-й тип & 1-й выбор & 2-й выбор & 3-й выбор
\\ \hline
L & Y & Y & 1 \\ L & Y & N & 2 \\ L & N & Y & 4 \\ L & N & N & 3 \\
C & Y & Y & 3 \\ C & Y & N & 7 \\ C & N & Y & 9 \\ C & N & N & 7 \\
R & Y & Y & 3 \\ R & Y & N & 4 \\ R & N & Y & 2 \\
R & N & N & 1 \\ \hline
\end{tabular}

a. Изобразите развернутую форму этой игры.

b. Определите чистые cтратегии, доступные каждому
человеку.

с. Изобразите стратегическую форму ситуации.

b. Дайте пример равновесия Байеса. Используйте
правило Байеса, чтоб подтвердить свой пример.

с. Найдите решение для игры, если второй знает тип
первого, когда выбирает.

9. Воспроизведите наш анализ Игры Сороконожки,
учитывая, что вероятность или нелогичность каждого
игрока скорее .3 чем .03.

10. В грядущих выборах по реформе тарифа
страхования, общеизвестно, что Ваш голос -
решающий. Вы не уверены относительно тождества
спонсора реформы, но уверены (и правильным), что
есть возможность7-из-1O, что эта реформа "за"
страхование (СЛУЖБА ИММИГРАЦИИ И НАТУРАЛИЗАЦИИ) и
как следствие, это поднимет Ваши тарифы страхования
(сделает Вас беднее). Имеется возможность 3-из-1O,
что реформа является (консоль) "за" потребителя,
эта реформа сохранит Ваши тарифы страхования на их
изначальном уровне. Активный участник кампании
знает, какого типа эта реформа - страхования или
потребителя, (в любом случае, ему платят только
если билль проходит) и он должен решить, идти ли к
Вашей фирме, чтобы сообщить, чтобы Вы голосовали за
билль, или нет. (Активный участник кампании
выбирает "Фирма" или "НЕТ.") В день выборов Вы
должны решить, голосовать ли "За" или "Против"
реформы. Выигрыши определены следующим образом:
активному участнику кампании стоит \$5 придти к
Вашей фирме: активному участнику кампании платят
\$15, если реформа проходит и \$0, если реформа
терпит неудачу. Если тип страхования проходит или
тип потребителя проигрывает, Ваши тарифы повышаются
- Вы теряете \$10. Если реформа типа потребителя
проигрывает или реформа типа страхования проходит,
Ваши тарифы остаются теми же самыми - Вы получаете
\$0.

стр 256

a. Выведите развернутую форму игры.

b. Определите чистые cтратегии, доступные каждому
игроку.

c. Изобразите стратегическую форму ситуации.

c. Какие из следующих пяти пар стратегий являются
равновесием Байеса?

ИЗОБРАЖЕНИЕ ???

стр 257

1. Политическая философия - авторитарная или
демократическая? Президент Балски из Нижней
Слобовии никому не известен, включая своего
главного конкурента, Многопьющего. Люди учитывают,
что Балски ценит демократические принципы с
вероятностью $q$, $.5<q<1$. В поисках поддержки
(покровительства) членов НОКР (Нации, объединенные
для Капиталистического Развития), Балски обещал
позволить национальные выборы для президента -
выборы, которые могут идти любым путем с равной
вероятностью, если Многопьющий выставляется против
него. Теперь Балски должен решить, использовать ли
недавнее этническое волнение как оправдание
нарушению своего слова. Если он изменяет своему
слову, он сохраняет свою позицию, но его жена может
не ожидать приглашения делать покупки на Катании
Родео на следующей встрече на высшем уровне, в то
время как Многопьющий получит международное
сочувствие и сможет появиться на Ночной линии. Если
Балски сдержит слово, Многопьющий должен решить,
оспаривать ли его. Неоспариваемый Балски побеждает
тонким(слабым) полем в неоспариваемых выборах. Если
Многопьющий оспаривает и проигрывает, он выглядит
как кто-то, кто не может нанести поражение,
являющееся обязательным, кто регулировал его страну
при безработице 40\% и -20\% ВВП тариф роста.
Оценка этого результата зависит также от
блокирования Балски демократией, так как проигрыш
выборов исторически считается "очень плохим делом"
в Нижней Слобовии. Многопьющий твердо
придерживается демократических принципов, но такой
проигрыш для людей подобных Балски - сокрушительный
психологический удар, который ведет к его
полуотставке как вспомогательного руководителя
Крапски Устройства подачи и Экрана (растения)
Двери. Тогда у нас есть следующие результаты:

01: Балски отменяет 1-е выборы.

02: Балски позволяет выборы, но Многопьющий
отказывается конкурировать.

03: Балски позволяет выборы, а Многопьющий
выступает против него, но проигрывает.

04: Балски позволяет выборы; Многопьющий выступает
против него, но побеждает.

В терминах всемирной передовой твердой валюты --
Долларах Диснея - допустим оценки, основываясь на
философии Балски, - следующим образом (с
вознаграждениями Балски и Многопьющего
соответственно и учтите, как всегда было в таких
случаях в Нижней Слобовии, что Доллары Диснея и
полезность образовывают дугу эквивалентно):

\begin{tabular}{lll}
& Авторитаритарный & Демократ \\ \cline{2-3}
01 & \multicolumn{1}{|l}{(-1,1)} & \multicolumn{1}{|l|}{(-1,1)} \\
\cline{2-3} 02 & \multicolumn{1}{|l}{(0,0)} &
\multicolumn{1}{|l|}{(0,0)} \\ \cline{2-3} 03 &
\multicolumn{1}{|l}{(2,-6)} &
\multicolumn{1}{|l|}{(2,0)} \\ \cline{2-3}
04 & \multicolumn{1}{|l}{(-2,4)} & \multicolumn{1}{|l|}{(-8,4)} \\
\cline{2-3}
\end{tabular}

a. Изобразите эту ситуацию в развернутой и
стратегической формах.

b. Опишите cтратегии равновесия игроков как функцию
q.

стр 316

5. Рассмотрите следующую систему правила
большинства среди двадцати семи избирателей:
имеются три области, каждая разделена на три
района. От каждого района по три избирателя.
Представители (компьютеры, которые не учитываются в
подсчете голосов) на региональном и районном
уровнях. Представители на районном уровне получают
(в любой паре вариантов) предпочтения большинства
избирателей в своих районах, а региональные
представители получают предпочтения большинства
представителей в своих областях. Результат
стратегии выбирается голосами большинства
региональных представителей.

a. Каково минимальное число избирателей, которые
могут формировать решающую коалицию (т.е. Будут ли
их привилегированные позиции стратегии
результатом)? Приведите пример, чтобы
проиллюстрировать Ваш ответ.

b. Как Ваш ответ на (a) изменится, если
представителей уровня районов нет?

с. Допустим, что все предпочтения одиночно достигли
максимума над одиночной выдачей(проблемой), зависит
ли заключительный результат от существования
представителей уровня района? Почему?

6. Рассмотрите симметрическую игру из шести людей,
где значения коалиций С, v (C), даны

v (C)= ???

Где /C/ обозначает число членов С. Учитывая, что
то, что коалиция заслуживает, бесконечно делится
среди ее членов, имеет ли эта игра ядро?

7. В экономике, общее представление полезности для
двух предметов потребления - функция вставка
формулы , где и вставка формулы и и положительные
константы. Допустим, что двое людей заключают
сделку по распределению десяти единиц каждого из
товаров, где оба товара бесконечно делимы. Пусть \_
= = 2 для обоих людей, и допустим, что начальные
вклады обоих товаров l человека - 2 и 7,
соответственно, так, чтобы начальные вклады обоих
товаров 2 человека были 8 и 3, соответственно.
Каково ядро в этой игре с двумя людьми?

стр 317

8. Используя пространственные позиций стратегии в
проблеме 4, определите, имеет ли коалиция \{1,3,4\}
устойчивое предложение коалиции в заключающей
сделке. (Подсказка: Допустим, что игроки 1, 4 и 3
экспериментально соглашаются на предложение,
которое является Парето оптимальным для них и
приблизительно на полпути между 1 и 4 идеальным.
Допустим, объект 1 выступает против 3 с
предложением, которое является Парето оптимальным
для \{l,2,5\}, проверьте, может ли 3 закончик с
любой коалицией, которая исключает 1 и с
предложением, которое дает 3 столько же, сколько 3
получает от первоначального предложения с
\{l,3,4\}.

9. Допустим, три здания, каждое с большим
находящимся перед передней стороной окном, стоят в
треугольнике так, что человек, сидящий в любой
фирме, может видеть одну пятую собственного сада,
весь сад фирмы слева, и ни одного сада,
принадлежащего фирме справа. Допустим, что каждый
местный владелец обеспечен одним мешком удобрения и
обязан в соответствии c законом использовать его
(пригородные ценности остаются тем, что они есть).
Мешок удобрения, распределенный по одному саду,
улучшает этот сад так, что человек, видящий сад
целиком, наслаждается пользой пяти единиц
полезности, человек, видящий сад только на одну
пятую наслаждается одной единицей полезности, а
человек, не видящий сада не получает полезности
вообще. Допустим, что местные владельцы могут
объединиться, чтобы распределить свои удобрения, и
что люди не могут удобрять сад другого без
разрешения. Допустим также, что, если сад удобрить
двумя мешками, он будет в два раза красивее.

a. Имеет ли игра ядро?

b. Как изменится Ваш ответ, если предположить, что
коалиция двух людей может согласиться на действие
до действий игрока? (Подсказка: Допустим, что
коалиция позволяет исключенному игроку распределить
удобрение наилучшим образом для действий коалиции.)

10. Для примера в Рисунке 6.8 определите, влияет ли
порядок голосования на устойчивый результат. То
есть рассмотрите возможность, которую комитет
утверждает сначала на 1 выдаче(проблеме), затем на
2 выдаче(проблеме).

11. Допустим, существует пробел стратегии с двумя
размерными, учтите, что никакие два члена комитета
не используют одну и ту же идеальную отметку
совместно, и что все контуры безразличия - круги.
Используя область определения ядра в терминах
Парето оптимальных полисов для выигрывающий
коалиций, какие общие значения можете Вы вывести
относительно распределения идеальных пунктов(точек)
по правилу большинства в пространственных играх с
ядрами?

стр. 318

12. Набросайте доказательство утверждения, что ядро
большинства игр по голосованию, будучи связанным с
любой выигрывающей коалицией, является
конкурирующим решением. Также, набросайте
доказательство утверждения, что основная-простая
V-последовательность в сотрудничестве с
соответствующими минимальными выигрывающими
коалициями, тоже является конкурирующим решением.

13. Допустим, трое людей 1, 2 и 3 должны
использовать правило большинства, чтобы выбрать
правило, чтоб поделить \$1,000 между собой, и
учтите, что только два правила доступны: (1) сделка
заключается лицом к лицу, и люди 1 и 2 имеют по два
голоса, а 3 человек имеет один голос, и (2)
процедура, посредством которой 1 делает
предложение, по которому он получает \$500, а 2 и 3
делят оставшиеся \$500 на части по \$250 (т.e. 2
получает \$500 или \$250 или \$0). Если предложение
принимается 2 или 3, оно выполняется. Если оно
отклоняется, 3 может сделать подобное предложение
(т.e. \$500 для 2 человека, а люди 1 и 3 получают
оставшиеся части по \$250). Если оно отклоняется,
выполняется деление(333, 333, 333). Каким будет
заключительный результат?

14. Опишите V-последовательность и конкурирующее
решение для следующей игры, где торгуют голосами
(исключая лотереи, разрешая поражению каждого
закона стоить ноль каждому законодателю, и
учитывая, что вознаграждения по законам отделимы):

\begin{tabular}{|l|l|l|l|l|l|l|}
\hline Законодатель & A & B & C & D & E & F \\
\hline
1 & 3 & 3 & 2 & -4 & -4 & 2 \\ 2 & 2 & -4 & -4 & 2 & 3 & 3 \\
3 & -4 & 2 & 3 & 3 & 2 & -4 \\ \hline
\end{tabular}

15. Рассмотрите предпочтения следующего комитета из
четырех человек по трем законам A, В и С:

\begin{tabular}{|l|l|l|l|l|}
\hline L$_{1}$ & L$_{3}$ & L$_{5}$ & L$_{7}$ &
U$_{1}$ \\ \hline A & B & C & A & 2 \\ B & C & A &
B & 1 \\ C & A & B & C & 0 \\ \hline
\end{tabular}

стр 319

Допустим, что комитет может пройти один и только
один из трех законов. закон, чтобы пройти, должен
получить по крайней мере две трети голосов
Голосование - тяжелое так, что голос L1 считается
один раз, L7 - считается семь раз и так далее.
Стратегические законодатели голосуют одновременно,
и голосуют только за один закон, и могут общаться и
формировать совместные(кооперативные) соглашения.
Законодатель получает вознаграждение 2 если
наиболее для него предпочтительная альтернатива
проходит, 1, если вторая наиболее для него
предпочтительная альтернатива проходит, и 0 в
остальных случаях. Полезность - передаваемая и
бесконечно делимая. Все совместные(кооперативные)
соглашения -- бесплатны и осуществимы.

a. Какой из биллей пройдет, если вообще пройдет
хоть один?

b. Если голос L1 считается три раза вместо одного,
как изменится Ваш ответ на часть (a)?

16. Бикамерализм (законодательный орган с двумя
палатами) принят, чтобы стимулировать стабильность.
Рассмотрите пространственную ситуацию с тремя
законодателями в каждой из двух палат, и учтите,
что все предпочтения даны простым расстоянием
Евклида. Контуры Безразличия - круги.

a. Используя правило большинства, приведите пример
с двумя выдачами, в которых не имеется никакого
ядра внутри каждой палаты, но имеется полное ядро,
если любое движение, чтобы опрокинуть статус кво,
наносит поражение этому в каждой палате.

b. Могут ли Парето неэффективные результаты быть
поддержанными как равновесие при таком
расположении?

с. Может ли быть какое-либо обстоятельство, при
котором статус-кво является ядром при
Бикамерализме, но не является ядром, если две
палаты объединены?

17. В следующей игре с двумя участниками в
стратегической форме игра заканчивается с
назначенными выигрышами, если игроки выбирают что
угодно, но не ($a_{2},b_{1}$). Однако, если они
выбирают ($a_{2}$, $b_{1}$), они должны играть еще
раз. Определите значение продолжения для ($a_{2}$,
$b_{1}$), которое позволит Вам решить игру.

\begin{tabular}{lll}
& $b_{1}$ & $b_{2}$ \\ \cline{2-3}
$a_{1}$ & \multicolumn{1}{|l}{продолжение} & \multicolumn{1}{|l|}{2,5} \\
\cline{2-3} $a_{2}$ & \multicolumn{1}{|l}{5,5} &
\multicolumn{1}{|l|}{0,6} \\ \cline{2-3}
\end{tabular}

%\section{Э.Мулен}
%
%{\Large Задачи из книги
%Э.Мулен ``Теория игр (с примерами из мат. экономики)'' }
%\input ma80litr.ltx

%\protect\addcontentsline{toc}{section}{ {Литература} }

\begin{thebibliography}{99}

\bibitem{Kreps} David M. Kreps. 1990.  A Course in Microeconomic Theory.- Princeton University Press,   
Princeton.

\bibitem{Ordesh} Peter C. Ordeshook. 1992. A Political Theory Primer.- Routledge, N.-Y., London.

\bibitem{Myerson} R.B.Myerson. 1991. Game Theory (Analysis of Conflict).- Harvard U.P., Camridge, 
London.

\bibitem{Fudenberg}  Fudenberg, Drew \& Jean Tirole. 1991. Game Theory.- MIT Press.

\bibitem{Rasmusen}  Eric Rasmusen. 1989. Games and Information (An
Introduction to Game Theory).-  Blackwell. Cambridge MA, Oxford UK.

\bibitem{Tirole}  Jean Tirole. 1988. The Theory of Industrial Organization.-
 MIT Press.   Cambridge, Massachusets.

\bibitem{Heywood} Andrew Heywood. 1997. Politics.- London, Macmillan.

\bibitem{Lane}. J.-E.Lane \& S.Ersson. 1994.
 Comparative politics.- Cambridge, Blackwell.

\bibitem{Moulin85} Э.Мулен. 1985.  Теория игр  (с примерами из математической
экономики).- М., Мир.

\bibitem{Moulin95} Э.Мулен. 1995?. Кооперативное принятие решений:
 аксиомы и проблемы.- М., Мир.

\bibitem{Varian} H.Varian ``Microec.Analysis''

\bibitem{BusGel99} В.Бусыгин, С.Коковин, Е.Желободько, А.Цыплаков.
1999. ``Микроэкономический анализ несовершенных
рынков''.- TEMPUS (TACIS), NSU, Новосибирск.

\bibitem{BusCyp96} В.Бусыгин, С.Коковин,  А.Цыплаков. 1996.
 ``Методы микроэкономического анализа: фиаско рынка''.- TEMPUS (TACIS),
 NSU, Новосибирск.

\end{thebibliography}

1) Рекомендуемая студентам литература:

1. J. Tirole. 1988. Industrial organization.- MIT
Press. Cambridge, Massachusets. (Ж.Тироль. Теория
отраслевых рынков.- М.Экономика, 1999.) (Глава 11).

2. В.Бусыгин, С.Коковин, Е.Желободько, А.Цыплаков.
1999. ``Микроэкономический анализ несовершенных
рынков".- TEMPUS (TACIS), NSU, Новосибирск. (Глава 1).

3. В.Бусыгин, С.Коковин, А.Цыплаков. 1996. ``Методы
микроэкономического анализа'' - TEMPUS (TACIS), NSU, Новосибирск. (Глава 1).

4. D.M. Kreps. 1990.  A Course in Microeconomic
Theory.- Princeton University Press,   Princeton.
(Part III, Chapters 11-15)

5. Peter C. Ordeshook. 1992. A Political Theory 
Primer.- Routledge, N.-Y., London.

6. Andrew Heywood. 1997. Politics .- London, Macmillan.

7. R.B.Myerson. 1991. Game Theory (Analysis of
Conflict).- Harvard U.P., Camridge, London.

8.  Fudenberg, Drew and Jean Tirole. 1991. Game
theory.- MIT Press.       Cambridge, Massachusets.

9. J.-E.Lane and S.Ersson. 1994. Comparative
politics.- Cambridge, Blackwell.

2) Дополнительная литература используемая в курсе:

9. Э.Мулен. 1985. Теория игр, с примерами из мат.
экономики.- пер. с англ. Москва, Мир.

10. Э.Мулен. 1991. Кооперативное принятие решений:
аксиомы и модели.- пер. с англ. Москва, Мир.

11. Э.Экланд. 1985. Математическая экономика.- пер.
с англ. Москва, Мир.

12. В.Маракулин. 2001. Равновесный анализ
математических моделей экономики с не-стандартными
ценами (Теория игр - часть 3).- Новосибирск НГУ.

13. Р.Льюс, Э.Райфа. 1971.// Игры и решения.- пер.
с англ. Москва, Мир. //

14. Р. Оуен. 1971// Теория игр.- пер. с англ.
Москва, Наука. //

15. Дж.фон Нейман, О.Моргенштерн. 1970. Теория игр
и экономическое поведение.- пер. с англ. Москва,
Наука.

3) Источники задач и упражнений, используемые в
курсе:

Книги: J.Tirole 1988, D.M. Kreps 1990, Э.Мулен.
1985, 1991, P.C.Ordeshook 1992. Подборки задач
университетов (из Интернета и личных контактов):
Harward, Central European University (Budapest),
New Economic School (Moscow).

\end{document} 