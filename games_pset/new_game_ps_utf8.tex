\documentclass[pdftex, 12pt, a4paper]{article}


\input{/home/boris/science/tex_general/title_bor_utf8}

%\usepackage{showkeys} % показывать метки
\usepackage{chessboard}
%\usepackage{verbatim}

%\input{prob_and_sol_utf8}
\input{/home/boris/science/tex_general/prob_and_sol_utf8}

\title{Задачник для тигров\footnote{Можно всегда скачать c \url{http://demeshev.wordpress.com/materials}}}
\author{Тигр и все-все-все}
\date{\today}


\newcommand{\source}[1]{\problemtext{\par Источник: #1. \par}}


\makeindex % команда для создания предметного указателя
\bibliographystyle{plain} % стиль оформления ссылок


% from БЗИ:
\def\nsk{\vspace{-2mm}}
\def\hi{\vphantom{\Big)}}
\def\hidr#1#2{\frac{\displaystyle#1\hi}{\displaystyle#2\hi}}
\def\phm{\hphantom{-}}
\def\st{{\rm st}\,}
\def\Par{Pareto} % сам дописал, как правильно это Парето обозначать? ...
\def\hence{\Rightarrow} % чтобы работал кусок из БЗИ



\begin{document}

\pagestyle{myheadings} \markboth{Задачник для Тигров }{Задачник для Тигров }

\nocite{winkler:gpdp} \nocite{colell:mt} \nocite{binmore:fg}
\nocite{ilf:12} \nocite{kino:mind} \nocite{jowas:incorrect} \nocite{redondo:etg} \nocite{sekei:paradox}
\nocite{polisci:lectures} \nocite{cramton:lectures} \nocite{mti:lectures} \nocite{kockesen:lectures} \nocite{brams:dps} \nocite{squintani:nncgt} \nocite{zade:rn} \nocite{osborne:igt} \nocite{osborne:cgt} \nocite{miller:gtw} \nocite{gintis:gte} \nocite{danilov:lte} \nocite{gt.net}

\maketitle
\tableofcontents{}

\parindent=0 pt % отступ равен 0

\section{Introduction}
% gt_introduction

Задачи взяты из различных источников, по возможности, указан автор.

В коллекцию (Леша разрешил!) перекочевали задачи из «Большого Задачника Игр» (С. Коковкин, А. Тонис, А. Савватеев и др.). Эти задачи отмечены так «БЗИ».

Если есть вопросы по этой коллекции - boris.demeshev@gmail.com




%ver 04.06.07 Все сброшено в tex с потерей форматирования и рисунков \par
%ver 06.06.07 Восстановлено кое-как форматирование \par
%ver 18.06.07 Восстановление картинок к играм и оформления продолжается \par
%ver 18.11.07 Остались деревья: rayles, к берегу водохранилища, несколько деревьев в лесу
%ver 15.05.08 Ковбои \par
%ver 05.06.08 Rayles таки сдан! Дележ рутинной работы, ссылки для дележа пирога \par

\subsection{Quotes}

Ralph: When she put two potatoes on the table, the big one and the small one, you immediately took the big one without asking me what I wanted.\par
Norton: What would you have done?\par
Ralph: I would have taken the small one, of course.\par
Norton: You would? (in disbelief)\par
Ralph: Yes, I would!\par
Norton: So, what are complaining about? You GOT the little one!\par
{\it The Honeymooners, James Q. Wilson}\par

There are two important concepts in economics. The first is «Buy low, sell high», which is self-explanatory. The second is opportunity cost, the highest valued alternative that must be sacrificed to attain something or otherwise satisfy a want. I discovered this concept as an undergraduate at Caltech. I was never very in to computer games, but I found myself randomly playing tetris. Out of the blue I was struck by a revelation: «I could be having sex right now.» I haven't played a computer game since.\par
{\it Introduction to Methods of Applied Mathematics, Sean Mauch}\par

Бюджетное ограничение следует называть принципом кота Матроскина: «Чтобы продать что-нибудь ненужное, надо сначала купить что-нибудь ненужное».\par
{\it идея Юры Автономова}\par

Правильно хватать самый маленький кусок торта! Его можно съесть раньше, чем сестры доедят свои куски, и тогда успеешь взять еще и второй!\par
{\it по мотивам «Делим по справедливости», Брамс}\par



In mastering the material in this book, you are going to have to do a lot of work. This will consist mainly of chewing a pencil or pen as you struggle to do some sums. Maths is like that. Hours of your life will pass doing this, when you could be watching the X-files or playing basketball, or whatever.\par
{\it Michael D. Alder, An Introduction to  Complex Analysis for Engineers}\par


Well of course I didn't do any at first ... then someone suggested I try just a little sum or two, and I thought «Why not? ... I can handle it». Then one day someone said «Hey, man, that's kidstuff - try some calculus» ... so I tried some differentials ... then I went on to integrals ... even the occasional volume of revolution ... but I can stop any time I want to ... I know I can. OK, so I do the odd bit of complex analysis, but only a few times ... that stuff can really screw your head up for days ... but I can handle it ... it's OK really ... I can stop any time I want ...\par
{\it tim@bierman.demon.co.uk (Tim Bierman)}\par



\subsection{About}

Задачник не вызывает сонливости и не имеет противопоказаний.\par
В случае крайне маловероятной посадки на воду задачник может быть использован в качестве спасательного средства. \par
Задачник оборудован тремя аварийными выходами: в передней, средней и хвостовой частях.\par
Отпускается без рецепта.\par
Срок годности не ограничен.\par
Не содержит мелких деталей, которые могут быть проглочены детьми до 3-х лет.\par
{\bf }Предисловие:\par
Задачи упорядочены только по типу!\par

Условия задач кроме составителя комментировал Тигр - тотем теории игр.\par


Краткий словарь:\par
Nash Equilibrium, NE - равновесие по Нэшу. В задачнике этот термин используется максимально широко, в том числе и для игр с неполной информацией.\par
Subgame Perfect Nash Equilibrium, SPNE - равновесие по Нэшу совершенное в подыграх\par
Weak Sequential Equilibrium - слабое секвенциальное равновесие \par
Sequential Equilibrium, SE - секвенциальное равновесие\par
Common knowledge of information - всеобщность знания\par
Correlated Equilibrium - Коррелированное равновесие\par

Многие задачи можно решать, руководствуясь только здравым смыслом. Т.е. если Вы все-таки так и не поняли, что такое равновесие по Нэшу, попытайтесь ответить на вопрос «Как бы я играл в эту игру?»\par
По умолчанию предполагается, что все сказанное в условии является всеобщим знанием.\par

Что означают пометки у задач?\par
$[$Т$]$ - Трудная задача\par
$[$О$]$ - Особенная задача. У особенной задачи может быть смешное условие, а ее решение может кардинально изменить геополитическую обстановку в мире.




Маршруты для туристов (нужно разработать):

$[$Сюда также что-то типа: эти задачи (список) предлагались в курсе ВШЭ, т.е. несколько маршрутов для самостоятельного изучения$]$

«Последний двоечник». Задачи для тех, кто хочет гарантировать себе один или, если повезет, два балла из десяти:

«PG-13» Parents strongly cautioned. Some material may be inappropriate for children under 13:\par

«NC-17» No one 17 and under admitted. May contain explicit sex scenes, and/or scenes of excessive violence:\par

«Для тех, кто и вправду крут». {\it Тигр:  По-моему, это задачи, которые составитель задачника не смог решить сам...} \par

«Для прессы» {\it Тигр: Подполковник Киселевич говорил, что любая аудитория делится на три части. Спереди сидят лошади, они все время  пашут, посередине сидят бездельники, они ничего не делают, а сзади сидит пресса, она ждет сенсаций.}\par


\problemonly
\input{gt_problems_utf8}

\restoresection
\section{Решения}
\solutiononly
\addtocounter{secsolution}{1} % 1 - число section до команды \problemonly
\input{gt_problems_utf8}


\restoresection

\section{Названия концепций решения (Коковин+)}
\textbf{Максимин (ММ)} - исход игры (профиль
стратегий) при осторожном поведении всех, то есть
при максимизации гарантированных выигрышей, не
учитывая в своих расчетах целей и текущих решений
партнеров.\vspace{2mm}

\textbf{Решение в (слабо-) доминирующих стратегиях
(WDE)} или слабо-доминирующее равновесие - исход
игры в случае наличия у каждого
"абсолютно-оптимальной" стратегии, то есть
стратегии, (слабо) доминирующей над всеми другими
его стратегиями независимо от ходов партнеров, их
целей и текущих решений. [Аналогично и определение
сильно-доминирующего равновесия
\textbf{SDE}.]\vspace{2mm}

\textbf{Решение в итерационно-
(слабо-)недоминируемых стратегиях (IND${}_W$)} -
исход игры в случае одновременного итерационного
отбрасывания (слабо-) доминируемых стратегий каждым
игроком и соответствующего редуцирования игры:
исключения отброшенных стратегий из рассмотрения
ВСЕМИ игроками. Требует знания или целей партнеров
или факта отбрасывания стратегий. [Аналогично
определяется Решение в итерационно-
сильно-недоминируемых стратегиях
(IND${}_S$).]\vspace{2mm}

\textbf{Равновесие Нэша (NE}) -  исход игры
(профиль стратегий), при котором ни у одного игрока
нет стимула отступить от своей текущей стратегии,
при знании текущих стратегий партнеров и гипотезе,
что партнеры не отступят. [Эквивалентный вариант:
Равновесие Нэша - исход, когда все сходили
одновременно вслепую, имея лишь некоторые ожидания
о запланированном ходе партнеров, а когда карты
открылись, то все ожидания
оправдались.]\vspace{2mm}

\textbf{Совершенное в Подыграх Равновесие (Нэша)
(SPE} = SPNE) - это равновесие Нэша в развернутой
форме игры, являющееся также равновесием Нэша во
всех ее подыграх.
% бред какой-то
%(Внимание: оно может не являться
%NE этой же игры в нормальной форме, поэтому не
%всегда $SPE\subseteq NE$!)
\vspace{2mm}


\textbf{Слабое секвенциальное равновесие} Weak sequential equilibrium, WSE

\vspace{2mm}


\textbf{Секвенциальное равновесие} Sequential equilibrium, SE

\vspace{2mm}

\textbf{Совершенное байесовское равновесие} Perfect bayesian equilibrium, PBE
Применимо только к динамическим байесовским играм.
\vspace{2mm}






\textbf{Слабый оптимум Парето} ($W\Par$) -
возможный исход, который нельзя улучшить для всех
игроков сразу, даже согласовав их ходы.
\textbf{Сильный оптимум Парето} ($\Par$) - исход,
который нельзя улучшить для кого-то, не ухудшив для
других. \vspace{2mm}

\textbf{Элемент (слабого) Ядра игры} (${\bf C}$) -
возможный исход, который не блокируется ни одной
коалицией в переговорах. Коалиция блокирует в
переговорах (отвергает) вариант, если имеет другой,
строго более желательный для всех своих членов,
среди СВОИХ возможностей (среди вариантов,
достижимых независимо от действий вне-коалиционных
игроков). Т.е. Ядро - множество вариантов, вне
которого соглашений быть не может. \vspace{2mm}

{\bf Сокращения:}  $MM$ -- MaxiMin, $DE$ --
Dominant Equilibrium, $SDE$ -- Strong Dominant
Equilibrium, $IND_W$ -- Iterative (Weakly)
Non-Dominant Equilibrium,  $SoE$ -- Sophisticated
Equilibrium, $NE$ - Nash Equilibrium,  $NE_m$ --
Nash Equilibrium in Mixed stratagies, $SP(N)E$ --
Subgame Perfect (Nash) Equilibrium, $StE$ --
Stackelberg Equilibrium, $\Par$ - Pareto, $C$ --
Core. \vspace{2mm}






\section{Названия некоторых стратегий в повторяющейся дилемме заключенного}

Обозначения:

 $a^{t} $  - исход базовой игры с номером  $t$ ;

 $a_{i}^{t} $  - ход сделанный  $i$ -ым игроком в базовой игре с номером  $t$ .

 $s_{i} $  - стратегия  $i$ -го игрока;

 $h^{t} $  - предыстория игры к моменту времени  $t$ :  $h^{t} =\left\{a^{1} ,a^{2} ,...a^{t-1} \right\}$ ;

 $s_{i} \left(h^{t} \right)$  - ход, предписываемый стратегией  $s_{i} $  после истории  $h^{t} $  (в момент  $t$ );

{\bf Стратегия «Всегда кооперироваться»} (always cooperate)

Предписывает всегда играть ход  $c$ :  $s_{i} \left(h^{t} \right)=c,\quad \forall t$

{\bf Наивная стратегия переключения} (naive grim trigger)

Предписывает играть ход  $c$  в первой партии и далее до тех пор, пока противник играет ход  $c$ :  $s_{i} \left(h^{t} \right)=\left\{\begin{array}{l} {c,\quad t=1} \par {c,\quad t>1,\quad \forall \tau <t\Rightarrow a_{j}^{\tau } =c} \par {d,\quad otherwise} \end{array}\right. $

{\bf Стратегия переключения} (grim trigger)

Предписывает играть ход  $c$  в первой партии и далее до тех пор, пока оба игрока играют ход  $c$ :  $s_{i} \left(h^{t} \right)=\left\{\begin{array}{l} {c,\quad t=1} \par {c,\quad t>1,\quad \forall \tau <t\Rightarrow a^{\tau } =\left(c;c\right)} \par {d,\quad otherwise} \end{array}\right. $

{\bf Стратегия «Зуб за зуб»} (Tit for Tat)

Предписывает играть ход  $c$  в первой партии и далее повторять ход противника в предыдущей партии:  $s_{i} \left(h^{t} \right)=\left\{\begin{array}{l} {c,\quad t=1} \par {a_{j}^{t-1} ,\quad t>1} \end{array}\right. $

{\bf Стратегия Кнута и Пряника} (Win-Stay, Lose-Shift; Pavlov strategy)

Предписывает играть ход  $c$  в первой партии и далее играть ход  $c$ , если в предыдущей партии действия игроков совпали:  $s_{i} \left(h^{t} \right)=\left\{\begin{array}{l} {c,\quad t=1} \par {c,\quad t>1,\quad a^{t-1} \in \left\{\left(c;c\right),\left(d;d\right)\right\}} \par {d,\quad otherwise} \end{array}\right. $
{\it Тигр:  Эта хитрая стратегия была внедрена известными специалистами по теории игр Кнутом Б.Б. и Пряником В.Л.}

{\bf Стратегия ограниченного возмездия} (limited retaliation)

Предписывает играть ход  $c$ , пока все игроки кооперируются. Если произошло нарушение, то в течение  $k$  ходов играть  $d$ , затем вернуться в исходное состояние. Состоит из трех фаз:
Фаза 1: сыграть ход  $c$  и переключиться в фазу 2;
Фаза 2: играть ход  $c$  до тех пор, пока все игроки играют ход  $c$ , в противном случае переключиться в фазу 3, положив  $\tau :=0$ ;
Фаза 3: пока  $\tau \le k$ , положить  $\tau :=\tau +1$  и играть ход  $d$ , иначе переключиться в фазу 1.



\bibliography{/home/boris/Dropbox/Public/tex_general/opit}


\printindex % печать предметного указателя здесь






\end{document}


%%% Local Variables:
%%% mode: latex
%%% TeX-master: t
%%% End:
