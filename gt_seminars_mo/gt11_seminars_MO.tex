\documentclass[pdftex,12pt,a4paper]{article}

% jan 2012

% sudo yum install texlive-bbm texlive-bbm-macros texlive-asymptote texlive-cm-super texlive-cyrillic texlive-pgfplots texlive-subfigure
% yum install texlive-chessboard texlive-skaknew % for \usepackage{chessboard}
% yum install texlive-minted texlive-navigator texlive-yax texlive-texapi

% растягиваем границы страницы
%\emergencystretch=2em \voffset=-2cm \hoffset=-1cm
%\unitlength=0.6mm \textwidth=17cm \textheight=25cm

\usepackage{makeidx} % для создания предметных указателей
\usepackage{verbatim} % для многострочных комментариев
\usepackage{cmap} % для поиска русских слов в pdf
\usepackage[pdftex]{graphicx} % для вставки графики 
% omit pdftex option if not using pdflatex


%\usepackage{dsfont} % шрифт для единички с двойной палочкой (для индикатора события)
\usepackage{bbm} % шрифт - двойные буквы

\usepackage[colorlinks,hyperindex,unicode,breaklinks]{hyperref} % гиперссылки в pdf


\usepackage[utf8]{inputenc} % выбор кодировки файла
\usepackage[T2A]{fontenc} % кодировка шрифта
\usepackage[russian]{babel} % выбор языка

\usepackage{amssymb}
\usepackage{amsmath}
\usepackage{amsthm}
\usepackage{epsfig}
\usepackage{bm}
\usepackage{color}

\usepackage{multicol}


\usepackage{textcomp}  % Чтобы в формулах можно было русские буквы писать через \text{}

\usepackage{embedfile} % Чтобы код LaTeXа включился как приложение в PDF-файл

\usepackage{subfigure} % для создания нескольких рисунков внутри одного

\usepackage{tikz,pgfplots} % язык для рисования графики из latex'a
\usetikzlibrary{trees} % прибамбас в нем для рисовки деревьев
\usetikzlibrary{arrows} % прибамбас в нем для рисовки стрелочек подлиннее
\usepackage{tikz-qtree} % прибамбас в нем для рисовки деревьев


\usepackage{ifpdf} % чтобы проверять, запускаем мы pdflatex или просто latex

\ifpdf
	\usepackage[pdftex]{graphicx} 
	\DeclareGraphicsRule{*}{mps}{*}{} % все неупомянутые ps файлы объявляем упрощенными, т.е. mps типа. Просто ps графику нельзя использовать, но без некоторых спец. команд - можно. Например, результат работы metapost - это ps файлы простого (mps) типа. Собственно ради использования metapost эта строка и введена.
\else
	\usepackage{graphicx}
\fi



% конец добавки

\usepackage{asymptote} % After graphicx!, пакет для рисования графиков и прочего
%\usepackage{sagetex} % i suppose after graphicx also..., для связи с sage



\embedfile[desc={Исходный LaTeX файл}]{\jobname.tex} % Включение кода в выходной файл
\embedfile[desc={Стилевой файл}]{/home/boris/science/tex_general/title_bor_utf8.tex}



% вместо горизонтальной делаем косую черточку в нестрогих неравенствах
\renewcommand{\le}{\leqslant}
\renewcommand{\ge}{\geqslant} 
\renewcommand{\leq}{\leqslant}
\renewcommand{\geq}{\geqslant}

% делаем короче интервал в списках 
\setlength{\itemsep}{0pt} 
\setlength{\parskip}{0pt} 
\setlength{\parsep}{0pt}

% свешиваем пунктуацию (т.е. знаки пунктуации могут вылезать за правую границу текста, при этом текст выглядит ровнее)
\usepackage{microtype}

% более красивые таблицы
\usepackage{booktabs}
% заповеди из докупентации: 
% 1. Не используйте вертикальные линни
% 2. Не используйте двойные линии
% 3. Единицы измерения - в шапку таблицы
% 4. Не сокращайте .1 вместо 0.1
% 5. Повторяющееся значение повторяйте, а не говорите "то же"


% DEFS
\def \mbf{\mathbf}
\def \msf{\mathsf}
\def \mbb{\mathbb}
\def \tbf{\textbf}
\def \tsf{\textsf}
\def \ttt{\texttt}
\def \tbb{\textbb}

\def \wh{\widehat}
\def \wt{\widetilde}
\def \ni{\noindent}
\def \ol{\overline}
\def \cd{\cdot}
\def \bl{\bigl}
\def \br{\bigr}
\def \Bl{\Bigl}
\def \Br{\Bigr}
\def \fr{\frac}
\def \bs{\backslash}
\def \lims{\limits}
\def \arg{{\operatorname{arg}}}
\def \dist{{\operatorname{dist}}}
\def \VC{{\operatorname{VCdim}}}
\def \card{{\operatorname{card}}}
\def \sgn{{\operatorname{sign}\,}}
\def \sign{{\operatorname{sign}\,}}
\def \xfs{(x_1,\ldots,x_{n-1})}
\def \Tr{{\operatorname{\mbf{Tr}}}}
\DeclareMathOperator*{\argmin}{arg\,min}
\DeclareMathOperator*{\argmax}{arg\,max}
\DeclareMathOperator*{\amn}{arg\,min}
\DeclareMathOperator*{\amx}{arg\,max}
\def \cov{{\operatorname{Cov}}}

\def \xfs{(x_1,\ldots,x_{n-1})}
\def \ti{\tilde}
\def \wti{\widetilde}


\def \mL{\mathcal{L}}
\def \mW{\mathcal{W}}
\def \mH{\mathcal{H}}
\def \mC{\mathcal{C}}
\def \mE{\mathcal{E}}
\def \mN{\mathcal{N}}
\def \mA{\mathcal{A}}
\def \mB{\mathcal{B}}
\def \mU{\mathcal{U}}
\def \mV{\mathcal{V}}
\def \mF{\mathcal{F}}

\def \R{\mbb R}
\def \N{\mbb N}
\def \Z{\mbb Z}
\def \P{\mbb{P}}
%\def \p{\mbb{P}}
\def \E{\mbb{E}}
\def \D{\msf{D}}
\def \I{\mbf{I}}

\def \a{\alpha}
\def \b{\beta}
\def \t{\tau}
\def \dt{\delta}
\def \e{\varepsilon}
\def \ga{\gamma}
\def \kp{\varkappa}
\def \la{\lambda}
\def \sg{\sigma}
\def \sgm{\sigma}
\def \tt{\theta}
\def \ve{\varepsilon}
\def \Dt{\Delta}
\def \La{\Lambda}
\def \Sgm{\Sigma}
\def \Sg{\Sigma}
\def \Tt{\Theta}
\def \Om{\Omega}
\def \om{\omega}


\def \ni{\noindent}
\def \lq{\glqq}
\def \rq{\grqq}
\def \lbr{\linebreak}
\def \vsi{\vspace{0.1cm}}
\def \vsii{\vspace{0.2cm}}
\def \vsiii{\vspace{0.3cm}}
\def \vsiv{\vspace{0.4cm}}
\def \vsv{\vspace{0.5cm}}
\def \vsvi{\vspace{0.6cm}}
\def \vsvii{\vspace{0.7cm}}
\def \vsviii{\vspace{0.8cm}}
\def \vsix{\vspace{0.9cm}}
\def \VSI{\vspace{1cm}}
\def \VSII{\vspace{2cm}}
\def \VSIII{\vspace{3cm}}


\newcommand{\grad}{\mathrm{grad}}
\newcommand{\dx}[1]{\,\mathrm{d}#1} % для интеграла: маленький отступ и прямая d
\newcommand{\ind}[1]{\mathbbm{1}_{\{#1\}}} % Индикатор события
%\renewcommand{\to}{\rightarrow}
\newcommand{\eqdef}{\mathrel{\stackrel{\rm def}=}}
\newcommand{\iid}{\mathrel{\stackrel{\rm i.\,i.\,d.}\sim}}
\newcommand{\const}{\mathrm{const}}

%на всякий случай пока есть
%теоремы без нумерации и имени
%\newtheorem*{theor}{Теорема}

%"Определения","Замечания"
%и "Гипотезы" не нумеруются
%\newtheorem*{defin}{Определение}
%\newtheorem*{rem}{Замечание}
%\newtheorem*{conj}{Гипотеза}

%"Теоремы" и "Леммы" нумеруются
%по главам и согласованно м/у собой
%\newtheorem{theorem}{Теорема}
%\newtheorem{lemma}[theorem]{Лемма}

% Утверждения нумеруются по главам
% независимо от Лемм и Теорем
%\newtheorem{prop}{Утверждение}
%\newtheorem{cor}{Следствие}


%\usepackage{showkeys} % показывать метки

%\input{/home/boris/Dropbox/Public/tex_general/prob_and_sol_utf8}


\newcounter{zadacha}[section]
%новые счетчик "zadacha" будет автоматом сбрасываться на 0 при старте нового раздела

\newcommand{\zad}{\par\refstepcounter{zadacha}\arabic{zadacha}. }

\renewcommand{\thezadacha}{\thesection.\arabic{zadacha}}


%\renewcommand{\thedefin}{\thesection.\arabic{defini}}


\begin{document}
\parindent=0 pt % отступ равен 0


%\section{Методические рекомендации для преподавателей и студентов при проведении семинарских занятий по курсу теории игр ГУ-ВШЭ}

%\newpage

\section{Матричная форма, осторожные стратегии, обратная индукция (начало)}

$[$Slantchev$]$ - глава Elements of basic models, $[$IGT$]$ - глава 2,5 \\

\small
Информационное множество --- один или несколько узлов, которые игрок не может различить между собой (пунктир на дереве) \\
Стратегия -- инструкция, указывающая какой ход нужно выбирать в каждом узле (информационном множестве). Стратегия позволяет игроку продолжить игру с любой позиции, не только с начальной. \\
Профиль стратегий или исход -- набор стратегий по одной от каждого игрока \\
Игра с совершенной информацией -- игра, где каждый игрок знает всю предысторию игры;  на дереве есть только одноточечные информационные множества\\
Осторожная стратегия -- стратегия, которую выберет игрок, предполагающий, что при любом его выборе для него произойдет самое худшее\\
Антагонистическая игра -- игра двух игроков, в которой выигрыш одного игрока равен проигрышу другого\\
Природа или Случай -- особый игрок, у которого изначально заданы вероятности ходов (буква N на дереве).\\
Суть метода обратной индукции -- начиная с конца игры определяем кому как выгодно пойти. Каждый игрок выбирает больший средний выигрыш. Применим в играх с совершенной информацией. \\
\normalsize

Задачи 1.1.	1.2.	1.3. 
\begin{figure}[!htbp]
    \includegraphics{game.23}
    \includegraphics{game.25}
    \includegraphics{game.24}
\end{figure} \\
а) Сколько стратегий у каждого игрока? Сколько в игре исходов?\\
б) Выпишите все профили стратегий, при которых, игра оканчивается в узле (-1;-1).\\
в) Переведите игру в матричную форму\\
г) Решите игру методом обратной индукции \\

Задачи 2.1.	2.2.	2.3.
\begin{figure}[!htbp]
    \includegraphics{game.13}
    \includegraphics{game.22}
    \includegraphics{game.28}
\end{figure} \\
a) Укажите число стратегий каждого игрока \\
б) Переведите игры в матричную форму\\
Примечание: в играх с природой в матрицу пишут средний выигрыш \\

\setcounter{zadacha}{2}
\zad Задана антагонистическая игра (в матрице указывается выигрыш для первого игрока):\\
$\begin{array}{|c|c|c|c|}
\hline
    {} &  d & e & f   \\
\hline
    a &  { - 2} & 1 & 3   \\
    b &  3 & x & { - 5}   \\
    c &  2 & { - 3} & 0   \\
\hline
\end{array}$\\
а) Допишите в матрицу выигрыши второго игрока;\\
б) Найдите осторожные стратегии обоих игроков при $x = 0$\\
г) Найдите осторожные стратегии обоих игроков при произвольном $x$\\

\zad Задана матрица игры:\\
$\begin{array}{|c|c|c|c|}
\hline
    {} &  {t_1 } & {t_2 } & {t_3 }   \\
\hline
    {l_1 } &  {\left( {0;0} \right)} & {\left( {0;0} \right)} & {\left( {0;0} \right)}   \\
    {l_2 } &  {\left( {0;0} \right)} & {\left( {1; - 3} \right)} & {\left( { - 3;1} \right)}   \\
    {l_3 } &  {\left( {0;0} \right)} & {\left( { - 3;1} \right)} & {\left( {1; - 3} \right)}   \\
\hline
\end{array}$\\
а) Существует ли последовательная игра с совершенной информацией (<<дерево без пунктиров>>) с такой матрицей? \\
б) Существует ли последовательная игра с несовершенной информацией (<<дерево с пунктирами>>) с такой матрицей? \\

\zad Deep Blue\\
Представим, что Вы играете партию в шахматы против компьютера. Сколько стратегий использует компьютер во время одной партии?\\

\zad Есть несколько кучек камней. Два игрока ходят по очереди. За ход разрешается: либо взять любое положительное количество камней из одной кучки, либо поделить любую кучку на две новые непустые кучки. Проигрывает тот, кто не может сделать ход. Проигравший платит победителю 1 рубль. Игроки ходят по очереди.\\
Нарисуйте дерево игры, начинающейся с единственной кучки из 3 камней.\\

\zad В кучке лежит 5 камней. Петя и Вася по очереди забирают камни из кучки. Петя ходит первым. За свой ход Петя может взять 1 или 3 камня, а Вася 2 или 3 камня. Проигрывает тот, кто не может сделать ход. Проигравший платит победителю рубль.\\
а) Нарисуйте дерево игры;\\
б) Переведите игру в матричную форму.\\
в) Найдите осторожные стратегии игроков.\\
г) Решите игру методом обратной индукции\\

\zad Допустим, что задана некая антагонистическая игра в матричной форме.\\
Определите, верно или не верно каждое утверждение:\\
а) При увеличении всех чисел матрицы на единицу осторожные стратегии игроков не изменятся.\\
б) При умножении всех чисел на минус единицу осторожные стратегии игроков не изменятся.\\
в) При транспонировании матрицы осторожные стратегии игроков поменяются местами (т.е. осторожные стратегии второго игрока станут осторожными стратегиями первого игрока и наоборот).\\

\zad С чего все начиналось... \\
Париж, Людовик XIV, 1654 год, высшее общество говорит о рождении новой науки - теории вероятностей. Ах, кавалер де Мере, <<fort honnete homme sans etre mathematicien>> <<благородный человек, хотя и не математик>>... Старая задача, неправильные решения которой предлагались тысячелетиями (например, одно из неправильных решений предлагал изобретатель двойной записи, кумир бухгалтеров, Лука Пачоли) наконец решена правильно!\\
Два игрока одинаковой силы играют в некую игру до шести побед. Игрок первым выигравший шесть партий (не обязательно подряд) получает 800 луидоров. К текущему моменту первый игрок выиграл 5 партий, а второй - 3 партии. Они вынуждены прервать игру в данной ситуации.\\
Как им поделить приз по справедливости?\\

\zad Кортес\\
Кортес с бандой головорезов высадился на берегу. Кортес выбирает, нападать ли на деревушку или нет. Местная деревушка может либо сразу перейти в подчинение Кортеса, либо принять бой. Если деревушка примет бой, то выбор появится у Кортеса: либо драться до победного конца, либо после первых потерь бежать на кораблях обратно. Ценность деревушки для Кортеса – одна единица, ценность собственных головорезов – 2 единицы. Если Кортес будет драться до конца, то деревушка будет взята, но большинство головорезов погибнет в бою. Для жителей деревушки – главное остаться в живых, сохранить при этом независимость, конечно, желательно. \\
а) Нарисуйте дерево игры и найдите обратно-индукционный исход.\\
б) Нарисуйте дерево игры и найдите обратно-индукционный исход в случае, если Кортес ограничил свои возможности – сжег корабли.\\


\section{Антагонистический игры}

%Надо было на антагонистические игры не обращать внимания...

\zad Какая-нибудь антагонистическая игра. В следующем году имеет смысл не рассказывать, а подробно разобрать 2x3 неантагонистическую

\zad Камень, Ножницы, Бумага и Колодец тоже надо




\section{Доминирование, NE в чистых  }

\textbf{Равновесие по Нэшу}. Nash equilibrium. NE.\\
-- Профиль стратегий, в котором ни один игрок не захочет сменить свою стратегию, даже если узнает стратегии других игроков. 



\zad Петя, Вася и Волшебная Шкатулочка \\
Вася и Петя нашли волшебную шкатулочку. Если в нее положить деньги и сказать <<Ахалай Махалай>>, то сумма, лежащая в шкатулке увеличивается в полтора раза. Один недостаток: работает только раз! Петя и Вася решили поступить так: каждый положит в шкатулку сколько хочет, потом они скажут <<Ахалай Махалай>> и поделят всю сумму поровну.\\
а) Что является стратегией игрока?\\
б) Представьте эту игру в \textit{нормальной форме}. \\
в) Можно ли представить игру в матричной форме? Почему?\\
г) Найдите равновесие по Нэшу;\\


\zad Игра в половину среднего. Три раунда. Победители делят шоколадку поровну.
Можно было спросить, что изменится, если каждый победитель получает по конфете?

\zad Делим пирог\\
Мама одновременно спрашивает двух братьев, какую долю пирога каждый хотел бы получить. Братья получают то, что попросили, если пирога хватает (остаток при этом мама уносит на работу). Если братья вместе запросили больше, чем целый пирог, то они не получают ничего.\\
а) Представьте игру в нормальной форме;\\
б) Найдите все равновесия по Нэшу в чистых стратегиях.\\
в) Как изменится ответ, если братья завидуют друг другу, т.е. если удовольствие равно размеру собственного куска минус размер чужого куска?

\zad
Задача. Два человека пришли в кабак. У одного из них 10 золотых, у второго
-- 6 золотых. Каждый может тратить деньги на выпивку или на музыку.
Музыка является общественным благом - ее слышат все. Выпивка -
частным. Полезности равны $u_{1}=(m_{1}+m_{2})d_{1}$ и
$u_{2}=(m_{1}+m_{2})d_{2}$, где $m_{i}$ и $d_{i}$ - расходы $i$-го
человека на музыку и выпивку. Предположим, что деньги бесконечно
делимы. \par
а) Найдите равновесие Нэша \par
б) Что изменится в случае, если у второго 2 золотых? \par
в) Что изменится, если функции полезности имеют вид $u_{1}=(m_{1}-m_{2})v_{1}+20v_1$ и
$u_{2}=(m_{2}-m_1)v_{2}+20v_2$?


\section{Биматричные игры, NE в смешанных }

\zad Биматричная игра 3х3, сводимая к 2х2 вычеркиванием строго доминируемых стратегий

\zad Первый игрок называет число $x\in [0;10]$, одновременно второй называет число $y\in [0;5]$. Выигрыши определяются по формулам
\begin{equation}
\begin{array}{l}
u_1(x,y)=-x^2+6xy+\cos(y) \\
u_2(x,y)=-y^2+4xy+4y+\sin(x)
\end{array}
\end{equation}

Найдите равновесие Нэша в чистых стратегиях

\zad Конверты \\
В пяти конвертах спрятаны суммы в 10\$, 20\$, 40\$, 80\$ и 160\$. Случайным образом выбираются два конверта с соседними суммами и выдаются игрокам, по одному конверту игроку. {\it Тигр: А в конвертах - это взятки?} Каждый игрок открывает свой конверт и выбирает, хочет ли он оставить себе сумму или хочет обменяться. Обмен происходит, если оба игрока согласны обменяться.\\
а)	Сколько чистых стратегий у каждого игрока?\\
б)	Найдите все равновесия по Нэшу в чистых и стратегиях\\



\zad Вопросы о разном... \\
а) Может ли в равновесие по Нэшу входить строго доминируемая стратегия? \\
б) Может ли в равновесие по Нэшу входить нестрого доминируемая стратегия? \\
в) Могут ли все исходы игры быть равновесными по Нэшу? \\
г) Может ли в произвольной игре не быть ни одного равновесия по Нэшу в чистых стратегиях? А в биматричной? \\
д) Может ли в произвольной игре не быть ни одного равновесия по Нэшу в смешанных стратегиях? А в биматричной? \\
е) Может ли существовать в произвольной игре стратегия, на которую не существует оптимального ответа? А в биматричной? \\
ё) Может ли игра в развернутой форме быть антагонистической?\\
ж) Сколько смешанных стратегий у первого игрока, если у первого игрока две чистых стратегии, а у второго – три чистых стратегии?\\

\zad
Задача. Limbo.

До весны 2007 года в Швеции существовала необычная лотерея <<Limbo>>. Правила выглядят следующим образом. Вы можете выбрать любое натуральное число. Победителем объявляется тот, кто назвал самое маленькое число, никем более не названное. Например, если игроки назвали числа 1, 3, 1, 2, 4, то победителем будет тот, кто назвал число 2. Если наименьшего никем более не названного нет, то приз остается у организаторов. \par
а) Опишите все равновесия по Нэшу в чистых стратегиях для $n$ игроков


\zad
Задача. Мост \par
Из пункта $A$ в пункт $D$ можно попасть двумя путями - через B или
через C. Если по дороге AB едет $x$ машин, то время в пути каждой
из них будет равно $f_{AB}(x)=x+32$. Для других отрезков пути
функции равны: $f_{BD}(x)=5x+1$, $f_{CD}(x)=x+32$ и
$f_{AC}(x)=5x+1$.
Каждое утро из города $A$ в город $D$ едет 6 машин. \par
\begin{figure}[ht]
    \includegraphics{game.5}
\end{figure} \par
а) Сколько машин и по какой дороге едет в равновесии по Нэшу?
Сколько им требуется времени, чтобы добраться из $A$ в $D$? \par
б) Как изменятся ответы, если между городами $B$ и $C$ построен
удобный мост, такой что $f_{BC}=0$?




\end{document}

$[$Slantchev$]$ - глава Dominance, Nash Equilibrium, Symmetry; $[$IGT$]$ - глава 2,3,4, 11\\



\textbf{Парето оптимум}. Pareto optimum. PO.\\
-- Вектор платежей называется Парето-оптимальным, если любой другой вектор платежей хуже хотя бы для одного игрока.

%Исход, в котором платеж ни одного игрока нельзя улучшить, не ухудшив при этом платеж хотя бы одного другого игрока.\\

Разные замечания: 
\begin{itemize}
\item Смешивать стратегии $a$ и $b$ оптимально тогда и только тогда, когда $a$ и $b$ приносят одинаковый платеж, а остальные стратегии - не больший. \\
\item Если предложить игрокам перейти из Парето оптимального исхода в любой другой, то хотя бы один игрок не согласится.\\
\end{itemize}




Развернутая форма - дерево\\
Нормальная форма включает: список игроков; для каждого игрока: список стратегий, зависимость выигрыша от выбранных всеми игроками стратегий. \\
Стратегия <<a>> строго доминирует стратегию <<b>>, если вне зависимости от действий других игроков <<a>> приносит больший выигрыш, чем <<b>>.\\

Смешанная стратегия -- случайный эксперимент, в результате которого выбирается одна из чистых стратегий.\\


\zad Рассмотрим игру $\begin{array}{|c|c|c|c|}
\hline
    {} &  {t_1 } & {t_2 } & {t_3 }   \\
\hline
    {l_1 } &  {\left( {2;8} \right)} & {\left( {1;6} \right)} & {\left( {9;20} \right)}   \\
    {l_2 } &  {\left( {7;7} \right)} & {\left( {6;8} \right)} & {\left( {2;6} \right)}   \\
\hline
\end{array}$\\

а) Какая стратегия является наилучшим ответом второго игрока на стратегию $l_2 $?\\
б) Дополните матрицу смешанной стратегий первого игрока $\frac{1}{3}l_1  + \frac{2}{3}l_2 $\\
в) Какая стратегия является наилучшим ответом второго игрока на стратегию $\frac{1}{3}l_1  + \frac{2}{3}l_2 $?\\
г) Найдите все смешанные стратегии (смешиваются $t_2 $  и $t_3 $), которые строго доминируют чистую стратегию $t_1 $ в исходной матрице. \\


\zad Найдите равновесия по Нэшу в смешанных стратегиях: \\
а) $\begin{array}{|c|c|c|c|}
\hline
& t_{1} & t_{2} & t_{3} \\
\hline
s_{1} & (2;4) & (3;4) & (0;0) \\
s_{2} & (1;2) & (4;7) & (1;8) \\
\hline
\end{array}$;
б) $\begin{array}{|c|c|c|c|}
\hline
& t_{1} & t_{2} & t_{3} \\
\hline
s_{1} & (3;3) & (1;1) & (5;3) \\
s_{2} & (1;2) & (2;2) & (1;1) \\
\hline
\end{array}$;
в) $\begin{array}{|c|c|c|}
\hline
& t_{1} & t_{2} \\
\hline
s_{1} & 5 & 1 \\
s_{2} & 2 & 3 \\
s_{3} & 1 & 6 \\
\hline
\end{array}$ \\


\zad Два игрока играют в статическую игру. Первый выбирает $x\in [0;3]$, второй одновременно выбирает $y\in [0;4]$. Платежные функции имеют вид $U_{1}(x,y)=-x^{2}+2xy+6y+cos(y)$, $U_{2}(x,y)=-y^{2}-8xy+6y+arctg(x)$. \\
а) Изобразите функции наилучших ответов в осях $(x,y)$ \\
б) Найдите равновесие по Нэшу в чистых стратегиях \\


\zad $\begin{array}{|c|c|c|c|}
\hline
& d & e & f \\
\hline
a & (3;5) & (1;0) & (2;2)\\
b & (-1;0) & (4;4) & (7;2)\\
c & (1;7) & (2;3) & (4;7)\\
\hline
\end{array}$ \\
a) Придумайте смешанную стратегию, строго доминирующую стратегию $c$ \\
б) Последовательно вычеркивая строго доминируемые стратегии сведите матрицу к размеру $(2\times 2)$ \\
в) Найдите все равновесия по Нэшу в смешанных стратегиях \\


\zad Два тигра \\
Два тигра заметили двух антилоп. Маленькую, весом в один условный килограмм, и большую, весом в $a > 1$ условных килограммов. Они одновременно принимают решение, за какой антилопой погнаться. Тигры всегда догоняют антилоп. \textit{Тигр: Если хотят, то конечно.} Если тигры выберут одну антилопу, то они поделят ее поровну. \\
а) Запишите игру в матричной форме;\\
б) Найдите все равновесия по Нэшу в чистых и смешанных стратегиях в зависимости от $a$\\

\zad Для игр: \\
$\begin{array}{|c|c|c|c|}
\hline
    {} &  d & e & f   \\
\hline
    a &  (2;2) & (1;1) & (7;1)   \\
    b &  (0;1) & (3;3) & (-3;2)   \\
    c &  (1;4) & (2;2) & (0;0)   \\
\hline
\end{array}$;
 $\begin{array}{|c|c|c|c|}
\hline
    {} &  d & e & f   \\
\hline
    a &  (0;2) & (2;1) & (-2;1)   \\
    b &  (0;1) & (3;3) & (-3;4)   \\
    c &  (1;0) & (2;2) & (0;0)   \\
\hline
\end{array}$;
 $\begin{array}{|c|c|c|c|}
\hline
    {} &  d & e & f   \\
\hline
    a &  (2;2) & (3;2) & (7;1)   \\
    b &  (0;1) & (3;3) & (-3;1)   \\
    c &  (0;4) & (-1;2) & (0;0)   \\
\hline
\end{array}$;\\
а) Вычеркнув строго доминируемые стратегии сведите данные игры к размеру $(2\times 2)$ \\
Подсказка: при вычеркивании можно использовать смешанные стратегии! \\
б) Найдите равновесия по Нэшу в смешанных стратегиях;\\
в) Найдите Парето-оптимальные точки в чистых стратегиях \\


\zad Придумайте биматричную игру размером $\left( {3 \times 3} \right)$, в которой с помощью вычеркивания нестрого доминируемых стратегий можно оставить ровно один исход, причем результат зависит от порядка вычеркивания.\\

\zad Два игрока одновременно называют натуральное число от 1 до 5. Первый игрок получает выигрыш, равный квадрату разности чисел. Второй игрок получает выигрыш равный наименьшему числу. \\
Найдите равновесия по Нэшу в чистых стратегиях и Парето-оптимальные точки.\\



\zad Три игрока одновременно называют одно из чисел: ноль или один. Если все трое называют единицу, то их выигрыш равен 10, если все трое называют ноль, то их выигрыш равен 5.\\
а) Найдите Парето-оптимальные точки в чистых стратегиях;\\
б) Найдите равновесия по Нэшу в чистых стратегиях \\
в) Найдите равновесия по Нэшу в смешанных стратегиях \\ \\

\zad Экология \\
Три фирмы, использующие воду из одного озера, одновременно решают, очищать ли им сточные воды, сбрасываемые в то же озеро. Очистка воды означает издержки равные единице. Если воду не очищают две или три фирмы, то каждая из трех фирм несет дополнительные издержки в размере трех единиц.\\
а) Найдите все равновесия по Нэшу в чистых стратегиях\\
б) Найдите все равновесия по Нэшу в смешанных стратегиях\\

\zad Рассмотрим игру $\begin{array}{|c|c|c|c|}
\hline
    {} &  {t_1 } & {t_2 } & {t_3 }   \\
\hline
    {l_1 } &  {\left( {2;x} \right)} & {\left( {1;4} \right)} & {\left( {9;20} \right)}   \\
    {l_2 } &  {\left( {7;y} \right)} & {\left( {6;8} \right)} & {\left( {2;4} \right)}   \\
\hline
\end{array}$\\
а) Пусть $y = 5$. При каких $x$  существует смешанная стратегия, строго доминирующая чистую стратегию $t_1 $? \\
б) Изобразите на плоскости множество таких пар $\left( {x,y} \right)$, при которых существует чистая стратегия, строго доминирующая стратегию $t_1 $. \\
в) Изобразите на плоскости множество таких пар $\left( {x,y} \right)$, при которых существует смешанная стратегия, строго доминирующая стратегию $t_1 $ \\


\zad Вопросы о разном... \\
а) Может ли в равновесие по Нэшу входить строго доминируемая стратегия? \\
б) Может ли в равновесие по Нэшу входить нестрого доминируемая стратегия? \\
в) Могут ли все исходы игры быть равновесными по Нэшу? \\
г) Может ли в произвольной игре не быть ни одного равновесия по Нэшу в чистых стратегиях? А в биматричной? \\
д) Может ли в произвольной игре не быть ни одного равновесия по Нэшу в смешанных стратегиях? А в биматричной? \\
е) Может ли существовать в произвольной игре стратегия, на которую не существует оптимального ответа? А в биматричной? \\
ё) Может ли игра в развернутой форме быть антагонистической?\\
ж) Сколько смешанных стратегий у первого игрока, если у первого игрока две чистых стратегии, а у второго – три чистых стратегии?\\



\zad
Задача. Limbo.

До весны 2007 года в Швеции существовала необычная лотерея <<Limbo>>. Правила выглядят следующим образом. Вы можете выбрать любое натуральное число. Победителем объявляется тот, кто назвал самое маленькое число, никем более не названное. Например, если игроки назвали числа 1, 3, 1, 2, 4, то победителем будет тот, кто назвал число 2. Если наименьшего никем более не названного нет, то приз остается у организаторов. \par
а) Опишите все равновесия по Нэшу в чистых стратегиях для $n$ игроков


\zad
Задача. Мост \par
Из пункта $A$ в пункт $D$ можно попасть двумя путями - через B или
через C. Если по дороге AB едет $x$ машин, то время в пути каждой
из них будет равно $f_{AB}(x)=x+32$. Для других отрезков пути
функции равны: $f_{BD}(x)=5x+1$, $f_{CD}(x)=x+32$ и
$f_{AC}(x)=5x+1$.
Каждое утро из города $A$ в город $D$ едет 6 машин. \par
\begin{figure}[ht]
    \includegraphics{game.5}
\end{figure} \par
а) Сколько машин и по какой дороге едет в равновесии по Нэшу?
Сколько им требуется времени, чтобы добраться из $A$ в $D$? \par
б) Как изменятся ответы, если между городами $B$ и $C$ построен
удобный мост, такой что $f_{BC}=0$?


\zad
Задача. В приемной комиссии три человека, включая председателя. Обсуждается, что поставить по теории игр тунеядцу Сидорову. Члены комиссии одновременно предлагают оценку. Ставится оценка, получившая наибольшее число голосов. Если ни одна из оценок не получила большинства голосов, то ставится ДВА. \par
Предпочтения членов комиссии выглядят следующим образом:
Председатель:  $2\succ 3\succ 4$ . Двоечник он, и точка! \par
Второй член комиссии:  $4\succ 2\succ 3$ . Да он знает на четверку! Если что, пересдаст на четыре!\par
Третий член комиссии:  $3\succ 4\succ 2$ . Нормальная тройка, ну чуть ближе к четверке.\par
Найдите все равновесия по Нэшу в этой игре;




\end{document}

\section{Практикум по NE}

\zad Дуополия Бертрана \\
Две фирмы назначают цены на свою продукцию. Предельные издержки обеих фирм равны нулю. Рыночный спрос описывается функцией $Q = \max \left\{ {1 - P,0} \right\}$. Весь спрос достается фирме, назначившей наименьшую цену; если фирмы назначили одинаковую цену, то спрос делится между ними поровну.\\
а) Изобразите на плоскости стратегии равновесные по Нэшу и Парето оптимальные точки;\\
б) Грандиозное предложение! Фирмы снова одновременно назначают цены, но каждая фирма обязуется вернуть покупателю разницу в цене товара, если конкурент продает дешевле. Как изменятся множества равновесных по Нэшу стратегий и Парето оптимальных точек?\\

\zad Старику на остановке плохо. Рядом находятся еще $n$  человек. Каждый из них, независимо от других, может либо вызвать скорую с помощью мобильного, либо ничего не делать. Если никто не вызовет скорую, то старик умрет. Если скорая будет вызвана, то старик будет спасен. Если старик умрет, то полезность каждого равна 0, если старик остается в живых, то полезность каждого равна 1. Издержки телефонного звонка равны 0.001.\\
а) Найдите все равновесия по Нэшу в чистых стратегиях;\\
б) Найдите симметричное равновесие по Нэшу в смешанных стратегиях;\\
в) Как зависит от  $n$  вероятность получения помощи?\\


\zad Выкуп доли\\
Доля Маши в ЗАО <<Красивое платье>> составляет  $s$, а доля Кати -  $\left(1-s\right)$. Маша и Катя решили отказаться от дальнейшего сотрудничества. У ЗАО должна быть одна владелица! 
Ценность ЗАО для каждой владелицы - случайная величина  $\nu _{i} $  равномерно распределенная на  $\left[0;1\right]$ .\\
а)	Маша и Катя решили попробовать аукцион. Оба игрока одновременно называют цену ЗАО. Тот, кто назвал более высокую цену, должен выкупить акции партнера по предложенной им самим цене (более высокой). Найдите равновесие по Нэшу в линейных стратегиях, т.е. предлагаемая каждым игроком цена должна иметь вид  $b_{i} =\alpha +\beta \nu _{i} $ , где  $v_{i} $  - его оценка стоимости ЗАО, а  $\alpha $  и  $\beta $  - константы. Не забудьте случай  $s=0$ , т.е. Маша продавец, а Катя - покупатель. Возможно ли, что акции достанутся тому, кто их меньше ценит?\\
б)	Решите вариант задачи: тот, кто назвал более высокую цену, должен выкупить акции партнера по предложенной партнером цене (более низкой). \\

\zad Что предложил Warren Buffet? \\
Парламент поделен на две партии: республиканцы и демократы. Для принятия реформы необходима ее поддержка обеими партиями. Реформа безразлична обеим партиям. Warren Buffet предложил, чтобы какой-нибудь миллиардер выступил со следующим обещанием: Если реформа не будет принята, то партия, поддержавшая реформу во время голосования получит 1000000000\$ (Один миллиард долларов). Партии голосуют одновременно (можно проголосовать только за или против реформы). Каждая партия хотела бы получить деньги и не хотела бы, чтобы деньги достались конкурентам. Допустим, что нашелся миллиардер, выступивший с таким заявлением, а партии поверили заявлению миллиардера.\\
а) Представьте игру в нормальной (матричной) форме; \\
б) Найдите равновесие по Нэшу\\
в) Чего добился миллиардер? \\
Тигр: сам Buffet с подобным заявлением не выступил!\\
Source: Game theory at work \\

\zad Конверты \\
В пяти конвертах спрятаны суммы в 10\$, 20\$, 40\$, 80\$ и 160\$. Случайным образом выбираются два конверта с соседними суммами и выдаются игрокам. {\it Тигр: А в конвертах - это взятки?} Каждый игрок открывает свой конверт и выбирает, хочет ли он оставить себе сумму или хочет обменяться. Обмен происходит, если оба игрока согласны обменяться.\\
а)	Сколько чистых стратегий у каждого игрока?\\
б)	Найдите все равновесия по Нэшу в чистых и смешанных стратегиях.\\
Source: exam Jannssen \\


\zad На аукционе продаются 12 стульев работы мастера Гамбса (одним лотом). Ценность стульев для каждого из $n$ покупателей - случайная величина, равномерная на $[0;1]$. Покупатели одновременно назначают цены. Победитель аукциона - тот, кто назвал самую высокую цену. Стулья достаются победителю. Победитель платит названную им цену. \\
а) Найдите симметричное равновесие по Нэшу в чистых стратегиях. Найдите ожидаемую выручку устроителей аукциона. \\
б) Изменим правила: победитель платит названную им цену плюс комиссионный сбор в 10\%. Найдите равновесие по Нэшу и ожидаемую выручку устроителей. \\
в) Изменим правила: победитель платит вторую по величине цену (т.е. если игроки назвали цены 100, 200, 300 и 400, то победитель - тот, кто сказал 400, но платит он 300). Найдите равновесие по Нэшу и ожидаемую выручку устроителей. \\

\zad Морской бой в узком проливе \\
Первый игрок располагает корабль размером $1\times 2$ на поле размером $1\times 4$. Второй игрок не зная выбор первого делает выстрел в одну из четырех клеток поля. Задача первого - спасти корабль, задача второго - потопить. \\
а) Постройте матрицу игры \\
в) Найдите все равновесия по Нэшу в смешанных стратегиях \\

\zad (*) Мусоросжигательный завод \\
В стране  $n$  городов. Около одного из них нужно построить большой мусоросжигательный завод. Предположим, что ущерб от мусоросжигательного завода для жителей каждого города - случайная величина, равномерно распределенная на отрезке  $\left[0;1\right]$ . Каждый город объявляет компенсацию, требуемую за постройку мусоросжигательного завода поблизости. Завод строят около города, запросившего наименьшую компенсацию. Деньги выплачивают остальные города в равной пропорции.\\
Найдите симметричное равновесие по Нэшу в чистых стратегиях;\\

\zad Мост \\
Из пункта $A$ в пункт $D$ можно попасть двумя путями - через B или
через C. Если по дороге AB едет $x$ машин, то время в пути каждой
из них будет равно $f_{AB}(x)=x+32$. Для других отрезков пути
функции равны: $f_{BD}(x)=5x+1$, $f_{CD}(x)=x+32$ и
$f_{AC}(x)=5x+1$.
Каждое утро из города $A$ в город $D$ едет 6 машин. \\
\begin{figure}[h]
    \includegraphics{game.5}
\end{figure} \\
а) Сколько машин и по какой дороге едет в равновесии по Нэшу?
Сколько им требуется времени, чтобы добраться из $A$ в $D$? \\
б) Как изменятся ответы, если между городами $B$ и $C$ построен
удобный мост, такой что $f_{BC}=0$? \\

\zad Два человека пришли в кабак. У одного из них 10 золотых, у второго
- 6 золотых. Каждый может тратить деньги на выпивку или на музыку.
Музыка является общественным благом - ее слышат все. Выпивка -
частным. Полезности равны $u_{1}=(m_{1}+m_{2})d_{1}$ и
$u_{2}=(m_{1}+m_{2})d_{2}$, где $m_{i}$ и $d_{i}$ - расходы $i$-го
человека на музыку и выпивку. Предположим, что деньги бесконечно
делимы. \\
а) Найдите равновесие по Нэшу \\
б) Что изменится в случае, если у второго 2 золотых? \\

\zad Президент \\
Два гражданина борются за пост президента страны. Каждый из них выбирает свою политическую позицию. Под позицией мы будем понимать натуральное число от 1 до 99, где число один означает крайне левую позицию, а 99 – крайне правую. Если оба гражданина занимают одну политическую позицию, то голоса делятся поровну. Если позиции различны, то каждый житель (в стране 99 жителей) выбирает того кандидата, к которому он ближе расположен. Если жителю все равно, то его голос делится поровну между кандидатами.\\
а) Найдите все исходы, которые остаются в результате последовательного вычеркивания нестрого доминируемых стратегий.\\
б) Найдите равновесие по Нэшу\\


\zad Всеобщность знания\\
$\begin{array}{|c|c|c|}
\hline
    {} &  s & c   \\
\hline
    s &  {\left( {1;1} \right)} & {\left( { - 2; - 1} \right)}   \\
    c &  {\left( { - 1; - 2} \right)} & {\left( { - 1; - 1} \right)}   \\
\hline
\end{array}$\\
В уездном городе $N$ живут игроки двух типов: <<безумцы>> и рациональные. При встрече в городе $N$ принято играть в игру, изображенную слева. Рациональные игроки играют стратегию, приносящую наибольший выигрыш, а безумцы – стратегию $c$.\\
Как-то случилось Петя попасть в этот город и встретится с одним рациональным аборигеном. Они никогда раньше не виделись и никогда больше не увидятся. Петя знает, что абориген - рационален. Абориген знает, что Петя - рационален. Петя ошибочно полагает, что абориген считает его безумцем. Абориген знает о Петиной ошибке.\\
Найдите все равновесия по Нэшу в этой матрице. Какое из них будет сыграно?\\


\zad О пользе гадалок замолвите слово... \\
Маша пишет на бумажках два любых различных натуральных числа по своему выбору. Одну бумажку она прячет в левую руку, а другую - в правую. Саша выбирает любую Машину руку. Маша показывает число, написанное на выбранной бумажке. Саша высказывает свою догадку о том, открыл ли он большее из двух чисел или меньшее. Если Саша не угадал, то Маша выиграла.\\
а) Докажите, что у Саши есть смешанная стратегия, гарантирующая ему вероятность выигрыша строго более 50\% вне зависимости от стратегии Маши (!) \\
Подсказка: представьте себе, что перед выбором руки и высказыванием догадки Саша может обратиться к потомственной гадалке в пятом поколении Глафире Лукитичне Пуассоновой (500\% гарантия, снятие порчи и сглаза без греха и ущерба для здоровья, исправляет некачественную работу шарлатанов). Глафира Лукитична (ничего не зная о Маше!) называет наугад число $X=N+0.5$, где $N$ - Пуассоновская случайная величина с $\lambda=1$.  \\
Source: Winkler, games people don't play \\


\section{SPNE (Subgame Perfect Nash Equilibrium), обратная индукция продолжение}

$[$IGT$]$ - главы 5,6,7; $[$Slantchev$]$ - глава Perfect Equilibria in Extensive Form Games (кроме части 2) \\

Подыгра – часть игры, начинающаяся с определенного узла (конкретный узел и все узлы <<ниже>> него). При <<отрезании>> подыгры от всей игры не должны <<разрываться>> информационные множества\\
Равновесие по Нэшу, совершенное в подыграх (SPNE) – профиль стратегий, оказывающийся равновесием по Нэшу в каждой подыгре включая всю игру. В играх с совершенной информацией совпадает с методом обратной индукции. \\

Задачи 1.1.	1.2.	1.3. \\
\begin{figure}[!htbp]
    \includegraphics{game.17}
    \includegraphics{game.22}
    \includegraphics{game.16}
\end{figure} \\
а) Посчитайте число подыгр;\\
б) Найдите равновесия по Нэшу, совершенные в подыграх \\

\zad 
Первый игрок называет неотрицательное число $x$. Затем второй игрок называет неотрицательное число $y$. Выигрыши игроков определяются по формулам: $u^I \left( {x,y} \right) =  - x^2  + 4xy - y^2  + 11$ и $u^{II} \left( {x,y} \right) =  - 2y^2  + xy - 4x^2  + 12 + x + 2y$.\\
а) Что представляют собой стратегии игроков?\\
б) Найдите равновесие по Нэшу, совершенное в подыграх (примените метод обратной индукции)\\
в) Приведите пример равновесия по Нэшу, не совершенного в подыграх\\
г) Как изменится ответ, если второй игрок выбирает $y$ одновременно с первым?\\
е) Решите пункты б-г для случая $u^{II} \left( {x,y} \right) =  - y^2  + 6xy - 4x^2  + 12 + x - 2y$\\
з) Решите пункты б-г для случая $u^{II} \left( {x,y} \right) =  - y^2  + 8xy - 4x^2  + 12 + x + 2y$\\ 

\zad Четыреста лет назад\\
В 1612 г. в Лионе появилась книга поэта и математика Баше де Мезирьяка (Bachet de Meziriac) <<Занимательные и приятные числовые задачи>> (Problemes plaisants et delectables qui se font par les nombres). В ней была предложена следующая игра. Двое по очереди называют числа от 1 до 10, числа складываются. Выигрывает тот, кто первым доведет сумму до 100. Кто выигрывает при правильной игре?\\

\zad 
В кучке 121 камень. Двое игроков ходят по очереди. Первый игрок за один ход может взять 1 или 3 камня, а второй – 1 или 4 камня. Проигрывает тот, кто не может сделать ход по правилам (выигрывает взявший последний камень). Кто выигрывает при правильной игре (если никто не делает ошибок)?\\

\zad <<Набери чет>>\\
В кучке 135 камней, двое игроков по очереди забирают себе от 1 до 4 камней. Выигрывает тот, кто к концу игры наберет четное число камней. Кто выигрывает при правильной игре?\\
Подсказка: обратная индукция на «двух линиях», как в предыдущей задаче.\\

\zad 
Первоначально фишка расположена в узле А. Двое игроков по очереди могут двигать фишку на один ход в любом направлении, которое указано стрелочкой. Тот, кто не сможет сделать очередной ход, проиграл. Кто выигрывает при правильной игре?\\
\begin{figure}[!htbp]
    \includegraphics{game.27} 
\end{figure}\\

\zad <<Одинокий ферзь>>\\
Шахматная доска, одинокий раненый ферзь стоит на h6 (вверху справа). Раненый ферзь может двигаться или влево, или вниз, или влево-вниз (на любое число клеток). Двое игроков ходят по очереди, тот кто переставит ферзя на а1 выиграл. Кто выигрывает при правильной игре?\\
Подсказка: попробуйте обратную индукцию на шахматной доске\\

\zad 
В первой кучке – 6 камней, во второй – 7 камней. Вася и Петя ходят по очереди. Петя ходит первым.  Игрок делающий ход может выбрать одну из кучек и из нее взять один, два или три камня.\\
а) Кто выигрывает, Вася или Петя, если цель игры – взять самый последний камень?\\
б) Кто выигрывает, Вася или Петя, если проигрывает тот, кто взял самый последний камень?\\
Подсказка: обратная индукция на плоскости.\\

\zad 
Определите, истинно или ложно каждое утверждение:\\
а) Равновесие, совершенное в подыграх, всегда является равновесием по Нэшу.\\
б) Обратно-индукционный исход всегда является равновесием, совершенным в подыграх.\\
в) В конечной последовательной игре с полной совершенной информацией обязательно существует равновесие, совершенное подыграх.\\
г) Если перевести последовательную игру в матричную форму и вычеркнуть нестрого доминируемую стратегию, то можно потерять равновесие, совершенное в подыграх.\\

\zad 
Игра трех игроков представлена на рисунке \\
а) Найдите все равновесия по Нэшу в чистых стратегиях;\\
Подсказка: нарисуйте две матрицы; первый игрок выбирает матрицу, второй – строку, третий – столбец.\\
б) Найдите все равновесия в чистых стратегиях, совершенные в подыграх;\\
\begin{figure}[!htbp]
    \includegraphics{game.21}
\end{figure}\\


\zad Судья и потерпевший\\
Ущерб – случайная величина $v$, равновероятно принимающая любое значение из $\left\{ {0;1;...;99} \right\}$. Потерпевший точно знает величину ущерба $v$, а судья знает лишь распределение. Потерпевший выбирает один из двух вариантов: честно задекларировать величину ущерба или не говорить ничего. Судья выбирает величину компенсации $R$.\\
Полезность потерпевшего $U_1  = R - v$. Полезность судьи $U_2  =  - \left( {v - R} \right)^2$.\\
Найдите равновесия по Нэшу, совершенные в подыграх.\\

\zad Развод \\
Джон и Бэтти нужно поделить виллу, яхту, акции (Тигр: Контрольный пакет акций свечного заводика в Самаре) и автомобиль. Они условились на следующей процедуре дележа: каждый забирает себе один предмет по очереди.\\
Предпочтения выглядят так (в порядке убывания ценности):\\
Бэтти: яхта, вилла, акции, автомобиль. Джон: вилла, акции, автомобиль, яхта\\
а) Найдите равновесие, совершенное в подыграх, если очередность выбор предметов Бэтти-Джон-Бэтти-Джон;\\
б) Во время дележа имущества Джон и Бэтти помирились, и решили культурно отдохнуть. \\
Предпочтения (снова в порядке убывания ценности).\\
Бэтти: балет, казино, бассейн, бокс. Джон: бокс, бассейн, казино, балет.\\
Они по очереди вычеркивают нежелательную альтернативу, до тех пор, пока не останется только одна. Найдите равновесия по Нэшу, совершенные в подыграх, для того же варианта очередности.\\
Source: Брамс \\

\zad 
Неправильная монетка выпадает <<орлом>> с вероятностью $0<p<1$. Первый игрок знает результат выпадения монетки, второй – нет. Первый игрок объявляет второму, как выпала монетка (при этом он может соврать). Затем второй делает свою догадку о том, как в действительности выпала монетка. За свою правдивость первый игрок получает единицу полезности и еще две единицы получает в том случае, если второй скажет «орел». Второй игрок получает единицу полезности, если верно угадает, как выпала монетка. Найдите все равновесия по Нэшу, совершенные в подыграх.\\

\zad Две сверхдержавы\\
Две сверхдержавы играют в одновременную игру и выбирают, нажимать ли красную кнопку запуска стратегических ядерных ракет, или нет. Эффективных систем защиты от ядерного удара нет. Матрица игры имеет вид:\\
$\begin{array}{|c|c|c|}
\hline
    {} &  {press} & {not}   \\
\hline
    {press} &  {\left( { - a; - a} \right)} & {\left( {1; - a} \right)}   \\
    {not} &  {\left( { - a;1} \right)} & {\left( {0;0} \right)}   \\
\hline
\end{array}$
а) Найдите все равновесия по Нэшу в этой игре;\\
б) Допустим, обе сверхдержавы создали системы раннего обнаружения запуска ядерных ракет. Если одна из сверхдержав не запустила ракеты, а другая – запустила, у незапустившей появляется возможность изменить свою решение. Для простоты будем считать, что если ни одна из держав не нажала красную кнопку, то игра сразу оканчивается. Отменить запуск по-прежнему нельзя. \\
б1) Изобразите дерево игры; \\
б2) Найдите равновесия по Нэшу совершенные в подыграх в измененной игре; \\
Тигр: они доиграются... \\


\section{Повторяемые игры, SPNE} 

$[$IGT$]$ - главы 14,15; $[$Slantchev$]$ - глава Repeated games \\

\zad Рассмотрим повторяемую игру с дисконт-фактором $\delta $
 и матрицей базовой игры вида\\
$\begin{array}{|c|c|c|}
\hline
    {} &  c & d   \\
\hline
    c &  {2;6} & { - 1;3}   \\
    d &  {4;2} & {1;5}   \\
\hline
\end{array}$

Первый игрок использует следующую стратегию: \\
В нечетной партии сделать ход $c$. В четной партии скопировать ход противника в предыдущей партии\\
Второй игрок использует следующую стратегию:\\
В 1-ой партии сделать ход $d$. Во 2-ой партии сделать ход $c$. В $n$-ой партии ($n \ge 3$) скопировать ход противника в $\left( {n - 2} \right)$-ой партии. \\
а) Найдите дисконтированные платежи игроков в игре в целом.\\
б) Найдите дисконтированные платежи игроков в подыгре, начинающейся после $\left\{ {\left( {dd} \right),\left( {dd} \right)} \right\}$.\\
в) Верно ли, что указанные две стратегии являются равновесием по Нэшу? Равновесием по Нэшу, совершенным в подыграх?\\


\zad Задана бесконечно повторяемая игра с дисконт-фактором $\delta$ и матрицей \\
$\begin{array}{|c|c|c|}
\hline 
 & c & d \\
\hline
c & 5;4 & 0;7 \\
d & 6;0 & 2;3 \\
\hline
\end{array}$ \\
а) Определите, при каких $\delta$ будут равновесными по Нэщу профили стратегий:\\
А1. (всегда <<c>>, всегда <<c>>) \\ 
А2. (всегда <<d>>, всегда <<d>>) \\
А3. (стратегия переключения, стратегия переключения) \\
А4. (наивная стратегия переключения, наивная стратегия переключения) \\
А5. (зуб за зуб, зуб за зуб) \\
А6. (стратегия кнута и пряника, стратегия кнута и пряника) \\
А7. (стратегия переключения наоборот, стратегия переключения наоборот) \\
А8. (стратегия переключения наоборот, зуб за зуб) \\
б) Определите, при каких $\delta$ эти профили будут равновесиями по Нэщу, совершенными в подыграх \\


\zad Придумайте 5 различных пар стратегий, являющимися равновесными по Нэшу в бесконечноповторяемой игре с дисконт-фактором $\delta=0,999$ и матрицей \\
$\begin{array}{|c|c|c|}
\hline 
 & c & d \\
\hline
c & 4;5 & 0;6 \\
d & 6;0 & 2;3 \\
\hline
\end{array}$ \\


\zad Рассмотрим повторяемую игру $G$ с дисконт-фактором $\delta $ и матрицей базовой игры вида\\
$\begin{array}{|c|c|c|}
\hline
    {} &  c & d   \\
\hline
    a &  {6;0} & {1;2}   \\
    b &  {4;3} & {0;4}   \\
\hline
\end{array}$\\
а) Найдите в матрице партии равновесие по Нэшу и исход, Парето-доминирующий это равновесие.\\
б) Самостоятельно сформулируйте стратегии переключения, так, чтобы в фазе наказания игралось равновесие по Нэшу из матрицы, а в фазе кооперации – исход, доминирующий его по Парето. \\
в) При каких значениях дисконт фактора пара стратегий переключения будет равновесием по Нэшу, совершенным в подыграх?\\


\zad Рассмотрим бесконечноповторяемую игру с дисконт фактором $\delta$\\
$\begin{array}{|c|c|c|c|}
\hline
    {} &  A & B & C   \\
\hline
    A &  {\left( {3;3} \right)} & {\left( {3;5} \right)} & {\left( {0;0} \right)}   \\
    B &  {\left( {5;3} \right)} & {\left( {2;2} \right)} & {\left( {0;0} \right)}   \\
    C &  {\left( {0;0} \right)} & {\left( {0;0} \right)} & {\left( {1;1} \right)}   \\
\hline
\end{array}$\\
\textbf{Стратегия $BA - C$}. Играть ход $a$ в четных партиях и ход $b$ в нечетных до тех пор,
пока исходом игры является $\left( {a;b} \right)$ или $\left( {b;a} \right)$. Если произойдет исход, отличный от $\left( {a;b} \right)$ или $\left( {b;a} \right)$ всегда далее играть ход $c$.\\
\textbf{Стратегия $AB - C$}. Играть ход $a$ в нечетных партиях и ход $b$ в четных до тех пор,
пока исходом игры является $\left( {a;b} \right)$ или $\left( {b;a} \right)$. Если произойдет исход, отличный от $\left( {a;b} \right)$ или $\left( {b;a} \right)$ всегда далее играть ход $c$.\\
а) При каких значениях дисконт фактора пара стратегий $\left( {BA - C;AB - C} \right)$ является равновесием по Нэшу?\\
б) Совершенным подыгровым равновесием по Нэшу?\\


\zad Матрица базовой игры имеет вид:\\ $\begin{array}{|c|c|c|}
\hline
    {} &  {t_1 } & {t_2 }   \\
\hline
    {l_1 } &  {\left( {4;1} \right)} & {\left( {4;0} \right)}   \\
    {l_2 } &  {\left( {6;5} \right)} & {\left( {3;1} \right)}   \\
\hline
\end{array}$.\\
а) Изобразите графически, какие платежи достижимы в повторяемой игре;\\
б) Являются ли достижимыми в повторяемой игре платежи $\left( {4;2} \right)$, $\left( {3,5;0,5} \right)$, $\left( {5;2} \right)$ и $\left( {5;3} \right)$?  \\
в) Какие из указанных платежей могут быть реализованы на множестве смешанных стратегий в отдельной базовой игре?\\


\zad 
Базовая игра повторяется два раза без дисконтирования. Может ли платеж $\left( {4;4} \right)$
 быть получен в первой партии в совершенном подыгровом равновесии по Нэшу? Если да, то укажите соответствующий профиль стратегий.\\
$\begin{array}{|c|c|c|c|}
\hline
    {} &  a & b & c   \\
\hline
    a &  {\left( {3;1} \right)} & {\left( {0;0} \right)} & {\left( {5;0} \right)}   \\
    b &  {\left( {2;1} \right)} & {\left( {1;2} \right)} & {\left( {3;1} \right)}   \\
    c &  {\left( {1;2} \right)} & {\left( {0;1} \right)} & {\left( {4;4} \right)}   \\
\hline
\end{array}$\\
Source: London School of Economics exam 1996 \\

\zad Решите задачи 439.1; 454.3; 459.1; 459.2; 459.3 из $[$IGT$]$ \\

\section{Несовершенная информация, Байесовские игры}

$[$IGT$]$ - глава 9; $[$Slantchev$]$ - глава Static and Dynamic Games of Incomplete Information (часть 1) \\


\zad Саша (первый игрок) не совсем уверен, предпочитает ли Маша его компанию, или склонна избегать его. С точки зрения Саши:\\
с $p=\frac{2}{3}$ 
 игра имеет вид: $\begin{array}{|c|c|c|}
\hline
    {} &  F & T   \\
\hline
    F &  {\left( {2;1} \right)} & {\left( {0;0} \right)}   \\
    T &  {\left( {0;0} \right)} & {\left( {1;2} \right)}   \\
\hline
\end{array}$;
с $p=\frac{1}{3}$
 игра имеет вид: $\begin{array}{|c|c|c|}
\hline
    {} &  F & T   \\
\hline
    F &  {\left( {2;0} \right)} & {\left( {0;2} \right)}   \\
    T &  {\left( {0;1} \right)} & {\left( {1;1} \right)}   \\
\hline
\end{array}$\\
Маша в отличие от Саши точно знает, в какую игру она играет. \\
а) Укажите количество типов каждого игрока; сформулируйте чистые стратегии Саши и Маши;\\
б) Найдите равновесие по Нэшу в чистых стратегиях и (*) в смешанных \\

% 
%$\begin{array}{|c|c|c|c|c|}
%\hline
%& TT & TF & FT & FF \\
%\hline
%F & (0;\frac{2}{3}) & (\frac{2}{3};0) & (\frac{4}{3};\frac{4}{3}) & (2;\frac{2}{3}) \\
%T & (1;\frac{5}{3}) & (\frac{2}{3};\frac{5}{3}) & (\frac{1}{3};\frac{1}{3}) & %(0;\frac{1}{3}) \\
%\hline
%\end{array}$, NE: (T,TT), (F,FT), (T,TF) \\

\zad Значение $\theta _1 $ известно первому игроку, а значение $\theta _2 $ - второму.\\
$\begin{array}{|c|c|c|}
\hline
    {} &  F & T   \\
\hline
    F &  {\left( {3;2 + \theta _2 } \right)} & {\left( {2;1} \right)}   \\
    T &  {\left( {1;0} \right)} & {\left( {4 + \theta _1 ;1} \right)}   \\
\hline
\end{array}$
, общеизвестно, что $\theta _1  \sim U\left[ {0;2} \right]$, $\theta _2  \sim U\left[ {1;2} \right]$.\\
а) Найдите равновесие по Нэшу (если это возможно), в котором все четыре исхода играются с положительной вероятностью, а игроки используют стратегии вида: <<если известное мне $\theta$ выше определенного порога, то я сделаю один ход, если ниже, то другой>> \\
б) Найдите равновесие по Нэшу другого вида \\
в) Рассмотрите следующие вариации задачи:\\
В1. Значение $\theta _1 $ будет известно второму игроку, а $\theta _2 $ - первому\\
В2. Второй игрок будет ошибаться и считать, что $\theta _1  \sim U\left[ {1;2} \right]$, а на самом деле $\theta _1  \sim U\left[ {0;2} \right]$.\\
В3. Платеж второго игрока в ситуации $\left( {T;F} \right)$ будет равен 2, а не 0.\\
В4. Матрица платежей имеет вид:\\
$\begin{array}{|c|c|c|}
\hline
    {} &  F & T   \\
\hline
    F &  {\left( {0;2} \right)} & {\left( {2;3 + \theta _2 } \right)}   \\
    T &  {\left( {1 + \theta _1 ;4} \right)} & {\left( {1;1} \right)}   \\
\hline
\end{array}$ \\ \\

% a) $x_{1}=\frac{4}{3}(\sqrt{15}-3)$, $x_{2}=\frac{1}{2}(\sqrt{15}-1)$ \\
% b) например (при любом $\theta_{1}$ ходить $F$;при любом $\theta_{2}$ ходить $F$) \\

\zad Два игрока одновременно выбирают действительные числа $x_1 $ и $x_2 $, соответственно. Платежные функции могут иметь один из двух видов:\\
$\left( \begin{array}{l}
 u_1  \\ 
 u_2  \\ 
 \end{array} \right) = \left( \begin{array}{l}
  - x_1^2  + x_2 x_1  \\ 
  - x_2^2  + x_1 x_2  + 4x_2  \\ 
 \end{array} \right)$
, с  $p=\frac{7}{{10}}$ или 
$\left( \begin{array}{l}
 u_1  \\ 
 u_2  \\ 
 \end{array} \right) = \left( \begin{array}{l}
  - x_1^2  + 4x_2 x_1  \\ 
  - x_2^2  \\ 
 \end{array} \right)$
, с  $p=\frac{3}{{10}}$\\
Первый игрок точно знает, какой вид имеют платежные функции, второй знает только закон распределения.\\
а) Найдите равновесие по Нэшу в чистых стратегиях;\\
б) (*) Предположим, что первый игрок до того, как второй игрок сделает свой ход, может послать ему один из двух сигналов <<А>> или <<Б>> (не обязательно достоверный!). Найдите равновесие по Нэшу в таком варианте игры. \\


\zad Два игрока одновременно выбирают действительные числа $x_1$ и $x_2$, соответственно.\\
$\left( \begin{array}{l}
 u_1  \\ 
 u_2  \\ 
 \end{array} \right) = \left( \begin{array}{l}
  - x_1^2  + 2x_2 x_1  + \theta _1 x_1  \\ 
  - x_2^2  + 4x_1 x_2  + 2\theta _2 x_2  \\ 
 \end{array} \right)$
, где $\theta _1  \sim U\left[ {0;2} \right]$
, $\theta _2  \sim U\left[ {1;2} \right]$\\

Значение $\theta _1 $ известно первому игроку, а значение $\theta _2 $ - второму.\\
a) Найдите равновесие по Нэшу в чистых стратегиях; \\
b) Является ли оно совершенным в подыграх? \\

% a) $x_{1}=-2,5+\theta_{1}/2$, $x_{2}=-4+\theta_{2}$ \\
% b) да, т.к. нет подыгр \\

\zad Матрица игры имеет один из двух видов (вероятность каждого вида равна 0,5).\\ 
$\begin{array}{|c|c|c|c|}
\hline
    {} &  k & l & m   \\
\hline
    a &  {\left( 1;2/3 \right)} & {\left( {1;0} \right)} & {\left( {1;1} \right)}   \\
    b &  {\left( {2;2} \right)} & {\left( {0;0} \right)} & {\left( {0;3} \right)}   \\
\hline
\end{array}$
 или $\begin{array}{|c|c|c|c|}
\hline
    {} &  k & l & m   \\
\hline
    a &  (1;2/3) & {\left( {1;1} \right)} & {\left( {1;0} \right)}   \\
    b &  {\left( {2;2} \right)} & {\left( {0;3} \right)} & {\left( {0;0} \right)}   \\
\hline
\end{array}$\\
Первый игрок знает матрицу игры, второй – нет\\
а) Найдите равновесие по Нэшу и платеж второго игрока;\\
б) Допустим, что второй игрок также знает матрицу игры. Найдите равновесие по Нэшу;\\
в) Почему не срабатывает рассуждение <<второй игрок может отказаться от лишней информации,  поэтому его платеж не может упасть>>?\\


\zad Два партнера инвестируют $x_1 $ и $x_2 $ в совместное предприятие. Значение случайной величины  $\theta _1 $ известно первому игроку, а значение $\theta _2 $ - второму. Оба игрока знают, что $\theta _1  \sim U\left[ {0;1} \right]$, $\theta _2  \sim U\left[ {0;1} \right]$.\\
Полезности игроков имеют вид $U_1  = \theta _1 x_1 x_2  - x_1^3 $ и $U_2  = \theta _2 x_1 x_2  - x_2^3 $.\\
Найдите равновесие по Нэшу \\

%, в котором стратегии игроков имеют вид $x_i \left( {\theta _i } \right) = a_i  + b_i \sqrt {\theta _i } $, где $a_i $ и $b_i $ - некоторые константы. \\

\zad Василий, покажите публике <<Правосудие>>\\
В пассаже на Петровке на аукцион выставлена <<Фигура, изображающая правосудие>> (бронзовая, в полном порядке). \textit{Тигр: Я в полный рост с весами в лапе.} Ценность фигуры для каждого из двоих покупателей $v_i$ – случайная величина, распределенная равномерно на отрезке $\left[ {0;1} \right]$. Игроки одновременно подают заявки с указанием цены покупки $b_i$. Фигура достается тому, кто указал наибольшую цену. Если игроки указали одинаковую цену $b$, то их платежи равны $\frac{1}{2}\left( {v_i  - b} \right)$.\\
а) Укажите количество типов каждого игрока;\\
Пусть первый игрок использует линейную стратегию $b_1 \left( {v_1 } \right) = kv_1  + l$\\
б) Найдите ожидаемый выигрыш второго игрока, при условии, что ценность «Правосудия» для него равна $v_2 $, а указал он цену $b_2 $\\
в) Найдите равновесие по Нэшу, в котором оба игрока используют одинаковую линейную стратегию. Укажите среднюю выручку продавца в этом равновесии.\\


\zad Простой покер\\
Юля и Петя делают ставку по 1\$. Далее они по очереди тянут из шляпы бумажки с числами. В шляпе лежат натуральные числа от 1 до $N$. Затем Юля и Петя одновременно выбирают, доложить ли еще по 5\$, или сдаться. Если оба игрока сдаются, то оба теряют свою первоначальную ставку в пользу казино; если один сдался, а другой увеличил ставку, то увеличивший забирает себе все, что находится на кону; если оба игрока увеличили ставку, то победителем считается тот, у кого число больше. \\
а) Найдите все равновесия по Нэшу в чистых стратегиях для $N = 3$\\
б) (*) Найдите все равновесия по Нэшу в смешанных стратегиях для $N = 3$\\
в) (*) Решите задачу при произвольном $N$ \\
%Решение: \\
%в) Оптимальная стратегия должна иметь вид: если вижу число меньше $n$, то сдаться, если вижу число больше или равно $n$, то повысить ставку. Почему, кстати? \\
%Находим ожидаемый выигрыш игрока если он видит число $k$ и удваивает ставку. \\
%При $k<n$ повышение должно быть невыгодно. \\
%При $k\ge n$ повышение должно быть выгодно. \\
%Получаем двойное неравенство: \\
%$n\in \left[\frac{5N+2}{7};\frac{5N+9}{7}\right]$ \\




\zad Простой покер - вариация \\
Предположим, что числа, которые узнают Юля и Петя независимы и равномерно распределены на отрезке $[0;1]$. Остальные правила - без изменений. \\
а) Найдите все равновесия по Нэшу в чистых стратегиях\\
б) (*) Найдите все равновесия по Нэшу в смешанных стратегиях\\


\zad Компенсация ущерба \\
Простая модель арбитража: потерпевший называет свою оценку ущерба $h$, адвокаты ответчика одновременно предлагают свою оценку $l$. Затем арбитр выбирает тот вариант, который ему кажется более справедливым, ближе к некоторому идеальному $x$. Арбитр знает $x$, стороны - не знают.\\ Найдите равновесие по Нэшу, если $x \sim U\left[ {0;1} \right]$;\\

\zad На рынке корову старик продавал... \\
Покупатель и продавец одновременно называют цены $p_b $ и $p_s $. Если $p_s  \le p_b $, то обмен происходит по цене $\frac{{p_b  + p_s }}{2}$; если нет, то обмена не происходит. Ценность коровы для покупателя и продавца – независимые случайные величины $v_b $ и $v_s $, распределенные равномерно на $\left[ {0;1} \right]$. \\
а) Пусть игроки используют линейные стратегии \\
а1) Найдите равновесие по Нэшу; \\
а2) В осях $(v_{b},v_{s})$ нарисуйте ситуации, при которых обмен происходит; \\
а3) С какой вероятностью происходит обмен? \\
а4) Всегда ли происходит обмен в случае $v_s  < v_b $? \\
б) Пусть покупатель использует стратегию $p_b=\left\{\begin{array}{l}q, v_b\ge q \\ 0, v_b<q \end{array}\right.$, а продавец $p_s=\left\{\begin{array}{l}1, v_s\ge q \\ q, v_s<q \end{array}\right.$. \\
Ответьте на аналогичные вопросы. \\


\section{Динамические игры с несовершенной информацией} 

$[$IGT$]$ - глава 10; $[$Slantchev$]$ - 
глава Perfect Equilibria in Extensive Form Games (часть 2) \\

WSE – weak sequential equilibrium, слабое секвенциальное равновесие\\
WSE состоит из профиля стратегий (profile of strategies) и вер (beliefs) игроков. Каждый игрок, попав в информационное множество должен иметь свое мнение о вероятностях нахождения в каждом из узлов информационного множества.\\

\zad 
\begin{figure}[!htbp]
    \includegraphics{game.28}
    \includegraphics{game.14}
    \includegraphics{game.26}
\end{figure} \\
\begin{figure}[!htbp]
    \includegraphics{game.29}
    \includegraphics{game.31}
    \includegraphics{game.21}
\end{figure} \\
\begin{figure}[!htbp]
    \includegraphics{game.32}
\end{figure} \\
Найдите для каждой игры NE, SPNE, WSE и (*) SE \\

\zad Простейший покер \\
Маша и Саша положили на кон по одному доллар. Маша берет из колоды одну карту. Известно, что выигрышная для Маши карта придет с вероятностью $p$. Маша может либо сразу открыть карту, либо удвоить ставку. Если ставка удвоена, то Саша может либо отказаться от удвоения ставки (и проиграть один доллар), либо поддержать удвоение ставки. Затем карта открывается.\\
а) Найдите слабое секвенциальное равновесие в зависимости от $p$ (в чистых и смешанных стратегиях) \\
в) Постройте график зависимости цены игры от $p$\\
д) При каком $p$ игра справедлива, если ставка увеличивается не вдвое, а в $n$  раз?\\
з) Попробуйте поиграть со своей девушкой (молодым человеком)!\\

\zad Настоящие ковбои заказывают бифштекс!\\
У первого игрока два типа: настоящий ковбой () и сладкоежка ($W$
), которые природа выбирает с вероятностями 0,9 и 0,1 соответственно. Оба игрока знают эти вероятности, но только первый игрок видит ход Природы. Первый игрок делает заказ у стойки бара в тот момент, когда в бар входит второй игрок.\\
Второй игрок - довольно буйный тип. \textit{Тигр: тип не в смысле теории игр, а просто тип.} Он заходит в бар, чтобы подраться, однако он трус и не хотел бы встретиться с настоящим ковбоем. У второго игрока есть выбор: завязывать драку (F) или не завязывать (N). Перед своим ходом второй игрок видит выбор первого игрока. 
Первый игрок может заказать бифштекс (S) или пудинг (P). Настоящие ковбои предпочитают бифштекс, а сладкоежки - пудинг.\\
\begin{figure}[htbp]
    \includegraphics{game.30}
\end{figure} \\
Найдите слабое секвенциальное равновесие.\\


\zad Это война!\\
Две страны, Левая и Правая, хотят поделить территорию в виде отрезка $\left[ {0;1} \right]$. Столицы стран находятся на концах отрезка. Природа выбирает издержки ведения войны:  - для Левой и $c_r $ - для Правой, величины $c_l $ и $c_r $ независимы и равномерно распределены на отрезке $\left[ {0;1} \right]$. Издержки $c_l $ известны только Левой стране. Издержки $c_r $ известны обеим странам. Правая страна выдвигает свои требования к границе $x \in \left[ {0;1} \right]$. Левая страна либо удовлетворяет требования Правой, либо развязывает войну. Правая страна побеждает в войне с экзогенной вероятностью $p \in \left( {0;1} \right)$. Стране-победительнице достается вся территория.\\
а) Найдите равновесия по Нэшу в чистых стратегиях; слабые секвенциальные равновесия.\\
б) Найдите вероятность войны и средние выигрыши игроков в равновесии.\\
в) Предположим, что издержки ведения войны были бы общеизвестны. Какова была бы вероятность войны и средние выигрыши стран в равновесии?\\
г) Ответьте на пункты <<а>> и <<б>>, если требования выдвигает Левая страна.\\

\zad Выкуп доли-2\\
Доля Маши в ЗАО «Красивое платье» составляет $s$, а доля Кати - $\left( {1 - s} \right)$. Маша и Катя решили отказаться от дальнейшего сотрудничества. У ЗАО должна быть одна владелица! Сначала Маша называет цену $p$ - сколько по ее мнению стоит все ЗАО. Затем Катя выбирает выкупить ли за $sp$ Машину долю или продать Маше свою долю за $\left( {1 - s} \right)p$. 
Ценность ЗАО для каждой владелицы – случайная величина равномерная на $\left[ {0;1} \right]$.\\
Найдите слабое секвенциальное равновесие;\\


\section{Дополнительные задачи}

\zad Заработок короля\\
В стране  $n$  жителей, каждый из которых получает заработную плату в одну монету. Когда в стране победила демократия, король потерял свою власть, даже был лишен права голоса. Единственное, что он может - так это предлагать перераспределение заработной платы. Зарплата каждого жителя должна выражаться неотрицательным количеством монет, в сумме все зарплаты должны равняться  $n$ . Когда король предлагает перераспределение зарплаты, каждый житель, кроме самого короля, может проголосовать за, против или вообще не приходить на голосование. Новое распределение одобряется, если число голосов <<за>> строго больше числа голосов <<против>>.\\
Каждый житель эгоистичен, голосует <<за>>, если в новом проекте его зарплата растет, <<против>>, если падает, и не приходит на голосование, если ему предлагается одинаковая зарплата.\\
Какую зарплату в результате получит хитрый король и сколько голосований ему потребуется?\\


\zad (*) Покер в чате\\
Трое заядлых игроков в покер сидят в чате. Предложите процедуру раздачи карт, при которой каждый игрок знает свои карты и не знает карт соперника. Игроки абсолютно рациональны и обладают безграничными вычислительными возможностями, поэтому использование кодов с открытым ключом (типа RSA) недопустимо. В чате можно посылать сообщения, адресованные как всем сразу, так и конкретному лицу.\\


\zad Игры с моделью Кидленда-Прескотта \\
Правительство пообещало, что инфляция будет равна нулю в следующем году! После сенсационного объявления народ определяет свои ожидания инфляции $\pi ^e $. Зная ожидания народа, правительство выбирает фактический уровень инфляции $\pi $ в будущем году. Функция полезности правительства $U =  - \left( {\pi  - \pi ^p } \right)^2  - \left( {y - y^p } \right)^2 $
, где $y^p $ - желаемый правительством уровень выпуска, а $\pi ^p  > 0$ - желаемый правительством уровень инфляции. \\
Совокупное предложение задано функцией $y = y^{*}  + b\left( {\pi  - \pi ^e } \right)$, где $y^{*}$ - потенциальный уровень выпуска. Константа $b$ - положительная.\\
Найдите равновесие по Нэшу и платежи, получаемые игроками в следующих ситуациях:\\
а) Население верит правительству, а правительство не обязано исполнять свои обязательства.\\
б) Население верит правительству, а правительство исполняет свои обязательства.\\
в) Функция полезности населения имеет вид $U =  - \left( {\pi  - \pi ^e } \right)^2 $, а правительство исполняет свои обязательства. Тигр: С такой функцией полезности население догадается!\\
д) Функция полезности населения имеет вид $U =  - \left( {\pi  - \pi ^e } \right)^2 $, а правительство не обязано исполнять свои обязательства.\\
ж) Какой должна быть величина штрафа за неисполнение обязательств, чтобы в пункте г) правительство было заинтересовано в их исполнении?\\
з) Сравните выигрыш правительства в случаях <<в>> и <<г>>. Обратите внимание на то, что в случае <<в>> у правительства нет возможности нарушать обещание, а в случае <<г>> такая возможность появляется.\\


\zad Передача информации в модели Курно.\\
Две фирмы по очереди выбирают объемы производства. Вторая фирма при выборе своего объема производства не знает объема, выбранного первой фирмой. После того, как обе фирмы произвели товар, рынок определяет цены в соответствии с функцией спроса $P\left( Q \right) = 1 - Q$
, где $Q = q_1  + q_2 $.\\
а) Что представляет собой стратегия второй фирмы? Найдите равновесие по Нэшу. Совпадает ли оно с равновесием по Нэшу совершенным в подыграх?\\
б) Предположим теперь, что первая фирма честно декларирует выбранный ею объем выпуска. Что представляет собой стратегия второй фирмы? Найдите равновесие по Нэшу, совершенное в подыграх. Будут ли другие равновесия по Нэшу?\\
в) Сравните платежи второй фирмы в обоих вариантах модели. Почему наличие дополнительной информации снижает платеж второй фирмы? Почему не корректно рассуждение <<Полученную информацию можно не учитывать, поэтому платеж не может упасть>>?\\


\zad (*) Пираты делят золото \\
Есть золотой песок и три пирата, но нет весов.\\
Предпочтения пиратов субъективны (действительно равные кучи могут казаться пирату неравными). Но предпочтения стабильны: если пират считал две кучи равными,  и видит, что к первой досыпали песок, то он будет считать первую кучу большей. Каждый пират может делить кучу песка на равные кучи, сравнивать несколько куч, отсыпать из большей кучи песок так, чтобы она сравнялась с меньшей.\\
Предложите конечную процедуру справедливого дележа, при которой у пиратов не было бы зависти (каждый считал бы, что его часть не меньше, чем у других).\\

\zad Футбол\\
Четыре команды играют в футбол. Сначала проводится полуфинал (первая команда играет со второй, третья - с четвертой), затем две команды победительницы участвуют в финале. Выигравшая чемпионат команда получает приз равный  $S$ . Перед началом чемпионата каждая команда выбирает затраты на тренировки  $a$  (неотрицательное число). Если встречаются команды с затратами  $a$  и  $b$ , то вероятность выигрыша первой равна  $\frac{a}{a+b} $ .\\
{\it Тигр: А может быть это фирмы дают взятки чиновникам?}\\
а) Найдите симметричное  равновесие  по Нэшу (т.е. равновесие в котором все команды выбирают одинаковый уровень усилий);\\
б) Пусть команд будет восемь, а игра проходит по схеме четвертьфинал-полуфинал-финал. Докажите, что симметричного равновесия по Нэшу не существует.\\
{\it Тигр: А если вторая производная  окажется равной нулю?}\\
{\it  Автор: Тсс!}\\
в) Докажите, что симметричное равновесие по Нэшу отсутствует, если число туров превышает два.\\
г) (*) Докажите, что если число туров превышает два, то в любом равновесии по Нэшу будет команда с нулевым уровнем тренировки.\\
{\it Тигр: Среди кандидатов в президенты есть те, кто не надеется победить, среди студентов есть те, кто не готовится к экзамену... Все это звенья одной цепи!}\\


\zad (*) Русская рулетка \\
Два гусара, одна дама, одна пуля в шестизарядном револьвере...\\
Сначала первый гусар выбирает, отправиться ли ему в кабак (тогда он получит полезность ноль, а дама достанется другому) или приставить револьвер к виску и нажать на курок. Смерть означает полезность равную минус единице (при этом дама, естественно, достается другому). Если гусар остается жив, то право и обязанность выбора кабак/стреляться переходит ко второму. Барабан револьвера при этом не перекручивается, т.е. вероятность смерти увеличивается. Ценность дамы для одного гусара  $a$ , для другого -  $b$ .\\
а) Найдите равновесия в зависимости от  $a$  и  $b$ .\\
б) Каким игроком лучше быть, первым или вторым, в зависимости от   $a$  и  $b$ .\\
{\it Тигр: Настоящего джентльмена, конечно, интересует настоящая дама, } $a\gg 1$, $b\gg 1$ \\

\zad Limbo \\
До весны 2007 года в Швеции существовала необычная лотерея <<Limbo>>. Правила выглядят следующим образом. Вы можете выбрать любое натуральное число. Победителем объявляется тот, кто назвал самое маленькое число, никем более не названное. Например, если игроки назвали числа 1, 3, 1, 2, 4, то победителем будет тот, кто назвал число 2. \\
а) Опишите все равновесия по Нэшу в чистых стратегиях для $n$ игроков \\
б) Найдите симметрическое равновесие для трех игроков (т.е. равновесие, в котором все игроки используют одинаковые стратегии) \\
в) Почему была закрыта эта лотерея? \\
Подсказка для <<б>>: какие есть известные законы распределения на $\mathbb{N}$? \\

\zad (**) Continuous Colonel Blotto \\
Two generals have the same amount of units (continuous) to compete over three battlefields in a campaign. In each battlefield, the general who assigns more units wins. The general winning 2 battlefields wins the entire campaign. They simultaneously announce their decision of unit distribution over the three battlefields. Find NE \\
Hint1: is it possible that $U_{1}+U_{2}+U_{3}=1$ and $U_{i}$ - uniform? \\
Hint2: сумма высот проведенных из точки $N$ к трем сторонам треугольника не зависит от выбора точки $N$ \\

\zad (*) Петя выбирает место где спрятаться внутри круга единичного радиуса. Одновременно Вася выбирает место, где будет искать Петю. Вася замечает Петю, если расстояние между ними не превосходит половины радиуса. Если Вася нашел Петю, то Петя платит Васе рубль. Если Вася не нашел Петю, то Вася - Пете. \\
Найдите цену игры. \\


\newpage 
\textbf{Варианты контрольных - 2007.} \\

\textbf{Контрольная номер 1} \\
Задача 1. \\
В игре участвуют три игрока. Сначала первый игрок выбирает $x\in R$; затем второй игрок, зная $x$, выбирает $y\in R$; затем третий игрок, зная $x$ и $y$, выбирает $z\in \left[1;2\right]$. Функции выигрыша имеют вид: \\
$U_{1} =xyz+\frac{9x}{z} -zx^{2}$ \\
$U_{2} =\left(z-2\right)y^{2} -4xy+\left(5+z\right)y+5x$\\
$U_{3} =2x^{3} -y^{4} -3z-z^{2}$  \\
Какие $x$, $y$ и $z$ будут реализованы игроками при использовании ими метода обратной индукции? \\
Ответ: $x=2$, $y=-1$ и $z=1$ \\
Задача 2. \\
Найдите равновесие в смешанных стратегиях и цену игры в матричной игре с противоположными интересами: \\
$
\begin{array}{|c|c|c|c|}
\hline
1 & 12 & -10 & -8 \\
\hline
-9 & -10 & 2 & -4 \\
\hline
\end{array}
$ \\
Ответ: $p=5/14$, $v=-38/7$, $q_{1}=2/7$, $q_{4}=5/7$ \\
Задача 3 \\
Найдите равновесные по Нэшу исходы (в чистых и смешанных стратегиях) и Парето оптимальные исходы (в чистых стратегиях) следующих биматричных игр:\\
a) 
$\begin{array}{|c|ccc|} 
\hline
& d & e & f\\
\hline
a & (1;2) & (4;4) & (-4;3) \\
b & (4;3) & (1;0) & (2;1) \\
c & (2;3) & (6;6) & (3;7) \\
\hline
\end{array}$ 
б) 
$\begin{array}{|c|ccc|} 
\hline
& d & e & f\\
\hline
a & (3;-1) & (1;5) & (4;1) \\
b & (2;1) & (-1;7) & (5;0) \\
c & (1;5) & (1;-1) & (5;1) \\
\hline
\end{array}$ \\
Ответ: \\
а) NE: $(b,d)$, $(c,f)$, $(\frac{2}{3}b+\frac{1}{3}c,\frac{1}{3}d+\frac{2}{3}f)$ \\
PO: $(6;6)=(c,e)$, $(3;7)=(c,f)$ \\
б) NE $(ta+(1-t)c,e)$, где $t\in[\frac{1}{2},1]$ \\
PO: $(5;1)=(c,f)$, $(1;5)=(c,d)$, $(1;5)=(a,e)$, $(-1;7)=(b,e)$ \\
В этом пункте вычеркивание начинается с $\frac{1}{2}d+\frac{1}{2}e\succ f$ \\

Задача 4. \\
На рисунке представлена последовательная игра двух игроков: I и II. \\
\begin{figure}[htbp]
    \includegraphics{game.38}
\end{figure}\\

а) Представьте игру в нормальной (матричной) форме \\
б) Найдите все равновесия по Нэшу (в чистых стратегиях)	\\
в) Найдите все равновесия по Нэшу, совершенные в подыграх (в чистых стратегиях) \\
Ответ: \\
б) NE: $(ad,cf)$, $(ad,cg)$, $(ae,nf)$, $(ae,ng)$, $(bd,nf)$, $(be,nf)$ \\
в) SPNE: $(ad,cf)$, $(ae,nf)$, $(be,nf)$ \\
Задача 5.	На столе лежат $n$ фишек. Два игрока (Саша и Маша) по очереди убирают некоторое количество фишек. Саша может убрать либо 1, либо 2 фишки. Маша может убрать либо 3, либо 4 фишки. Проигрывает тот, кто не сможет сделать очередной ход. Кто победит в следующих четырех случаях, если игроки при выборе стратегий используют метод обратной индукции? \\
a) Первым ходит Саша, $n=132$  \\
б) Первым ходит Саша, $n=175$  \\
в) Первой ходит Маша, $n=132$  \\
г) Первой ходит Маша, $n=175$  \\
Ответ: \\
а) Саша \\
б) Маша \\
в) Саша \\
г) Саша \\


\textbf{Контрольная номер 2} \\
Задача 1. \\
Саша и Маша поссорились и предпочитают развлекаться отдельно друг от друга. Матрица игры имеет вид: \\
$\begin{array}{|c|cc|}
\hline
& $Футбол$ & $Балет$ \\
\hline
$Футбол$ & (0;0) & (2+t_{1};2) \\
$Балет$ & (1;3+t_{2}) & (0;0)\\
\hline
\end{array}$\\
В матрице Саша - первый игрок, Маша - вторая. \\
Величина $t_1$ известна Саше, но неизвестна Маше. Величина $t_2$ известна Маше, но неизвестна Саше. Обоим известно, что $t_1$ и $t_2$ - случайные величины, равномерно распределенные на промежутках $[0;12]$ и $[0;7]$ соответственно. \\
Найдите разделяющее Байесовское равновесие $(s, m)$, для которого: \\
Саша выбирает Футбол, если $t_{1}>s$, и Балет в противном случае, \\
а Маша выбирает Футбол, если $t_{2}>m$, и Балет в противном случае. \\
Ответ: $(4;1)$, $7-3m=sm$, $24-5s=sm$ \\
Задача 2. \\
Задана бесконечно повторяемая игра $G(\infty;\delta)$ \\
$\begin{array}{|c|cc|}
\hline
& t_{2} & t_{2} \\
\hline
s_{1} & (2;1) & (6;-2) \\
s_{2} & (0;5) & (4;2) \\
\hline
\end{array}$
a) Сформулируйте стратегии переключения, при которых игроки будут играть $(s_2;t_2)$ во всех партиях \\
б) При каких значениях $\delta$ эти стратегии составляют совершенное подыгровое равновесие по Нэшу? \\
Ответ: \\
а) Для первого игрока: \\
В первой партии сделать ход $s_{2}$ \\
В $n$-ой партии ($n\ge2$) сделать ход $s_{2}$, если во всех предыдущих партиях был сыгран исход $(s_{2};t_{2})$; иначе сделать ход $s_{1}$ \\
б) $\delta\in [0.75;1)$ \\

Задача 3. \\
Два игрока одновременно выбирают действительные числа $x_{1}$ и $x_{2}$  соответственно. Платежные функции игроков могут иметь один из двух видов: \\
$\left\{
\begin{array}{c}
U_{1}=-x_{1}^{2}-x_{1}x_{2}+2x_{1} \\
U_{2}=-x_{2}^{2}+2x_{1}x_{2}-x_{2}
\end{array}\right.$ с вероятностью 0,8;\\
$\left\{
\begin{array}{c}
U_{1}=-x_{1}^{2}+2x_{1}x_{2}+3x_{1} \\
U_{2}=-x_{2}^{2}+x_{1}x_{2}+x_{2}
\end{array}\right.$ с вероятностью 0,2;\\
Первый игрок точно знает, какой вид имеют платежные функции. Оба игрока знают законы распределения. Найти равновесие по Нэшу в чистых стратегиях.\\
Ответ: $x_{1a}=\frac{3}{4}$, $x_{1b}=2$, $x_{2}=\frac{1}{2}$ \\
Задача 4. \\
Найдите слабые секвенциальные равновесия (в чистых стратегиях) в игре: \\
\begin{figure}[htbp]
    \includegraphics{game.39}
\end{figure}\\
Ответ: $\left\{(lL,uD); p=0.2; q\in [0;1]\right\}$, 
$\left\{(rR,dD);p\ge \frac{1}{4} ;q=0.2\right\}$\\
Задача 5. \\
Найдите слабые секвенциальные равновесия (в чистых стратегиях) в игре: \\
\begin{figure}[htbp]
    \includegraphics{game.40}
\end{figure}\\
Ответ: $\left\{(a,d); \mu=\frac{3}{5}; \nu=0\right\}$, 
$\left\{(b,c); \mu=\frac{3}{5}; \nu=\frac{2}{7}\right\}$\\




\newpage
\textbf{Названия некоторых стратегий в повторяющейся дилемме заключенного }\\

$\begin{array}{|c|c|c|}
\hline 
 & c & d \\
\hline
c & 5;5 & 0;6 \\
d & 6;0 & 3;3 \\
\hline
\end{array}$ \\
Стратегия $d$ строго доминирует стратегию $c$, однако (c;c) - одна из Парето оптимальных точек. \\


{\bf Всегда <<c>>} \\
В любой партии делать ход c независимо от предыстории. \\
{\bf Всегда <<d>>} \\
В любой партии делать ход d независимо от предыстории. \\
{\bf Цикл <<ccd>>} \\
В первой партии сделать ход c, во второй - c, в третьей d, далее снова играется c, c, d... \\
{\bf Стратегия переключения} (grim trigger)\\
В первой партии cделать ход c. В $n$-ой партии сделать ход $c$, если во всех предыдущих партиях был исход $(c;c)$. В $n$-ой партии cделать ход $d$, если хотя бы в одной предыдущей партии не был сыгран исход $(c;c)$. \\
{\bf Наивная стратегия переключения} (naive grim trigger)\\
В первой партии cделать ход c. В $n$-ой партии сделать ход $c$, если во всех предыдущих партиях противник делал ход $c$. В $n$-ой партии сделать ход $d$, если хотя бы в одной предыдущей партии противник сделал ход $d$. \\
{\bf Стратегия <<Зуб за зуб>>} (Tit for Tat)\\
В первой партии сделать ход $c$. В $n$-ой партии повторить ход противника в предыдущей партии. \\
{\bf Стратегия Кнута и Пряника} (Win-Stay, Lose-Shift; Pavlov strategy)\\
В первой партии сделать ход $c$. В $n$-ой партии сделать ход  $c$, если в предыдущей партии действия игроков совпали. В $n$-ой партии сделать ход  $d$, если в предыдущей партии действия игроков не совпали. \\
{\bf Стратегия переключения наоборот}\\
В первой партии cделать ход d. В $n$-ой партии сделать ход $d$, если во всех предыдущих партиях был исход $(d;d)$. В $n$-ой партии сделать ход $c$, если хотя бы в одной предыдущей партии не был сыгран исход $(d;d)$. \\
{\bf Стратегия ограниченного возмездия} (limited retaliation)\\
Играть ход  $c$ в первой партии и далее до тех пор, пока исходом игры является $(c;c)$. Если произошло исход, отличный от $(c;c)$, то в течение  $k$  ходов подряд делать ход $d$, затем вернуться в исходное состояние. \\ 








\newpage

Основные источник, которые я старался не замутить: \\
1. $[$IGT$]$ - Osborne, Introduction to game theory \\
учебник с кучей примеров и задач \\
2. Захаров, \\

Другие рекомендуемые источники: \\
2. Game theory at work \\
Почти без математики; показывает взгляд на многие жизненные ситуации с точки теории игр\\
3. Binmore, Fun and Games или Binmore, Playing for real \\
Увлекательная книга бакалаврского уровня \\
4. $[$Slantchev$]$ - Slantchev, Lecture notes \\
Курс лекций, задачи, решения, очень похож на вышкинский \\
Будьте осторожны! у Сланчева нестрогое определение PBE \\
Его PBE занимает промежуточное положение между WSE и SE \\
5. Gintis, Game theory evolving \\
Сборник задач, решая которые можно довольно глубоко освоить теорию игр. \\

Для тех, кто понял, что теория игр - это круто: \\
6. Данилов, \\
Хороший курс по теории игр для математиков на русском \\
7. Osborne, A. Rubinstein, A course in game theory, The MIT Press, 1994 \\
Следующий уровень после $[$IGT$]$ \\
8. Squintani, Notes for Non-Cooperative Game Theory \\
Лекции магистерского уровня. \\

Где достать? \\
Slanchev, Данилов - свободно распространяются в интернете \\
Злостные нарушители копирайта скачивают, например, на www.avaxhome.ws \\

Binmore, Gintis - есть в библиотеке вышки в hardcore \\


Статьи, понятные второкурснику: \\
Chess are dominance solvable in at most two steps \\
Escalation and dollar auction \\
Recent advances in combinatorial game \\
General Blotto \\
Brams, Taylor, Dividing cake \\
Dividing chore \\


Прочий интернет, имеющий отношение к теории игр: \\
www.gametheory.net \\
www.xion.ru (учеба, 2 курс, теория игр) \\


\newpage

\section{Ответы}

1.4. а) нет б) да (рисунок) \\
1.5. одну \\

1.8. а) верно б) неверно в) неверно \\
1.9. 7:1 \\

2.1. а) $t_{2}$ \\
2.5. возможная матрица: $\begin{array}{|c|c|c|}
\hline
& + & - \\
\hline
+ & (0;0) & (1;-1) \\
\hline
- & (-1;1) & (0;0) \\
\hline
\end{array}$, NE - $(+;+)$ \\
Мораль: можно добится нужного исхода голосования пообещав деньги, но не выплачивая ни копейки. \\
2.7. $u_{1}(x_{1},x_{2})=\frac{3}{4}(x_{1}+x_{2})-x_{1}$ \\
$u_{2}(x_{1},x_{2})=\frac{3}{4}(x_{1}+x_{2})-x_{2}$ \\
NE - $(x_{1}=0,x_{2}=0)$ \\
2.8. а) верно б) верно в) Если стратегию можно вычеркнуть, то сделать это никогда не поздно \\
2.9. нет \\
2.10. а) называемая им цена б) верно \\

3.1.1. Одна подыгра (не считая игры в целом), NE: $(LX,M)$, $(LY,M)$, $(RX,N)$. SPNE: $(LY,M)$, $(RX,N)$ \\
3.1.2. Одна подыгра (не считая игры в целом), NE: $(LX,N)$, $(LY,N)$, $(RX,M)$. SPNE: $(RX,M)$ \\
3.1.3. Одна подыгра (не считая игры в целом), NE: $(RX,B)$, SPNE: $(RX,B)$ \\
3.6. A, C, D, E, H = +; B, F, G, I = -\\
3.9. а) верно б) верно в) верно г) неверно \\
3.10. NE: $(In,R,r)$, $(Out,L,l)$, $(Out,L,r)$, $(Out,R,l)$, SPNE: $(In,R,r)$ \\
4.1. \\
a) $BR(l_{2})=t_{2}$ \\
б) $t_{3}$ \\
в) $qt_{2}+(1-q)t_{3}$, где $q\in(1/2;6/7)$ \\
г) $(l_{1},t_{3})$, $(l_{2},t_{2})$, $(\frac{1}{8}l_{1}+\frac{7}{8}l_{2},\frac{7}{12}t_{2}+\frac{5}{12}t_{3})$ \\
4.2. a) да; б) $\infty$ \\
4.3. б) все пары $(x_{1},x_{2})$, такие, что $x_{1}+x_{2}=1$, $x_{1}\in [0;1]$ \\
4.7. в) блицкриг - отрицательно,завтрак - положительно  \\
4.5. в) все исходы Парето-оптимальны \\
5.1. в) не NE, не SPNE (второму игроку выгодно отклонится) \\
5.2. \\
A1: NE - $\emptyset$, SPNE - $\emptyset$ \\
A2: NE - $(0;1)$, SPNE - $(0;1)$ \\
A3: NE - $[\frac{3}{4};1)$, SPNE - $[\frac{3}{4};1)$ ($\frac{1}{4}$ - граница для первого игрока) \\
A4: NE - $[\frac{3}{4};1)$, SPNE - $\emptyset$ \\
5.4. в) $[0.5;1)$ \\
6.1. а) у Маши 2 типа, у Саши 1 тип \\
b) $\begin{array}{|c|c|c|c|c|}
\hline
& TT & TF & FT & FF \\
\hline
F & (0;\frac{2}{3}) & (\frac{2}{3};0) & (\frac{4}{3};\frac{4}{3}) & (2;\frac{2}{3}) \\
T & (1;\frac{5}{3}) & (\frac{2}{3};\frac{5}{3}) & (\frac{1}{3};\frac{1}{3}) & (0;\frac{1}{3}) \\
\hline
\end{array}$, NE: (T,TT), (F,FT), (T,TF) \\
6.2.  a) $x_{1}=\frac{4}{3}(\sqrt{15}-3)$, $x_{2}=\frac{1}{2}(\sqrt{15}-1)$ \\
b) например (при любом $\theta_{1}$ ходить $F$;при любом $\theta_{2}$ ходить $F$) \\
6.4. a) $x_{1}=-2,5+\theta_{1}/2$, $x_{2}=-4+\theta_{2}$  b) да, т.к. нет подыгр \\
6.5. в) первый игрок знает о том, что второй знает \\

6.8. в) Оптимальная стратегия должна иметь вид: если вижу число меньше $n$, то сдаться, если вижу число больше или равно $n$, то повысить ставку. Почему, кстати? \\
Находим ожидаемый выигрыш игрока если он видит число $k$ и удваивает ставку. \\
При $k<n$ повышение должно быть невыгодно. \\
При $k\ge n$ повышение должно быть выгодно. \\
Получаем двойное неравенство: \\
$n\in \left[\frac{5N+2}{7};\frac{5N+9}{7}\right]$ \\



\end{document}