\documentclass[pdftex,12pt,a4paper]{article}

% jan 2012

% sudo yum install texlive-bbm texlive-bbm-macros texlive-asymptote texlive-cm-super texlive-cyrillic texlive-pgfplots texlive-subfigure
% yum install texlive-chessboard texlive-skaknew % for \usepackage{chessboard}
% yum install texlive-minted texlive-navigator texlive-yax texlive-texapi

% растягиваем границы страницы
%\emergencystretch=2em \voffset=-2cm \hoffset=-1cm
%\unitlength=0.6mm \textwidth=17cm \textheight=25cm

\usepackage{makeidx} % для создания предметных указателей
\usepackage{verbatim} % для многострочных комментариев
\usepackage{cmap} % для поиска русских слов в pdf
\usepackage[pdftex]{graphicx} % для вставки графики 
% omit pdftex option if not using pdflatex


%\usepackage{dsfont} % шрифт для единички с двойной палочкой (для индикатора события)
\usepackage{bbm} % шрифт - двойные буквы

\usepackage[colorlinks,hyperindex,unicode,breaklinks]{hyperref} % гиперссылки в pdf


\usepackage[utf8]{inputenc} % выбор кодировки файла
\usepackage[T2A]{fontenc} % кодировка шрифта
\usepackage[russian]{babel} % выбор языка

\usepackage{amssymb}
\usepackage{amsmath}
\usepackage{amsthm}
\usepackage{epsfig}
\usepackage{bm}
\usepackage{color}

\usepackage{multicol}


\usepackage{textcomp}  % Чтобы в формулах можно было русские буквы писать через \text{}

\usepackage{embedfile} % Чтобы код LaTeXа включился как приложение в PDF-файл

\usepackage{subfigure} % для создания нескольких рисунков внутри одного

\usepackage{tikz,pgfplots} % язык для рисования графики из latex'a
\usetikzlibrary{trees} % прибамбас в нем для рисовки деревьев
\usetikzlibrary{arrows} % прибамбас в нем для рисовки стрелочек подлиннее
\usepackage{tikz-qtree} % прибамбас в нем для рисовки деревьев


\usepackage{ifpdf} % чтобы проверять, запускаем мы pdflatex или просто latex

\ifpdf
	\usepackage[pdftex]{graphicx} 
	\DeclareGraphicsRule{*}{mps}{*}{} % все неупомянутые ps файлы объявляем упрощенными, т.е. mps типа. Просто ps графику нельзя использовать, но без некоторых спец. команд - можно. Например, результат работы metapost - это ps файлы простого (mps) типа. Собственно ради использования metapost эта строка и введена.
\else
	\usepackage{graphicx}
\fi



% конец добавки

\usepackage{asymptote} % After graphicx!, пакет для рисования графиков и прочего
%\usepackage{sagetex} % i suppose after graphicx also..., для связи с sage



\embedfile[desc={Исходный LaTeX файл}]{\jobname.tex} % Включение кода в выходной файл
\embedfile[desc={Стилевой файл}]{/home/boris/science/tex_general/title_bor_utf8.tex}



% вместо горизонтальной делаем косую черточку в нестрогих неравенствах
\renewcommand{\le}{\leqslant}
\renewcommand{\ge}{\geqslant} 
\renewcommand{\leq}{\leqslant}
\renewcommand{\geq}{\geqslant}

% делаем короче интервал в списках 
\setlength{\itemsep}{0pt} 
\setlength{\parskip}{0pt} 
\setlength{\parsep}{0pt}

% свешиваем пунктуацию (т.е. знаки пунктуации могут вылезать за правую границу текста, при этом текст выглядит ровнее)
\usepackage{microtype}

% более красивые таблицы
\usepackage{booktabs}
% заповеди из докупентации: 
% 1. Не используйте вертикальные линни
% 2. Не используйте двойные линии
% 3. Единицы измерения - в шапку таблицы
% 4. Не сокращайте .1 вместо 0.1
% 5. Повторяющееся значение повторяйте, а не говорите "то же"


% DEFS
\def \mbf{\mathbf}
\def \msf{\mathsf}
\def \mbb{\mathbb}
\def \tbf{\textbf}
\def \tsf{\textsf}
\def \ttt{\texttt}
\def \tbb{\textbb}

\def \wh{\widehat}
\def \wt{\widetilde}
\def \ni{\noindent}
\def \ol{\overline}
\def \cd{\cdot}
\def \bl{\bigl}
\def \br{\bigr}
\def \Bl{\Bigl}
\def \Br{\Bigr}
\def \fr{\frac}
\def \bs{\backslash}
\def \lims{\limits}
\def \arg{{\operatorname{arg}}}
\def \dist{{\operatorname{dist}}}
\def \VC{{\operatorname{VCdim}}}
\def \card{{\operatorname{card}}}
\def \sgn{{\operatorname{sign}\,}}
\def \sign{{\operatorname{sign}\,}}
\def \xfs{(x_1,\ldots,x_{n-1})}
\def \Tr{{\operatorname{\mbf{Tr}}}}
\DeclareMathOperator*{\argmin}{arg\,min}
\DeclareMathOperator*{\argmax}{arg\,max}
\DeclareMathOperator*{\amn}{arg\,min}
\DeclareMathOperator*{\amx}{arg\,max}
\def \cov{{\operatorname{Cov}}}

\def \xfs{(x_1,\ldots,x_{n-1})}
\def \ti{\tilde}
\def \wti{\widetilde}


\def \mL{\mathcal{L}}
\def \mW{\mathcal{W}}
\def \mH{\mathcal{H}}
\def \mC{\mathcal{C}}
\def \mE{\mathcal{E}}
\def \mN{\mathcal{N}}
\def \mA{\mathcal{A}}
\def \mB{\mathcal{B}}
\def \mU{\mathcal{U}}
\def \mV{\mathcal{V}}
\def \mF{\mathcal{F}}

\def \R{\mbb R}
\def \N{\mbb N}
\def \Z{\mbb Z}
\def \P{\mbb{P}}
%\def \p{\mbb{P}}
\def \E{\mbb{E}}
\def \D{\msf{D}}
\def \I{\mbf{I}}

\def \a{\alpha}
\def \b{\beta}
\def \t{\tau}
\def \dt{\delta}
\def \e{\varepsilon}
\def \ga{\gamma}
\def \kp{\varkappa}
\def \la{\lambda}
\def \sg{\sigma}
\def \sgm{\sigma}
\def \tt{\theta}
\def \ve{\varepsilon}
\def \Dt{\Delta}
\def \La{\Lambda}
\def \Sgm{\Sigma}
\def \Sg{\Sigma}
\def \Tt{\Theta}
\def \Om{\Omega}
\def \om{\omega}


\def \ni{\noindent}
\def \lq{\glqq}
\def \rq{\grqq}
\def \lbr{\linebreak}
\def \vsi{\vspace{0.1cm}}
\def \vsii{\vspace{0.2cm}}
\def \vsiii{\vspace{0.3cm}}
\def \vsiv{\vspace{0.4cm}}
\def \vsv{\vspace{0.5cm}}
\def \vsvi{\vspace{0.6cm}}
\def \vsvii{\vspace{0.7cm}}
\def \vsviii{\vspace{0.8cm}}
\def \vsix{\vspace{0.9cm}}
\def \VSI{\vspace{1cm}}
\def \VSII{\vspace{2cm}}
\def \VSIII{\vspace{3cm}}


\newcommand{\grad}{\mathrm{grad}}
\newcommand{\dx}[1]{\,\mathrm{d}#1} % для интеграла: маленький отступ и прямая d
\newcommand{\ind}[1]{\mathbbm{1}_{\{#1\}}} % Индикатор события
%\renewcommand{\to}{\rightarrow}
\newcommand{\eqdef}{\mathrel{\stackrel{\rm def}=}}
\newcommand{\iid}{\mathrel{\stackrel{\rm i.\,i.\,d.}\sim}}
\newcommand{\const}{\mathrm{const}}

%на всякий случай пока есть
%теоремы без нумерации и имени
%\newtheorem*{theor}{Теорема}

%"Определения","Замечания"
%и "Гипотезы" не нумеруются
%\newtheorem*{defin}{Определение}
%\newtheorem*{rem}{Замечание}
%\newtheorem*{conj}{Гипотеза}

%"Теоремы" и "Леммы" нумеруются
%по главам и согласованно м/у собой
%\newtheorem{theorem}{Теорема}
%\newtheorem{lemma}[theorem]{Лемма}

% Утверждения нумеруются по главам
% независимо от Лемм и Теорем
%\newtheorem{prop}{Утверждение}
%\newtheorem{cor}{Следствие}



\begin{document}
\parindent=0 pt

По поводу неразберихи с PBE. (ver 23.01.06)

Мы будем рассматривать конечные игры с полной несовершенной
информацией. \\

Равновесие по Нэшу (Nash Equilibrium, NE) \\

Равновесие по Нэшу, совершенное в подыграх (Subgame Perfect Nash
Equilibrium, SPNE) \\

Слабое совершенное Байесовское равновесие (Weak Perfect Bayesian
Equilibrium, WPВE).  \\
Cистема (профиль стратегий, веры) $(\beta, \mu)$ называется WPBE,
если выполнены
требования: \\
(Sequential rationality, Последовательная рациональность): \\
Обозначим $U_{i}(\beta,\mu|I)$ - ожидаемый выигрыш $i$-го игрока,
при использовании игроками профиля стратегий $\beta$, вер $\mu$ и
при условии достижения информационного множества $I$. \\
Для любого игрока $i$ и для любого его информационного множества $I$: \\
$U_{i}((\beta_{i},\beta_{-i}),\mu|I)\geq
U_{i}((\beta_{i}^{'},\beta_{-i}),\mu|I)$

При фиксированных верах и стратегиях других игроков, ни один игрок
не может увеличить свой условный ожидаемый выигрыш.




(Weak consistency, Слабая состоятельность):  \\
Если $x \in I$ и $P(I)>0$ то $\mu(x)=\frac{P(x)}{P(I)}$ \\
Если вероятность попасть в множество $I$ строго положительна, то
веры в $I$ определяются по формуле условной вероятности. \\


Альтернативные названия: \\

Assessment Equilibrium - Binmore, FG \\
Perfect Bayesian Equilibrium - Squintani, Lecture notes \\
Perfect Bayesian Equilibrium - Larry Blume, Lecture notes \\
Совершенное Байесовское равновесие - Коковкин, НГУ \\
Weak Sequential Equilibrium - Osborne, IGT \\
Слабое секвенциальное равновесие - Данилов, РЭШ \\
WPBE - MWG \\

Секвенциальное равновесие (Sequential Equilibrium, SE). \\

Cистема (профиль стратегий, веры) $(\beta, \mu)$ называется SE,
если выполнены
требования: \\
(Sequential rationality):

(Consistency, состоятельность): Существует последовательность
систем (профиль стратегий, веры), которая сходится к
$(\beta,\mu)$, $\beta_{n}$ является полностью смешанным профилем
стратегий, и $\mu_{n}$ слабо
согласовано с $\beta_{n}$. \\



Это общепринятое название. \\
Сильное секвениальное равновесие - Данилов \\

Формулу условной вероятности часто называют формулой Байеса. \\

SPNE может не являться WPBE. Т.е. профиль стратегий $\beta$ может
быть совершенным в подыграх, однако ни при каких верах $\mu$,
система $(\beta,\mu)$ не будет являться WPBE. \\
Пример: \\

\begin{figure}[h]
    \includegraphics{game.3}
\end{figure}
Упражнение. Убедитесь в том, что профиль $(Out,r)$ является NE, и
следовательно, SPNE, т.к. в игре нет собственных подыгр. Однако ни
при каких верах $\mu$ система $\{(Out,r),\mu\}$ не будет WPBE. \\

WPBE может не являться SPNE. Т.е. система из профиля $\beta$ и вер
$\mu$ может быть WPBE, но сам профиль $\beta$ может не быть
совершенным в подыграх. Пример после двух определений PBE. \\

Верны следующие утверждения: \\
$SE\subseteq WPBE \subseteq NE$ \\
$SE\subseteq SPNE \subseteq NE$ \\
В конечной игре с несовершенной информацией существует SE в
смешанных стратегиях. \\


Существует также понятие Совершенное Байесовское равновесие
(Perfect Bayesian Equilibrium, PBE). Будьте осторожны, под этим
понятием разные авторы подразумевают разные вещи! \\

Есть три основные группы лиц:

1. Osborne-Rubinstein, Fudenberg-Tirole, под PBE подразумевают
равновесие в динамических играх с неполной информацией! Этот класс
игр сводится к частному случаю игр с несовершенной информацией. \\

Т.е. собственно PBE оказывается не определенным для произвольной
игры с несовершенной информацией!!! PBE определен для иного класса
игр, которые в принципе можно свести к играм с несовершенной
информацией. \\

2. Squintani и ряд других авторов, под PBE подразумевает просто WPBE. \\

3. Gibbons, Slantchev и ряд других авторов НЕ дают строго
определения PBE, ограничиваясь оборотом вроде "веры должны
определяться по правилу Байеса везде, где это возможно". При этом
они подчеркивают, что PBE - это не то же самое, что WPBE. \\

Остановимся более подробно на случае 1. \\

Игра с неполной информацией. \\
Природа присваивает каждому игроку тип. Каждый игрок знает только
свой тип. Затем игроки играют в последовательную игру с
наблюдаемыми ходами. Допускается, что некоторые ходы делаются
одновременно. Платежи зависят от сделанных ходов и типов. \\
Каждую игру с неполной информацией можно представить в виде игры с
несовершенной информацией. \\
Cуществуют игры с несовершенной информацией, которым не
соответствует ни одной игры с неполной информацией. \\
Например: \\
\begin{figure}[h]
  \includegraphics{game.1}
\end{figure}


Если у $i$-го игрока несколько типов, то на каждом этапе игры, все
остальные игроки должны иметь мнение о вероятности каждого из
типов. Эти мнения должны быть согласованы (разные игроки должны
быть одинакового мнения). \\
Эти мнения также называются верами, однако они не совсем
соответствуют верам в играх с несовершенной информацией, какие
используются в WPBE или SE. \\
Например, в игре 2 игрока, у каждого один тип, ходят одновременно. \\

\begin{figure}[h]
  \includegraphics{game.4}
\end{figure}

Если рассматривать эту игру как игру с неполной информацией, то
веры тождественно равны единице. \\
Если рассматривать эту игру как игру с несовершенной информацией,
то, то у второго игра есть веры. \\






В произвольной игре с несовершенной информацией PBE можно
доопределить так: \\
PBE(1) это WPBE, удовлетворяющее дополнительному требованию: \\

Допустим, что вероятность попасть в информационное множество $I$
равна нулю, а узлы-предшественники лежат в нескольких
информационных множествах $I_{1}^{pr}$,...$I_{n}^{pr}$. Само $I$
оказывается естественным образом разбито на множества
$I_{1}$,...$I_{n}$. Пусть узел $x \in I_{k}$. Веру в узле $x$
обозначим $\mu(x)$, а $\mu(I_{k})=\sum_{x \in I_{k}} \mu(x)$. Если
$\mu(I_{k})>0$, то применяем формулу условной вероятности, а
именно, $ \frac{\mu(x)}{\mu(I_{k})}$ должна равняться вероятности
выбора узла $x$ в информационном множестве $I_{k}^{pr}$, которая
рассчитывается исходя из вер в $I_{k}^{pr}$. \\

Если рассмотреть игру с неполной информацией, то PBE в ней
(определяемое по Osborne) совпадет с PBE(1) для
соответствующей игры с несовершенной информацией. \\

PBE(1) не обязательно является совершенным в подыграх в
произвольное игре с несовершенной информацией. Для частного случая
игр с неполной информацией, PBE(1) является совершенным в
подыграх.

Существует альтернативное определение PBE в произвольной игре с
несовершенной информацией: \\
PBE(2) - это такая система (профиль стратегий, веры), которая
является WPBE  в каждой подыгре. \\

Оно не совпадает с PBE(1). Очевидно, что PBE(2) являются
совершенными в подыграх.

Что имеют в виду лица третьего типа не известно  (скорее всего,
PBE(1)), т.к. примеры, приводимые ими подходят как под PBE(1), так
и под PBE(2).
Собственно, приводят они один и тот же пример: \\
\begin{figure}[h]
  \includegraphics{game.2}
\end{figure}
\\
Упражнение. \\
Убедитесь в том, что в данной игре WPBE это $\{(In,R,r),\mu=0\}$ и
$\{(Out,R,l),\mu \geq 0.5\}$. \\
Убедитесь в том, что SE, PBE(1) и PBE(2) это только $\{(In,R,r),\mu=0\}$. \\
Заметьте также и то, что только профиль $(In,R,r)$ является SPNE \\


Чтобы еще более смутить невинные души студентов, следует отметить,
что к первой группе лиц (определяющим PBE только в играх с
неполной информацией) примыкают те, кто определяют PBE только в
играх с неполной информацией, но включают туда дополнительные
требования. \\

Окончательно запутать смущенные души можно с помощью критерия
структурной состоятельности вер. (Structural consistency of beliefs). \\

(Structural consistency of beliefs): Веры $\mu$ являются
структурно состоятельными, если для любого информационного
множества $I$ существует такой профиль стратегий $\beta$, что $I$
достигается с положительной вероятностью, и веры на $I$
определяются по формуле условной вероятности. \\

Т.е. веры в разных множествах могут определяться разными
профилями. \\

Однако это интуитивное требование  может быть несовместимо с
требованием состоятельности вер. Существуют структурно
состоятельные веры, которые не являются даже слабо состоятельными.
Существуют состоятельные веры (даже дополняемые профилем до
секвенциального равновесия), которые не являются структурно
состоятельными. \\



\end{document}