\documentclass{article}

\usepackage[utf8]{inputenc}
\usepackage[russian]{babel}

\begin{document}

Рассмотрим бесконечно-повторяемую игру с базовой матрицей

\begin{tabular}{c|cc} 
 & c & d \\ 
\hline
a & 4;4 & 0;5 \\ 
b & 5;0 & 1;1 \\ 
\end{tabular} 

Выигрыш каждого игрока равен предельному среднему выигрышей в отдельной партии. Игроки обладают короткой памятью и помнят только результат одной предыдущей партии. Стационарная стратегия первого игрока описывается четырьмя вероятностями выбора стратегии \verb|a| в следующей партии, в зависимости от результата предыдущей, $p_{ac}$, $p_{ad}$, $p_{bc}$ и $p_{bd}$. Стратегия второго игрока по аналогии описывается четырьмя вероятностями выбора стратегии \verb|c|, $q_{ac}$, $q_{ad}$, $q_{bc}$ и $q_{bd}$. 

\begin{enumerate}
\item Чему будет равен средний предельный выигрыш каждого игрока, если $p_{ad}=p_{ac}=p_{bc}=p_{bd}=1$, а $q_{ac}=0.1$, $q_{ad}=0.2$, $q_{bc}=0.3$ и $q_{bd}=0.4$? Как часто в долгосрочном периоде партия будет оканчиваться исходами \verb|ac|, \verb|ad|, \verb|bc| и \verb|bd|?
\item Может ли первый игрок в одиночку добиться того, чтобы между его средним предельным выигрышем $s_x$ и средним предельным выигрышем второго игрока $s_y$ выполнялось соотношение $(s_x-1)=4(s_y-1)$?

\end{enumerate}




\end{document}