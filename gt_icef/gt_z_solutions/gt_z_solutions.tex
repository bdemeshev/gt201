\RequirePackage[l2tabu, orthodox]{nag} % проверка плохого стиля, но что-то она не очень пашет

\documentclass[pdftex,12pt,a4paper]{article}

\usepackage[utf8]{inputenc}
\usepackage[russian]{babel}
\usepackage{booktabs}

\usepackage{amsmath}

\begin{document}

1.1.

\begin{tabular}{c|ccc}
 & $L$ & $C$ & $R$ \\
\midrule
$u$ & 4;1 & 0;0 & 0;3 \\
$s$ & 1;6 & 5;5 & 4;3 \\
$d$ & 2;5 & 7;3 & 6;0
\end{tabular}

Стратегия $d$ сильно доминирует стратегию $s$. После вычеркивания $s$ стратегия $L$ сильно доминирует стратегию $C$. Получаем матрицу

\begin{tabular}{c|cc}
 & $L$ & $R$ \\
\midrule
$u$ & 4;1 & 0;3 \\
$d$ & 2;5 & 6;0
\end{tabular}

Полезности:
\[
u_1(p,q) = 4pq + 2(1-p)q + 6(1-p)(1-q) = p(8q - 6) + 6 - 4q
\]

\[
u_2(p,q) = pq + 5(1-p)q + 3p(1-q) = q(5-7p)+3p
\]

Функции оптимального ответа:

\[
\check{p}(q) =
\begin{cases}
0, \; q < 3/4 \\
[0;1], \; q = 3/4 \\
1, \; q > 3/4
\end{cases}
\]

\[
\check{q}(p) =
\begin{cases}
0, \; p > 5/7 \\
[0;1], \; p = 5/7 \\
1, \; p < 5/7
\end{cases}
\]

Ответ: единственное равновесие в смешанных стратегиях, $(p=5/7, q = 3/4)$



\end{document}
