\documentclass[10pt,a4paper]{article}
\usepackage[utf8]{inputenc}
\usepackage[russian]{babel}
\usepackage[OT1]{fontenc}
\usepackage{amsmath}
\usepackage{amsfonts}
\usepackage{amssymb}
\let\P\relax
\DeclareMathOperator{\P}{\mathbb{P}}
\begin{document}

Double round auction

Четыре игрока, ценности товара независимы и равномерны на $[0;1]$. Аукцион организован так: два полуфинала (аукционы первой цены) и финал между победителями (ещё один аукцион первой цены). Победитель получает чудо-швабру. В каждом туре победитель платит. Найдите равновесные стратегии.

Solution 

Стратегия каждого игрока: сколько ставить в зависимости от своей субъективной оценки чудо-швабры в первом туре и во втором, то есть формально стратегия описывается двумя функциями, $b^{I}(v)$ и $b^{II}(v)$.

Предположим, что эти фукнции возрастающие. Тогда в итоге товар получает тот игрок, который ценит чудо-швабру выше других. 

Игрок, проходящий во второй тур, является победителем первого тура, поэтому распределение субъективных оценок во втором туре уже не является равномерным. Субъективная ценность игрока $A$, прошедшего во второй тур, является максимумом двух равномерных случайных величин:

\[
v_A=\max\{v_1,v_2\}
\] 

Найдём её функцию распределения:

\begin{multline}
F_{v_A}(t)=\P(v_A \leq t)=\P(\max\{v_1,v_2\} \leq t)=\P(v_1 \leq t, v_2 \leq t)=\\
=\P(v_1 \leq t)\cdot \P(v_2 \leq t)=t^2
\end{multline}

Естественно, функция плотности равна $f_{v_A}(t)=2t$.

Решаем с конца, со второго тура. Допустим во второй тур вышли игроки $A$ и $B$. Зафиксируем стратегию второго игрока, $b_B=b(v_B)$. Максимизируем ожидаемый выигрыш $A$:

\[
(v_A-b_A)\cdot \P(b(v_B) < b_A | v_A) 
\]

Мы ищем оптимальную ставку $b_A$. Сделаем замену неизвестного числа $b_A$ на $b(t)$. Не страшно, что функция $b()$ пока еще неизвестна. Зато формулы упрощаются:

\[
(v_A-b(t))\cdot  \P(b(v_B) < b(t))=(v_A - b(t))\cdot \P( v_B < t)=(v_A - b(t))\cdot t^2
\]

Берем производную по $t$:

\[
(v_A-b(t)) \cdot 2t - b'(t) t^2=0
\]

В силу симметрии оптимальная стратегия игрока $A$ должна равняться $b()$, то есть $b(t)=b(v_A)$ и получаем:

\[
(t-b(t)) \cdot 2 - t \cdot b'(t)=0
\]

Решая это линейное уравнение находим, что $b(t)=\frac{2}{3}t + c\cdot t^{-2}$. Нам нужно возрастающее решение, поэтому берём $c=0$. 

Итого, оптимальная стратегия во втором туре $b(v)=\frac{2}{3}v$. Можно было немного сжульничать, предположить, что стратегия линейна и найти оптимальную линейную стратегию.

Если игрок $A$ с оценкой товара равной $v_A$ только вошёл во второй тур, то он ожидает, что его выигрыш будет равен:

\[
(v_A-2/3\cdot v_A)\cdot v_A^2=v_A^3/3
\]

Находим оптимальную стратегию в первом туре. На этот раз $b()$ неизвестная оптимальная стратегия игроков в первом туре. Ожидаемый выигрыш в начале игры равен вероятности выйти во второй тур помножить на условный ожидаемый выигрыш, если я дошел до второго тура.

\[
(v_1^3/3-b_1) \cdot \P(b(v_2) < b_1 | v_1)
\]

Делаем замену $b_1=b(t)$:

\[
(v_1^3/3-b(t)) \cdot \P(v_2 < t )=(v_1^3/3-b(t))\cdot t
\]

Берем производную по $t$ и вспоминаем, что в оптимуме $b(t)=b(v_1)$:

\[
t^3/3-b(t)-b'(t)\cdot t=0
\]

Решаем дифференциальное уравнение $b(t)=t^3/12+c/t$, отбираем возрастающее решение, получаем $b(t)=t^3/12$.

Итого: Нэш: $b_I(v)=v^3/12$, $b_{II}(v)=2v/3$.

\end{document}