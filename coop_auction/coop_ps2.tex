\subsection{Дз 2}

\vspace{5pt} \indef{Задача 1.}

Какие из рассмотренных примеров игр являются супермодулярными?

Ботинки-2, Малое Гадюкино, 3 player unanimity game, Гномы и золото, Нефтепровод, Угадай цену пирога, Помещик и крестьяне


\vspace{5pt} \indef{Задача 2.}

Приведите пример несупераддитивной игры. Желательно не абстрактный, а жизненный. Желательно смешной.


\vspace{5pt} \indef{Задача 3.}

а) Докажите, что из супермодулярности следует супераддитивность.

б) Приведите пример супераддитивной, но не супермодулярной игры. Желательно не абстрактный, а жизненный. Желательно смешной.


\vspace{5pt} \indef{Задача 4.}

Разложите игры Ботинки, Гномы и золото, Ботинки-2 на простые игры.

