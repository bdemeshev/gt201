\section{Экзамен. Кооперативная теория игр.}

Задача 1. ООН

Совет Безопасности ООН состоит из 15 членов. Пять членов Совета — постоянные (Россия, США, Великобритания, Франция и Китай), остальные десять членов - периодически меняются. Для принятия решения о применении санкций необходимо одобрение не менее 9 членов включая всех постоянных. В этой кооперативной игре выигрышем коалиции можно считать 1, если она может принять санкции, и 0, если не может.

Найдите вектор Шепли. Во сколько раз влияние постоянного члена сильнее, чем не постоянного?

Задача 2. Количество голосов

В стране N есть 5 провинций, разных по численности населения: 100, 100, 200, 300, 400 (тыс. чел.) Руководство страны состоит из 5 человек. Им даны голоса пропорционально численности провинции, т.е. 1, 1, 2, 3, 4 голоса, соответственно. Решение принимается, если за него подано не менее 8 голосов (из 11 возможных). В этой кооперативной игре выигрышем коалиции можно считать 1, если она может одобрить решение и 0, если не может.

а) Найдите вектор Шепли. Соответствует ли он численности населения?

б) (возможно явное использование компьютера). Подберите численности голосов так, чтобы вектор Шепли был максимально пропорционален численности населения.



Задача 3. Игра <<Мусор>>

Есть $n$ дворов. На каждом дворе скопился один мешок мусора. Вывоз одного мешка мусора стоит 1 рубль. Но можно же втихаря подкинуть его соседу! Поэтому: $v(N)=-n$ (т.к. большой коалиции придется по-любому убирать со своей территории $n$ мешков). Если какая-то коалиция захочет отсоединится от большой коалиции, то мы будем считать, что остаток большой коалиции будет играть против <<отколовшихся>>. Поэтому при $S\neq N$: $v(S)=|S|-n$, где $|S|$ - это численность коалиции $S$ (ровно столько мешков коалиции $S$ удастся выкинуть на соседские дворы).

Найдите ядро этой игры при произвольном $n$


Задача 4. Банковский вклад.

У Ани - 70 рублей, у Бори - 80 рублей, у Вовы - 150 рублей. Процентная ставка по вкладу: 5\% при сумме вклада в диапазоне $[0;100)$, 6\% при сумме вклада в диапазоне $[100;200)$ и 7\% при сумме вклада в диапазоне $[200;\infty)$. Если они соберутся вместе, то смогут расчитывать на ставку в 7\%. Как им поделить прибыль?

Найдите вектор шепли, нуклеолус, около-ядро (при поиске около-ядра придется уходить в отрицательные $\varepsilon$)


Задача 5. Добровольное страхование

Есть группа А из 100 человек. У каждого из них страховой случай (потеря 1-го рубля) наступает независимо с вероятностью 0.1. Они решили объединится и застраховать сами себя, так чтобы вероятность банкротства группы была равна всего 0.001. 
Есть группа Б из 100 человек. У каждого из них страховой случай (потеря 1-го рубля) наступает независимо с вероятностью 0.15. Они решили объединится и застраховать сами себя, так чтобы вероятность банкротства группы была равна всего 0.001. 

а) Каков должен быть резерв страховой компании А?

б)  Каков должен быть резерв страховой компании Б?

в) Две группы решили объединится и застраховать сами себя, так чтобы вероятность банкротства группы была равна всего 0.001. Каков должен быть резерв объединенной страховой компании? 

г) Как правильно поделить расходы между игроками в случае объедения двух страховых групп? (Найдите ядро, вектор Шепли, нуклеолус, К-ядро)

Задача 6. Обмен рисками.

Есть два фонда, А и Б. Активы А можно считать случайной величиной со средним 6 и дисперсией 5, активы Б - случайной величиной со средним 10 и дисперсией 15. Предположим для простоты, что они не коррелированы. Компании хотят обменятся активами так, чтобы средние не поменялись, а дисперсии упали. т.е. компания А отдает часть своих активов компании Б, а компания Б отдает часть своих активов компании А, возможно также, что одна из компаний платит другой какую-то сумму денег (безрисковый актив).

а) Нарисуйте на плоскости возможные комбинации дисперсий

б) Найдите решение Нэша и решение Калаи-Смородинского


Задача 7. Кое-какие свойства ядра

Рассмотрии игру 4-х игроков, $v$:

Ценность большой коалиции равна 2, $v(N)=2$

Ценность любой коалиции из 3-х игроков равна 1, $v(S)=1$ при $|S|=3$.

Ценность каждого отдельного игрока равна нулю, $v(i)=0$ при любых $i$.

$v(\{1,2\})=v(\{3,4\})=0$, ценность других коалиций из 2-х игроков равна 1.

а) Найдите ядро

б) Что произойдет с ядром, если ценность коалиции $\{1,3,4\}$ возрастет с 1 до 2?

в) Прокомментируйте то, что происходит с выигрышами 1-го, 3-го и 4-го игроков


Задача 8. Вектор Шепли и нуклеолус. Почувствуйте разницу!

Рассмотрим вариант игры <<Ботинки>>. В игре 4 игрока. У первого - 2 правых ботинка, у второго - 1 правый, у остальных - по одному левому. Полная пара стоит 1 рубль, отдельный ботинок ничего не стоит.

а) Найдите вектор Шепли и нуклеолус

б) Прокомментируйте разницу с игрой, в которой у первого игрока 1 правый ботинок.





