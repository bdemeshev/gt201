\section{Переговоры.}

В этой главе Винни-Пух узнает несколько решений кооперативных игр, основанных на моделировании переговоров. В этом зачарованном месте мы его и оставим.

На самом деле все подходы к определению равновесия могут быть сформулированы на языке <<возражение-контр-возражение>>. Что такое равновесие по Нэшу в некооперативных играх? Исход, против которого никто в отдельности не возражает. Что такое ядро в кооперативных? Набор платежей, против которого не возражает ни одна коалиция. Проблема заключается в том, что эти концепции оказываются слишком сильными или слишком слабыми. Тогда их можно ослабить или усилить с помощью понятия <<контр-возражения>>.

\subsection{Переговорное множество. Ах если ты так, то я...}

Мы снова возвращаемся к кооперативным играм нескольких игроков в характеристической форме. Несколько следующих концепций решения основаны на моделировании переговоров. Два самых интуитивно понятных требования к дележу - это \hyperref[Pareto]{Парето-оптимальность} и \hyperref[irationality]{индивидуальная рациональность}. Будем рассматривать только те дележи, где эти требования выполнены.

\begin{mydef} \index{Импутация} \indef{Импутация}  (imputation) - любой Парето-оптимальный и индивидуально-рациональный дележ.
\end{mydef}

\begin{mydef}
Импутация $x$ \indef{доступна для коалиции $S$}, если отсоединившись коалиция $S$ может обеспечить своим игрокам выигрыши из импутации $x$, т.е. $\sum_{i\in S}x_{i}\leq v(S)$ или $x(S)\leq v(S)$.
\end{mydef}

Представим себе спор двух игроков, $i$ и $j$ о том, как поделить общественное богатство между всеми игроками (уточним, что всего игроков $n$, а спорят двое). Эти двое будут агитировать разные коалиции за свой дележ.

Предположим, что игрок $j$ предложил импутацию $x$. 

Как ему может возразить игрок $i$? Например, угрозой:

<<Давайте лучше вместо дележа $x$ выберем дележ $y$! А если ты настаиваешь на $x$, то я тогда сколочу против тебя группу $S$, мы отсоединимся и заработаем свою часть дележа $y$ сами, а это больше, чем ты нам предлагаешь в дележе $x$!>>

\begin{mydef}
Пара $(y,S)$ называется \indef{возражением игрока $i$ игроку $j$ на импутацию $x$} если:
\begin{itemize}
\item[-] импутация $y$ доступна для коалиции $S$
\item[-] игрок $i$ входит в $S$, а $j$ не входит в $S$
\item[-] импутация $y$ обеспечивает каждому члену коалиции $S$ более высокий выигрыш чем импутация $x$
\end{itemize}
\end{mydef}

Как игрок $j$ может ответить на это возражение? Например, ответной угрозой: 

<<Если ты пытаешься сколотить группу $S$, то я в ответ сформирую группу $T$, так что им достанется не меньше, чем раньше предлагалось мной или тобой!>>

\begin{mydef}
Пара $(z,T)$ называется \indef{контр-возражением игрока $j$ игроку $i$ на возражение $(y,S)$} если:

\begin{itemize}
\item[-] импутация $z$ доступна для коалиции $T$
\item[-] игрок $j$ входит в $T$, а $i$ не входит в $T$
\item[-] игрокам входящим в $T$ импутация $z$ гарантирует выигрыш не меньший, чем изначальная импутация $x$.
\item[-] игрокам входящим в $T$, и в $S$ импутация $z$ гарантирует выигрыш не меньший, чем обещан возражением $y$.
\end{itemize}
\end{mydef}

Подчеркнем следующий момент. В определение контр-возражения не требуется, чтобы коалиция $T$ пересекалась с коалицией $S$. Если коалиция $T$ из контр-возражения $(z,T)$ пересекается с коалицией $S$, то она действительно мешает формированию коалиции $S$. Но возможна ситуация, когда $T$ не пересекается с $S$ и тогда понятие контр-возражения оказывается довольно слабым. В этом случае игрок $i$ может сказать, <<Ну и ладушки, ты иди со своей $T$, а я пойду со своей $S$, а остальные пусть между собой сами разбираются>>. Это не лишает определение смысла, т.к. в этом случае первый игрок может предложить возражение с коалицией $S\cup T$ и соответствующими платежами.

Используя понятия возражения можно по новому определить ядро:

\begin{mydef}
Ядро \index{Ядро} - это множество импутаций, на которые никто никому не может возразить.
\end{mydef}

Ядро часто бывает пусто, поэтому возникло такое послабление:

\begin{mydef}
Переговорное множество \index{Переговорное множество} - это множество импутаций $x$ таких, что на любое возражение $(y,S)$ игрока $i$ игроку $j$ на $x$, найдется контрвозражение игрока $j$ на возражение $(y,S)$.
\end{mydef}


\begin{myex} Игра <<Носки>>. Индивидуальная рациональность и эффективность дают условия: $x_{1}\geq 60$, $x_{2}\geq 120$, $x_{1}+x_{2}=240$. Далее мы легко замечаем, что ни у кого нет возражений. В этой игре оказалось, что переговорное множество совпало с ядром.
\end{myex}


В общем случае из определения непосредственно вытекает теорема:
\begin{myth}
Ядро всегда лежит в переговорном множестве
\end{myth}

Оказывается, что переговорное множество всегда непусто, что хорошо. Но оказывается внутри него можно выделить K-ядро, которое тоже всегда непусто, а внутри К-ядра еще и единственную точку нуклеолус, которая всегда существует. Более того, в отличие от вектора Шепли, который всегда существует но может не попадать в ядро, нуклеолус всегда существует и попадает в ядро (если, конечно, ядро не пусто). Этим мы сейчас и займемся!

\subsection{К-ядро.}

Сразу скажем, ядро и К-ядро это не одно и то же. В английском это core и kernel. В русском переводе иногда говорят про C-ядро и К-ядро. Мы будем говорить ядро (core) и К-ядро (kernel).

Для краткости определим:
\begin{mydef}
Эксцесс коалиции \index{Эксцесс коалиции} $S$ при импутации $x$ - это разница между суммой, которая коалиция может заработать сама и тем, что ей предлагается в импутации $x$, $e(S,x):=v(S)-x(S)$.
\end{mydef}

Эксцесс можно рассматривать как жертву, приносимую коалицией $S$ при формировании большой коалиции. Если считать, что есть некие издержки формирования коалиции, то вполне логично, что в равновесии эта жертва может быть положительной. Ядро считает, что издержек формирования коалиции нет, и для любой импутации $x$ из ядра эксцесс любой коалиции неположительный.

Заметим, что эксцесс также можно рассматривать как силу коалиции. Если при дележе $x$ у коалиции $S$ большой эксцесс, то у нее больше стимулов уходить, чтобы ее удержать надо менять $x$ в ее пользу.

Чтобы мотивировать К-ядро представим снова представим себе спор двух игроков. Игрок $j$ предложил импутацию $x$. 

Игрок $i$ недоволен и говорит примерно так: 

<<Вот ты, $j$, при дележе получаешь больше чем в одиночку можешь заработать ($x_{j}>v(j)$), а коалиция $S$, где я есть, а тебя - нет, жертвует аж целых $e(S,x)$!!!>>

\begin{mydef}
Коалиция $S$ называется \indef{возражением игрока $i$ игроку $j$} если:
\begin{itemize}
\item[-] игрок $i$ входит в $S$, а $j$ не входит в $S$
\item[-] игрок $j$ получает в импутации $x$ больше, чем может заработать в одиночку, $x_{j}>v(j)$
\end{itemize}
\end{mydef}

На что игрок $j$ возражает,

<<Ха, да коалиция $T$ куда вхожу я, а ты - не входишь, жертвует не меньше!>>

\begin{mydef}
Коалиция $T$ называется \indef{контр-возражением игрока $j$ игроку $i$ на возражение $S$} если:
\begin{itemize}
\item[-] игрок $j$ входит в $S$, а $i$ не входит в $S$
\item[-] эксцесс коалиции $T$ не меньше эксцесса коалиции $S$, $e(T,x)\geq e(S,x)$
\end{itemize}
\end{mydef}


И мы опять потребуем того, чтобы на всякое возражение нашлось контр-возражение:

\begin{mydef}
К-ядро \index{К-ядро} (kernel) - это множество импутаций $x$, таких что на любое возражение $S$ игрока $i$ игроку $j$, игрок $j$ может контр-возразить $T$. 
\end{mydef}


Можно определить К-ядро другим эквивалентным способом. Когда требование индивидуальной рациональности выполнено, игроки делятся на двух типов\footnote{Желающие могут упростить определение в Википедии, ибо два неравенства - это ужас!}: тех, которые получают ровно столько, сколько могут заработать самостоятельно, и тех, которые получают больше. К тем, кто получает ровно столько, сколько может заработать сам, не придерешься. А вот остальные должны как-то мотивировать свой доход. К-ядро требует, чтобы те, кто получает больше чем может заработать сам, были членами коалиций с большим эксцессом. Более формально:

Обозначим с помощью $s_{ij}$ наибольшую возможную жертву, которую приносят коалиции, содержащие игрока $i$ и не содержащие игрока $j$:

\begin{equation}
s_{ij}(x):=\max\{e(S,x)|i\in S,j\notin S\}
\end{equation}

Эту величину можно рассматривать как силу, с которой игрок $i$ может убеждать игрока $j$. 

\begin{mydef}
\indef{К-ядро} \index{К-ядро} (kernel) \index{Kernel} - это множество импутаций $x$, таких, что для любой пары игроков $i$ и $j$ выполнено \indef{хотя бы одно} из двух условий:
\begin{itemize}
\item[-] игрок $j$ получает ровно столько, сколько может заработать, если отсоединится, $x_{j}=v(j)$.
\item[-] игрок $j$ способен убеждать игрока $i$ не меньше, чем игрок $i$ способен убеждать игрока $j$, $s_{ji}(x)\geq s_{ij}(x)$.
\end{itemize}
\end{mydef}

\begin{myex} Игра <<Носки>>. Индивидуальная рациональность и эффективность дают условия: $x_{1}\geq 60$, $x_{2}\geq 120$, $x_{1}+x_{2}=240$. Эксцесс для первого игрока: $(60-x_{1})$, эксцесс второго игрока: $(120-x_{2})$.

Проверяем краевые решения: 

Если $x_{1}=60$, то $x_{2}=180$ и остается проверить, что второй игрок способен убеждать первого сильнее чем первый второго: $120-180\geq 60-60$. Невозможно.

Если $x_{2}=120$, то $x_{1}=120$ и остается проверить, что первый игрок способен убеждать второго, сильнее чем второй первого: $60-120\geq 120-120$. Невозможно.

Внутреннее решение. Силы убеждения должны быть равны. $60-x_{1}=120-x_{2}$. В данном случае К-ядро состоит из единственной точки $(90,150)$.
\end{myex}

Как и было обещано:

\begin{myth}
В кооперативной игре в характеристической форме К-ядро всегда лежит в переговорном множестве.
\end{myth}
\begin{proof}
Рассмотрим произвольную импутацию $x$ из К-ядра. Чтобы доказать, что она лежит в переговорном множестве достаточно предъявить контр-возражение $(z,T)$ на любое возражение $(y,S)$.

Пусть $(y,S)$ возражение игрока $i$ игроку $j$.
\begin{itemize}
\item Если игрок $j$ при дележе $x$ получает ровно столько, сколько может заработать сам, то в качестве контр-возражения годится пара $(z,T)$, где $z=x$, а $T$ состоит только из игрока $j$. В споре это аргумент типа <<Да ты, $i$, глянь, я же себе вообще ничего общественного не взял!>>.
\item Если игрок $j$ при дележе $x$ получает больше, чем может заработать сам, $x_{j}>v(j)$, то в качестве контр-возражения подойдет пара $(z,T)$, где $T$ - коалиция с наибольшим эксцессом среди всех коалиций, содержащих $j$, но не содержащих $i$. Давайте убедимся в этом. Итак, $v(T)-x(T)=s_{ji}(x)$. 

Из того что $x$ лежит в К-ядре следует:
\begin{equation}
v(T)-x(T)=s_{ji}(x) \geq s_{ij}(x)
\end{equation}

Т.к. $s_{ij}$ - максимальный эксцесс:
\begin{equation}
s_{ij}(x) \geq v(S)-x(S)
\end{equation}

Т.к. импутация $y$ доступна для коалиции $S$:
\begin{equation}
v(S)-x(S)\geq y(S)-x(S)
\end{equation}

Неравенство $v(T)\geq x(T)+y(S)-x(S)$ легко довести до готовности напильником:

Во-первых, 
\begin{equation}
y(S)=y(S\cap T)+y(S\backslash T)
\end{equation}

Во-вторых,
\begin{equation}
x(T)-x(S)=(x(S\cap T)+x(T\backslash S))-(x(S\cap T)+x(S\backslash T))=x(T\backslash S)-x(S\backslash T)
\end{equation}

Следовательно, 
\begin{equation}
v(T)\geq y(S\cap T)+y(S\backslash T)+x(T\backslash S)-x(S\backslash T)
\end{equation}. 

По определению возражения, $y(S\backslash T)-x(S\backslash T)\geq 0$. Значит $v(T)\geq y(S\cap T) + x(T\backslash S)$. А это и означает, что у коалиции $T$ хватит денег, чтобы заплатить своим членам не меньше, чем они получают при $x$, а тем членам, кого соблазняли возражением $(y,S)$ заплатить не ниже того, что обещано при дележе $y$, т.е. соответствующий $z$ существует.
\end{itemize}


\end{proof}

Отметим, что переговорное множество, как и обычное ядро легко обобщаются на случай игры с нетрасферабельной полезностью. Связано это с тем, что переговорное множество и ядро не сравнивают полезности разных коалиций, а К-ядро сравнивает. Поэтому эта теорема не обобщается на игры с нетрансферабельной полезностью. 

\subsection{Нуклеолус}

Еще одна пара возражение-контрвозражение. На этот раз представим спор двух коалиций, $S$ и $T$. Коалиция $T$ предложила импутацию $x$.

Коалиция $S$ может возразить:

<<Давайте лучше выберем дележ $y$, т.к. мы получим при этом больше денег, $y(S)>x(S)$!>>
\begin{mydef}
Пара $(S,y)$ называется возражением коалиции $S$ на дележ $x$, если при дележе $y$ коалиция $S$ получает больше, чем при дележе $x$, $y(S)>x(S)$.
\end{mydef}

На что коалиция $T$ может контр-возразить:

<<Разбежались! В своем дележе вы нас аж дважды обидели! Мало того, что мы меньше получаем, $y(T)<x(T)$, так еще и наши стимулы отсоединяться становяться больше, чем были ваши при изначальном дележе, $e(T,y)\geq e(S,x)$.>>
\begin{mydef}
Коалиция $T$ называется контр-возражением на возражение $(S,y)$, если:
\begin{itemize}
\item[-] коалиция $T$ получает при дележе $y$ меньше, чем при дележе $x$, $y(T)<x(T)$
\item[-] эксцесс коалиции $T$ при дележе новом дележе $y$ больше, чем эксцесс коалиции $S$ при дележе $x$, $e(T,y)\geq e(S,x)$
\end{itemize}
\end{mydef}


\begin{mydef}
Нуклеолус \index{Нуклеолус} - это множество импутаций $x$ таких, что на любое возражение $(S,y)$ на $x$ найдется контр-возражение $T$.
\end{mydef}


\begin{myex} Рассмотрим игру <<Носки>>. Пара носков стоит 60 рублей. Один носок ничего не стоит. У Андрея - три носка, у Бориса - пять носков, поэтому $N=\{$Андрей,Борис$\}$, $v($Андрей$)=60$, $v($Борис$)=120$, $v($Андрей, Борис$)=240$. 

Пусть $x$ - импутация. Пусть на эту импутацию есть возражение $y$ от первого игрока, т.е. $y_{1}>x_{1}$. Когда у второго будет контр-возражение? Для контр-возражения нужно, чтобы он получал меньше и чтобы его эксцесс становился больше, чем был у первого. Импутации $y$ и $x$ эффективны, первый получает больше при $y$, значит второй автоматом получает меньше при $y$. Эксцесс второго при $y$ равен $(120-y_{2})=120-(240-y_{1})=y_{1}-120$, эксцесс первого при $x$ равен $60-x_{1}$.

Итого: из неравенства $y_{1}>x_{1}$ должно следовать $y_{1}-120\geq 60-x_{1}$.

Аналогично для возражения второго и контр-возражения первого:

Из неравенства $y_{1}<x_{1}$ должно следовать $60-y_{1}\geq x_{1}-120$.

Значит $x_{1}=180-x_{1}$. Следовательно, нуклеолус - это $(90,150)$. Что говорит в пользу нуклеолуса, т.к. снова соответствует интуитивному дележу 3:5.
\end{myex}

\begin{myth}
Нуклеолус всегда всегда существует и состоит из одной точки.
\end{myth}

\begin{myth}
Нуклеолус всегда лежит в К-ядре
\end{myth}

Из этих двух теорем непосредственно следует, что К-ядро всегда непусто, а следовательно и переговорное множество всегда непусто.

\begin{myth}
Если ядро непусто, то нуклеолус лежит в ядре.
\end{myth}

Мы оставим доказательства этих трех теорем за пределами нашего краткого курса. Но (опять же без доказательств) мы дадим другое описание нуклеолуса, чтобы стало понятнее, что же это за зверь, и почему эти теоремы похожи на правду.

Когда задана импутация $x$ для каждой коалиции можно выписать ее эксцесс. Упорядочим эксцессы по убыванию. Получим, что каждой импутации $x$ однозначно сопоставлен вектор эксцессов $E(x)=(e(S_{1},x),e(S_{2},x),...)$, где $e(S_{k},x)\geq e(S_{k+1},x)$.

Большие значения эксцессов означают нестабильность дележа $x$: чем больше эксцессы, тем больше стимулов у коалиций отказаться от дележа $x$. Поэтому равновесным логично считать дележ с наименьшим вектором эксцессов. Вопрос лишь в том, как сравнить два вектора с разными числами, где иногда больше компонента первого вектора, а иногда - второго.

В начале вектора $E(x)$ идут более крупные, следовательно, более важные эксцессы. Поэтому разумно сравнивать векторы эксцессов по лексикографическому принципу:

\begin{mydef}
Вектор $a$ \indef{лексикографически больше} \index{Лексикографический принцип} вектора $b$ если первая несовпадающая цифра больше в векторе $a$.
\end{mydef}

Пример: вектор (5,5,4,4,1) лексикографически больше вектора (5,4,4,4,4).

Оказывается можно дать такое определение нуклеолуса:
\begin{mydef}
Нуклеолус \index{Нуклеолус} - это дележ с самым маленьким (в лексикографическом смысле) вектором эксцессов.
\end{mydef}


Доказать эквивалентность данных двух определений нуклеолуса непросто. Зато из второго определения легко следует его существование, единственность и способ нахождения. Сейчас мы его и продемонстрируем.

Почему ядро бывает пусто? Потому, что оно требует, чтобы эксцесс любой коалиции не превосходил бы ноль. Естественно, можно ослабить эти неравенства и рассмотреть более широкое $\varepsilon$-ядро:

\begin{mydef}
$\varepsilon$-ядро \index{$\varepsilon$-ядро} - это множество импутаций $x$, таких что эксцесс любой коалиции не превосходит $\varepsilon$, т.е. для любой $S$ верно неравенство $e(S,x)\leq \varepsilon$.
\end{mydef}

Другими словами, $\varepsilon$-ядро - моделирует ситуацию, когда ни одной коалиции не выгодно отсоединяться, при условии, что издержки отсоединения равны $\varepsilon$. По определению, ядро является $0$-ядром. 

Допустим, что ядро пусто. Будем увеличивать $\varepsilon$ до тех пор пока в $\varepsilon$-ядре не появится какой-нибудь дележ. такой момент рано или поздно наступит, поскольку заработок любой коалиции ограничен заработком большой коалиции $v(N)$. Поскольку неравенства нестрогие, будет существовать и наименьшее $\varepsilon$ при котором $\varepsilon$-ядро непусто.

\begin{mydef}
Около-ядро \index{Около-ядро} (near-core, least-core) - наименьшее непустое $\varepsilon$-ядро.
\end{mydef}

Оказывается нуклеолус можно нащупать в около-ядре таким способом:

Начнем с около-ядра. 

Шаг 1. Запомним наименьшее $\varepsilon$ и те коалиции, для которых дальнейшее уменьшение $\varepsilon$ приводит к пустоте $\varepsilon$-ядра. Этим коалициям при дальнейшем увеньшении $\varepsilon$ выгодно отсоединится.
Получим некое пороговое $\varepsilon_{1}$ и некий список коалиций $B_{1}$, для которых это $\varepsilon_{1}$ является наименьшими издержками для невыгодности выхода из большой коалиции. Снижая $\varepsilon$ мы добились, что ни один эксцесс не превосходит $\varepsilon_{1}$ и улучшить этот результат нельзя.

Шаг 2. Для остальных коалиций будем уменьшать $\varepsilon$ в неравенстве $e(S,x)\leq \varepsilon$. Наступит такой момент, что дальнейшей уменьшение $\varepsilon$ снова приводит пустоте множества дележей, т.к. некоторым другим коалициям становится выгоднее отсоединиться.
Получим некое $\varepsilon_{2}$ и некий список коалиций $B_{2}$. Сейчас мы добились того, что все эксцессы (кроме тех, что нельзя сделать ниже $\varepsilon_{1}$) не превосходят $\varepsilon_{2}$.

Шаги 3-... Будем повторять шаг 2 до тех пор, пока дальнейшее уменьшение $\varepsilon$ хотя бы для одной коалиции будет приводит к пустоте множества дележей.

В результате получаем некий единственный дележ. Он будет единственным т.к. в противном случае, можно было бы <<урезать>>  множество снизив $\varepsilon_{k}$ в одном из неравенств. По построению мы получили лексикографически минимальный элемент. Значит это нуклеолус.


Помимо этого дележа получаем убывающую последовательность эксцессов $\varepsilon_{1}$, $\varepsilon_{2}$, ... и непересекающиеся списки коалиций $B_{1}$, $B_{2}$, ... Число коалиций в списке $B_{i}$ обозначим как $b_{i}$. Нуклеолусу будет соответствовать вектор эксцессов:

\begin{equation}
(\underbrace{\varepsilon_{1},\varepsilon_{1} ..., \varepsilon_{1}}_{\mbox{$b_{1}$ раз}}, \underbrace{\varepsilon_{2},\varepsilon_{2} ..., \varepsilon_{2}}_{\mbox{$b_{2}$ раз}}, ...) 
\end{equation}


\begin{myex} Покажем эту процедуру на примере игры <<Носки>>.

Пусть $x$ - импутация, т.е. $x_{1}+x_{2}=240$, $x_{1}\geq 60$, $x_{2}\geq 120$.

Мы должны минимизировать наибольший эксцесс, $\min \varepsilon$ при ограничениях:

$60-x_{1}\leq \varepsilon$ и $120-x_{2}\leq \varepsilon$

Находим, что $\varepsilon_{1}=-30$. При этом оба неравенства превращаются в равенство. Т.е. решение получилось уже после первого шага. Нуклеолус, как и при первом способе, равен $(90,150)$. Вектор эксцессов, $E(x)=(-30,-30)$.
\end{myex}

Если верить в эквивалентность двух определений нуклеолуса, то теорема о том, что нуклеолус лежит в ядре, если оно непусто, оказывается автоматически доказанной, т.к. в этом случае первый шаг начинается с ядра.



\subsection{Решение Неймана-Моргенштерна. С чего все начиналось?}

Осталось лишь рассказать об одной из первых концепций решения кооперативных игр. Речь идет об устойчивом множестве (решении Неймана-Моргенштерна). Для этого нам потребуется еще одно понятие возражения. 

\begin{mydef}
Импутация $y$ называется \indef{возражением коалиции $S$ на импутацию $x$} если: 
\begin{itemize}
\item[-] импутация $y$ обеспечивает каждому члену коалиции $S$ более высокий выигрыш чем импутация $x$
\item[-] импутация $y$ доступна для коалиции $S$, $v(S)\geq y(S)$.
\end{itemize}
Обозначаем $y\succ_{S} x$
\end{mydef}

Используя это понятие, легко например определить уже известное нам ядро:

\begin{mydef}
\index{Ядро} - это множество всех импутаций, на которые ни у одной коалиции нет возражения.
\end{mydef}

Чуть ослабив требования мы получим решение Неймана-Моргенштерна (устойчивое множество)

\begin{mydef}
Множество импутаций $Y$ называется решением Неймана-Моргенштерна или устойчивым множеством \index{Решение Неймана-Моргенштерна} \index{Устойчивое множество} (stable set, NM-solution) если выполнены два условия:

Внутренняя стабильность \index{Внутренняя стабильность}: Если $x$ и $y$ - две импутации из $Y$, то не существует коалиции $S$ для которой $y$ было бы возражением на $x$.

Внешняя стабильность \index{Внешняя стабильность}: Если $x$ - импутация не входящая в $Y$, то найдется коалиция $S$, которая сможет предъявить в качестве возражения на $x$ некоторую импутацию $y$ из $Y$.
\end{mydef}

Сразу заметим, что устойчивых множеств может быть много. Устойчивое множество представляет собой разбиение множества возможных импутаций на две части: импутации внутри $Y$ не являются возражением друг на друга ни с чьей точки зрения, а на импутации вне $Y$ можно найти возражение из $Y$ (с чьей-нибудь точки зрения). Впрочем, импутации извне $Y$ могут быть возражениями на импутации внутри $Y$, это не запрещается.

\begin{myex}
Илья Муромец, Добрыня Никитич и Алеша Попович охотятся на одного единственного Змея Горыныча. В одиночку никто не может его поймать. А любые два богатыря уже могут поймать Змея Горыныча.

$A=\{(0.5,0.5,0),(0.5,0,0.5),(0,0.5,0.5)\}$ - это устойчивое множество. Проверяем внутреннюю стабильность. В одиночку игроки возражать не могут, т.к. не могут завалить Змея Горыныча. А среди данных дележей нет варианта, чтобы при смене $x$ на $y$ становилось лучше сразу двум игрокам. Проверяем внешнюю стабильность. Пусть есть некий дележ $z$ не равный этим трем. Тогда в этом дележе $z$ обязательно найдется два игрока, получающих строго меньше половины. И дележ $z$ будет доминироваться дележом из $A$, где эти двое получают по 0.5.

$B=\{(0.3,x_{2},x_{3}|x_{2}+x_{3}=0.7\}$ - тоже устойчивое множество. Проверяем внутреннюю стабильность. Снова в одиночку никто не может ничего получить, а увеличить сразу два платежа невозможно. Проверяем внешнюю стабильность. Пусть есть некий дележ $z$ не входящий в $B$. Если в дележе $z$ первый игрок получает больше 0.3, то мы его сможем доминировать дележом, где второй и третий игроки получают больше, чем в $z$. Допустим в дележе $z$ первый игрок получает меньше 0.3. Рассмотрим два дележа из $B$: $(0.3,0.7,0)$ и $(0.3,0,0.7)$. Один из них обязательно доминирует $z$, т.к. первому игроку любой дележ из $B$ предпочтительнее, а одновременное выполнение неравенств $z_{2}\geq 0.7$ и $z_{3}\geq 0.7$ невозможно.

Упражнение. Какие еще устойчивые множества есть в этой игре?
\end{myex}

Устойчивые множества не могут быть вложены друг в друга:
\begin{myth}
Если $X$ - устойчивое множество, то не существует устойчивого множества $Y$ целиком лежащего внутри $X$ и не совпадаюющего с $X$.
\end{myth}
\begin{proof}
Допустим, что $Y$ целиком лежит в $X$ и не совпадает с $X$. Значит можно выбрать две точки $x\neq y$, так, что $x\in X\backslash Y$, $y\in Y$. Получаем, что с одной стороны, $y$ должно быть возражением на $x$ (внешняя стабильность $Y$), а с другой стороны, $y$ не может быть возражением на $x$ (внутренняя стабильность $X$). Противоречие.
\end{proof}


\begin{myth}
Любое устойчивое множество содержит внутри себя ядро. Если ядро является устойчивым множеством, то других устойчивых множеств не существует.
\end{myth}
\begin{proof}
Определение ядра более строгое, чем определение устойчивого множества: в первом случае не должно быть ни одного возражения, во втором случае на возражения наложены ограничения. Поэтому ядро входит в любое устойчивое множество.

Как мы только что доказали, устойчивые множества не могут быть вложены друг в друга, а ядро входит в каждое устойчивое множество. Значит если ядро само по себе устойчиво, то других устойчивых множеств быть не может.
\end{proof}

К сожалению, решение Неймана-Моргенштерна иногда бывает пусто. Контрпример в виде игры с 10 игроками есть в статье Лукаса \cite{lucas:gns}.
\nocite{lemaire:cgtia}



