% для создания отдельной главы...
\documentclass[pdftex,12pt,a4paper]{article}

% jan 2012

% sudo yum install texlive-bbm texlive-bbm-macros texlive-asymptote texlive-cm-super texlive-cyrillic texlive-pgfplots texlive-subfigure
% yum install texlive-chessboard texlive-skaknew % for \usepackage{chessboard}
% yum install texlive-minted texlive-navigator texlive-yax texlive-texapi

% растягиваем границы страницы
%\emergencystretch=2em \voffset=-2cm \hoffset=-1cm
%\unitlength=0.6mm \textwidth=17cm \textheight=25cm

\usepackage{makeidx} % для создания предметных указателей
\usepackage{verbatim} % для многострочных комментариев
\usepackage{cmap} % для поиска русских слов в pdf
\usepackage[pdftex]{graphicx} % для вставки графики 
% omit pdftex option if not using pdflatex


%\usepackage{dsfont} % шрифт для единички с двойной палочкой (для индикатора события)
\usepackage{bbm} % шрифт - двойные буквы

\usepackage[colorlinks,hyperindex,unicode,breaklinks]{hyperref} % гиперссылки в pdf


\usepackage[utf8]{inputenc} % выбор кодировки файла
\usepackage[T2A]{fontenc} % кодировка шрифта
\usepackage[russian]{babel} % выбор языка

\usepackage{amssymb}
\usepackage{amsmath}
\usepackage{amsthm}
\usepackage{epsfig}
\usepackage{bm}
\usepackage{color}

\usepackage{multicol}


\usepackage{textcomp}  % Чтобы в формулах можно было русские буквы писать через \text{}

\usepackage{embedfile} % Чтобы код LaTeXа включился как приложение в PDF-файл

\usepackage{subfigure} % для создания нескольких рисунков внутри одного

\usepackage{tikz,pgfplots} % язык для рисования графики из latex'a
\usetikzlibrary{trees} % прибамбас в нем для рисовки деревьев
\usetikzlibrary{arrows} % прибамбас в нем для рисовки стрелочек подлиннее
\usepackage{tikz-qtree} % прибамбас в нем для рисовки деревьев


\usepackage{ifpdf} % чтобы проверять, запускаем мы pdflatex или просто latex

\ifpdf
	\usepackage[pdftex]{graphicx} 
	\DeclareGraphicsRule{*}{mps}{*}{} % все неупомянутые ps файлы объявляем упрощенными, т.е. mps типа. Просто ps графику нельзя использовать, но без некоторых спец. команд - можно. Например, результат работы metapost - это ps файлы простого (mps) типа. Собственно ради использования metapost эта строка и введена.
\else
	\usepackage{graphicx}
\fi



% конец добавки

\usepackage{asymptote} % After graphicx!, пакет для рисования графиков и прочего
%\usepackage{sagetex} % i suppose after graphicx also..., для связи с sage



\embedfile[desc={Исходный LaTeX файл}]{\jobname.tex} % Включение кода в выходной файл
\embedfile[desc={Стилевой файл}]{/home/boris/science/tex_general/title_bor_utf8.tex}



% вместо горизонтальной делаем косую черточку в нестрогих неравенствах
\renewcommand{\le}{\leqslant}
\renewcommand{\ge}{\geqslant} 
\renewcommand{\leq}{\leqslant}
\renewcommand{\geq}{\geqslant}

% делаем короче интервал в списках 
\setlength{\itemsep}{0pt} 
\setlength{\parskip}{0pt} 
\setlength{\parsep}{0pt}

% свешиваем пунктуацию (т.е. знаки пунктуации могут вылезать за правую границу текста, при этом текст выглядит ровнее)
\usepackage{microtype}

% более красивые таблицы
\usepackage{booktabs}
% заповеди из докупентации: 
% 1. Не используйте вертикальные линни
% 2. Не используйте двойные линии
% 3. Единицы измерения - в шапку таблицы
% 4. Не сокращайте .1 вместо 0.1
% 5. Повторяющееся значение повторяйте, а не говорите "то же"


% DEFS
\def \mbf{\mathbf}
\def \msf{\mathsf}
\def \mbb{\mathbb}
\def \tbf{\textbf}
\def \tsf{\textsf}
\def \ttt{\texttt}
\def \tbb{\textbb}

\def \wh{\widehat}
\def \wt{\widetilde}
\def \ni{\noindent}
\def \ol{\overline}
\def \cd{\cdot}
\def \bl{\bigl}
\def \br{\bigr}
\def \Bl{\Bigl}
\def \Br{\Bigr}
\def \fr{\frac}
\def \bs{\backslash}
\def \lims{\limits}
\def \arg{{\operatorname{arg}}}
\def \dist{{\operatorname{dist}}}
\def \VC{{\operatorname{VCdim}}}
\def \card{{\operatorname{card}}}
\def \sgn{{\operatorname{sign}\,}}
\def \sign{{\operatorname{sign}\,}}
\def \xfs{(x_1,\ldots,x_{n-1})}
\def \Tr{{\operatorname{\mbf{Tr}}}}
\DeclareMathOperator*{\argmin}{arg\,min}
\DeclareMathOperator*{\argmax}{arg\,max}
\DeclareMathOperator*{\amn}{arg\,min}
\DeclareMathOperator*{\amx}{arg\,max}
\def \cov{{\operatorname{Cov}}}

\def \xfs{(x_1,\ldots,x_{n-1})}
\def \ti{\tilde}
\def \wti{\widetilde}


\def \mL{\mathcal{L}}
\def \mW{\mathcal{W}}
\def \mH{\mathcal{H}}
\def \mC{\mathcal{C}}
\def \mE{\mathcal{E}}
\def \mN{\mathcal{N}}
\def \mA{\mathcal{A}}
\def \mB{\mathcal{B}}
\def \mU{\mathcal{U}}
\def \mV{\mathcal{V}}
\def \mF{\mathcal{F}}

\def \R{\mbb R}
\def \N{\mbb N}
\def \Z{\mbb Z}
\def \P{\mbb{P}}
%\def \p{\mbb{P}}
\def \E{\mbb{E}}
\def \D{\msf{D}}
\def \I{\mbf{I}}

\def \a{\alpha}
\def \b{\beta}
\def \t{\tau}
\def \dt{\delta}
\def \e{\varepsilon}
\def \ga{\gamma}
\def \kp{\varkappa}
\def \la{\lambda}
\def \sg{\sigma}
\def \sgm{\sigma}
\def \tt{\theta}
\def \ve{\varepsilon}
\def \Dt{\Delta}
\def \La{\Lambda}
\def \Sgm{\Sigma}
\def \Sg{\Sigma}
\def \Tt{\Theta}
\def \Om{\Omega}
\def \om{\omega}


\def \ni{\noindent}
\def \lq{\glqq}
\def \rq{\grqq}
\def \lbr{\linebreak}
\def \vsi{\vspace{0.1cm}}
\def \vsii{\vspace{0.2cm}}
\def \vsiii{\vspace{0.3cm}}
\def \vsiv{\vspace{0.4cm}}
\def \vsv{\vspace{0.5cm}}
\def \vsvi{\vspace{0.6cm}}
\def \vsvii{\vspace{0.7cm}}
\def \vsviii{\vspace{0.8cm}}
\def \vsix{\vspace{0.9cm}}
\def \VSI{\vspace{1cm}}
\def \VSII{\vspace{2cm}}
\def \VSIII{\vspace{3cm}}


\newcommand{\grad}{\mathrm{grad}}
\newcommand{\dx}[1]{\,\mathrm{d}#1} % для интеграла: маленький отступ и прямая d
\newcommand{\ind}[1]{\mathbbm{1}_{\{#1\}}} % Индикатор события
%\renewcommand{\to}{\rightarrow}
\newcommand{\eqdef}{\mathrel{\stackrel{\rm def}=}}
\newcommand{\iid}{\mathrel{\stackrel{\rm i.\,i.\,d.}\sim}}
\newcommand{\const}{\mathrm{const}}

%на всякий случай пока есть
%теоремы без нумерации и имени
%\newtheorem*{theor}{Теорема}

%"Определения","Замечания"
%и "Гипотезы" не нумеруются
%\newtheorem*{defin}{Определение}
%\newtheorem*{rem}{Замечание}
%\newtheorem*{conj}{Гипотеза}

%"Теоремы" и "Леммы" нумеруются
%по главам и согласованно м/у собой
%\newtheorem{theorem}{Теорема}
%\newtheorem{lemma}[theorem]{Лемма}

% Утверждения нумеруются по главам
% независимо от Лемм и Теорем
%\newtheorem{prop}{Утверждение}
%\newtheorem{cor}{Следствие}

%% jan 2012

% sudo yum install texlive-bbm texlive-bbm-macros texlive-asymptote texlive-cm-super texlive-cyrillic texlive-pgfplots texlive-subfigure
% yum install texlive-chessboard texlive-skaknew % for \usepackage{chessboard}
% yum install texlive-minted texlive-navigator texlive-yax texlive-texapi

% растягиваем границы страницы
%\emergencystretch=2em \voffset=-2cm \hoffset=-1cm
%\unitlength=0.6mm \textwidth=17cm \textheight=25cm

\usepackage{makeidx} % для создания предметных указателей
\usepackage{verbatim} % для многострочных комментариев
\usepackage{cmap} % для поиска русских слов в pdf
\usepackage[pdftex]{graphicx} % для вставки графики 
% omit pdftex option if not using pdflatex


%\usepackage{dsfont} % шрифт для единички с двойной палочкой (для индикатора события)
\usepackage{bbm} % шрифт - двойные буквы

\usepackage[colorlinks,hyperindex,unicode,breaklinks]{hyperref} % гиперссылки в pdf


\usepackage[utf8]{inputenc} % выбор кодировки файла
\usepackage[T2A]{fontenc} % кодировка шрифта
\usepackage[russian]{babel} % выбор языка

\usepackage{amssymb}
\usepackage{amsmath}
\usepackage{amsthm}
\usepackage{epsfig}
\usepackage{bm}
\usepackage{color}

\usepackage{multicol}


\usepackage{textcomp}  % Чтобы в формулах можно было русские буквы писать через \text{}

\usepackage{embedfile} % Чтобы код LaTeXа включился как приложение в PDF-файл

\usepackage{subfigure} % для создания нескольких рисунков внутри одного

\usepackage{tikz,pgfplots} % язык для рисования графики из latex'a
\usetikzlibrary{trees} % прибамбас в нем для рисовки деревьев
\usetikzlibrary{arrows} % прибамбас в нем для рисовки стрелочек подлиннее
\usepackage{tikz-qtree} % прибамбас в нем для рисовки деревьев


\usepackage{ifpdf} % чтобы проверять, запускаем мы pdflatex или просто latex

\ifpdf
	\usepackage[pdftex]{graphicx} 
	\DeclareGraphicsRule{*}{mps}{*}{} % все неупомянутые ps файлы объявляем упрощенными, т.е. mps типа. Просто ps графику нельзя использовать, но без некоторых спец. команд - можно. Например, результат работы metapost - это ps файлы простого (mps) типа. Собственно ради использования metapost эта строка и введена.
\else
	\usepackage{graphicx}
\fi



% конец добавки

\usepackage{asymptote} % After graphicx!, пакет для рисования графиков и прочего
%\usepackage{sagetex} % i suppose after graphicx also..., для связи с sage



\embedfile[desc={Исходный LaTeX файл}]{\jobname.tex} % Включение кода в выходной файл
\embedfile[desc={Стилевой файл}]{/home/boris/science/tex_general/title_bor_utf8.tex}



% вместо горизонтальной делаем косую черточку в нестрогих неравенствах
\renewcommand{\le}{\leqslant}
\renewcommand{\ge}{\geqslant} 
\renewcommand{\leq}{\leqslant}
\renewcommand{\geq}{\geqslant}

% делаем короче интервал в списках 
\setlength{\itemsep}{0pt} 
\setlength{\parskip}{0pt} 
\setlength{\parsep}{0pt}

% свешиваем пунктуацию (т.е. знаки пунктуации могут вылезать за правую границу текста, при этом текст выглядит ровнее)
\usepackage{microtype}

% более красивые таблицы
\usepackage{booktabs}
% заповеди из докупентации: 
% 1. Не используйте вертикальные линни
% 2. Не используйте двойные линии
% 3. Единицы измерения - в шапку таблицы
% 4. Не сокращайте .1 вместо 0.1
% 5. Повторяющееся значение повторяйте, а не говорите "то же"


% DEFS
\def \mbf{\mathbf}
\def \msf{\mathsf}
\def \mbb{\mathbb}
\def \tbf{\textbf}
\def \tsf{\textsf}
\def \ttt{\texttt}
\def \tbb{\textbb}

\def \wh{\widehat}
\def \wt{\widetilde}
\def \ni{\noindent}
\def \ol{\overline}
\def \cd{\cdot}
\def \bl{\bigl}
\def \br{\bigr}
\def \Bl{\Bigl}
\def \Br{\Bigr}
\def \fr{\frac}
\def \bs{\backslash}
\def \lims{\limits}
\def \arg{{\operatorname{arg}}}
\def \dist{{\operatorname{dist}}}
\def \VC{{\operatorname{VCdim}}}
\def \card{{\operatorname{card}}}
\def \sgn{{\operatorname{sign}\,}}
\def \sign{{\operatorname{sign}\,}}
\def \xfs{(x_1,\ldots,x_{n-1})}
\def \Tr{{\operatorname{\mbf{Tr}}}}
\DeclareMathOperator*{\argmin}{arg\,min}
\DeclareMathOperator*{\argmax}{arg\,max}
\DeclareMathOperator*{\amn}{arg\,min}
\DeclareMathOperator*{\amx}{arg\,max}
\def \cov{{\operatorname{Cov}}}

\def \xfs{(x_1,\ldots,x_{n-1})}
\def \ti{\tilde}
\def \wti{\widetilde}


\def \mL{\mathcal{L}}
\def \mW{\mathcal{W}}
\def \mH{\mathcal{H}}
\def \mC{\mathcal{C}}
\def \mE{\mathcal{E}}
\def \mN{\mathcal{N}}
\def \mA{\mathcal{A}}
\def \mB{\mathcal{B}}
\def \mU{\mathcal{U}}
\def \mV{\mathcal{V}}
\def \mF{\mathcal{F}}

\def \R{\mbb R}
\def \N{\mbb N}
\def \Z{\mbb Z}
\def \P{\mbb{P}}
%\def \p{\mbb{P}}
\def \E{\mbb{E}}
\def \D{\msf{D}}
\def \I{\mbf{I}}

\def \a{\alpha}
\def \b{\beta}
\def \t{\tau}
\def \dt{\delta}
\def \e{\varepsilon}
\def \ga{\gamma}
\def \kp{\varkappa}
\def \la{\lambda}
\def \sg{\sigma}
\def \sgm{\sigma}
\def \tt{\theta}
\def \ve{\varepsilon}
\def \Dt{\Delta}
\def \La{\Lambda}
\def \Sgm{\Sigma}
\def \Sg{\Sigma}
\def \Tt{\Theta}
\def \Om{\Omega}
\def \om{\omega}


\def \ni{\noindent}
\def \lq{\glqq}
\def \rq{\grqq}
\def \lbr{\linebreak}
\def \vsi{\vspace{0.1cm}}
\def \vsii{\vspace{0.2cm}}
\def \vsiii{\vspace{0.3cm}}
\def \vsiv{\vspace{0.4cm}}
\def \vsv{\vspace{0.5cm}}
\def \vsvi{\vspace{0.6cm}}
\def \vsvii{\vspace{0.7cm}}
\def \vsviii{\vspace{0.8cm}}
\def \vsix{\vspace{0.9cm}}
\def \VSI{\vspace{1cm}}
\def \VSII{\vspace{2cm}}
\def \VSIII{\vspace{3cm}}


\newcommand{\grad}{\mathrm{grad}}
\newcommand{\dx}[1]{\,\mathrm{d}#1} % для интеграла: маленький отступ и прямая d
\newcommand{\ind}[1]{\mathbbm{1}_{\{#1\}}} % Индикатор события
%\renewcommand{\to}{\rightarrow}
\newcommand{\eqdef}{\mathrel{\stackrel{\rm def}=}}
\newcommand{\iid}{\mathrel{\stackrel{\rm i.\,i.\,d.}\sim}}
\newcommand{\const}{\mathrm{const}}

%на всякий случай пока есть
%теоремы без нумерации и имени
%\newtheorem*{theor}{Теорема}

%"Определения","Замечания"
%и "Гипотезы" не нумеруются
%\newtheorem*{defin}{Определение}
%\newtheorem*{rem}{Замечание}
%\newtheorem*{conj}{Гипотеза}

%"Теоремы" и "Леммы" нумеруются
%по главам и согласованно м/у собой
%\newtheorem{theorem}{Теорема}
%\newtheorem{lemma}[theorem]{Лемма}

% Утверждения нумеруются по главам
% независимо от Лемм и Теорем
%\newtheorem{prop}{Утверждение}
%\newtheorem{cor}{Следствие}

%\input{e:/Documents/tex_general/prob_and_sol_utf8}
%\input{prob_and_sol_utf8}

%\usepackage{showkeys} % показывать метки
%\usepackage{verbatim}

\newcommand{\textbf}[1]{\textbf{#1}}

\numberwithin{equation}{page} % уравнения нумеруются на каждой стр. отдельно

\newtheorem{theorem}[equation]{Теорема} % нумерация сквозная с уравнениями
\theoremstyle{definition} % убирает курсив и что-то еще наверное делает ;)
\newtheorem{definition}[equation]{Определение}
\theoremstyle{definition}
\newtheorem{myex}[equation]{Пример}
%\newtheorem{assertion}{Утверждение}
%\newtheorem{lemma}{Лемма}
\theoremstyle{definition}
\newtheorem*{myproof}{Доказательство}

\title{Дистанционная авантюра.}
\author{Борис Демешев}
\date{\today}

%\makeindex % команда для создания предметного указателя
%\bibliographystyle{plain} % стиль оформления ссылок

\begin{document}

%\maketitle
%\tableofcontents{}

\section{Ядро и вектор Шепли. Первое знакомство}

\subsection{Игра в характеристической форме}

Начнем с обозначений:

$N$ - множество всех игроков, а $n$ - их количество.

\begin{definition}
Коалиция (coalition) - подмножество игроков.
\end{definition}

\begin{definition}
Большая коалиция (grand coalition) - синоним для множества всех игроков.
\end{definition}


\begin{definition}
Коалиционная игра в характеристической форме (coalitional game или cooperative game in characteristic form) это:

1. Множество игроков $N$.

2. Характеристическая функция (characteristic function), $v$, сопоставляющая каждой коалиции сумму денег, которую эта коалиция может заработать самостоятельно.
\end{definition}

Характеристическая функция может принимать отрицательные значения (например, при дележе расходов). Мы считаем, что пустая коалиция (куда никто не входит), не может заработать денег и никому ничего не должна, т.е. $v(\emptyset)=0$. Именно характеристическая функция полностью описывает игру в нашем случае.

\begin{myex} "Ботинки". Пара ботинок (левый плюс правый) стоит 600 рублей. Один ботинок без пары не стоит ничего. У Лени есть левый ботинок, у Левы - еще один такой же левый, а у Паши - правый. Здесь $N=\{$Леня,Лева,Паша$\}$, $v($Леня$)=v($Лева$)=v($Паша$)=0$ (в одиночку никто не может получить 600 рублей); $v($Леня,Лева$)=0$ (у них нет правого); для любой другой коалиции $S$, $v(S)=600$, т.к. есть и правый и левый ботинки. 
\end{myex}

\begin{myex} "Носки". Левые и правые носки ничем не отличаются. Пара носков стоит 60 рублей. Один носок ничего не стоит. У Андрея - три носка, у Бориса - пять носков. Здесь $N=\{$Андрей,Борис$\}$, $v($Андрей$)=60$, $v($Борис$)=120$, $v($Андрей, Борис$)=240$.
\end{myex}


Все упомянутые примеры обладают свойством, которое называется супераддитивностью (superadditivity):

Если несколько непересекающихся коалиций объединяются, то вместе как одна коалиция они заработают не меньше, чем по отдельности.

\begin{definition}
Игра называется супераддитивной, если для любых непересекающихся коалиций $S_{1}$, $S_{2}$, верно неравенство $v(S_{1}\cup S_{2})\geq v(S_{1})+v(S_{2})$.
\end{definition}

В такой игре у игроков есть интерес в создании большой коалиции и дележе полученного $v(N)$.

Вопрос в том, как поделить $v(N)$?

Мы обсудим две концепции решения - ядро и вектор Шепли.

\subsection{Ядро.}

Предположим, что большая коалиция решила каким-то образом разделить $v(N)$. С математической точки зрения, дележ - это вектор $x=(x_{1},x_{2},...,x_{n})$. Поскольку большая коалиция может заработать $v(N)$, то любой дележ обязан удовлетворять бюджетному ограничению $x_{1}+x_{2}+x_{3}+...+x_{n}\leq v(N)$. Чего мы хотим от дележа?

Собственно, всегда хочется двух условий: эффективность и устойчивости (в каких-нибудь смыслах).

Эффективность - это отсутствие потерь: вся доступная сумма $v(N)$ должна распределяться между игроками без остатка, т.е. неравенство $x_{1}+x_{2}+x_{3}+...+x_{n}\leq v(N)$ должно быть выполнено как равенство.

\begin{definition}. Условие \textbf{эффективности}. Дележ называется эффективным, если $x_{1}+x_{2}+x_{3}+...+x_{n}=v(N)$.
\end{definition}


Об устойчивости. Хочется, чтобы среди игроков не было сепаратистских тендеций. Допустим какой-нибудь коалиции $S$ при дележе достается меньше, чем та сумма, которую она может заработать самостоятельно. В таком случае игроки входящие в $S$ не захотят участвовать в большой коалиции, отсоединятся и получат больше. Отсоединившись, коалиция $S$ получает $v(S)$; а соглашаясь на дележ - получает $\sum_{i\in S} x_{i}$. Отсюда возникает:

\begin{definition}
Условие \textbf{отсутствия сепаратистских тенденций}. Дележ удовлетворяет условию отсутствия сепаратистских тендеций если для любой коалиции $S$: $\sum_{i\in S} x_{i}\geq v(S)$.
\end{definition}

\begin{definition} Ядро (Core) - это множество дележей, удовлетворяющих условиям:

1) Эффективности

2) Отсутствие сепаратистских тендеций.
\end{definition}

Находим ядро в наших примерах. Исходя из определения - ядро можно найти решая систему из одного уравнения и нескольких неравнеств.

\begin{myex}. Ботинки. $x_{l1}+x_{l2}+x_{r}=1$, $x_{l1}+x_{r}\geq 1$, $x_{l2}+x_{r}\geq 1$. Единственное решение - это $x_{l1}=x_{l2}=0$, $x_{r}=1$. Обладатель редкого ресурса получает все!
\end{myex}

\begin{myex}. Носки. $x_{1}+x_{2}=240$, $x_{1}\geq 60$, $x_{2}\geq 120$. Решение: любой дележ вида: $(x_{1},240-x_{1})$, где $x_{1}\in [60;120]$.
\end{myex}

Недостатки ядра. Во-первых, ядро бывает пустым. Оно бывает пустым из-за того, что условие полного отсутствия сепаратистских тенденций слишком сильное. Во-вторых, ядро бывает не единственным.

Эти две проблемы исправляет другая концепция - вектор Шепли (Shapley value). Он всегда существует и всегда единственный.

\subsection{Вектор Шепли}

Допустим игроки у нас занумерованы в некотором порядке $\pi$. Здесь $\pi$ - это не 3.14..., а некая последовательность чисел от 1 до $n$, например, $\{1,3,2,4,5\}$. Будем формировать большую коалицию добавляя игроков по одному в указанном порядке. Когда мы добавляем $i$-го игрока у нас уже сформирована некоторая коалиция $S$. Присоединяясь к этой коалиции $S$, игрок $i$ увеличивает достижимый выигрыш на $v(S\cup \{i\})-v(S)$. Назовем эту прибавку вкладом $i$-го игрока в большую коалицию, обозначим ее $Add(i,\pi)$. Обозначение не общепринятое, но мне кажется его введение оправдано.

Конечно же, вклад $i$-го игрока в большую коалицию, $Add(i,\pi)$, зависит от порядка формирования большой коалиции $\pi$. 

Например, в игре "Ботинки". Если формировать большую коалицию в порядке Леня, Лева, Паша, то вклад Левы равен нулю. Если формировать большую коалицию в порядке Паша, Лева, Леня, то вклад Левы равен 600 рублей.

Если же формировать большую коалицию добавляя игроков по одному в случайном порядке $\pi$, то прибавка, вносимая $i$-м игроком, $Add(i)$ - будет случайной величиной.

Вектор Шепли - это вектор $(\phi_{1},\phi_{2},...,\phi_{N})$, где выигрыш $i$-го игрока, $\phi_{i}$ определяется по принципу:

$\phi_{i}=E(Add(i))$.

\begin{definition} Вектор Шепли - это математическое ожидание вклада каждого игрока, если большая коалиция формируется в случайном порядке.
\end{definition}

Из определения ясно, что вектор Шепли всегда (по крайней мере при конечном числе игроков) существует и всегда единственный.

В наших примерах:

\begin{myex} Ботинки. Найдем $E(Add(r))$. Если Паша входит первым, то его вклад равен нулю, иначе его вклад равен 600. Значит $E(Add(r))=\frac{2}{3}600=400$. Вклад Лени равен 600 только если первым вошел Паша, а вторым - Леня. Значит $E(Add(l1))=\frac{1}{6}600=100$. Аналогично для Левы. Значит вектор Шепли равен: $(100,100,400)$.
\end{myex}

\begin{myex}. Носки. Возможно всего два порядка формирования большой коалиции: Андрей-Борис и Борис-Андрей. В первом случае вклады игроков равны $Add(a,ab)=60$, $Add(b,ab)=180$, во втором - $Add(b,ba)=120$ и $Add(a,ba)=120$. И вектор Шепли: $\phi_{a}=90$, $\phi_{b}=150$. Что соответствует интуитивному дележу в пропорции $3/5$.
\end{myex}


\begin{theorem} Вектор Шепли удовлетворяет требованию эффективности. $\sum \phi_{i}=v(N)$.
\end{theorem}
\begin{proof}
$$\sum \phi_{i}=\sum E(Add(i))=E (\sum Add(i));$$
Мы определяли $Add(i)$ как вклад $i$-го игрока при пошаговом построении большой коалиции. Поэтому суммарный вклад всех игроков всегда равен стоимости большой коалиции, $\sum Add(i)=v(N)$. Каждое $Add(i)$ - случайная величина, но их сумма (вне зависимости от порядка формирования большой коалиции) равна константе $v(N)$.
$$E(\sum Add(i))=E(v(N))=v(N)$$
\end{proof}

Лирическое отступление. Немного критики. В играх, которые мы рассматривали, предполагалось, что выигрыш (или полезность) можно перераспределять между игроками. Любая коалиция $S$ заработав $v(S)$ могла разделить этот выигрыш $v(S)$ между своими участниками произвольным образом. 

Насколько это реалистично? 

Деньги, конечно, можно делить. Но вот полезность они приносят разную, достаточно представить себе, что в игре "Носки" Андрей - бедный пенсионер, а Борис - богатый менеджер. 

Передаваема ли полезность? 

В одной из серий мультфильма про Удава, Попугая, Мартышку и Слоненка, Удав передавал Мартышке привет. У него было хорошое настроение. Как известно, первые два "Привета" кое-кто потерял, и только третий "Привет" достался Мартышке.

Отсюда вывод: деньги легко передаются, но не отражают полезность; полезность передеатся плохо.




\end{document}