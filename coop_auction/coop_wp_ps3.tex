\subsection{Дз 3}

 \textbf{Задача 1.}
 У первого игрока есть $l$ литров левой полуфилософской
жидкости. У второго игрока есть $m>l$ литров правой полуфилософской
жидкости. При смешивании 1 литра левой и одного литра правой полуфилософской
жидкостей получается 1 кг золота. Полуфилосовская жидкость стоит 1
рубль за литр, золото - 3 рубля за килограмм. Полезность от денег
задана функцией $u(m)=\sqrt{m}$. Как поделить полезность между игроками?
(найдите и решение Нэша и решение Калаи-Смородински).

 \textbf{Задача 2.}
 Рассмотрим коалиционную игру двух игроков в характеристической
форме.

Верно ли, что решение Нэша всегда совпадает с вектором Шепли? Докажите
или приведите контр-пример.

Верно ли, что решение Калаи-Смородински всегда совпадает с вектором
Шепли? Докажите или приведите контр-пример.

Верно ли, что решение Нэша и Калаи-Смородински всегда совпадают? Докажите
или приведите контр-пример.

 \textbf{Задача 3.}
 Пусть имеется задача торга $(X,d).$Рассмотрим связанную
с ней некооперативную игру.

Первый игрок предлагает дележ $x^{I}\in X$

Второй игрок предлагает дележ $x^{II}\in X$ и вероятность $p\in[0;1]$.

С вероятностью $p$ игра заканчивается и игроки получают точку несогласия
$d$. С вероятностью $(1-p)$ игра продолжается:

Первый игрок выбирает в качестве финального дележа либо предложенный
им в начале игры дележ $x^{I}$, либо лотерею $px^{II}$. 

Верно ли, что совершенное в подыграх равновесие в этой игре совпадает
с решением Нэша задачи торга? С решением Калаи-Смородински?

 \textbf{Задача 4.} Докажите, что решение Калаи-Смородинского - единственное
решение, удовлетворяющее условиям эффективности, симметрии, нечувствительности
к смене масштаба, индивидуальной рациональности и индивидуальной монотонности. 

 \textbf{Задача 5.} Какое решение получится, если известно, что оно удовлетворяет
условиям индивидуальной рациональности, эффективности, симметрии,
индивидуальной монотонности и независимости от третьих альтернатив?