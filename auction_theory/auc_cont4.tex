\documentclass[pdftex,12pt,a4paper]{article}

\input{/home/boris/Dropbox/Public/tex_general/title_bor_utf8}

%\usepackage{showkeys} % показывать метки

\input{/home/boris/Dropbox/Public/tex_general/prob_and_sol_utf8}

%\title{Задачи по элементарной теории вероятностей и матстатистике}
%\author{Составитель: Борис Демешев, boris.demeshev@gmail.com}
%\date{\today}

\begin{document}

%\pagestyle{myheadings} \markboth{ТВИМС-задачник. Демешев Борис. roah@yandex.ru }{ТВИМС-задачник. Демешев Борис. roah@yandex.ru }
%\maketitle
%\tableofcontents{}

%\parindent=0 pt % отступ равен 0

\begin{Large}
Моделирование аукционов. Контрольная работа 4.
\end{Large}

\begin{enumerate}
\item Можно пользоваться калькулятором. Вопрос в том, нужно ли?
\item Можно решать задачи в любом порядке.
\item С собой можно принести один лист А4, где заранее могут быть написаны (именно написаны, а не напечатаны) любые формулы, теоремы или комментарии.
\item Продолжительность работы 1 час 20 минут.
\item Условия нельзя забрать с собой. Условия и решения открыто доступны на \url{auctiontheory.wordpress.com} после окончания контрольной.
\item Обсуждать задачи во время работы нельзя.
\item Человек проводящий контрольную не будет отвечать на вопросы по тексту задач. 
\item Скорее всего, в задачах нет очепяток. Если, по твоему мнению, опечатка есть, то ее нужно исправить самому исходя из своего представления о хорошей задаче. При этом нужно четко отразить этот факт перед началом решения. Например, <<По-моему, в тексте есть опечатка и вместо ... должно быть ...>>. Твоя гипотеза об опечатках является личной и не подлежит обсуждению во время работы.
\item Насколько подробно все расписывать --- решай сам исходя из конкретной ситуации. Очевидно, что в примере $ 1+2+3=? $ ответ можно написать сразу, а взятие интеграла $ \int x^{5}\cos(x)dx $ требует каких-то промежуточных записей.
\item Паниковать на контрольной строжайше запрещено!
%\item Каждая из 5 задач весит 5 баллов.
\item Для каждой задачи обязательно нужно спрогнозировать свою оценку. Не надо скромничать, лучше попытаться объективно оценить свое решение.  За неверное оценивание баллы снижаться не будут, а верное оценивание даст возможность чему-то научиться. Опыт показывает, что оценка своих собственных решений позволяет резко улучшить их качество. Прогноз своей оценки пишем в табличку!
\item Не забудь подписать свою работу. Пожалуйста!

\end{enumerate}

\begin{large}
Имя:
\end{large}

\vspace{4pt}

\begin{large}
Отчество:
\end{large}

\vspace{4pt}

\begin{large}
Фамилия:
\end{large}

\vspace{4pt}

\begin{large}
Группа:
\end{large}

\vspace{4pt} 

\begin{tabular}{|c|c|c|c|c|c|c|}
\hline  & Задача 1 & Задача 2 & Задача 3 & Задача 4 & Итого \\ 
\hline Прогноз оценки &  &  &  &  &   \\ 
\hline Оценка (от 0 до 5) &  &  &  &  &   \\ 
\hline 
\end{tabular} 
\newpage

\begin{enumerate}

\item На аукционе участвуют $ n $ игроков. Пусть функция распределения сигналов имеет вид $ F(x)=x^{a} $ на $ [0;1] $, где $ a $ --- это некая константа, $ a\geq 1 $. 
\begin{enumerate}
\item Найдите $ MR(x) $. Является ли $ MR(x) $ возрастающей?
\item Постройте оптимальный аукцион.
\end{enumerate}

\item Петя переезжает на новую квартиру, поэтому продает свои старые шкаф и комод (варианта взять их с собой у него нет).  Потенциальных покупателей двое. Первый покупатель знает значение $ X_{1} $, второй --- значение $ X_{2} $. Величины  $ X_{1} $ и  $ X_{2} $ независимы и равномерны на $ [0;1] $. Полезности первого игрока: от шкафа --- $ 0.5 $, от комода --- $ 0.8X_{1} $, от шкафа и комода --- $ 0.5+X_{1} $. Полезности второго игрока: от шкафа --- $ 0.8 $, от комода --- $ X_{2} $, от шкафа и комода --- $ 0.8+0.8X_{2}$
\begin{enumerate}
\item Четко опишите механизм VCG применительно к этой задаче.
\item Какова средняя прибыль продавца при использовании механизма VCG?
\end{enumerate}


\item Есть $ n $ городов. Рядом с одним из них нужно построить мусоросжигательный завод. Жители города рядом с которым будет построен завод получат отрицательную полезность $ U_{i}=-X_{i} $. Остальные получат полезность 0. Величины $ X_{i}\sim U[0;1] $ и независимы. Каждый город знает свое $ X_{i} $. 
\begin{enumerate}
\item Опишите механизм VCG применительно к этой задаче. Т.е. предполагается, что игроки объявляют числа $ b_{i}\in [0;1] $ и механизм должен определять, у какого города строить завод и какие платежи должны сделать игроки в зависимости от $ b_{i} $.
\item Выпишите функцию плотности для компенсации, которую получают жители города рядом с которым будет построен мусоросжигательный завод.
\item Сходится ли баланс у механизма VCG в этом случае? Если нет, то сколько в среднем нужно вложить средств извне в этот механизм?
\item Что больше: компенсация или ущерб от строительства завода в механизме VCG?
\end{enumerate}


%\item Есть $ 3 $ города. Рядом с одним из них нужно построить мусоросжигательный завод. Жители города рядом с которым будет построен завод получат отрицательную полезность $ U_{i}=-X_{i} $. Остальные получат полезность 0. Величины $ X_{i}\sim U[0;1] $ и независимы. Каждый город знает свое $ X_{i} $. Города одновременно называют требуемую компенсацию $ b_{i} $. Завод строится у того города, у которого $ b_{i} $ меньше. Остальные города выплачивают компенсацию поровну.
%\begin{enumerate}
%\item Найдите равновесие Нэша
%\item Как надо изменить этот механизм, чтобы он стал правдивым?
%\end{enumerate}



%\item Есть один покупатель и один продавец. Ценности товара: $ X_{1} $ --- для покупателя, $ X_{2} $ --- для продавца. Величины $ X_{i} $ независимы и равномерны на $ [0;1] $.
%\begin{enumerate}
%\item Опишите механизм VCG применительно к этой задаче. Т.е. предполагается, что игроки объявляют числа $ b_{i}\in [0;1] $ и механизм должен определять, кому отдать товар и какие платежи должны сделать игроки в зависимости от $ b_{i} $.
%\item Каков средний баланс механизма VCG в этой задаче?
%\item Предположим, что вместо VCG используется такой механизм: игроки одновременно называют желаемые цены, $ b_{1} $ и $ b_{2} $. Если $ b_{1}>b_{2} $, то обмен происходит по цене $ 0.5(b_{1}+b_{2}) $. Найдите равновесие Нэша.
%\item Верно ли, что при втором механизме обмен происходит если и только если $ X_{1}>X_{2} $?
%\end{enumerate}

%Hint: равновесие Нэша будет в линейных стратегиях



\item Кнопочный аукцион и три игрока. Ценности $ V_{1} $, $ V_{2} $ и $ V_{3} $ равномерны на $ [0;1] $ и независимы. Первый и второй игрок знают значение своих ценностей, т.е. $ X_{1}=V_{1} $ и $ X_{2}=V_{2} $. А третий игрок --- не знает! 
\begin{enumerate}
\item Что собой представляют стратегии игроков в этом случае? Почему их можно упростить?
\item Найдите равновесие Нэша
\end{enumerate}
\end{enumerate}

\bibliography{/home/boris/Dropbox/Public/tex_general/opit}
\printindex % печать предметного указателя здесь

\end{document}