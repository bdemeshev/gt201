\documentclass[pdftex,12pt,a4paper]{article}

\input{/home/boris/Dropbox/Public/tex_general/title_bor_utf8}

%\usepackage{showkeys} % показывать метки

\input{/home/boris/Dropbox/Public/tex_general/prob_and_sol_utf8}

%\title{Задачи по элементарной теории вероятностей и матстатистике}
%\author{Составитель: Борис Демешев, boris.demeshev@gmail.com}
%\date{\today}

\begin{document}

%\pagestyle{myheadings} \markboth{ТВИМС-задачник. Демешев Борис. roah@yandex.ru }{ТВИМС-задачник. Демешев Борис. roah@yandex.ru }
%\maketitle
%\tableofcontents{}

%\parindent=0 pt % отступ равен 0

\begin{Large}
Моделирование аукционов. Контрольная работа 3.
\end{Large}

\begin{enumerate}
\item Можно пользоваться калькулятором. Вопрос в том, нужно ли?
\item Можно решать задачи в любом порядке.
\item С собой можно принести один лист А4, где заранее могут быть написаны (именно написаны, а не напечатаны) любые формулы, теоремы или комментарии.
\item Продолжительность работы 1 час 20 минут.
\item Условия нельзя забрать с собой. Условия и решения открыто доступны на \url{auctiontheory.wordpress.com} после окончания контрольной.
\item Обсуждать задачи во время работы нельзя.
\item Человек проводящий контрольную не будет отвечать на вопросы по тексту задач. 
\item Скорее всего, в задачах нет очепяток. Если, по твоему мнению, опечатка есть, то ее нужно исправить самому исходя из своего представления о хорошей задаче. При этом нужно четко отразить этот факт перед началом решения. Например, <<По-моему, в тексте есть опечатка и вместо ... должно быть ...>>. Твоя гипотеза об опечатках является личной и не подлежит обсуждению во время работы.
\item Насколько подробно все расписывать --- решай сам исходя из конкретной ситуации. Очевидно, что в примере $ 1+2+3=? $ ответ можно написать сразу, а взятие интеграла $ \int x^{5}\cos(x)dx $ требует каких-то промежуточных записей.
\item Паниковать на контрольной строжайше запрещено!
%\item Каждая из 5 задач весит 5 баллов.
\item Для каждой задачи обязательно нужно спрогнозировать свою оценку. Не надо скромничать, лучше попытаться объективно оценить свое решение.  За неверное оценивание баллы снижаться не будут, а верное оценивание даст возможность чему-то научиться. Опыт показывает, что оценка своих собственных решений позволяет резко улучшить их качество. Прогноз своей оценки пишем в табличку!
\item Не забудьде подписать свою работу. Пожалуйста!

\end{enumerate}

\begin{large}
Имя:
\end{large}

\vspace{4pt}

\begin{large}
Отчество:
\end{large}

\vspace{4pt}

\begin{large}
Фамилия:
\end{large}

\vspace{4pt}

\begin{large}
Группа:
\end{large}

\vspace{4pt} 

\begin{tabular}{|c|c|c|c|c|c|c|}
\hline  & Задача 1 & Задача 2 & Задача 3 & Задача 4 & Задача 5 & Итого \\ 
\hline Прогноз оценки &  &  &  &  &  & \\ 
\hline Оценка (от 0 до 5) &  &  &  &  &  & \\ 
\hline 
\end{tabular} 
\newpage

\begin{enumerate}


\item Пусть $  V $ --- общая ценность товара для двух игроков, равномерна на $ [0;1] $. ... ... ... ... ... ... ... ... ... ... ... ... ... ... ... ...... ... ... ... ... ... ... ...... ... ... ... ... ... ... ...

\begin{enumerate}
\item Найдите совместную функцию плотности $ X_{1} $ и $ X_{2} $. Верно ли, что $ X_{1} $ и $ X_{2} $ аффилированны?
\item Найдите $ v(x,y)=E(V|X_{1}=x,Y_{1}=y) $ 
\item Найдите совместную функцию плотности $ X_{1} $ и $ Y_{1} $, $ g(x,y) $
\end{enumerate}


\item На аукционе продается картина, которая равновероятно является <<Джокондой>> Леонардо да Винчи или ее подделкой. За нее торгуются $ n $ покупателей. Ценность картины для всех покупателей одинакова, $ V_{1}=V_{2}=...=V_{n}=V $ и равна 1, если это оригинал и 0, если подделка.

... ... ... ... ... ... ... ... ... ... ... ... ... ... ... ...... ... ... ... ... ... ... ...... ... ... ... ... ... ... ...
... ... ... ... ... ... ... ...... ... ... ... ... ... ... ...
\begin{enumerate}
\item Найдите совместную функцию плотности всех $ X_{i} $. Верно ли, что все $ X_{i} $ аффилированны?
\item Найдите $ v(x,y)=E(V|X_{1}=x,Y_{1}=y) $
\item Найдите совместную функцию плотности $ X_{1} $ и $ Y_{1} $, $ g(x,y) $
\end{enumerate}

\item На аукционе ... присутствуют $ n $ покупателей. Ценности совпадают с сигналами, $ V_{i}=X_{i} $; сигналы $ X_{i} $ независимы и равномерны на $ [0;1] $. На аукционе продается $k$ одинаковых чудо-швабр, $ 1<k<n $. Каждому покупателю нужна только одна чудо-швабра. Покупатели одновременно делают свои ставки. Чудо-швабры достаются по одной каждому из $ k $ покупателей с самыми высокими ставками. ...

Найдите ...


\item На аукционе ... присутствуют $ n $ покупателей. Ценности совпадают с сигналами, $ V_{i}=X_{i} $; сигналы $ X_{i} $ независимы и равномерны на $ [0;1] $. На аукционе продается $k$ одинаковых чудо-швабр, $ 1<k<n $. Каждому покупателю нужна только одна чудо-швабра. Покупатели одновременно делают свои ставки. Чудо-швабры достаются по одной каждому из $ k $ покупателей с самыми высокими ставками. ...

Найдите ...

Hint: Когда продавался один товар, то условие победы первого игрока --- $ Y_{1}<a $, а если продается $ k $ товаров, то условие победы первого игрока $ Y_{?}<a $.



\item Существуют ли неаффилированные случайные величины $ X_{1} $ и $ X_{2} $ такие, что ... ?


\end{enumerate}


Подсказка: по-моему, задача 2 дольше задачи 1, задача 4 дольше задачи 3.


\bibliography{/home/boris/Dropbox/Public/tex_general/opit}
\printindex % печать предметного указателя здесь

\end{document}