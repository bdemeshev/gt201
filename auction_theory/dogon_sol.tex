\begin{enumerate}
\item Кнопка <<Buy now!>>

Два игрока торгуются за товар на кнопочном аукционе с возможностью немедленной покупки товара. Ценности $ X_{i}=V_{i} $ независимы и равномерны на $ [0;1] $. Каждый игрок знает свою ценность $ X_{i} $. Продавец дает игрокам возможность купить товар немедленно по фиксированной цене $ a $. Подробнее. В начале аукциона текущая цена равна нулю и оба игрока жмут на свои кнопки. Текущая цена растет с течением времени. Кто первый отпустил свою кнопку, тот проиграл. В этот момент аукцион заканчивается и победитель получает товар по текущей цене. Но в любой момент пока аукцион не закончился, любой игрок может сказать: <<Покупаю по цене $ a $>>. В этом случае ему достается товар по цене $ a $ и аукцион заканчивается.
\begin{enumerate}
\item Что является стратегией игрока на этом аукционе?
\item Применима ли в данном случае теорема об одинаковой доходности?
\item Найдите равновесие Нэша
\end{enumerate}

\item Есть шесть покупателей. У продавца две чудо-швабры. Каждый покупатель хочет только одну чудо-швабру. Продавец решил продавать эти две чудо швабры путем двух последовательных аукционов первой цены, на каждом из которых будет выставляться одна чудо-швабра. Каждый игрок знает ценность чудо-швабры для себя, $ X_{i}=V_{i} $. Ценности независимы и равномерны на $ [0;1] $. Ценности не меняются со временем. Когда проводится второй аукцион известна только ставка, которую сделал победитель первого.

\begin{enumerate}
\item Применима ли в данном случае теорема об одинаковой доходности или ее небольшая вариация?
\item Что является стратегией игрока в этой игре?
\item Найдите равновесие Нэша
\item Верно ли, что средние цены на обоих аукционах равны?
\item Какова вероятность того, что на первом аукционе цена будет больше, чем на втором?
\end{enumerate}

\item В моделях аукциона первой и второй цены с независимыми, равномерными на $ [0;1] $ ценностями покупателей сравните дисперсию выигрыша продавца. 

%Можно ли сказать сделать какой-то вывод\footnote{Этот вопрос является исследовательским. Возможно он очень легкий или наоборот очень сложный --- не знаю. Оценивается любое продвижение вперед.}  для произвольного регулярного распределения ценностей?

\item Может ли цена расти с ростом предложения?

Рассмотрим кнопочный аукцион, в котором участвуют три игрока. Продавец продает  одинаковых чудо-швабр. Каждому игроку нужна только одна чудо-швабра. Ценность чудо-швабры для всех игроков одинакова и равна $ V=X_{1}+X_{2}+X_{3} $. Каждый из игроков знает только свое $ X_{i} $. Сигналы $ X_{i} $ независимы и имеют регулярное распределение $ F(t) $. Чудо-швабры по одной достаются тем игрокам, кто отпустил кнопку позже всех. 

\begin{enumerate}
\item Найдите равновесие Нэша для случая $ k=1 $
\item Найдите равновесие Нэша для случая $ k=2 $
\item Приведите пример распределения $ F(t) $ при котором средняя цена растет с увеличением предложения с $ k=1 $ до $ k=2 $
\end{enumerate}

$ k=1 $: $ b^{3}(x)=3x $, $ b^{2}(x,p_{3})=2x+p/3$

$ k=2 $. Поскольку аукцион заканчивается при выходе первого игрока, то стратегия определяется функцией $ b^{3}(x)$. 

Поскольку мы такой аукцион не решали, то используем стандартный подход с максимизацией прибыли:
\begin{equation}
E(Profit_{1}|X_{1}=x,Bid_{1}:=b_{1})=E((X_{1}+X_{2}+X_{3}-b(Y_{2}))1_{b_{1}>b(Y_{2})}|X_{1}=x,Bid_{1}:=b_{1})
\end{equation}

Чудо-замена $b_{1}=b(a)$ и независимость $ X_{i} $ дают нам:
\begin{equation}
\pi_{1}(x,b(a))=E((x+X_{2}+X_{3})-b(Y_{2}))1_{a>Y_{2}})=xP(Y_{2}<a)+2E(X_{2}\cdot 1_{Y_{2}<a})-E(b(Y_{2})\cdot 1_{Y_{2}<a})
\end{equation}

Сосредоточимся на $ E(X_{2}\cdot 1_{Y_{2}<a}) $:
\begin{equation}
E(X_{2}\cdot 1_{Y_{2}<a})=E(X_{2}\cdot 1_{X_{2}\wedge X_{3}<a})=E(X_{2}\cdot 1_{X_{2}<a}\cdot 1_{X_{2}<X_{3}})+E(X_{2}\cdot 1_{X_{3}<a}\cdot 1_{X_{3}<X_{2}})
\end{equation}




\item Может ли цена расти с падением спроса?

Рассмотрим кнопочный аукцион, в котором могли бы участвовать три игрока. Продается одна чудо-швабра. Ценность чудо-швабры для всех игроков одинакова и равна $ V=X_{1}+X_{2}+X_{3} $. Каждый из трех потенциальных игроков знает только свое $ X_{i} $. Сигналы $ X_{i} $ независимы и имеют регулярное распределение $ F(t) $. Перед началом аукциона продавец случайным образом выбирает одного игрока и говорит: <<Ты мне не нравишься, поэтому ты в аукционе не участвуешь>>. Оставшиеся двое участвуют в аукционе. 

\begin{enumerate}
\item Найдите равновесие Нэша
\item Приведите пример распределения $ F(t) $ при котором средняя цена растет с падением спроса
\end{enumerate}

Hint: Задача 7 из лекции 3




\end{enumerate}
