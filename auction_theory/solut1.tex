

\begin{enumerate}
\item Предположим, что условия теоремы об одинаковых доходностях выполнены. 
\begin{enumerate}
\item Может ли выбор механизма проведения аукциона влиять на ковариацию выплат двух разных игроков?
\item  Найдите ковариацию выплат первого и второго игрока в аукционе первой цены с независимыми и равномерными на $ [0;1] $ ценностями. Hint: можно пользоваться тем, что средняя выплата равна $ \frac{n-1}{n(n+1)} $.
\end{enumerate}

Solution:

\begin{equation}
Cov(Pay_{1},Pay_{2})=\E(Pay_{1}\cdot Pay_{2})-\E(Pay_{1})\E(Pay_{2}) 
\end{equation}

На вычитаемое способ аукциона влиять не может в силу теоремы об одинаковой доходности. Сосредоточимся на $\E(Pay_{1}\cdot Pay_{2})$ . В аукционе первой цены никакие два игрока не могут платить одновременно, поэтому произведение выплат всегда равно нулю, т.е. $ \E(Pay_{1}\cdot Pay_{2})=0 $. В аукционе <<Платят все>> произведение выплат строго положительно, поэтому $ \E(Pay_{1}Pay_{2})>0 $. Значит способ проведения аукциона может влиять на ковариацию.




\item <<Наследство>>\index{Наследство} по типу аукциона второй цены. Двум сыновьям достался земельный участок в наследство. Отец не хотел, чтобы участок был разделен, поэтому по завещанию установлены следущие правила: два брата одновременно делают ставки. Участок получает тот, кто сделал большую ставку. При этом получивший участок выплачивает проигравшему меньшую из двух ставок. Ценности участка для игроков независимы и равномерны на $ [0;1] $. 

Найдите равновесие Нэша.

Solution.

Ожидаемая прибыль:

\begin{multline}
\pi(x,b_{1})=(x-\E(b(X_{2})|b(X_{2})<b_{1}))\cdot \P(b(X_{2})<b_{1})+\\
+b_{1}\P(b(X_{2})>b_{1})
\end{multline}

После чудо-замены $ b_{1}=b(a) $ и упрощения вероятностей:

\begin{equation}
\pi=(x-\E(b(X_{2})|X_{2}<a))\P(X_{2}<a)+b(a)(1-\P(X_{2}<a))
\end{equation}

Или:
\begin{equation}
\pi=xF(a)-\E(b(X_{2})\cdot 1_{X_{2}<a}))+b(a)(1-F(a))
\end{equation}

В записи с интегралом:
\begin{equation}
\pi=xF(a)-\int_{0}^{a}b(t)f(t)dt+b(a)(1-F(a))
\end{equation}

Приравниваем производную к нулю:
\begin{equation}
xf(a)-b(a)f(a)-b(a)f(a)+b'(a)(1-F(a))=0
\end{equation}

Для случая равномерного распределения:
\begin{equation}
x-2b(x)+b'(x)(1-x)=0
\end{equation}

Подбором коэффициентов находим линейное решение:
\begin{equation}
b(x)=\frac{1}{3}x+\frac{1}{6}
\end{equation}


\item Рассмотрим аукцион второй цены. Предположим, что ценности независимы и имеют регулярное распределение. Агенты не нейтральны к риску. Их отношение к риску отражается функцией полезности $ u() $. Про $ u() $ известно, что она непрерывна, строго возрастает и для удобства $ u(0)=0 $. Т.е. если игрок получает товар ценностью $ x $ и платит продавцу $ m $, то его полезность равна $ u(x-m) $. 

Найдите равновесие Нэша.

Как в лекции, составляем табличку и видим, что $ b_{1}=X_{1} $ нестрого доминирует остальные стратегии.


\item Рассмотрим аукцион второй цены с резервной ставкой $ r $. Резервная ставка --- это минимальная цена за которую продавец согласен расстаться с товаром. Если все игроки сделали ставки ниже $ r $, товар остается у продавца, никто ничего не платит. Если хотя бы один игрок сделал ставку выше $ r $, то товар достается игроку сделавшему самую высокую ставку и платит он максимум между второй по величине ставкой и $ r $. Константа $ r $ общеизвестна всем игрокам. Ценности независимы и имеют регулярное распределение. Агенты нейтральны к риску. 

Найдите равновесие Нэша.


Как в лекции, составляем табличку и видим, что $ b_{1}=X_{1} $ нестрого доминирует остальные стратегии.

\item Рассмотрим аукцион первой цены с двумя игроками. Ценности независимы и равномерны на $ [0;1] $. Но ставку можно сделать только 0 или 0.5. Если ставки игроков совпали, то товар достается случайно выбираемому игроку за соответствующую плату. 

Найдите равновесие Нэша.

Предположим, что стратегия имеет вид:

Если ценность ниже порога $ x^{*} $, то делать ставку 0, иначе делать ставку $ 0.5 $.

Осталось найти $ x^{*} $.

Допустим, что второй игрок использует такую стратегию.

Если первый сделает ставку ноль, то его ожидаемый выигрыш будет равен:
\begin{equation}
\pi(x,0)=x^{*}\frac{1}{2}x
\end{equation}

Если первый сделает ставку $0.5$, то его ожидаемый выигрыш будет равен:
\begin{equation}
\pi(x,0.5)=(x-0.5)x^{*}+(x-0.5)\frac{1}{2}(1-x^{*})=(x-0.5)\frac{1}{2}(x^{*}+1)
\end{equation}

Находим условие, при котором $ \pi(x,0.5)>\pi(x,0) $, получаем:

\begin{equation}
x>\frac{1}{2}(x^{*}+1)
\end{equation}

Значит правая часть представляет собой $ x^{*} $. Решаем уравнение $x^{*}=\frac{1}{2}(x^{*}+1)  $, получаем $ x^{*}=1 $. Т.е. вне зависимости от ценности игрокам имеет смысл ставить 0.


\end{enumerate}

