\begin{enumerate}


\item Пусть $  V $ --- общая ценность товара для двух игроков, равномерна на $ [0;1] $. Величины $ R_{1} $ и $ R_{2} $ --- независимы между собой и с $ V $ и равномерны на $ [0.5;1.5] $. Игроки получают сигналы $ X_{i}=V\cdot R_{i} $. 
\begin{enumerate}
\item Найдите совместную функцию плотности $ X_{1} $ и $ X_{2} $. Верно ли, что $ X_{1} $ и $ X_{2} $ аффилированны?
\item Найдите $ v(x,y)=\E(V|X_{1}=x,Y_{1}=y) $ 
\item Найдите совместную функцию плотности $ X_{1} $ и $ Y_{1} $, $ g(x,y) $
\end{enumerate}

По условию, при фиксированном $ v $ величина равномерна на $ [0.5v;1.5v] $. Длина этого отрезка равна $ v $, значит условная функция плотности $ X_{1} $ при фиксированном $ v $ имеет вид:
\begin{equation}
p(x_{1}|v)=\frac{1}{v}\quad x_{1}\in [0.5v;1.5v]
\end{equation}


Т.к. при фиксированном $ v $ величины $X_{1}  $ и $ X_{2} $ независимы, то выписываем $ p(x_{1},x_{2}|v) $:
\begin{equation}
p(x_{1},x_{2}|v)=\frac{1}{v}\cdot \frac{1}{v}, \quad x_{1},x_{2}\in [0.5v;1.5v]
\end{equation}

Т.к. $ p(x_{1},x_{2},v)=p(x_{1},x_{2}|v)p(v) $:
\begin{equation}
p(x_{1},x_{2},v)=\frac{1}{v}\cdot \frac{1}{v}\cdot 1, \quad x_{1},x_{2}\in [0.5v;1.5v]
\end{equation}

Условие $ x_{1},x_{2}\in [0.5v;1.5v] $ записываем как: $ x_{1}\wedge x_{2} > 0.5v $ и $ x_{1}\vee x_{2} <1.5v $. Или как $ v\in [\frac{x_{1}\vee x_{2}}{1.5};\frac{x_{1}\wedge x_{2}}{0.5}] $. Для краткости обозначим этот интервал: $ [v_{min};v_{max}] $.

Интегрируем по $ v $ в указанных пределах и получаем:
\begin{equation}
p(x_{1},x_{2})=\int_{v_{min}}^{v_{max}}\frac{1}{v^{2}}dv=\frac{1}{v_{min}}-\frac{1}{v_{max}}
\end{equation}

Есть точки, где функция недифференциируема, поэтому проверять супермодулярность нужно будет по определению. Проверка супермодулярности пропущена. Она сводится к аккуратному рассмотрению нескольких случаев.

Здесь $ Y_{1}=X_{2} $, поэтому третий пункт уже решен, осталось найти:

\begin{multline}
\E(V|X_{1}=x_{1}, X_{2}=x_{2})=\int v p(v|x_{1},x_{2})dv=\\
\int v \frac{p(x_{1},x_{2},v)}{p(x_{1},x_{2})} dv=\frac{\int v p(x_{1},x_{2},v) dv }{p(x_{1},x_{2})}
\end{multline}

В числителе:
\begin{equation}
\int_{v_{min}}^{v_{max}}\frac{1}{v}dv=\ln(v_{max})-\ln(v_{min})
\end{equation}

Значит в итоге: 
\begin{equation}
v(x_{1},x_{2})=\frac{\ln(v_{max})-\ln(v_{min})}{\frac{1}{v_{min}}-\frac{1}{v_{max}}}
\end{equation}

\item На аукционе продается картина, которая равновероятно является <<Джокондой>> Леонардо да Винчи или ее подделкой. За нее торгуются $ n $ покупателей. Ценность картины для всех покупателей одинакова, $ V_{1}=V_{2}=\ldots=V_{n}=V $ и равна 1, если это оригинал и 0, если подделка.

Если $ V=0 $, то сигналы $ X_{i} $ условно независимы и равномерны на $ [0;1] $. Если $ V=1 $, то сигналы $ X_{i} $ условно независимы и имеют функцию плотности $ f(x|V=1)=2x $ при  $x\in [0;1] $
\begin{enumerate}
\item Найдите совместную функцию плотности всех $ X_{i} $. Верно ли, что все $ X_{i} $ аффилированны?
\item Найдите $ v(x,y)=\E(V|X_{1}=x,Y_{1}=y) $
\item Найдите совместную функцию плотности $ X_{1} $ и $ Y_{1} $, $ g(x,y) $
\end{enumerate}


Возьмем событие $ A=\{X_{1}\in[x_{1};x_{1}+\Delta] \cap \ldots \cap X_{n}\in[x_{n};x_{n}+\Delta]\} $. Поскольку $ \P(A)>0 $ действуют старые правила:
\begin{multline}
\P(A)=\P(A\cap V=1)+\P(A\cap V=0)=\\
=\P(A|V=1)\P(V=1)+\P(A|V=0)\P(V=0)=\\
=0.5\P(A|V=1)+0.5\P(A|V=0)
\end{multline}

О-малые говорят нам, что плотности подчиняются такой же формуле, т.к. многомерная плотность есть вероятность поделить на $ \Delta^{n} $:
\begin{equation}
f(x_{1},\ldots,x_{n})=0.5f(x_{1},\ldots,x_{n}|V=0)+f(x_{1},\ldots,x_{n}|V=1)
\end{equation}

Поэтому совместная функция плотности имеет вид:
\begin{equation}
f(x_{1},\ldots,x_{n})=0.5\cdot 1\cdot 1 \ldots\cdot 1+0.5\cdot 2x_{1}\cdot 2x_{2}\cdot \ldots 2x_{n}=0.5+2^{n-1}x_{1}\cdot \ldots \cdot x_{n}
\end{equation}
Проверяем вторую смешанную производную логарифма. В силу симметрии достаточно по $ x_{1} $ и $ x_{2} $:
\begin{equation}
\frac{\partial^{2}\ln(f)}{\partial x_{1}\partial x_{2}}=\ldots=\frac{0.5}{f(x_{1},\ldots,x_{2})^{2}}\geq 0
\end{equation}


Найдем сначала третий пункт:

Для этого представим себе событие $ A=\{X_{1}\in [x_{1};x_{1}+\Delta], Y_{1}\in[y_{1};y_{1}+\Delta]\} $. Для него $ \P(A)= 0.5\P(A|V=1)+0.5\P(A|V=0)$. О-малые говорят нам, что плотности подчиняются такой же формуле!

Поэтому находим две условные функции плотности $ X_{1} $ и $ Y_{1} $:
\begin{equation}
g(x,y|V=0)=(n-1)\cdot 1 \cdot y^{n-2}
\end{equation}
И
\begin{equation}
g(x,y|V=1)=(n-1)\cdot 2x \cdot 2y \cdot (y^{2})^{n-2}
\end{equation}
И получаем безусловную:
\begin{equation}
g(x,y)=0.5(n-1)\cdot 1 \cdot y^{n-2}+0.5(n-1)\cdot 2x \cdot 2y \cdot (y^{2})^{n-2}
\end{equation}

Поскольку $ V $ принимает значения только 0 и 1, то $ \E(V|A)=\P(V=1|A) $. По формуле условной вероятности:
\begin{equation}
\P(V=1|A)=\frac{\P(V=1 \cap A)}{\P(A)}=\frac{\P(A|V=1)\cdot \P(V=1)}{\P(A)}=\frac{0.5\P(A|V=1)}{\P(A)}
\end{equation}

И в итоге искомая функция $ v(x,y) $ равна:
\begin{multline}
v(x,y)=\P(V=1|X_{1}=x,Y_{1}=y)=\\
=\frac{0.5(n-1)\cdot 2x \cdot 2y\cdot (y^{2})^{n-2}}{0.5(n-1)\cdot 1 \cdot y^{n-2}+0.5(n-1)\cdot 2x \cdot 2y \cdot (y^{2})^{n-2}}=\\
=\frac{4xy^{n-1}}{1+4xy^{n-1}}
\end{multline}

\item На аукционе второй цены присутствуют $ n $ покупателей. Ценности совпадают с сигналами, $ V_{i}=X_{i} $; сигналы $ X_{i} $ независимы и равномерны на $ [0;1] $. На аукционе продается $k$ одинаковых чудо-швабр, $ 1<k<n $. Каждому покупателю нужна только одна чудо-швабра. Покупатели одновременно делают свои ставки. Чудо-швабры достаются по одной каждому из $ k $ покупателей с самыми высокими ставками. Каждый из $ k $ победителей платит организатору наибольшую проигравшую ставку.

Найдите равновесие Нэша.

Проверяем метод <<Авось старое решение подойдет>>. Строим табличку как в первой лекции и видим, что стратегия $ b(x)=x $ нестрого доминирует остальные стратегии. Единственное отличие: выиграю ли я аукцион зависит от сравнения моей ставки и  $m=b(Y_{k}) $, а не $m=b(Y_{1}) $ как в первой лекции.


\item На аукционе первой цены присутствуют $ n $ покупателей. Ценности совпадают с сигналами, $ V_{i}=X_{i} $; сигналы $ X_{i} $ независимы и равномерны на $ [0;1] $. На аукционе продается $k$ одинаковых чудо-швабр, $ 1<k<n $. Каждому покупателю нужна только одна чудо-швабра. Покупатели одновременно делают свои ставки. Чудо-швабры достаются по одной каждому из $ k $ покупателей с самыми высокими ставками. Эти $ k $ победителей платят свои ставки организатору.

Найдите дифференциальное уравнение, которому удовлетворяет равновесная стратегия.

Hint: Когда продавался один товар, то условие победы первого игрока --- $ Y_{1}<a $, а если продается $ k $ товаров, то условие победы первого игрока $ Y_{?}<a $.

Условие победы первого игрока: $ Y_{k}<a $. В функции прибыли мы можем убрать условие в силу независимости ценностей.
\begin{equation}
\pi(x,b(a))=(x-b(a))\E(1_{Y_{k}<a}|X_{1}=x)=(x-b(a))\P(Y_{k}<a)
\end{equation}

Применяем о-малые. Одна величина должна упасть около $ t $, $ (k-1) $ должна оказаться выше $ t $, и $ (n-1-k) $ должно оказаться ниже $ t $:
\begin{equation}
f_{Y_{k}}(t)=(n-1)\cdot C_{n-1}^{k-1}\cdot 1\cdot (1-t)^{k-1}\cdot t^{n-k-1}
\end{equation}

Значит:
\begin{equation}
\pi(x,b(a))=(x-b(a))\E(1_{Y_{k}<a}|X_{1}=x)=(x-b(a))\int_{0}^{a}f_{Y_{k}}(t)dt
\end{equation}

Получаем диф. ур:
\begin{equation}
(x-b(x))f_{Y_{k}}(x)-b'(x)\int_{0}^{x}f_{Y_{k}}(t)dt=0
\end{equation}
Всё. Дифференциальное уравнение получено. 

А дальше можно изолировать $ \int_{0}^{x}\ldots\, dt $ в правой части, взять производную по $ x $ и избавится от интеграла. Но это уже относится к решению дифференциального уравнения.


\item Существуют ли неаффилированные случайные величины $ X_{1} $ и $ X_{2} $ такие, что $Cov(X_{1},X_{2})>0  $?


Да. Возьмем пару аффилированных случайных величин с положительной корреляцией. У нее функция плотности всюду удовлетворяет условию 
\begin{equation}
\partial^{2}\ln (f(x_{1},x_{2}))/\partial x_{1}\partial x_{2} \geq 0
\end{equation}
А теперь на очень-очень маленьком участке нарушим это условие. Случайные величины перестали быть аффилированными. А ковариация от этого изменится очень-очень слабо, т.е. останется положительной.

Конкретный пример: $ X_{1} $ ---  равномерно на $ [0;1] $, $ D $ --- равномерно на $ [0;1] $. 
\begin{equation}
X_{2}=D+
\begin{cases}
X_{1}, X_{1}>0.00001 \\
-X_{1}, X_{1}<0.00001
\end{cases}
\end{equation} 


\end{enumerate}
