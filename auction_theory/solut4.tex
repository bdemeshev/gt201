\begin{enumerate}

\item На аукционе участвуют $ n $ игроков. Ценности независимы, $ X_{i}=V_{i}$. Пусть функция распределения сигналов имеет вид $ F(x)=x^{a} $ на $ [0;1] $, где $ a $ --- это некая константа, $ a\geq 1 $. 
\begin{enumerate}
\item Найдите $ MR(x) $. Является ли $ MR(x) $ возрастающей?
\item Постройте оптимальный аукцион.
\end{enumerate}

\begin{equation}
MR(x)=x-\frac{1-x^{a}}{ax^{a-1}}=x\left(1+\frac{1}{a}\right)-\frac{1}{ax^{a-1}}
\end{equation}
Даже без производной видно, что функция возрастает. Оптимальным будет аукцион второй цены с резервной ценой:
\begin{equation}
r=\left(\frac{1}{a+1}\right)^{1/a}
\end{equation}

\item Петя переезжает на новую квартиру, поэтому продает свои старые шкаф и комод (варианта взять их с собой у него нет).  Потенциальных покупателей двое. Первый покупатель знает значение $ X_{1} $, второй --- значение $ X_{2} $. Величины  $ X_{1} $ и  $ X_{2} $ независимы и равномерны на $ [0;1] $. Полезности первого игрока: от шкафа --- $ 0.5 $, от комода --- $ 0.8X_{1} $, от шкафа и комода --- $ 0.5+X_{1} $. Полезности второго игрока: от шкафа --- $ 0.8 $, от комода --- $ X_{2} $, от шкафа и комода --- $ 0.8+0.8X_{2}$
\begin{enumerate}
\item Четко опишите механизм VCG применительно к этой задаче.
\item Какова средняя прибыль продавца при использовании механизма VCG?
\end{enumerate}

Составляем табличку:

\begin{tabular}{c|cccc}
& (Ш,К) & (К,Ш) & (КШ,-) & (-,КШ) \\ 
\hline 
Покупатель 1 & 0.5 & $ 0.8X_{1} $ & $ 0.5+X_{1} $ & 0 \\ 
Покупатель 2 & $ X_{2} $ & 0.8 & 0 & $ 0.8+0.8X_{2} $ \\ 
Сумма & $ 0.5+X_{2} $& $ 0.8+0.8X_{1} $ & $ 0.5+X_{1} $ & $ 0.8+0.8X_{2} $ \\
\end{tabular} 

Покупатели одновременно декларируют свои значения $ X_{i} $. Мы знаем, что в механизме VCG им будет оптимально говорить правду. Механизм VCG максимизирует сумму полезностей. В данном случае мы замечаем, что $ 0.8+0.8X_{1}>0.5+X_{1} $ при любых $ X_{1} \in [0;1]$. И аналогично для $ X_{2} $. Поэтому правило выбора решения имеет вид:

Если $ X_{1}>X_{2} $, то комод --- первому, и шкаф --- второму. Если $ X_{1}<X_{2} $, то комод и шкаф --- второму.

Осталось правило платежей:

Если $ X_{1}>X_{2} $, то первый платит $ 0.8X_{2} $, а второй --- $ 0.5+0.2X_{1} $.

Если $ X_{1}<X_{2} $, то первый платит 0, а второй --- $ 0.5+X_{1} $.

Получаем выручку продавца:
\begin{equation}
R=(0.5+0.2X_{1}+0.8X_{2})1_{X_{1}>X_{2}}+(0.5+X_{1})1_{X_{1}<X_{2}}
%=0.5+(0.2X_{1}+0.8X_{2})1_{X_{1}>X_{2}}+X_{1}(1-1_{X_{1}<X_{2}})=0.5+X_{1}+(0.8X_{2}-0.8X_{1})1_{X_{1}>X_{2}}
\end{equation}

Находим:
\begin{equation}
\E(X_{1}1_{X_{1}>X_{2}})=\int_{0}^{1}\int_{0}^{x_{1}}x_{1} \cdot 1 \cdot dx_{2}dx_{1}=1/3
\end{equation}

Аналогично, $ \E(X_{1}1_{X_{1}<X_{2}})=1/6 $.

Получаем, что средняя выручка равна:
\begin{equation}
\E(R)=0.5\cdot \frac{1}{2}+0.2\cdot \frac{1}{3}+0.8\cdot \frac{1}{6}+0.5\cdot \frac{1}{2}+\frac{1}{6}=\frac{13}{15}
\end{equation}

\item Есть $ n $ городов. Рядом с одним из них нужно построить мусоросжигательный завод. Жители города рядом с которым будет построен завод получат отрицательную полезность $ U_{i}=-X_{i} $. Остальные получат полезность 0. Величины $ X_{i}\sim U[0;1] $ и независимы. Каждый город знает свое $ X_{i} $. 
\begin{enumerate}
\item Опишите механизм VCG применительно к этой задаче. Т.е. предполагается, что игроки объявляют числа $ b_{i}\in [0;1] $ и механизм должен определять, у какого города строить завод и какие платежи должны сделать игроки в зависимости от $ b_{i} $.
\item Выпишите функцию плотности для компенсации, которую получают жители города рядом с которым будет построен мусоросжигательный завод.
\item Сходится ли баланс у механизма VCG в этом случае? Если нет, то сколько в среднем нужно вложить средств извне в этот механизм?
\item Что больше: компенсация или ущерб от строительства завода в механизме VCG?
\end{enumerate}

Каждый город одновременно декларирует свой ущерб. 

Правило принятия решения: завод построить рядом с городом, сообщившим наименьший ущерб.

Правило платежей: Город рядом с которым строят завод должен получить компенсацию в размере минимума ущербов остальных городов. Остальные города ничего не платят и не получают. 

Автоматически получаем, что механизм VCG требует вливания средств извне. Т.к. компенсация равна не самому маленькому ущербу, а ущербу второму по малости, то: компенсация всегда больше ущерба.

Функция плотности: $ p(y)=n\cdot 1\cdot (n-1)y(1-y)^{n-2} $.

Средняя компесация равна (для взятия интеграла можно сделать замену $ z=1-y $): 
\begin{equation}
\E(K)=\int_{0}^{1}y\cdot n(n-1)y(1-y)^{n-2}dy=\frac{2}{n+1}
\end{equation}



%\item Есть $ 3 $ города. Рядом с одним из них нужно построить мусоросжигательный завод. Жители города рядом с которым будет построен завод получат отрицательную полезность $ U_{i}=-X_{i} $. Остальные получат полезность 0. Величины $ X_{i}\sim U[0;1] $ и независимы. Каждый город знает свое $ X_{i} $. Города одновременно называют требуемую компенсацию $ b_{i} $. Завод строится у того города, у которого $ b_{i} $ меньше. Остальные города выплачивают компенсацию поровну.
%\begin{enumerate}
%\item Найдите равновесие Нэша
%\item Как надо изменить этот механизм, чтобы он стал правдивым?
%\end{enumerate}



%\item Есть один покупатель и один продавец. Ценности товара: $ X_{1} $ --- для покупателя, $ X_{2} $ --- для продавца. Величины $ X_{i} $ независимы и равномерны на $ [0;1] $.
%\begin{enumerate}
%\item Опишите механизм VCG применительно к этой задаче. Т.е. предполагается, что игроки объявляют числа $ b_{i}\in [0;1] $ и механизм должен определять, кому отдать товар и какие платежи должны сделать игроки в зависимости от $ b_{i} $.
%\item Каков средний баланс механизма VCG в этой задаче?
%\item Предположим, что вместо VCG используется такой механизм: игроки одновременно называют желаемые цены, $ b_{1} $ и $ b_{2} $. Если $ b_{1}>b_{2} $, то обмен происходит по цене $ 0.5(b_{1}+b_{2}) $. Найдите равновесие Нэша.
%\item Верно ли, что при втором механизме обмен происходит если и только если $ X_{1}>X_{2} $?
%\end{enumerate}

%Hint: равновесие Нэша будет в линейных стратегиях



\item Кнопочный аукцион и три игрока. Ценности $ V_{1} $, $ V_{2} $ и $ V_{3} $ равномерны на $ [0;1] $ и независимы. Первый и второй игрок знают значение своих ценностей, т.е. $ X_{1}=V_{1} $ и $ X_{2}=V_{2} $. А третий игрок --- не знает значения своей ценности, а знает только закон распределения.
\begin{enumerate}
\item Что собой представляют стратегии игроков в этом случае? Почему их можно упростить?
\item Найдите равновесие Нэша
\end{enumerate}


Поскольку третий игрок ничего не знает, а только видит, сколько игроков осталось в игре, то его стратегия описывается двумя числами, $ b_{3}^{3} $ и $ b_{3}^{2} $. Эти числа говорят, до какой цены давить кнопку, если в игре осталось три и два игрока.

Стратегия первого игрока описывается тремя функциями: $ b_{1}^{3}(x) $ --- до какой цены давить кнопку, если в игре три игрока, $b_{1}^{2a}(x,p)$ --- до какой цены давить кнопку, если в игре двое: я и второй; $b_{1}^{2b}(x,p)$ --- до какой цены давить кнопку, если в игре двое: я и третий. Стратегия второго игрока имеет такой же вид. 

Поскольку ценности независимы, то никакой полезной информации от наблюдения за ценами выхода других игроков мы не получаем. Следовательно, стратегию третьего игрока можно заменить одним числом $ b_{3} $, а стратегию первого --- одной функцией $b_{1}(x)$.

Получаем аукцион второй цены. Игроки ориентируются на ожидаемый выигрыш. Поэтому с точки зрения третьего игрока его ценность равна 0.5. Т.е. равновесие Нэша имеет вид $ b_{3}=0.5 $; $ b_{1}(x)=x $; $ b_{2}(x)=x $.


\end{enumerate}

