\documentclass[pdftex,12pt,a4paper]{article}

\input{/home/boris/Dropbox/Public/tex_general/title_bor_utf8}

%\usepackage{showkeys} % показывать метки

\input{/home/boris/Dropbox/Public/tex_general/prob_and_sol_utf8}

%\title{Задачи по элементарной теории вероятностей и матстатистике}
%\author{Составитель: Борис Демешев, boris.demeshev@gmail.com}
%\date{\today}

\begin{document}

%\pagestyle{myheadings} \markboth{ТВИМС-задачник. Демешев Борис. roah@yandex.ru }{ТВИМС-задачник. Демешев Борис. roah@yandex.ru }
%\maketitle
%\tableofcontents{}

%\parindent=0 pt % отступ равен 0

\begin{Large}
Моделирование аукционов. Контрольная работа 2.
\end{Large}

\begin{enumerate}
\item Можно пользоваться калькулятором. Вопрос в том, нужно ли?
\item Можно решать задачи в любом порядке.
\item С собой можно принести один лист А4, где заранее могут быть написаны (именно написаны, а не напечатаны) любые формулы, теоремы или комментарии.
\item Продолжительность работы 1 час 20 минут.
\item Условия нельзя забрать с собой. Условия и решения открыто доступны на \url{auctiontheory.wordpress.com} после окончания контрольной.
\item Обсуждать задачи во время работы нельзя.
\item Человек проводящий контрольную не будет отвечать на вопросы по тексту задач. 
\item Скорее всего, в задачах нет очепяток. Если, по твоему мнению, опечатка есть, то ее нужно исправить самому исходя из своего представления о хорошей задаче. При этом нужно четко отразить этот факт перед началом решения. Например, <<По-моему, в тексте есть опечатка и вместо ... должно быть ...>>. Твоя гипотеза об опечатках является личной и не подлежит обсуждению во время работы.
\item Насколько подробно все расписывать --- решай сам исходя из конкретной ситуации. Очевидно, что в примере $ 1+2+3=? $ ответ можно написать сразу, а взятие интеграла $ \int x^{5}\cos(x)dx $ требует каких-то промежуточных записей.
\item Паниковать на контрольной строжайше запрещено!
%\item Каждая из 5 задач весит 5 баллов.
\item Для каждой задачи обязательно нужно спрогнозировать свою оценку. Не надо скромничать, лучше попытаться объективно оценить свое решение.  За неверное оценивание баллы снижаться не будут, а верное оценивание даст возможность чему-то научиться. Опыт показывает, что оценка своих собственных решений позволяет резко улучшить их качество. Прогноз своей оценки пишем в табличку!
\item Не забудьде подписать свою работу. Пожалуйста!

\end{enumerate}

\begin{large}
Имя:
\end{large}

\vspace{4pt}

\begin{large}
Отчество:
\end{large}

\vspace{4pt}

\begin{large}
Фамилия:
\end{large}

\vspace{4pt}

\begin{large}
Группа:
\end{large}

\vspace{4pt} 

\begin{tabular}{|c|c|c|c|c|c|c|}
\hline  & Задача 1 & Задача 2 & Задача 3 & Задача 4 & Задача 5 & Итого \\ 
\hline Прогноз оценки &  &  &  &  &  & \\ 
\hline Оценка (от 0 до 5) &  &  &  &  &  & \\ 
\hline 
\end{tabular} 
\newpage

\begin{enumerate}

\item Техническая задача. 
\begin{enumerate}
\item Известно, что $ f(\vec{x}) $ и $ g(\vec{x}) $ --- супермодулярные функции, а $ a>0 $ и $ b>0 $ --- константы. Верно ли, что $ af(\vec{x})+bg(\vec{x}) $ --- супермодулярная функция?
\item Пусть $ X_{1} $,..., $ X_{n} $ независимы и имеют функцию плотности $ f(t)=3t^{2} $ на $ [0;1] $. Случайные $ Y_{1} $, ..., $ Y_{n-1} $ --- это есть упорядоченные по убыванию случайные величины $ X_{2} $, ..., $ X_{n} $. С помощью о-малых или без нее найдите совместную функцию плотности $ f_{Y_{5},Y_{10}}(a,b) $.
\end{enumerate}

Следующие две задачи очень похожи, разница в них только в типе аукциона...

\item На аукционе первой цены продается участок. Потенциальных покупателей двое. Ценность участка для каждого игрока определяется его площадью. Первый покупатель знает ширину участка $ X_{1} $, а второй --- длину $X_{2}$. Совместная фукнция плотности $ X_{1} $ и $ X_{2} $ имеет вид $ f(x_{1},x_{2})=\frac{7}{8}+\frac{1}{2}x_{1}x_{2} $. Найдите дифференциальное уравнение, которому подчиняется равновесная стратегия игрока. 

\item На аукционе второй цены продается участок. Потенциальных покупателей двое. Ценность участка для каждого игрока определяется его площадью. Первый покупатель знает ширину участка $ X_{1} $, а второй --- длину $X_{2}$. Совместная фукнция плотности $ X_{1} $ и $ X_{2} $ имеет вид $ f(x_{1},x_{2})=\frac{7}{8}+\frac{1}{2}x_{1}x_{2} $. Найдите равновесие Нэша.


\item Найдите равновесие Нэша в случае кнопочного аукциона. Сигналы $ X_{i} $ игроков имеют совместную функцию плотности$ f(x_{1},x_{2},x_{3})=7/8+x_{1}x_{2}x_{3} $ при $ x_{1},x_{2},x_{3}\in[0;1] $. Ценности определяеются по формулам:
\begin{equation}
\begin{array}{c}
V_{1}=X_{1}(X_{2}+X_{3}) \\
V_{2}=X_{2}(X_{1}+X_{3}) \\
V_{3}=X_{3}(X_{1}+X_{2}) 
\end{array}
\end{equation}

\item Ценности игроков одинаково распределены, независимы, распределение ценностей дискретно: $ X_{i}$ равновероятно принимает натуральное значение от 1 до 100 включительно. Игроки одновременно делают ставки. Разрешаются только целые неотрицательные ставки. Товар достается игроку, сделавшему наивысшую ставку. Если таких игроков несколько, то победитель выбирается из них равновероятно. Победитель платит сделанную им ставку. Найдите хотя бы одно равновесие Нэша в чистых стратегиях, в котором игроки не используют нестрого доминируемых стратегий. 



Пример равновесия Нэша. Обозначим $ v_{max} $ --- максимальную ценность и $ v_{sec} $ --- вторую по величине ценность (возможно они совпадают). Те игроки, чья ценность $ v_{i}<v_{max} $ делают ставку $ b_{i}=v_{i}-1 $. Если лидер один, то он делает ставку $ b=v_{sec} $, иначе каждый лидер делает ставку $ b=v_{max}-1 $

\end{enumerate}


Подсказка: по моим ощущениям 1, 3 и 4-ая задачи легче 2 и 5-ой.


\bibliography{/home/boris/Dropbox/Public/tex_general/opit}
\printindex % печать предметного указателя здесь

\end{document}