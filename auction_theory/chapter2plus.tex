


\subsection{Немного про теорию вероятностей.}

\begin{mydef}
Индикатором события $ A $ называется случайная величина, которая равна 1, если $ A $ произошло и 0, если $ A $ не произошло. Обозначают индикатор $ A $ так: $ 1_{A} $.
\end{mydef}

Проверьте, что вы понимаете, что $ E(1_{A})=P(A) $.

С помощью индикаторов легко определить условное ожидание:

\begin{mydef}
Если $ P(A)>0 $, то условным ожидание случайной величины $ X $ при условии события $ A $ называют число:
\[ E(X|A):=\frac{E(X\cdot 1_{A})}{P(A)} \]
\end{mydef}

Если вам знакомо другое определение условного ожидания, убедитесь, что оно совпадает с этим!

Пример 1. Табличка. Найдите $ E(X|X>5) $.


Пример 2. Пусть $ X $ распределено экспоненциально с параметром $ \lambda=1 $, т.е. функция плотности $ X $ при $ t\geq 0 $ имеет вид: 
\[ p_{X}(t)=e^{-t} \]
Найдите $ E(X|X<5) $

На языке условных ожиданий и индикаторов очень легко интерпретировать наши знакомые формулы!

...





Использование бесконечно малых $o(dx)$

Пример с преобразованием функции плотности...

Пример с распределением одной порядковой статистики...

Пример с распределением двух порядковых статистик...

Пример с распределением $ n $ порядковых статистик...

Пример (?) с очередью и пуассоновским процессом



\subsection{Агенты несклонные к риску}

Что изменится, если агенты будут несклонны к риску? Ценности мы по прежнему считаем независимыми.

\begin{mydef}
Игрок не склонен к риску, если его функция полезности $ u() $ удовлетворяет условиями $ u'>0 $ и $ u''\leq 0 $.
\end{mydef}

Для удобства мы будем считать, что $ u(0)=0 $.

Если первый игрок выигрывает аукцион и платит $ b_{1} $, то его полезность равна $u(X_{1}-b_{1})$.

Рассмотрим первого игрока. Введем стандартные обозначения:

\begin{enumerate}
\item Аукцион первой цены.




\item Аукцион второй цены.
По сравнению со случаем нейтральности к риску в табличках произойдет только одно изменение: вместо $ X_{1}-b_{1} $ в табличке будет стоять $ u(X_{1}-b_{1}) $. Там, где был ноль, там ноль и останется: $ u(0)=0 $. В силу того, что $ u'>0 $ знак $ u(X_{1}-b_{1}) $ совпадает со знаком $ X_{1}-b_{1} $. Поэтому стратегия $ b_{1}=X_{1} $ по прежнему нестрого доминирует все остальные стратегии. Кстати, мы не использовали  тот факт, что $u''\leq 0 $, а значит тоже равновесие Нэша остается и для случая склонных к риску агентов.


\item Кнопочный аукцион.
Мы уже доказывали, что при независимых ценностях кнопочный аукцион эквивалентент аукциону второй цены. Значит, равновесие по Нэшу --- снова $ b_{i}=X_{i} $.
\end{enumerate}




