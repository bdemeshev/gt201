\documentclass[pdftex,12pt,a4paper]{article}

\input{/home/boris/Dropbox/Public/tex_general/title_bor_utf8}

%\usepackage{showkeys} % показывать метки

\input{/home/boris/Dropbox/Public/tex_general/prob_and_sol_utf8}

\title{Моделирование аукционов}
%\author{Составитель: Борис Демешев, boris.demeshev@gmail.com}
%\date{\today}

\begin{document}

%\pagestyle{myheadings} \markboth{ТВИМС-задачник. Демешев Борис. roah@yandex.ru }{ТВИМС-задачник. Демешев Борис. roah@yandex.ru }
\maketitle
%\tableofcontents{}

%\parindent=0 pt % отступ равен 0



Из курса можно будет узнать:

\begin{itemize}

\item Что аукцион - это не обязательно <<дядя с молоточком>> (например, фантастическая прибыль Google - это деньги полученные за счет продажи рекламы на аукционе)

\item  Как устроены самые крупные аукционы

\item Почему не всегда победитель аукциона платит ту сумму, которую поставил

\item Как влияют на прибыль организатора разные правила проведения аукциона

\item При каких условиях может нарушиться закон спроса (с ростом числа рациональных покупателей падает равновесная цена)

\end{itemize}


Для успешного освоения курса от желающих требуется знание (как минимум):

\begin{itemize}
\item Из теории игр: Равновесие по Нэшу.
\item Из теории вероятностей: Условная вероятность, математическое ожидание.
\end{itemize}


Формат проведения. Если кратко: лекции — заочно, контроль — очно.

Создается блог, где каждую неделю публикуется видеозапись и конспект очередной лекции, с упражнениями и решениями. С материалом лекции слушатель разбирается самостоятельно в удобное время, ответы на возникающие вопросы слушатель получает в том же блоге.

Каждую неделю проводится очная письменная работа. Результирующая оценка выставляется по результатам этих письменных работ.

Продолжительность курса 4 недели.


Если есть вопросы, смело спрашивайте: \url{boris.demeshev@gmail.com}. Встретиться со мной лично до начала курса можно будет в декабре.


\vskip 30pt


Борис Демешев








%\bibliography{/home/boris/Dropbox/Public/tex_general/opit}
%\printindex % печать предметного указателя здесь

\end{document}