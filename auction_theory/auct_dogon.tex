\documentclass[pdftex,12pt,a4paper]{article}

% jan 2012

% sudo yum install texlive-bbm texlive-bbm-macros texlive-asymptote texlive-cm-super texlive-cyrillic texlive-pgfplots texlive-subfigure
% yum install texlive-chessboard texlive-skaknew % for \usepackage{chessboard}
% yum install texlive-minted texlive-navigator texlive-yax texlive-texapi

% растягиваем границы страницы
%\emergencystretch=2em \voffset=-2cm \hoffset=-1cm
%\unitlength=0.6mm \textwidth=17cm \textheight=25cm

\usepackage{makeidx} % для создания предметных указателей
\usepackage{verbatim} % для многострочных комментариев
\usepackage{cmap} % для поиска русских слов в pdf
\usepackage[pdftex]{graphicx} % для вставки графики 
% omit pdftex option if not using pdflatex


%\usepackage{dsfont} % шрифт для единички с двойной палочкой (для индикатора события)
\usepackage{bbm} % шрифт - двойные буквы

\usepackage[colorlinks,hyperindex,unicode,breaklinks]{hyperref} % гиперссылки в pdf


\usepackage[utf8]{inputenc} % выбор кодировки файла
\usepackage[T2A]{fontenc} % кодировка шрифта
\usepackage[russian]{babel} % выбор языка

\usepackage{amssymb}
\usepackage{amsmath}
\usepackage{amsthm}
\usepackage{epsfig}
\usepackage{bm}
\usepackage{color}

\usepackage{multicol}


\usepackage{textcomp}  % Чтобы в формулах можно было русские буквы писать через \text{}

\usepackage{embedfile} % Чтобы код LaTeXа включился как приложение в PDF-файл

\usepackage{subfigure} % для создания нескольких рисунков внутри одного

\usepackage{tikz,pgfplots} % язык для рисования графики из latex'a
\usetikzlibrary{trees} % прибамбас в нем для рисовки деревьев
\usetikzlibrary{arrows} % прибамбас в нем для рисовки стрелочек подлиннее
\usepackage{tikz-qtree} % прибамбас в нем для рисовки деревьев


\usepackage{ifpdf} % чтобы проверять, запускаем мы pdflatex или просто latex

\ifpdf
	\usepackage[pdftex]{graphicx} 
	\DeclareGraphicsRule{*}{mps}{*}{} % все неупомянутые ps файлы объявляем упрощенными, т.е. mps типа. Просто ps графику нельзя использовать, но без некоторых спец. команд - можно. Например, результат работы metapost - это ps файлы простого (mps) типа. Собственно ради использования metapost эта строка и введена.
\else
	\usepackage{graphicx}
\fi



% конец добавки

\usepackage{asymptote} % After graphicx!, пакет для рисования графиков и прочего
%\usepackage{sagetex} % i suppose after graphicx also..., для связи с sage



\embedfile[desc={Исходный LaTeX файл}]{\jobname.tex} % Включение кода в выходной файл
\embedfile[desc={Стилевой файл}]{/home/boris/science/tex_general/title_bor_utf8.tex}



% вместо горизонтальной делаем косую черточку в нестрогих неравенствах
\renewcommand{\le}{\leqslant}
\renewcommand{\ge}{\geqslant} 
\renewcommand{\leq}{\leqslant}
\renewcommand{\geq}{\geqslant}

% делаем короче интервал в списках 
\setlength{\itemsep}{0pt} 
\setlength{\parskip}{0pt} 
\setlength{\parsep}{0pt}

% свешиваем пунктуацию (т.е. знаки пунктуации могут вылезать за правую границу текста, при этом текст выглядит ровнее)
\usepackage{microtype}

% более красивые таблицы
\usepackage{booktabs}
% заповеди из докупентации: 
% 1. Не используйте вертикальные линни
% 2. Не используйте двойные линии
% 3. Единицы измерения - в шапку таблицы
% 4. Не сокращайте .1 вместо 0.1
% 5. Повторяющееся значение повторяйте, а не говорите "то же"


% DEFS
\def \mbf{\mathbf}
\def \msf{\mathsf}
\def \mbb{\mathbb}
\def \tbf{\textbf}
\def \tsf{\textsf}
\def \ttt{\texttt}
\def \tbb{\textbb}

\def \wh{\widehat}
\def \wt{\widetilde}
\def \ni{\noindent}
\def \ol{\overline}
\def \cd{\cdot}
\def \bl{\bigl}
\def \br{\bigr}
\def \Bl{\Bigl}
\def \Br{\Bigr}
\def \fr{\frac}
\def \bs{\backslash}
\def \lims{\limits}
\def \arg{{\operatorname{arg}}}
\def \dist{{\operatorname{dist}}}
\def \VC{{\operatorname{VCdim}}}
\def \card{{\operatorname{card}}}
\def \sgn{{\operatorname{sign}\,}}
\def \sign{{\operatorname{sign}\,}}
\def \xfs{(x_1,\ldots,x_{n-1})}
\def \Tr{{\operatorname{\mbf{Tr}}}}
\DeclareMathOperator*{\argmin}{arg\,min}
\DeclareMathOperator*{\argmax}{arg\,max}
\DeclareMathOperator*{\amn}{arg\,min}
\DeclareMathOperator*{\amx}{arg\,max}
\def \cov{{\operatorname{Cov}}}

\def \xfs{(x_1,\ldots,x_{n-1})}
\def \ti{\tilde}
\def \wti{\widetilde}


\def \mL{\mathcal{L}}
\def \mW{\mathcal{W}}
\def \mH{\mathcal{H}}
\def \mC{\mathcal{C}}
\def \mE{\mathcal{E}}
\def \mN{\mathcal{N}}
\def \mA{\mathcal{A}}
\def \mB{\mathcal{B}}
\def \mU{\mathcal{U}}
\def \mV{\mathcal{V}}
\def \mF{\mathcal{F}}

\def \R{\mbb R}
\def \N{\mbb N}
\def \Z{\mbb Z}
\def \P{\mbb{P}}
%\def \p{\mbb{P}}
\def \E{\mbb{E}}
\def \D{\msf{D}}
\def \I{\mbf{I}}

\def \a{\alpha}
\def \b{\beta}
\def \t{\tau}
\def \dt{\delta}
\def \e{\varepsilon}
\def \ga{\gamma}
\def \kp{\varkappa}
\def \la{\lambda}
\def \sg{\sigma}
\def \sgm{\sigma}
\def \tt{\theta}
\def \ve{\varepsilon}
\def \Dt{\Delta}
\def \La{\Lambda}
\def \Sgm{\Sigma}
\def \Sg{\Sigma}
\def \Tt{\Theta}
\def \Om{\Omega}
\def \om{\omega}


\def \ni{\noindent}
\def \lq{\glqq}
\def \rq{\grqq}
\def \lbr{\linebreak}
\def \vsi{\vspace{0.1cm}}
\def \vsii{\vspace{0.2cm}}
\def \vsiii{\vspace{0.3cm}}
\def \vsiv{\vspace{0.4cm}}
\def \vsv{\vspace{0.5cm}}
\def \vsvi{\vspace{0.6cm}}
\def \vsvii{\vspace{0.7cm}}
\def \vsviii{\vspace{0.8cm}}
\def \vsix{\vspace{0.9cm}}
\def \VSI{\vspace{1cm}}
\def \VSII{\vspace{2cm}}
\def \VSIII{\vspace{3cm}}


\newcommand{\grad}{\mathrm{grad}}
\newcommand{\dx}[1]{\,\mathrm{d}#1} % для интеграла: маленький отступ и прямая d
\newcommand{\ind}[1]{\mathbbm{1}_{\{#1\}}} % Индикатор события
%\renewcommand{\to}{\rightarrow}
\newcommand{\eqdef}{\mathrel{\stackrel{\rm def}=}}
\newcommand{\iid}{\mathrel{\stackrel{\rm i.\,i.\,d.}\sim}}
\newcommand{\const}{\mathrm{const}}

%на всякий случай пока есть
%теоремы без нумерации и имени
%\newtheorem*{theor}{Теорема}

%"Определения","Замечания"
%и "Гипотезы" не нумеруются
%\newtheorem*{defin}{Определение}
%\newtheorem*{rem}{Замечание}
%\newtheorem*{conj}{Гипотеза}

%"Теоремы" и "Леммы" нумеруются
%по главам и согласованно м/у собой
%\newtheorem{theorem}{Теорема}
%\newtheorem{lemma}[theorem]{Лемма}

% Утверждения нумеруются по главам
% независимо от Лемм и Теорем
%\newtheorem{prop}{Утверждение}
%\newtheorem{cor}{Следствие}


%\usepackage{showkeys} % показывать метки

\input{/home/boris/Dropbox/Public/tex_general/prob_and_sol_utf8}

%\title{Задачи по элементарной теории вероятностей и матстатистике}
%\author{Составитель: Борис Демешев, boris.demeshev@gmail.com}
%\date{\today}

\begin{document}

%\pagestyle{myheadings} \markboth{ТВИМС-задачник. Демешев Борис. roah@yandex.ru }{ТВИМС-задачник. Демешев Борис. roah@yandex.ru }
%\maketitle
%\tableofcontents{}

%\parindent=0 pt % отступ равен 0

\begin{Large}
Моделирование аукционов. Догонялка.
\end{Large}

\begin{enumerate}
%\item Можно пользоваться калькулятором. Вопрос в том, нужно ли?
\item Можно решать задачи в любом порядке.
%\item С собой можно принести один лист А4, где заранее могут быть написаны (именно написаны, а не напечатаны) любые формулы, теоремы или комментарии.
%\item Продолжительность работы 1 час 20 минут.
%\item Условия нельзя забрать с собой. Условия и решения открыто доступны на \url{auctiontheory.wordpress.com} после окончания контрольной.
%\item Обсуждать задачи во время работы нельзя.
%\item Человек проводящий контрольную не будет отвечать на вопросы по тексту задач. 
\item При подозрении на опечатку --- спрашивайте в блоге!
\item Насколько подробно все расписывать --- решай сам исходя из конкретной ситуации. Очевидно, что в примере $ 1+2+3=? $ ответ можно написать сразу, а взятие интеграла $ \int x^{5}\cos(x)dx $ требует каких-то промежуточных записей.
\item Паниковать при решении догонялки строжайше запрещено!
%\item Каждая из 5 задач весит 5 баллов.
\item Для каждой задачи обязательно нужно спрогнозировать свою оценку. Не надо скромничать, лучше попытаться объективно оценить свое решение.  За неверное оценивание баллы снижаться не будут, а верное оценивание даст возможность чему-то научиться. Опыт показывает, что оценка своих собственных решений позволяет резко улучшить их качество. Прогноз своей оценки пишем в табличку!
\item Не забудь подписать свою работу. Пожалуйста!
\item Срок сдачи догонялки: 12 мая, четверг.

\end{enumerate}

\begin{large}
Имя:
\end{large}

\vspace{4pt}

\begin{large}
Отчество:
\end{large}

\vspace{4pt}

\begin{large}
Фамилия:
\end{large}

\vspace{4pt}

\begin{large}
Группа:
\end{large}

\vspace{4pt} 

\begin{tabular}{|c|c|c|c|c|c|c|}
\hline  & Задача 1 & Задача 2 & Задача 3 & Задача 4 & Задача 5 & Итого \\ 
\hline Прогноз оценки &  &  &  &  &  & \\ 
\hline Оценка (от 0 до 5) &  &  &  &  &  & \\ 
\hline 
\end{tabular} 
%\newpage

\begin{enumerate}
\item Кнопка <<Buy now!>>

Два игрока торгуются за товар на кнопочном аукционе с возможностью немедленной покупки товара. Ценности $ X_{i}=V_{i} $ независимы и равномерны на $ [0;1] $. Каждый игрок знает свою ценность $ X_{i} $. Продавец дает игрокам возможность купить товар немедленно по фиксированной цене $ a $. Подробнее. В начале аукциона текущая цена равна нулю и оба игрока жмут на свои кнопки. Текущая цена растет с течением времени. Кто первый отпустил свою кнопку, тот проиграл. В этот момент аукцион заканчивается и победитель получает товар по текущей цене. Но в любой момент пока аукцион не закончился, любой игрок может сказать: <<Покупаю по цене $ a $>>. В этом случае ему достается товар по цене $ a $ и аукцион заканчивается.
\begin{enumerate}
\item Что является стратегией игрока на этом аукционе?
\item Найдите равновесие Нэша
\item Изменится ли ожидаемый доход продавца, если аукцион будет проводится по обычным правилам аукциона второй цены? Применима ли теорема об одинаковой доходности?
\end{enumerate}

\item Есть шесть покупателей. У продавца две чудо-швабры. Каждый покупатель хочет только одну чудо-швабру. Продавец решил продавать эти две чудо швабры путем двух последовательных аукционов первой цены, на каждом из которых будет выставляться одна чудо-швабра. Каждый игрок знает ценность чудо-швабры для себя, $ X_{i}=V_{i} $. Ценности независимы и равномерны на $ [0;1] $. Ценности не меняются со временем. Когда проводится второй аукцион известна только ставка, которую сделал победитель первого.

\begin{enumerate}
\item Что является стратегией игрока в этой игре?
\item Найдите равновесие Нэша
\item Верно ли, что средние цены на обоих аукционах равны?
\item Какова вероятность того, что на первом аукционе цена будет больше, чем на втором?
\item Изменится ли ожидаемый доход продавца, если чудо-швабры будут продаваться на двух последовательных аукционах второй цены? Применима ли в данном случае теорема об одинаковой доходности или ее небольшая вариация?
\end{enumerate}

\item В моделях аукциона первой и второй цены с независимыми, равномерными на $ [0;1] $ ценностями покупателей сравните дисперсию выигрыша продавца. 

%Можно ли сказать сделать какой-то вывод\footnote{Этот вопрос является исследовательским. Возможно он очень легкий или наоборот очень сложный --- не знаю. Оценивается любое продвижение вперед.}  для произвольного регулярного распределения ценностей?

\item Может ли цена расти с ростом предложения?

Рассмотрим кнопочный аукцион, в котором участвуют три игрока. Продавец продает две одинаковых чудо-швабры. Каждому игроку нужна только одна чудо-швабра. Ценность чудо-швабры для всех игроков одинакова и равна $ V=X_{1}+X_{2}+X_{3} $. Каждый из игроков знает только свое $ X_{i} $. Сигналы $ X_{i} $ независимы и имеют регулярное распределение $ F(t) $ на отрезке $ [0;1] $. Чудо-швабры по одной достаются тем игрокам, кто отпустил кнопку позже всех. При этом платят они за нее цену, на которой отпустил кнопку самый слабый игрок.

\begin{enumerate}
\item Найдите равновесие Нэша
\item Найдите равновесие Нэша в случае когда продается всего одна чудо-швабра
\item Существует ли пример распределения $ F(t) $ при котором средняя цена чудо-швабры в  случае двух чудо-швабр выше, чем в случае одной чудо-швабры?
\end{enumerate}


\item Может ли цена расти с падением спроса?

Рассмотрим кнопочный аукцион, в котором хотят участвовать три игрока. Продается одна чудо-швабра. Ценность чудо-швабры для всех игроков одинакова и равна $ V=X_{1}+X_{2}+X_{3} $. Каждый из трех потенциальных игроков знает только свое $ X_{i} $. Сигналы $ X_{i} $ независимы и имеют регулярное распределение $ F(t) $ на отрезке $ [0;1] $. Перед началом аукциона продавец случайным образом выбирает одного игрока и говорит: <<Ты мне не нравишься, поэтому ты в аукционе не участвуешь>>. Оставшиеся двое участвуют в аукционе. 

\begin{enumerate}
\item Найдите равновесие Нэша
\item Существует ли пример распределения $ F(t) $ при котором средняя цена в случае удаления одного из игроков выше, чем в случае когда участвуют все трое желающих?
\end{enumerate}

Hint: Задача 7 из лекции 3




\end{enumerate}




\bibliography{/home/boris/Dropbox/Public/tex_general/opit}
\printindex % печать предметного указателя здесь

\end{document}