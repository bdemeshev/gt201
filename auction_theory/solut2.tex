\begin{enumerate}

\item Техническая задача. 
\begin{enumerate}
\item Известно, что $ f(\vec{x}) $ и $ g(\vec{x}) $ --- супермодулярные функции, а $ a>0 $ и $ b>0 $ --- константы. Верно ли, что $ af(\vec{x})+bg(\vec{x}) $ --- супермодулярная функция?

Да. Проверяем два свойства: 
\begin{enumerate}
\item $ af(\vec{x}) $ --- супермодулярна
\item $ f(\vec{x})+g(\vec{x}) $ --- супермодулярна
\end{enumerate}

\item Пусть $ X_{1} $,\ldots, $ X_{n} $ независимы и имеют функцию плотности $ f(t)=3t^{2} $ на $ [0;1] $. Случайные $ Y_{1} $, \ldots, $ Y_{n-1} $ --- это есть упорядоченные по убыванию случайные величины $ X_{2} $, \ldots, $ X_{n} $. С помощью о-малых или без нее найдите совместную функцию плотности $ f_{Y_{5},Y_{10}}(a,b) $.
\end{enumerate}

Сразу ответ: $ f(a,b)=(n-1)(n-2)C_{n-3}^{4}C_{n-7}^{4}3a^{2}3b^{2}(a^{3}-b^{3})^{4}(b^{3})^{n-11}(1-a^{3})^{4} $

Следующие две задачи очень похожи, разница в них только в типе аукциона\ldots

\item На аукционе первой цены продается участок. Потенциальных покупателей двое. Ценность участка для каждого игрока определяется его площадью. Первый покупатель знает ширину участка $ X_{1} $, а второй --- длину $X_{2}$. Совместная фукнция плотности $ X_{1} $ и $ X_{2} $ имеет вид $ f(x_{1},x_{2})=\frac{7}{8}+\frac{1}{2}x_{1}x_{2} $. Найдите дифференциальное уравнение, которому подчиняется равновесная стратегия игрока. 


Сразу начнем с чудо-замены, $ b_{1}=b(a) $. Сначала упростим событие $ W_{1} $:
\begin{equation}
W_{1}=\{Bid_{2}<b(a)\}=\{b(X_{2})<b(a)\}=\{X_{2}<a\}
\end{equation}

А теперь прибыль:
\begin{multline}
\pi(x,b(a))=\E(X_{1}X_{2}1_{W_{1}}|X_{1}=x)-b(a)\E(1_{W_{1}}|X_{1}=x)=\\
=x\E(X_{2}1_{X_{2}<a}|X_{1}=x)-b(a)\E(1_{X_{2}<a}|X_{1}=x)=\\
=x\int_{0}^{a}x_{2}\frac{f(x,x_{2})}{f(x)}dx_{2}-b(a)\int_{0}^{a}\frac{f(x,x_{2})}{f(x)}dx_{2}
\end{multline}


Сокращаем на $ f(x) $ и берем производную по $ a $:
\begin{equation}
\frac{\partial \pi}{\partial a}=xaf(x,a)-b'(a)\int_{0}^{a}f(x,x_{2})dx_{2}-b(a)f(x,a)=0
\end{equation}


Мы хотим, чтобы оптимальной стратегий первого была $ b_{1}=b(x) $, т.е. чтобы $ a=x $:
\begin{equation}
\frac{\partial \pi}{\partial a}=x^{2}f(x,x)-b'(x)\int_{0}^{x}f(x,x_{2})dx_{2}-b(x)f(x,x)=0
\end{equation}

Остается подставить:
\begin{equation}
\begin{array}{c}
f(x,x)=\frac{7}{8}+\frac{1}{2}x^{2} \\
\int_{0}^{x}f(x,x_{2})dx_{2}=\frac{7}{8}x+\frac{1}{4}x^{3}
\end{array}
\end{equation}


\item На аукционе второй цены продается участок. Потенциальных покупателей двое. Ценность участка для каждого игрока определяется его площадью. Первый покупатель знает ширину участка $ X_{1} $, а второй --- длину $X_{2}$. Совместная фукнция плотности $ X_{1} $ и $ X_{2} $ имеет вид $ f(x_{1},x_{2})=\frac{7}{8}+\frac{1}{2}x_{1}x_{2} $. Найдите равновесие Нэша.

Никакой разницы с кнопочным аукционом в данном случае нет. Игроков-то всего два! Значит равновесие Нэша имеет вид $ b(x)=x^{2} $. 

Доказательство:
Если второй игрок использует такую стратегию и первый выигрывает аукцион, то его прибыль равна:
\begin{equation}
X_{1}X_{2}-X_{2}^{2}=(X_{1}-X_{2})X_{2}
\end{equation}
Мы видим, что прибыль положительна, только если $ X_{1}>X_{2} $. Использование первым игроком функции $ b(x)=x^{2} $ будет обеспечивать его выигрыш только в ситуации $ X_{1}>X_{2} $, значит это и есть равновесие Нэша.


\item Найдите равновесие Нэша в случае кнопочного аукциона. Сигналы $ X_{i} $ игроков имеют совместную функцию плотности $ f(x_{1},x_{2},x_{3})=7/8+x_{1}x_{2}x_{3} $ при $ x_{1},x_{2},x_{3}\in[0;1] $. Ценности определяеются по формулам:
\begin{equation}
\begin{array}{c}
V_{1}=X_{1}(X_{2}+X_{3}) \\
V_{2}=X_{2}(X_{1}+X_{3}) \\
V_{3}=X_{3}(X_{1}+X_{2}) 
\end{array}
\end{equation}

Стратегия описывается двумя функциями: $ b^{3}(x)=2x^{2} $. Если при использовании такой стратегии игрок вышел на цене $ p $, значит его $ x=\sqrt{p/2} $. Получаем $ b^{2}(x,p)=x^{2}+x\sqrt{p/2}$. Доказательство аналогично лекции:

Если остальные игроки используют эти стратегии и первый выигрывает аукцион, то его выигрыш равен:
\begin{multline}
X_{1}(X_{2}+X_{3})-b^{2}(Y_{1},b^{3}(Y_{2}))=\\
=X_{1}(Y_{1}+Y_{2})-(Y_{1}^{2}+Y_{1}Y_{2})=(X_{1}-Y_{1})(Y_{1}+Y_{2})
\end{multline}
Мы видим, что выигрыш положительный, только если $ X_{1}>Y_{1} $. Использование первым игроком правил $ b^{3}() $ и $ b^{2}() $ приводит к выигрышу только если $ X_{1}>Y_{1} $, значит это и есть равновесие.


\item Ценности игроков одинаково распределены, независимы, распределение ценностей дискретно: $ X_{i}$ равновероятно принимает натуральное значение от 1 до 100 включительно. Игроки одновременно делают ставки. Значения всех ценностей общеизвестны всем игрокам еще до ставок! Разрешаются только целые неотрицательные ставки. Товар достается игроку, сделавшему наивысшую ставку. Если таких игроков несколько, то победитель выбирается из них равновероятно. Победитель платит сделанную им ставку. Найдите хотя бы одно равновесие Нэша в чистых стратегиях, в котором игроки не используют нестрого доминируемых стратегий. 


Пример равновесия Нэша. Обозначим $ v_{max} $ --- максимальную ценность и $ v_{sec} $ --- вторую по величине ценность (возможно они совпадают). Те игроки, чья ценность $ v_{i}<v_{max} $ делают ставку $ b_{i}=v_{i}-1 $. Если лидер один, то он делает ставку $ b=v_{sec} $, иначе каждый лидер делает ставку $ b=v_{max}-1 $.


\end{enumerate}
