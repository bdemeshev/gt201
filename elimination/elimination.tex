\documentclass[pdftex,12pt,a4paper]{article}

\input{/home/boris/Dropbox/Public/tex_general/title_bor_utf8}

%\usepackage{showkeys} % показывать метки

%\input{/home/boris/Dropbox/Public/tex_general/prob_and_sol_utf8}

\title{How long does it take to solve an antagonistic game by dominance?}
\author{Boris Demeshev, boris.demeshev@gmail.com}
\date{\today}

\begin{document}

%\pagestyle{myheadings} \markboth{ТВИМС-задачник. Демешев Борис. roah@yandex.ru }{ТВИМС-задачник. Демешев Борис. roah@yandex.ru }
\maketitle
%\tableofcontents{}

%\parindent=0 pt % отступ равен 0





Количество итераций при вычеркивании доминируемых стратегий \\

Под итерацией будем подразумевать вычеркивание всех нестрого доминируемых стратегий за одного игрока. \\

Теорема: в последовательной антагонистической двух игроков игре с $n$ исходами потребуется не более $\left\lfloor \frac{n+1}{2}\right\rfloor$ последовательных итераций. \\
Доказательство: \\
Занумеруем исходы: 1, 2,..., $n$ \\
Исход 1 - худший для первого игрока, худший для второго. \\
Исход $n$ - лучший для первого игрока, худший для второго. \\
Введем следующую терминологию: \\
Игрок 1 форсирует $k$-выигрыш. Если в некоей позиции игрок 1 может гарантировать исход с номером не меньше $k$, то он использует эту возможность. \\
Игрок 2 форсирует $k$-выигрыш. Если в некой позиции игрок 2 может гарантировать исход с номером не больше $k$, то он использует эту возможность. \\
Игрок 1 не делает $k$-ошибок. Если в некоей позиции игрок 1 может гарантировать исход с номером не меньше $k+1$, то он не допустит ошибки, позволяющей игроку 2 форсировать $k$-выигрыш. \\
Игрок 2 не делает $k$-ошибок. Если в некоей позиции игрок 2 может гарантировать исход с номером не больше $k-1$, то он не допустит ошибки, позволяющей игроку 1 форсировать $k$-выигрыш. \\
Собственно доказательство: \\
Итерация 1: \\
Первый игрок вычеркивает: \\
все стратегии, не форсирующие $n$-выигрыш. \\
Второй игрок вычеркивает: \\ 
все стратегии, не форсирующие $1$-выигрыш.\\
все стратегии, допускающие $n$-ошибки. \\
Итерация 2: \\
Первый игрок вычеркивает: \\
все стратегии, не форсирующие $n-1$-выигрыш. \\
все стратегии, допускающие $1$-ошибки. \\
Второй игрок вычеркивает: \\
все стратегии, не форсирующие $2$-выигрыш. \\
все стратегии, допускающие $n-1$-ошибки. \\
...
Итого: как максимум $\left\lfloor\frac{n+1}{2}\right\rfloor$ итераций \\

Данная теорема является обобщением результата полученного в работе Chess-like Games Are Dominance Solvable in at Most Two Steps, Christian Ewerhart, Games and Economic Behavior 33. \\
В работе Ewerhart утверждалось, что в последовательных антагонистических играх с тремя исходами (например, шахматы, шашки...) потребуется не более двух итераций. \\












\bibliography{/home/boris/Dropbox/tex_general/opit}
%\printindex % печать предметного указателя здесь

\end{document}